%-----------------------------------------------------------------------
% Notes on scattering off an embedded body
%-----------------------------------------------------------------------
\documentclass[11pt]{article}
% \usepackage[bookmarks=true]{hyperref}  % this changes the page location !
\usepackage[bookmarks=true,colorlinks=true,linkcolor=blue]{hyperref}

\hbadness=10000 
\sloppy \hfuzz=30pt

\usepackage{calc}
\usepackage[lmargin=.75in,rmargin=.75in,tmargin=.75in,bmargin=.75in]{geometry}

% \input homeHenshaw
\newcommand{\cgDoc}{../../../../cgDoc}% Here is where the cgDoc directory is 
\newcommand{\grad}{\nabla}

\newcommand{\blue}{\color{blue}}
\newcommand{\green}{\color{green}}
\newcommand{\red}{\color{red}}
\newcommand{\black}{\color{black}}


\usepackage{amsmath}
\usepackage{amssymb}

\usepackage{verbatim}
\usepackage{moreverb}

\usepackage{graphics}    
\usepackage{epsfig}    
\usepackage{calc}
\usepackage{ifthen}
\usepackage{float}
% the next one cause the table of contents to disappear!
% * \usepackage{fancybox}

\usepackage{makeidx} % index
\makeindex
\newcommand{\Index}[1]{#1\index{#1}}

\usepackage{tikz}
\input trimFig.tex


% ---- we have lemmas and theorems in this paper ----
\newtheorem{assumption}{Assumption}
\newtheorem{definition}{Definition}


% *** See http://www.eng.cam.ac.uk/help/tpl/textprocessing/squeeze.html
% By default, LaTeX doesn't like to fill more than 0.7 of a text page with tables and graphics, nor does it like too many figures per page. This behaviour can be changed by placing lines like the following before \begin{document}

\renewcommand\floatpagefraction{.99}
\renewcommand\topfraction{.99}
\renewcommand\bottomfraction{.99}
\renewcommand\textfraction{.01}   
\setcounter{totalnumber}{50}
\setcounter{topnumber}{50}
\setcounter{bottomnumber}{50}

\begin{document}

\input wdhDefinitions.tex
\newcommand{\eps}{\epsilon}
\newcommand{\ds}{\Delta s}

% \newcommand{\Bc}{{\mathcal B}}
% \newcommand{\Dc}{{\mathcal D}}
% \newcommand{\Ec}{{\mathcal E}}
% \newcommand{\Fc}{{\mathcal F}}
\newcommand{\Gc}{{\mathcal G}}
% \newcommand{\Hc}{{\mathcal H}}
% \newcommand{\Ic}{{\mathcal I}}
% \newcommand{\Jc}{{\mathcal J}}
% \newcommand{\Lc}{{\mathcal L}}
% \newcommand{\Nc}{{\mathcal N}}
% \newcommand{\Pc}{{\mathcal P}}
% \newcommand{\Rc}{{\mathcal R}}
% \newcommand{\Sc}{{\mathcal S}}

\newcommand{\bogus}[1]{}  % removes is argument completely

\vspace{5\baselineskip}
\begin{flushleft}
{\LARGE
Notes on Electromagnetic Scattering from an Embedded Body \\
using Overture's Cgmx Solver\\
}
\vspace{2\baselineskip}
Jordan Angel,  \\
William D. Henshaw, \\
% 
\vspace{2\baselineskip}
% 
Department of Mathematical Sciences, \\
Rensselaer Polytechnic Institute (RPI), \\
Troy, NY, USA, 12180. \\
\vspace{\baselineskip}
\today\\

\vspace{4\baselineskip}

\noindent{\bf\large Abstract:}

This document describes results from simulating an electromagetic wave scattering from a
body embedded in a dielectric using Overture's {\tt Cgmx} time-domain solver for
Maxwell's equations~\cite{CgmxUserGuide}. 

\end{flushleft}

% \clearpage
\tableofcontents
% \listoffigures

\clearpage


% ================================================================================================================
\section{Problem specification}

The following material is are based on some hand-written notes from Prof. Semyon Tsynkov. 

We consider a {\em plane wave} in the form of a {\em linear chirp} that impinges upon
a body embedded in a horizonal slab of a dielectric. 
If the material surface of the dielectric is the plane $z=0$ and
the angle of incidence from the $z$-axis is $\theta$ then 
the incident field is of the form ({\bf *check me*})
\begin{align*}
  & \Ev_0(x,t) = \hat\av~ P( t + \frac{x}{c} \sin(\theta) + {\frac{z}c} \cos(\theta) ) = \av~P(t - \frac{\hat{\kv}}{c}\cdot\xv), \\
  & \hat\av = [-\cos(\theta),~ 0, ~ \sin(\theta) ], \\
  & \hat{\kv} = [-\sin(\theta),~ 0, ~ -\cos(\theta) ]
\end{align*}
where $c$ is the speed of light in a vacuum and note that $\hat\av \cdot \hat\kv=$ ($\grad\cdot\Ev_0=0$). .

A linear {\em chirp} of unit amplitude and duration $\tau$ is the waveform
\begin{align*}
    & P(t) = \chi(t,\tau) e^{-i \phi(t) } , \\
    & \phi(t) =\omega_0 t +  \alpha t^2, 
\end{align*}
where the {\em indicator} function $\chi(t,\tau)$ is 
\begin{align*}
    & \chi(t) = \begin{cases}
                      1 & \text{ $t\in[-\tau/2,\tau/2$} \\
                      0 & \text{otherwise $t\in[-\tau/2,\tau/2]$} \\
                \end{cases}
\end{align*}
and where 
\begin{alignat*}{3}
&    \omega_0 && \text{: carrier frequency}, \\
&    \alpha = \frac{B}{2\tau} \quad && \text{: chirp rate}, \\
&    B        && \text{: band width}.
\end{alignat*}
The instantaneous frequency is
\[
  \phi'(t) = \omega_0 + 2\alpha t = \omega_0 + \frac{B}{\tau} t \qquad \text{ for $t\in[-\tau/2,\tau/2]$},
\]
and varies linearly along the chirp (hence a {em linear chirp}). 

Many radars use a narrow band chirp, e.g. $B/2\pi = 10MHz$. We will consider the parameters
 \begin{align*}
    & \frac{\omega_0}{2\pi} = 500 MHz - 3 GHz, \\
    &  \frac{B}{2\pi} = 10 MHz, \\
    & \tau = 5 \times 10^{-5} s. 
\end{align*}



% ================================================================================================================
\clearpage
\section{Two dimensional results} \label{sec:twoDimensionalResults}


In two space dimensions we consider a rectangular body of width $W=1$ and height $H=0.5$ with
top surface a distance $D=1/3$ from the surface $y=0$ as shown in Fig.~\ref{fig:embeddedBodyGrid2d}. 

A grid for this domain is shown in Figure~\ref{fig:embeddedBodyGrid2d}
and was generated with Ogen using the command file {\tt embeddedBodyGrid.cmd}. 
Finer grids are placed near the embedded body to more fully resolve the solution there while
coarser grids are used in the far field to keep down artificial reflections from
the far field boundaries. 

Let $\Gc^{(j)}$ denote the grid with resolution factor $j$ which has a grid spacing of approximately $\ds=1/(10 j)$ near 
the body. 

{% ------ 2D GRID ------
\newcommand{\figWidth}{7.5cm}
\newcommand{\trimfig}[2]{\trimFig{#1}{#2}{.0}{.0}{.0}{.0}}
\begin{figure}[htb]
\begin{center}
\begin{tikzpicture}[scale=1]
  \useasboundingbox (0.0,.75) rectangle (16.,7.75);  % set the bounding box (so we have less surrounding white space)
  \draw(0.0,0) node[anchor=south west,xshift=-4pt,yshift=+0pt] {\trimfig{fig/embeddedBodyGride4}{\figWidth}};
  \draw(8.0,0) node[anchor=south west,xshift=-4pt,yshift=+0pt] {\trimfig{fig/embeddedBodyGride4Zoom}{\figWidth}};
 % grid:
 % \draw[step=1cm,gray] (0,0) grid (16,7.5);
\end{tikzpicture}
\end{center}
  \caption{Composite $\Gc^{(4)}$ for the two-dimensional rectangular body embedded in a dielectric. The red and
green grids cover the top material ($\eps=1$) while the other grids cover the bottom material $\eps=6$.}
  \label{fig:embeddedBodyGrid2d}
\end{figure}
}

Fig.~\ref{fig:embeddedBodyFields} shows preliminary results for a plane wave (no chirp). 
The top material has $\eps=1$ (the speed of light is $1$) while the bottom
dielectric was chosen to have $\eps=6$.  
The domain was periodic in $x$ and the plane wave wave solution was specified as a time-dependent
condition on the top boundary. The bottom bounday is an ABC. 



{% ------ Fields   ------
\newcommand{\figWidth}{7.75cm}
\newcommand{\trimfig}[2]{\trimFigb{#1}{#2}{.0}{.0}{.2}{.05}}
\begin{figure}[htb]
\begin{center}
\begin{tikzpicture}[scale=1]
  \useasboundingbox (0.0,.75) rectangle (16.,14.5);  % set the bounding box (so we have less surrounding white space)
  \draw(0.0, 7) node[anchor=south west,xshift=-4pt,yshift=+0pt] {\trimfig{fig/ebG8Order4Angle60Ext10p0}{\figWidth}};
  \draw(8.0, 7) node[anchor=south west,xshift=-4pt,yshift=+0pt] {\trimfig{fig/ebG8Order4Angle60Eyt10p0}{\figWidth}};
%  
  \draw(0.0, 0) node[anchor=south west,xshift=-4pt,yshift=+0pt] {\trimfig{fig/ebG8Order4Angle60Ext20p0}{\figWidth}};
  \draw(8.0, 0) node[anchor=south west,xshift=-4pt,yshift=+0pt] {\trimfig{fig/ebG8Order4Angle60Eyt20p0}{\figWidth}};
%

% grid:
% \draw[step=1cm,gray] (0,0) grid (16,14.5);
\end{tikzpicture}
\end{center}
  \caption{Preliminary results for a plane wave (no chirp). $E_x$ and $E_y$ fields at time $t=10$ (top)
   and $t=20$ bottom. Results for $\Gc^{(8)}$ (periodic BC's in x). }
  \label{fig:embeddedBodyFields}
\end{figure}
}


% ================================================================================================================
\clearpage
\section{Three dimensional results} \label{sec:threeDimensionalResults}



\clearpage
\bibliography{\cgDoc/common/henshaw,\cgDoc/common/henshawPapers}
\bibliographystyle{siam}



\end{document}
% **************************************************************************************************************
% **************************************************************************************************************
% **************************************************************************************************************
% **************************************************************************************************************


