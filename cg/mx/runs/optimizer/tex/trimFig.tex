% -------------------------------------------------------------------------------------
%
% Trim a figure: **NEW VERSION**
%  In this new version we hard code the original width of the plot to match an Overture figure.
%  The clipping parameters then denote fractions of this width (width=7.5 in = 19.05 cm)
%
%  \trimPlot{file}{width}{xa}{xb}{ya}{yb}
% Trim a figure and show the bounding box:
%  \trimPlotb{file}{width}{xa}{xb}{ya}{yb}
% 
%  file = name of the graphics file for includegraphics : .eps or pdf or ...
%  width : final width (after trimming) of the figure e.g. 5cm
%  xa,xb,ya,yb : trim this fraction of the width from the left, right, bottom and top
% 
% -----------------------------------------------------------------------------------------
\newlength{\tfwidth}
\newlength{\tfheight}
\newlength{\tfxa}
\newlength{\tfxb}
\newlength{\tfya}
\newlength{\tfyb}
% 
\newcommand{\trimPlotWithBox}[6]{%
\setlength\fboxsep{0pt}%
\setlength\fboxrule{1.0pt}% border thickness
\fbox{\includegraphics[width=#2, clip, trim=#3 #4 #5 #6]{#1}}%
}
\newcommand{\trimPlotNoBox}[6]{%
\setlength\fboxsep{1pt}% note: make this 1pt and rule thickness zero so box size matches that below
\setlength\fboxrule{0.0pt}% border thickness
\fbox{\includegraphics[width=#2, clip, trim=#3 #4 #5 #6]{#1}}%
% \includegraphics[width=#2, clip, trim=#3 #4 #5 #6]{#1}
}
% Trim a figure and draw the bounding box 
% Overture figures are 7.5 inches wide (8.5 - .5 -.5) = 19.05 cm
\newcommand{\trimPlotb}[6]{%
\setlength{\tfwidth}{19.05cm}%   % width of un-clipped fig
\setlength{\tfxa}{\tfwidth*\real{#3}}%
\setlength{\tfxb}{\tfwidth*\real{#4}}%
\setlength{\tfya}{\tfwidth*\real{#5}}%
\setlength{\tfyb}{\tfwidth*\real{#6}}%
% trim=l b r t
\trimPlotWithBox{#1}{#2}{\tfxa}{\tfya}{\tfxb}{\tfyb}%
% \trimPlotWithBox{#1}{#2}{#3}{#4}{#5}{#6}%
}
% Trim a figure 
% 
\newcommand{\trimPlot}[6]{%
\setlength{\tfwidth}{19.05cm}%   % width of un-clipped fig
\setlength{\tfxa}{\tfwidth*\real{#3}}%
\setlength{\tfxb}{\tfwidth*\real{#4}}%
\setlength{\tfya}{\tfwidth*\real{#5}}%
\setlength{\tfyb}{\tfwidth*\real{#6}}%
% trim=l b r t
\trimPlotNoBox{#1}{#2}{\tfxa}{\tfya}{\tfxb}{\tfyb}%
}
% ----------------------------------------------------------------------------------------------
% -------------------------------------------------------------------------------------
%
% Trim a figure: 
%  \trimFig{file}{width}{xa}{xb}{ya}{yb}
% Trim a figure and show the bounding box:
%  \trimFigb{file}{width}{xa}{xb}{ya}{yb}
% 
%  file = name of the graphics file for includegraphics : .eps or pdf or ...
%  width : final width (after trimming) of the figure e.g. 5cm
%  xa,xb,ya,yb : trim this fraction of the width from the left, right, bottom and top
% 
% -----------------------------------------------------------------------------------------
\newcommand{\trimFigWithBox}[6]{%
\setlength\fboxsep{0pt}%
\setlength\fboxrule{1.0pt}% border thickness
\fbox{\includegraphics[width=#2, clip, trim=#3 #4 #5 #6]{#1}}%
}
% -----------------------------------------------------------------------------------------
\newcommand{\trimFigNoBox}[6]{%
\setlength\fboxsep{1pt}% note: make this 1pt and rule thickness zero so box size matches that below
\setlength\fboxrule{0.0pt}% border thickness
\fbox{\includegraphics[width=#2, clip, trim=#3 #4 #5 #6]{#1}}%
% \includegraphics[width=#2, clip, trim=#3 #4 #5 #6]{#1}
}
% ---------- Set height -----
\newcommand{\trimFigHeightWithBox}[6]{%
\setlength\fboxsep{0pt}%
\setlength\fboxrule{1.0pt}% border thickness
\fbox{\includegraphics[height=#2, clip, trim=#3 #4 #5 #6]{#1}}%
}
% ---------- Set height -----
\newcommand{\trimFigHeightNoBox}[6]{%
\setlength\fboxsep{1pt}% note: make this 1pt and rule thickness zero so box size matches that below
\setlength\fboxrule{0.0pt}% border thickness
\fbox{\includegraphics[height=#2, clip, trim=#3 #4 #5 #6]{#1}}%
% \includegraphics[width=#2, clip, trim=#3 #4 #5 #6]{#1}
}
% Trim a figure and draw the bounding box 
\newcommand{\trimFigb}[6]{%
\setlength{\tfwidth}{(#2+#2*\real{#3})+#2*\real{#4}}%   % width of un-clipped fig
\setlength{\tfheight}{(#2+#2*\real{#5})+#2*\real{#6}}%
\setlength{\tfxa}{\tfwidth*\real{#3}}%
\setlength{\tfxb}{\tfwidth*\real{#4}}%
\setlength{\tfya}{\tfheight*\real{#5}}%
\setlength{\tfyb}{\tfheight*\real{#6}}%
% trim=l b r t
\trimFigWithBox{#1}{#2}{\tfxa}{\tfya}{\tfxb}{\tfyb}%
}
% Trim a figure 
% 
\newcommand{\trimFig}[6]{%
\setlength{\tfwidth}{(#2+#2*\real{#3})+#2*\real{#4}}%   % width of un-clipped fig
\setlength{\tfheight}{(#2+#2*\real{#5})+#2*\real{#6}}%
\setlength{\tfxa}{\tfwidth*\real{#3}}%
\setlength{\tfxb}{\tfwidth*\real{#4}}%
\setlength{\tfya}{\tfheight*\real{#5}}%
\setlength{\tfyb}{\tfheight*\real{#6}}%
% trim=l b r t
\trimFigNoBox{#1}{#2}{\tfxa}{\tfya}{\tfxb}{\tfyb}%
}

%%%%%%%%%%%%%% TRIMW : trim and set width %%%%%%%%%%%%%%%%%%
%%%%%%%%%%%%%% THIS VERSION WORKS PROPERLY %%%%%%%%%%%%%%%%%%
%  Thanks to Longfei Li
% New:
% Trim a figure (working) 
%  trims #3 #4 #5 #6 percent of left, right, bottom, top of the original figure
\newsavebox\figBox

\newcommand{\trimw}[6]{%
%20160728 use \sbox to save one include graphics
\sbox\figBox{\includegraphics{#1}}
\setlength{\tfwidth}{\the\wd\figBox}
\setlength{\tfheight}{\the\ht\figBox}
\setlength{\tfxa}{\tfwidth*\real{#3}}%
\setlength{\tfxb}{\tfwidth*\real{#4}}%
\setlength{\tfya}{\tfheight*\real{#5}}%
\setlength{\tfyb}{\tfheight*\real{#6}}%
\trimFigNoBox{#1}{#2}{\tfxa}{\tfya}{\tfxb}{\tfyb}%
}

% ----------------------------------------------------------------------------------------------
%%%%%%%%%%%%%% TRIMWB : trim and set width %%%%%%%%%%%%%%%%%%
%%%%%%%%%%   Version of trim that also draws the bounding box %%%%%%%%%%%%
\newcommand{\trimwb}[6]{%
%old: 
%\setlength{\tfwidth}{(#2+#2*\real{#3})+#2*\real{#4}}%   % width of un-clipped fig
%\setlength{\tfheight}{(#2+#2*\real{#5})+#2*\real{#6}}%
%new: Longfei 20160727: we should get the size of the original picture:
%\settowidth\tfwidth{\includegraphics{#1}}
%\settoheight\tfheight{\includegraphics{#1}}

%20160728 use \sbox to save one include graphics
\sbox\figBox{\includegraphics{#1}}
\setlength{\tfwidth}{\the\wd\figBox}
\setlength{\tfheight}{\the\ht\figBox}
\setlength{\tfxa}{\tfwidth*\real{#3}}%
\setlength{\tfxb}{\tfwidth*\real{#4}}%
\setlength{\tfya}{\tfheight*\real{#5}}%
\setlength{\tfyb}{\tfheight*\real{#6}}%
\trimFigWithBox{#1}{#2}{\tfxa}{\tfya}{\tfxb}{\tfyb}%
}

% ----------------------------------------------------------------------------------------------
%%%%%%%%%%%%%% TRIMH : trim and set height %%%%%%%%%%%%%%%%%%
%%%%%%%%%%%%%% THIS VERSION WORKS PROPERLY %%%%%%%%%%%%%%%%%%
%  Thanks to Longfei Li
% New:
% Trim a figure (working) 
%  trims #3 #4 #5 #6 percent of left, right, bottom, top of the original figure
\newcommand{\trimh}[6]{%
%20160728 use \sbox to save one include graphics
\sbox\figBox{\includegraphics{#1}}
\setlength{\tfwidth}{\the\wd\figBox}
\setlength{\tfheight}{\the\ht\figBox}
\setlength{\tfxa}{\tfwidth*\real{#3}}%
\setlength{\tfxb}{\tfwidth*\real{#4}}%
\setlength{\tfya}{\tfheight*\real{#5}}%
\setlength{\tfyb}{\tfheight*\real{#6}}%
\trimFigHeightNoBox{#1}{#2}{\tfxa}{\tfya}{\tfxb}{\tfyb}%
}

% ----------------------------------------------------------------------------------------------
%%%%%%%%%%%%%% TRIMWB : trim and set height %%%%%%%%%%%%%%%%%%
%%%%%%%%%%   Version of trim that also draws the bounding box %%%%%%%%%%%%
\newcommand{\trimhb}[6]{%
%old: 
%\setlength{\tfwidth}{(#2+#2*\real{#3})+#2*\real{#4}}%   % width of un-clipped fig
%\setlength{\tfheight}{(#2+#2*\real{#5})+#2*\real{#6}}%
%new: Longfei 20160727: we should get the size of the original picture:
%\settowidth\tfwidth{\includegraphics{#1}}
%\settoheight\tfheight{\includegraphics{#1}}

%20160728 use \sbox to save one include graphics
\sbox\figBox{\includegraphics{#1}}
\setlength{\tfwidth}{\the\wd\figBox}
\setlength{\tfheight}{\the\ht\figBox}
\setlength{\tfxa}{\tfwidth*\real{#3}}%
\setlength{\tfxb}{\tfwidth*\real{#4}}%
\setlength{\tfya}{\tfheight*\real{#5}}%
\setlength{\tfyb}{\tfheight*\real{#6}}%
\trimFigHeightWithBox{#1}{#2}{\tfxa}{\tfya}{\tfxb}{\tfyb}%
}
% ----------------------------------------------------------------------------------------------
% 
