%-----------------------------------------------------------------------
% Simulating incompressible flow through a periodic channel
%-----------------------------------------------------------------------
\documentclass[11pt]{article}
% \usepackage[bookmarks=true]{hyperref}  % this changes the page location !
\usepackage[bookmarks=true,colorlinks=true,linkcolor=blue]{hyperref}

% \input documentationPageSize.tex
\hbadness=10000 
\sloppy \hfuzz=30pt

% \voffset=-.25truein
% \hoffset=-1.25truein
% \setlength{\textwidth}{7in}      % page width
% \setlength{\textheight}{9.5in}    % page height

\usepackage{calc}
\usepackage[lmargin=.75in,rmargin=.75in,tmargin=.75in,bmargin=.75in]{geometry}

\input homeHenshaw

% \input{pstricks}\input{pst-node}
% \input{colours}
\newcommand{\blue}{\color{blue}}
\newcommand{\green}{\color{green}}
\newcommand{\red}{\color{red}}
\newcommand{\black}{\color{black}}


\usepackage{amsmath}
\usepackage{amssymb}

\usepackage{verbatim}
\usepackage{moreverb}

\usepackage{graphics}    
\usepackage{epsfig}    
\usepackage{calc}
\usepackage{ifthen}
\usepackage{float}
% the next one cause the table of contents to disappear!
% * \usepackage{fancybox}

\usepackage{makeidx} % index
\makeindex
\newcommand{\Index}[1]{#1\index{#1}}

\usepackage{tikz}
% \usepackage{pgfplots}
\input trimFig.tex


% ---- we have lemmas and theorems in this paper ----
\newtheorem{assumption}{Assumption}
\newtheorem{definition}{Definition}

% \newcommand{\homeHenshaw}{/home/henshaw.0}

\newcommand{\Overture}{{\bf Over\-ture\ }}
\newcommand{\ogenDir}{\homeHenshaw/Overture/ogen}

\newcommand{\cgDoc}{\homeHenshaw/cgDoc}
\newcommand{\vpDir}{\homeHenshaw/cgDoc/ins/viscoPlastic}

\newcommand{\ovFigures}{\homeHenshaw/OvertureFigures}
\newcommand{\obFigures}{\homeHenshaw/res/OverBlown/docFigures}  % for figures
\newcommand{\convDir}{\homeHenshaw/cgDoc/ins/tables}
\newcommand{\insDocDir}{\homeHenshaw/cgDoc/ins}

% *** See http://www.eng.cam.ac.uk/help/tpl/textprocessing/squeeze.html
% By default, LaTeX doesn't like to fill more than 0.7 of a text page with tables and graphics, nor does it like too many figures per page. This behaviour can be changed by placing lines like the following before \begin{document}

\renewcommand\floatpagefraction{.99}
\renewcommand\topfraction{.99}
\renewcommand\bottomfraction{.99}
\renewcommand\textfraction{.01}   
\setcounter{totalnumber}{50}
\setcounter{topnumber}{50}
\setcounter{bottomnumber}{50}

\begin{document}

\input wdhDefinitions.tex

\def\comma  {~~~,~~}
\newcommand{\uvd}{\mathbf{U}}
\def\ud     {{    U}}
\def\pd     {{    P}}
\def\calo{{\cal O}}

\newcommand{\mbar}{\bar{m}}
\newcommand{\Rbar}{\bar{R}}
\newcommand{\Ru}{R_u}         % universal gas constant
% \newcommand{\Iv}{{\bf I}}
% \newcommand{\qv}{{\bf q}}
\newcommand{\Div}{\grad\cdot}
\newcommand{\tauv}{\boldsymbol{\tau}}
\newcommand{\thetav}{\boldsymbol{\theta}}
% \newcommand{\omegav}{\mathbf{\omega}}
% \newcommand{\Omegav}{\mathbf{\Omega}}

\newcommand{\Omegav}{\boldsymbol{\Omega}}
\newcommand{\omegav}{\boldsymbol{\omega}}
\newcommand{\sigmav}{\boldsymbol{\sigma}}
\newcommand{\cm}{{\rm cm}}

\newcommand{\ds}{\Delta s}
\newcommand{\dsbl}{\ds_{\rm bl}}


\newcommand{\sumi}{\sum_{i=1}^n}
% \newcommand{\half}{{1\over2}}
\newcommand{\dt}{{\Delta t}}

\def\ff {\tt} % font for fortran variables

% define the clipFig commands:
%% \input clipFig.tex

\newcommand{\Bc}{{\mathcal B}}
\newcommand{\Dc}{{\mathcal D}}
\newcommand{\Ec}{{\mathcal E}}
\newcommand{\Fc}{{\mathcal F}}
\newcommand{\Gc}{{\mathcal G}}
\newcommand{\Hc}{{\mathcal H}}
\newcommand{\Ic}{{\mathcal I}}
\newcommand{\Jc}{{\mathcal J}}
\newcommand{\Lc}{{\mathcal L}}
\newcommand{\Nc}{{\mathcal N}}
\newcommand{\Pc}{{\mathcal P}}
\newcommand{\Rc}{{\mathcal R}}
\newcommand{\Sc}{{\mathcal S}}

\newcommand{\bogus}[1]{}  % removes is argument completely

\vspace{5\baselineskip}
\begin{flushleft}
{\LARGE
Simulating Flow in a Periodic Channel using verture's Cgins Solver\\
}
\vspace{2\baselineskip}
Kyle. K. Chand,    \\
Lawrence Livermore National Laboratory      \\
Livermore, CA, 94551  \\
~ \\
William D. Henshaw, \\
Dept. of Mathematical Sciences, \\
RPI
\vspace{\baselineskip}
\today\\
\vspace{\baselineskip}
\vspace{4\baselineskip}

\noindent{\bf\large Abstract:}

This document includes some results from computing flow through a periodic channel.

\end{flushleft}

% \clearpage
\tableofcontents
% \listoffigures

\clearpage


\subsection{Flow in periodic channel}\label{sec:channelFlow}

The flow through a periodic channel is simulated with the AFS24 scheme.
Scripts for this case can be found in {\tt cg/ins/runs/periodicChannel}. 
The flow is driven by a pressure gradient. An initial perturbation is
added to excite the turbulence. The computations on finer
grids use coarser grid solutions as initial conditions.
The channel covers the domain $[-1,1]\times[0,2\pi]$. 
The grid $G^{(j)}$ for the channel has $64 j$ grid points in the axial direction. The gridlines
are stretched near the upper and lower walls by a factor of $5$. The viscosity is $\nu=10^{-5}$. 
For grid $G^{(16)}$ the finest grid-spacing normal to wall is apprximately 1.22e-3. 

{
\begin{figure}[hbt]
\newcommand{\figWidtha}{8.5cm}
\newcommand{\trimfig}[2]{\trimFig{#1}{#2}{0}{.21}{.45}{.40}}
\begin{center}
 \begin{tikzpicture}[scale=1]
 \useasboundingbox (0,-.75) rectangle (18,12.5);  % set the bounding box (so we have less surrounding white space)
%
 \draw (0,9.0) node[anchor=south west] {\trimfig{\insDocDir/fig/pChannel128Vorticity}{\figWidtha}};
 \draw (9,9.0) node[anchor=south west] {\trimfig{\insDocDir/fig/pChannel128StreamLines}{\figWidtha}};
 \draw (9,12.0) node[draw,fill=white,anchor=south] {\scriptsize Coarse grid: $G^{(2)}$};
%
 \draw (0,5.5) node[anchor=south west] {\trimfig{\insDocDir/fig/pChannel256Vorticity}{\figWidtha}};
 \draw (9,5.5) node[anchor=south west] {\trimfig{\insDocDir/fig/pChannel256StreamLines}{\figWidtha}};
 \draw (9,8.5) node[draw,fill=white,anchor=south] {\scriptsize Medium grid: $G^{(4)}$};
%
 \draw (0,2.0) node[anchor=south west] {\trimfig{\insDocDir/fig/pChannel512Vorticity}{\figWidtha}};
 \draw (9,2.0) node[anchor=south west] {\trimfig{\insDocDir/fig/pChannel512StreamLines}{\figWidtha}};
 \draw (9,5.0) node[draw,fill=white,anchor=south] {\scriptsize Fine grid: $G^{(8)}$};
%
\begin{scope}[yshift=-3.5cm]
 \draw (0,2.0) node[anchor=south west] {\trimfig{\insDocDir/fig/pChannel1024Vorticity}{\figWidtha}};
 \draw (9,2.0) node[anchor=south west] {\trimfig{\insDocDir/fig/pChannel1024StreamLines}{\figWidtha}};
 \draw (9,5.0) node[draw,fill=white,anchor=south] {\scriptsize Finer grid: $G^{(16)}$};
\end{scope}
%
%
%   \draw[step=1cm,gray] (0,0) grid (18,12);
%   \draw (current bounding box.south west) rectangle (current bounding box.north east);
 \end{tikzpicture}
\end{center}
\caption{Turbulent flow through a periodic channel on grids of increasing resolution. Left column: vorticity.
    Right column: streamlines. The vorticity is shown on the scale $[-5,5]$.} 
\label{fig:periodicChannelFlow}
\end{figure}
}

Commands to generate grids:
\begin{verbatim}
   ogen -noplot channelArg -order=4 -factor=2 -ml=2
   ogen -noplot channelArg -order=4 -factor=4 -ml=3
   ogen -noplot channelArg -order=4 -factor=8 -ml=4
   ogen -noplot channelArg -order=4 -factor=16 -ml=5
   ogen -noplot channelArg -order=4 -factor=32 -ml=5
   ogen -noplot channelArg -order=4 -factor=64 -ml=6
\end{verbatim}

Commands to run cgins:
\begin{verbatim}
Coarse grid (Yale solver)
nohup $ins/bin/cgins -noplot pChannel.cmd -g=channel2.order4.ml2.hdf -nu=1e-5 -order=4 -cfl=2 -ystr=14 -ad2=0 -ad4=1 -ad4l=0.5 -ad4n=0.5 -debug=3 -bc=noSlipWall -solver=yale -nullVector=channel2ml2_NullVector.hdf -show=pChannel2da.show -tp=1 -tf=200 -debug=3 -nopause >! pChannela.out &

finer, yale restart 
nohup $ins/bin/cgins -noplot pChannel.cmd -g=channel4.order4.ml3.hdf -nu=1e-5 -order=4 -cfl=2 -ystr=14 -ad2=0 -ad4=1 -ad4l=0.5 -ad4n=0.5 -debug=3 -bc=noSlipWall -solver=yale -nullVector=channel4ml3_NullVector.hdf -restart=pChannel2da.show -show=pChannel2db.show -tp=1 -tf=300 -debug=3 -nopause >! pChannelb.out &

even finer, yale restart 
nohup $ins/bin/cgins -noplot pChannel.cmd -g=channel8.order4.ml4.hdf -nu=1e-5 -order=4 -cfl=2 -ad2=0 -ad4=1 -ad4l=0.5 -ad4n=0.5 -debug=3 -bc=noSlipWall -solver=yale -nullVector=channel8ml4_NullVector.hdf -restart=pChannel2db.show -show=pChannel2dc.show -tp=1 -tf=400 -debug=3 -nopause >! pChannelc.out &

\end{verbatim}


% -------------------------------------------------------------------------------------------------
%\clearpage
%\bibliography{\homeHenshaw/papers/henshaw}
%\bibliographystyle{siam}


\printindex


\end{document}
