{% ------ BODY COMPARE TO TP ------
\newcommand{\figWidth}{7.cm}
\newcommand{\trimfig}[2]{\trimw{#1}{#2}{.0}{.0}{.0}{.0}}
\begin{figure}[htb]
\begin{center}
\resizebox{14cm}{!}{% START resize box
\begin{tikzpicture}[scale=1]
  \useasboundingbox (0.0,1) rectangle (16.,5.75);  % set the bounding box (so we have less surrounding white space)
%
   \draw(0.0,0) node[anchor=south west,xshift=-4pt,yshift=+0pt] {\trimfig{fig/risingDropPositionCompareTP}{\figWidth}};
   \draw(8.0,0) node[anchor=south west,xshift=-4pt,yshift=+0pt] {\trimfig{fig/risingDropVelocityCompareTP}{\figWidth}};
%%   \draw(4.,0) node[anchor=south west,xshift=-4pt,yshift=+0pt] {\trimfig{fig/risingDropAccelerationCompareTP}{\figWidth}};
%
%
% grid:
% \draw[step=1cm,gray] (0,0) grid (16,6);
\end{tikzpicture}
}% END resize box
\end{center}
  \caption{Rising disk in a counter-flow. Comparison of the position (left) and velocity (right) of solutions
  computed using the~\ampRB~scheme and the TP-RB scheme (with sub-time-step iterations) for the grid $\Gcrd^{(4)}$.
   % Results for grid G4 are in colour. Results for grid G8 are in black. 
%    Horizontal components of the displacement, $x_1$, velocity $v_1$ and acceleration $a_1$ of the rigid body.
%     Grid $\Gc^{(2)}$. 
% \dws{(As before, I suggest changing $\nu=1$ to $\mu=1$, $\theta_3$ to $\theta_b$ and $\omega_3$ to $\omega_b$ to be consistent with earlier changes.)}
     }
  \label{fig:risingLightDiskCompareTP}
\end{figure}
}
