% Falling disk settling onto a bottom wall
% \newcommand{\Gcrd}{\Gc_{\rm rd}}% Grid 
%\clearpage
\definecolor{jwbGreen}{rgb}{0, .6, 0}
\newcommand{\ampRB}{AMP-RB}% name of scheme
\newcommand{\old}[1]{{\color{red}old: #1}}
\newcommand{\jwb}[1]{{\color{jwbGreen}#1}}
\newcommand{\Gcrd}{\Gc_{\rm{rd}}}% grid for the rotating disk 
\newlength{\ycbTop}% For colour bar
\newlength{\ycbMid}%
\newcommand{\channelWidth}{{x_c}}
\newcommand{\profileWidth}{{x_p}}
\newcommand{\channelBottom}{{y_0}}
\newcommand{\channelTop}{{y_1}}
\newcommand{\diskRadius}{{R_b}}
\newcommand{\rampFunction}{{\cal R}}
\newcommand{\f}[2]{\frac{#1}{#2}}% \frac short form
\newcommand{\fvbe}{\fv_e}% extranal force on body
\newcommand{\gvbe}{\gv_e}% external torque on body


\subsection{Rising disk in a counter-flow} \label{sec:diskDrop}

\noindent{\blue Note: this section was dropped from the rbins paper.}
\medskip

Now consider a buoyant and light rigid disk translating and rotating in a vertical channel.
This problem is used to evaluate the~\ampRB~scheme for a light solid
undergoing significant translation and 
rotation. Due to these motions both added-mass and added-damping
effects are present, and their relative importance changes as the flow field evolves.
 A downward flow is imposed
to keep the disk close to the central portion of the computational domain over an extended period of time.

% ------------------ GRID FIGURE ----
\input tex/risingDiskGridFig
% ---------- STREAM LINES -------
\input tex/risingDiskStreamlinesFig


\input tex/risingDiskCurvesFig


The geometry of the problem is shown in Figure~\ref{fig:risingDiskGrid}.  The fluid channel
covers the rectangular domain $[-\channelWidth,\channelWidth]\times[\channelBottom,\channelTop]$,
where $\channelWidth=5/4$, $\channelBottom=-8$ and $\channelTop=6$.  
The solid disk, with radius $\diskRadius=1/2$, is initially centred at $(x,y)=(.5,.5)$. 
% The radius of the
% solid disk is $\diskRadius=1/2$ and it is centred at $(x,y)=(1/2,1/2)$ initially.
The flow in the channel and the motion of the solid disk are initiated from
rest by smoothly turning on a downward flow at the top of the channel and  smoothly turning on a body
force on the solid due to gravity.
%To avoid a non-smooth impulsive start (which would likely result in at best first-order accurate results), 
%the initial conditions are taken as zero for all variables and
%gravity and the inflow conditions are smoothly ramped
%over time using the ramp function~\eqref{eq:ramp}. 
At the top of the channel, a parabolic-like inflow profile is
specified as
\begin{align*}
    u(x,\channelTop,t)=0,\qquad 
    v(x,\channelTop,t) = V_0\,\rampFunction(t)\,
                \begin{cases} 
                    \f{(x+\channelWidth)^2}{(\channelWidth-\profileWidth)^2},   & \text{for $x\in[-\channelWidth,-\profileWidth]$}, \\
                    1    ,                      & \text{for $|x|< \profileWidth$}, \\
                    \f{(x-\channelWidth)^2}{(\channelWidth-\profileWidth)^2},   & \text{for $x\in[\profileWidth,\channelWidth]$},
                 \end{cases}
\end{align*}
where $\profileWidth=1/2$ measures the width of the uniform-velocity core of the profile, $\rampFunction(t)$ is the ramp function given in~\eqref{eq:ramp}, and the downward
velocity at inflow is taken as $V_0=-1.15$. 
No-slip boundary conditions are taken on the left and right sides of the channel, and on
the surface of the solid disk.  
An outflow boundary condition is taken at the bottom boundary of the channel, with the
pressure set to zero and the tangential components of the velocity extrapolated.
The fluid density and viscosity are taken to be $\rho=1$ and $\mu=1$, respectively, and the
density of the solid disk is $\rhob=0.001$.  The body force due to gravity is given by
\[
\fvbe(t)=\pi\diskRadius\sp2(\rho_b-\rho) \, \gv \, \rampFunction(t),
\]
where the acceleration due to gravity is taken to be $\gv=[0,\,-10]\sp{T}$ for the present simulation.
%For the simulations shown, gravity is downward with a magnitude of $g=10$. The viscosity 
%is taken as $\nu=1$, the fluid densiy is $\rho=1$ and the body density is $\rhob=10^{-3}$. 



The composite grid for the problem at $t=0$ is shown in Figure~\ref{fig:risingDiskGrid}.
The grid with resolution factor~$j$ is denoted by~$\Gcrd^{(j)}$, and 
consists of six component grids. Portions of the inlet and outlet regions
of the fluid domain are covered with coarser grids that span the channel.
These grids have grid points clustered near the vertical side walls, and they are also stretched
in the vertical direction to coarsen the grid away from the central portion of the computational domain.
Finer grids are used near the disk and they consist of two boundary layer grids,
a Cartesian background grid, and a stretched annular grid around the disk.  This latter
component grid translates and rotates with the solid disk.



Figure~\ref{fig:risingDiskStreamLines} shows (instantaneous) streamlines of the solution at a sequence
of times and the evolution of the centre of the solid disk.  This figure is complemented by the plots in Figure~\ref{fig:risingDropCurves} which show the position, velocity and acceleration of the body over time.
At early times, the disk experiences a torque due to the larger downward velocity in the central portion of the channel
and the resulting larger viscous shear on the surface of the disk.  This torque causes the disk to spin in a
counter-clockwise direction at early times as shown by the angular displacement in the top plot of Figure~\ref{fig:risingDropCurves}.  The fluid pressure is greater near the wall, and this tends to push the solid towards
the centre of the channel.  The upward force on the body due to its buoyancy approximately balances the viscous drag on
the disk so that it remains close to its initial vertical position.  It is interesting to observe that the body
experiences a decaying oscillation in both the horizontal position and its angular displacement (see Figure~\ref{fig:risingDropCurves}).
%and beings to spin in a counter-clockwise directon.
%At the same time, the disk beings to rise and move toward the centre of the channel. 
%Eventually the disk begins to move slowly upward as indicated by $y_b$ (recall that there is a downward flow), with a slight
%vertical oscillation (see $v_{b2}$). 
%The disk also settles into a slowly decaying left-right oscillation about the centre of the channel as indicated by $x_b$ and $v_{b1}$. 


% Figure~\ref{fig:risingDropPositionCurves}, \ref{fig:risingDropVelocityCurves}, and~\ref{fig:risingDropAccelerationCurves}.

%The plots of position, velocity and acceleration in Figure~\ref{fig:risingDropCurves}).
The plots of position, velocity and acceleration in Figure~\ref{fig:risingDropCurves}, and their corresponding
enlarged views, indicate the
convergence behaviour of the solutions computed using the composite grids, $\Gcrd^{(j)}$, $j=1$, 2, 4, 8 and~16.
%The results of a grid convergence study on the rigid-body variables are presented in Figure~\ref{fig:risingDropCurves}.
For all variables plotted, 
As noted previously for the rising rectangular solid in Section~\ref{sec:fallingBody}, the curves for acceleration are not as smooth as those for the position and velocity.
This is due to perturbations introduced by changes in the connectivity of the composite grid and by abrupt change in the time-step.  The amplitude of these perturbations decrease as the grid is refined.
It is noted that for later times, the vertical position of the disk, $y_b$, converges  slowly.
The vertical position is not strongly constrained by the geometry and it can slowly drift at later times due to small errors
at earlier times. 

% ------- compare to TP-SI ----------------
\input tex/risingDiskCompareTPFig


% \dws{(I am wondering whether this last plot is needed since we showed in the previous section that
% the results of the two schemes agree.  The value here just reconfirms the earlier result, although
% the 2000 sub-iteration is amusing.)}

Figure~\ref{fig:risingLightDiskCompareTP} compares results from the~\ampRB~scheme with the TP-RB~scheme
for the grid $\Gcrd^{(4)}$.  The results are nearly
identical. These results again confirm that the additional terms introduced into the~\ampRB~scheme, such as the added-damping tensors,
correctly represent the physics of the problem. This is quite a difficult case for the TP-RB~scheme, requiring a small relaxation parameter and
an average of 2000 sub-time-step iterations per step.
%(2000 iterations 
%Tris he TP-SI scheme required,  on average, more than 2000 sub-iterations per time-step in this simulation.


% --------------------------------------------------------------------------------------------------------------------
% --------------------------------------------------------------------------------------------------------------------
% \subsubsection{*NEWER* inlet-outlet grids $\rhob=.001$, $\nu=.1$}





%- % --------------------------------------------------------------------------------------------------------------
%- %----------------------------------OLD newer  -----------------------------------------------------------------
%- % --------------------------------------------------------------------------------------------------------------
%- \clearpage
%- \subsubsection{*NEW* $\rhob=.001$, $\nu=.1$}
%- 
%- {% ------ BODY CURVES  THETA, XB, YB ------
%- \newcommand{\figWidth}{10.cm}
%- \newcommand{\trimfig}[2]{\trimFig{#1}{#2}{.0}{.0}{.0}{.0}}
%- % zoom: 
%- \newcommand{\figWidthz}{5.5cm}
%- \newcommand{\trimfigz}[2]{\trimFig{#1}{#2}{.0}{.0}{.0}{.0}}
%- \begin{figure}[htb]
%- \begin{centre}
%- \begin{tikzpicture}[scale=1]
%-   \useasboundingbox (0.0,.25) rectangle (16.,9);  % set the bounding box (so we have less surrounding white space)
%- %
%-   \draw(0.0,0.0) node[anchor=south west,xshift=-4pt,yshift=+0pt] {\trimfig{diskDrop/risingLightDiskPosition}{\figWidth}};
%-   \draw(10.5,-.5) node[anchor=south west,xshift=-4pt,yshift=+0pt] {\trimfigz{diskDrop/risingLightDiskThetaZoom}{\figWidthz}};
%-   \draw(10.5,4.0) node[anchor=south west,xshift=-4pt,yshift=+0pt] {\trimfigz{diskDrop/risingLightDiskXbZoom}{\figWidthz}};
%- %
%- %
%- % grid:
%- % \draw[step=1cm,gray] (0,0) grid (16,9);
%- \end{tikzpicture}
%- \end{centre}
%-   \caption{Rising light disk in a counter flow. Body position and rotation.
%-    % Results for grid G4 are in colour. Results for grid G8 are in black. 
%- %    Horizontal components of the displacement, $x_1$, velocity $v_1$ and acceleration $a_1$ of the rigid body.
%- %     Grid $\Gc^{(2)}$. 
%-      }
%-   \label{fig:risingLightDiskPositionCurves}
%- \end{figure}
%- }
%- 
%- {% ------ BODY VELOCITY CURVES ------
%- \newcommand{\figWidth}{8.cm}
%- \newcommand{\trimfig}[2]{\trimFig{#1}{#2}{.0}{.0}{.0}{.0}}
%- % zoom: 
%- \newcommand{\figWidthz}{8cm}
%- \newcommand{\trimfigz}[2]{\trimFig{#1}{#2}{.0}{.0}{.0}{.0}}
%- \begin{figure}[htb]
%- \begin{centre}
%- \begin{tikzpicture}[scale=1]
%-   \useasboundingbox (0.0,.5) rectangle (16.,7);  % set the bounding box (so we have less surrounding white space)
%- %
%-   \draw(0.0,0.0) node[anchor=south west,xshift=-4pt,yshift=+0pt] {\trimfig{diskDrop/risingLightDiskVelocity}{\figWidth}};
%-   \draw(8.5,0) node[anchor=south west,xshift=-4pt,yshift=+0pt] {\trimfigz{diskDrop/risingLightDiskVelocityZoom}{\figWidthz}};
%- %
%- %
%- % grid:
%- % \draw[step=1cm,gray] (0,0) grid (16,7);
%- \end{tikzpicture}
%- \end{centre}
%-   \caption{Rising light disk in a counter flow. Body velocities.
%-    % Results for grid G4 are in colour. Results for grid G8 are in black. 
%- %    Horizontal components of the displacement, $x_1$, velocity $v_1$ and acceleration $a_1$ of the rigid body.
%- %     Grid $\Gc^{(2)}$. 
%-      }
%-   \label{fig:risingLightDiskVelocityCurves}
%- \end{figure}
%- }
%- {% ------ BODY ACCELERATION CURVES ------
%- \newcommand{\figWidth}{8.cm}
%- \newcommand{\trimfig}[2]{\trimFig{#1}{#2}{.0}{.0}{.0}{.0}}
%- % zoom: 
%- \newcommand{\figWidthz}{8cm}
%- \newcommand{\trimfigz}[2]{\trimFig{#1}{#2}{.0}{.0}{.0}{.0}}
%- \begin{figure}[htb]
%- \begin{centre}
%- \begin{tikzpicture}[scale=1]
%-   \useasboundingbox (0.0,.5) rectangle (16.,7);  % set the bounding box (so we have less surrounding white space)
%- %
%-   \draw(0.0,0.0) node[anchor=south west,xshift=-4pt,yshift=+0pt] {\trimfig{diskDrop/risingLightDiskAcceleration}{\figWidth}};
%-   \draw(8.5,0) node[anchor=south west,xshift=-4pt,yshift=+0pt] {\trimfigz{diskDrop/risingLightDiskAccelerationZoom}{\figWidthz}};
%- %
%- %
%- % grid:
%- % \draw[step=1cm,gray] (0,0) grid (16,7);
%- \end{tikzpicture}
%- \end{centre}
%-   \caption{Rising light disk in a counter flow. Body accelerations. BLIPS in G8 dues to bug in changing $\dt$ FIX ME.
%-    % Results for grid G4 are in colour. Results for grid G8 are in black. 
%- %    Horizontal components of the displacement, $x_1$, velocity $v_1$ and acceleration $a_1$ of the rigid body.
%- %     Grid $\Gc^{(2)}$. 
%-      }
%-   \label{fig:risingLightDiskAccelerationCurves}
%- \end{figure}
%- }
%- 
%- {% ------ BODY COMPARE TO TP ------
%- \newcommand{\figWidth}{7.5cm}
%- \newcommand{\trimfig}[2]{\trimFigb{#1}{#2}{.0}{.0}{.0}{.0}}
%- \begin{figure}[htb]
%- \begin{centre}
%- \begin{tikzpicture}[scale=1]
%-   \useasboundingbox (0.0,.5) rectangle (16.,13);  % set the bounding box (so we have less surrounding white space)
%- %
%-   \draw(0.0,6.5) node[anchor=south west,xshift=-4pt,yshift=+0pt] {\trimfig{diskDrop/risingLightDiskPositionCompareTP}{\figWidth}};
%-   \draw(8.0,6.5) node[anchor=south west,xshift=-4pt,yshift=+0pt] {\trimfig{diskDrop/risingLightDiskVelocityCompareTP}{\figWidth}};
%-   \draw(4.,0) node[anchor=south west,xshift=-4pt,yshift=+0pt] {\trimfig{diskDrop/risingLightDiskAccelerationCompareTP}{\figWidth}};
%- %
%- %
%- % grid:
%- % \draw[step=1cm,gray] (0,0) grid (16,13);
%- \end{tikzpicture}
%- \end{centre}
%-   \caption{Rising light disk in a counter flow. Comparison to TP-SI scheme for G4.
%-    % Results for grid G4 are in colour. Results for grid G8 are in black. 
%- %    Horizontal components of the displacement, $x_1$, velocity $v_1$ and acceleration $a_1$ of the rigid body.
%- %     Grid $\Gc^{(2)}$. 
%-      }
%-   \label{fig:risingLightDiskCompareTP}
%- \end{figure}
%- }
%- 
%- {% ------ BODY CURVES  ------
%- \newcommand{\figWidth}{8.cm}
%- \newcommand{\trimfig}[2]{\trimFig{#1}{#2}{.0}{.0}{.0}{.0}}
%- % zoom: 
%- \newcommand{\figWidthz}{5.cm}
%- \newcommand{\trimfigz}[2]{\trimFig{#1}{#2}{.0}{.0}{.0}{.0}}
%- \begin{figure}[htb]
%- \begin{centre}
%- \begin{tikzpicture}[scale=1]
%-   \useasboundingbox (0.0,.75) rectangle (16.,20.5);  % set the bounding box (so we have less surrounding white space)
%- %
%-   \draw(8.0,14) node[anchor=south west,xshift=-4pt,yshift=+0pt] {\trimfig{diskDrop/risingLightDiskAnimationCase4}{\figWidth}};
%- %
%-   \draw(0.0,7) node[anchor=south west,xshift=-4pt,yshift=+0pt] {\trimfig{diskDrop/risingLightDiskPosition}{\figWidth}};
%-   \draw(8.5,6.5) node[anchor=south west,xshift=-4pt,yshift=+0pt] {\trimfigz{diskDrop/risingLightDiskThetaZoom}{\figWidthz}};
%-   \draw(8.5,10.5) node[anchor=south west,xshift=-4pt,yshift=+0pt] {\trimfigz{diskDrop/risingLightDiskXbZoom}{\figWidthz}};
%- %
%-   \draw(0.0,0) node[anchor=south west,xshift=-4pt,yshift=+0pt] {\trimfig{diskDrop/risingLightDiskAcceleration}{\figWidth}};
%-   \draw(8.0,0) node[anchor=south west,xshift=-4pt,yshift=+0pt] {\trimfig{diskDrop/risingLightDiskVelocity}{\figWidth}};
%- %
%- % grid:
%-  \draw[step=1cm,gray] (0,0) grid (16,20);
%- \end{tikzpicture}
%- \end{centre}
%-   \caption{Rising light disk in a counter flow. 
%-    % Results for grid G4 are in colour. Results for grid G8 are in black. 
%- %    Horizontal components of the displacement, $x_1$, velocity $v_1$ and acceleration $a_1$ of the rigid body.
%- %     Grid $\Gc^{(2)}$. 
%-      }
%-   \label{fig:risingLightDiskCurves}
%- \end{figure}
%- }


%- % -----------------------------------------------------------------
%- \clearpage
%- \subsubsection{CASE III $\nu=.1$}
%- 
%- 
%- {% ------ BODY CURVES  ------
%- \newcommand{\figWidth}{8.cm}
%- \newcommand{\figWidthA}{12.cm}
%- \newcommand{\trimfig}[2]{\trimFig{#1}{#2}{.0}{.0}{.0}{.0}}
%- \begin{figure}[htb]
%- \begin{centre}
%- \begin{tikzpicture}[scale=1]
%-   \useasboundingbox (0.0,.75) rectangle (16.,20.5);  % set the bounding box (so we have less surrounding white space)
%- %
%- %%  \draw(8,14) node[anchor=south,xshift=-4pt,yshift=+0pt] {\trimfig{settlingDisk/settlingCylinderAnimationCase2}{\figWidthA}};
%- %
%-   \draw(0.0,7) node[anchor=south west,xshift=-4pt,yshift=+0pt] {\trimfig{diskDrop/risingDiskNu0p1Position}{\figWidth}};
%- %  \draw(8.0,7) node[anchor=south west,xshift=-4pt,yshift=+0pt] {\trimfig{diskDrop/risingDiskNu0p1AnimationCase1}{\figWidth}};
%- %
%- %  \draw(0.0,0) node[anchor=south west,xshift=-4pt,yshift=+0pt] {\trimfig{diskDrop/risingDiskNu0p1Acceleration}{\figWidth}};
%- %  \draw(8.0,0) node[anchor=south west,xshift=-4pt,yshift=+0pt] {\trimfig{diskDrop/risingDiskNu0p1Velocity}{\figWidth}};
%- %
%- % grid:
%- % \draw[step=1cm,gray] (0,0) grid (16,7.5);
%- \end{tikzpicture}
%- \end{centre}
%-   \caption{Rising disk in a counter flow. Case 3, $\nu=0.1$. 
%- %    Horizontal components of the displacement, $x_1$, velocity $v_1$ and acceleration $a_1$ of the rigid body.
%- %     Grid $\Gc^{(2)}$. 
%-      }
%-   \label{fig:risingDiskCaseIII}
%- \end{figure}
%- }
%- 


%- 
%- 
%- \clearpage
%- 
%- Figure~\ref{fig:settlingDiskCurves} shows results for $\rhos=2$. 
%- The disk initially accelerates downward. The downward speed increases to about $\vs_2\approx -.15$ ?? at time $t=1.8$ ??
%- and then gradually slows down as the disk approaches the bottom wall. 
%- The bottom of the disk slowly settles to the bottom wall *but* numerically will reach a finite distance based
%- due to the finite grid spacing. 
%- 
%- \noindent{\bf NOTES:}
%- \begin{itemize}
%-   \item For $\rhos=2$ the mass of the disk is $\ms=\pi a^2 \approx 1.571$
%- 
%-   \item The force due to gravity is normalized to $\gv = -1 \hat{y}$. 
%- 
%-   \item With no surrounding fluid the initial acceleration of the disk would be $\gv/\ms \approx -.6366 \hat{y}$.
%-     The actual initial acceleration is about $-0.3$ due to the initial pressure field.
%- 
%-   \item The TP scheme was unstable without sub-iterations.
%- 
%-   \item The TPSI scheme required on average 4? iterations while dropping and 17? when close to the bottom wall (G2). Could
%- maybe plot iterations over time?
%- 
%-   \item The AMP scheme had some trouble if the implicit time-step was too large. There is probably some time-step
%- restriction that arises in the AMP scheme dues to the predicted velocity used as a BC.
%- 
%-   \item The AMP and TP-SI have exactly the same blips in the acceleration {\bf WHY?} 
%- \end{itemize}

%- {% ------ BODY CURVES  ------
%- \newcommand{\figWidth}{8.cm}
%- \newcommand{\figWidthA}{12.cm}
%- \newcommand{\trimfig}[2]{\trimFig{#1}{#2}{.0}{.0}{.0}{.0}}
%- \begin{figure}[htb]
%- \begin{centre}
%- \begin{tikzpicture}[scale=1]
%-   \useasboundingbox (0.0,.75) rectangle (16.,20.5);  % set the bounding box (so we have less surrounding white space)
%- %
%- %%  \draw(8,14) node[anchor=south,xshift=-4pt,yshift=+0pt] {\trimfig{settlingDisk/settlingCylinderAnimationCase2}{\figWidthA}};
%- %
%-   \draw(0.0,7) node[anchor=south west,xshift=-4pt,yshift=+0pt] {\trimfig{diskDrop/diskDropPosition}{\figWidth}};
%-   \draw(8.0,7) node[anchor=south west,xshift=-4pt,yshift=+0pt] {\trimfig{diskDrop/diskDropAnimationCase1}{\figWidth}};
%- %
%-   \draw(0.0,0) node[anchor=south west,xshift=-4pt,yshift=+0pt] {\trimfig{diskDrop/diskDropAcceleration}{\figWidth}};
%-   \draw(8.0,0) node[anchor=south west,xshift=-4pt,yshift=+0pt] {\trimfig{diskDrop/diskDropVelocity}{\figWidth}};
%- %
%- % grid:
%- % \draw[step=1cm,gray] (0,0) grid (16,7.5);
%- \end{tikzpicture}
%- \end{centre}
%-   \caption{Rising disk in a counter flow. Results for grid G4 are in colour. Results for grid G8 are in black. 
%- %    Horizontal components of the displacement, $x_1$, velocity $v_1$ and acceleration $a_1$ of the rigid body.
%- %     Grid $\Gc^{(2)}$. 
%-      }
%-   \label{fig:diskDropCurves}
%- \end{figure}
%- }
%- 
%- {% ------ BODY CURVES - WIDER CHANNEL  ------
%- \newcommand{\figWidth}{8.cm}
%- \newcommand{\figWidthA}{12.cm}
%- \newcommand{\trimfig}[2]{\trimFig{#1}{#2}{.0}{.0}{.0}{.0}}
%- \begin{figure}[htb]
%- \begin{centre}
%- \begin{tikzpicture}[scale=1]
%-   \useasboundingbox (0.0,.75) rectangle (16.,20.5);  % set the bounding box (so we have less surrounding white space)
%- %
%- %%  \draw(8,14) node[anchor=south,xshift=-4pt,yshift=+0pt] {\trimfig{settlingDisk/settlingCylinderAnimationCase2}{\figWidthA}};
%- %
%-   \draw(0.0,7) node[anchor=south west,xshift=-4pt,yshift=+0pt] {\trimfig{diskDrop/risingDiskPosition}{\figWidth}};
%-   \draw(8.0,7) node[anchor=south west,xshift=-4pt,yshift=+0pt] {\trimfig{diskDrop/risingDiskAnimationCase2}{\figWidth}};
%- %
%-   \draw(0.0,0) node[anchor=south west,xshift=-4pt,yshift=+0pt] {\trimfig{diskDrop/risingDiskAcceleration}{\figWidth}};
%-   \draw(8.0,0) node[anchor=south west,xshift=-4pt,yshift=+0pt] {\trimfig{diskDrop/risingDiskVelocity}{\figWidth}};
%- %
%- % grid:
%- % \draw[step=1cm,gray] (0,0) grid (16,7.5);
%- \end{tikzpicture}
%- \end{centre}
%-   \caption{WIDER CHANNEL. Rising disk in a counter flow. Results for grid G4 are in colour. Results for grid G8 are in black. 
%- %    Horizontal components of the displacement, $x_1$, velocity $v_1$ and acceleration $a_1$ of the rigid body.
%- %     Grid $\Gc^{(2)}$. 
%-      }
%-   \label{fig:risingDiskCurves}
%- \end{figure}
%- }
%- 

