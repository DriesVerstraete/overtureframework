\section{Flow past a two-dimensional backward facing step}

The geometry of the backward facing step is shown in~\ref{fig:backStepGrids}. 
An inlet region starting at $x=x_a$ reaches the step at $x=0$. The step has height of $1$ and the bottom
of the step is located at $x=0$, $y=0$. The outlet region extends to $x=x_b$. 

The boundary conditions are chosen as inflow on the left at $x=x_a$. The inflow profile is a parabola
over some small height and then uniform above this. An outflow condition is used on the right.
The top wall is either a slip wall (symmetry) or an outflow or zero-traction condition $p=0$. 
The lower wall is a no-slip wall.


The overlapping grid is denoted by $\Gc^{(j)}$ where $j$ denotes the grid resolution factor: the grid spacing in the 
grids in vicinity of the step have a a grid spacing of approximately $\Delta s = 1/(10 j)$. The grids are stretched
in the direction normal to the boundary by a factor of about $5$.

Figure~\ref{fig:backStepGrids} shows a basic overlapping grid for the backward facing step.
The step is rounded and represented with the SmoothedPolygon mapping.
Two background Cartesian grids fill the inlet and outlet regions.

Figure~\ref{fig:backStepRefineGrids} shows an {\em improved} version of the grid including far-field grids
to treat a larger domain and a stretched boundary layer grid on the bottom wall that follows the step.

{
\begin{figure}[hbt]
\newcommand{\figWidth}{10cm}
\newcommand{\trimfig}[2]{\trimFigb{#1}{#2}{0.025}{.5}{.375}{.375}}
\newcommand{\figWidtha}{5.5cm}
\newcommand{\trimfiga}[2]{\trimFigb{#1}{#2}{0.0}{.0}{.0}{.0}}
\begin{center}\small
% ------------------------------------------------------------------------------------------------
\begin{tikzpicture}[scale=1]
  \useasboundingbox (0,0.25) rectangle (16,5.5);  % set the bounding box (so we have less surrounding white space)
  \draw ( 0, 0) node[anchor=south west,xshift=-4pt,yshift=-4pt] {\trimfig{\backStepDir/fig/backStepGrid}{\figWidth}};
  \draw (10.5, 0) node[anchor=south west,xshift=-4pt,yshift=-4pt] {\trimfiga{\backStepDir/fig/backStepGridZoom}{\figWidtha}};
%  \draw[step=1cm,gray] (0,0) grid (16,5);
\end{tikzpicture}
% ----------------------------------------------------------------------------------------
\caption{
Overlapping grid (version 1) 
for the backward facing step. Grid $\Gc^{(4)}$ (resolution factor $j=4$, grid for fourth-order accuracy). Right: zoomed view near rounded step.
Interpolation points (two layers) are marked a.cmds black squares.
% Two-dimensional overlapping grid for a cross-section through the terrain. Grid $\Gc^{(4)}$ (second-order accuracy) is shown
%    along with a magnified view of the grid near the surface.
}
\label{fig:backStepGrids}
\end{center}
\end{figure}
}


%- {
%- \begin{figure}[hbt]
%- \newcommand{\figWidth}{10cm}
%- \newcommand{\trimfig}[2]{\trimFigb{#1}{#2}{0.025}{.5}{.375}{.375}}
%- \newcommand{\figWidtha}{5.5cm}
%- \newcommand{\trimfiga}[2]{\trimFigb{#1}{#2}{0.0}{.0}{.0}{.0}}
%- \begin{center}\small
%- % ------------------------------------------------------------------------------------------------
%- \begin{tikzpicture}[scale=1]
%-   \useasboundingbox (0,0.25) rectangle (16,5.5);  % set the bounding box (so we have less surrounding white space)
%-   \draw ( 0, 0) node[anchor=south west,xshift=-4pt,yshift=-4pt] {\trimfig{\backStepDir/fig/backStepRefineGrid}{\figWidth}};
%-   \draw (10.5, 0) node[anchor=south west,xshift=-4pt,yshift=-4pt] {\trimfiga{\backStepDir/fig/backStepRefineGridZoom}{\figWidtha}};
%- %  \draw[step=1cm,gray] (0,0) grid (16,5);
%- \end{tikzpicture}
%- % ----------------------------------------------------------------------------------------
%- \caption{
%- Overlapping grid for the backward facing step with far-field grids. Grid $\Gc^{(4)}$ (fourth-order accuracy). Right: zoomed view near rounded step.
%- Interpolation points (two layers) are marked as black squares.
%- % Two-dimensional overlapping grid for a cross-section through the terrain. Grid $\Gc^{(4)}$ (second-order accuracy) is shown
%- %    along with a magnified view of the grid near the surface.
%- }
%- \label{fig:backStepRefineGrids}
%- \end{center}
%- \end{figure}
%- }


{
\begin{figure}[hbt]
\newcommand{\figWidth}{16cm}
\newcommand{\trimfig}[2]{\trimFigb{#1}{#2}{0.01}{.01}{.275}{.275}}
\newcommand{\figWidtha}{5.5cm}
\newcommand{\trimfiga}[2]{\trimFigb{#1}{#2}{0.0}{.0}{.0}{.0}}
\begin{center}\small
% ------------------------------------------------------------------------------------------------
\begin{tikzpicture}[scale=1]
  \useasboundingbox (0,0.25) rectangle (16,10.);  % set the bounding box (so we have less surrounding white space)
  \draw ( 0, 5.6) node[anchor=south west,xshift=-4pt,yshift=-4pt] {\trimfig{\backStepDir/fig/backStepRefineGrid}{\figWidth}};
  \draw ( 5, 0.0) node[anchor=south west,xshift=-4pt,yshift=-4pt] {\trimfiga{\backStepDir/fig/backStepRefineGridZoom}{\figWidtha}};
%   \draw (10.5, 0) node[anchor=south west,xshift=-4pt,yshift=-4pt] {\trimfiga{\backStepDir/fig/backStepRefineGrid2Zoom}{\figWidtha}};
%% \draw[step=1cm,gray] (0,0) grid (16,10.);
\end{tikzpicture}
% ----------------------------------------------------------------------------------------
\caption{
Overlapping grid (version 2) for the backward facing step with far-field grids. Grid $\Gc^{(4)}$ (fourth-order accuracy). Bottom: zoomed view near rounded step.
Interpolation points (two layers) are marked as black squares.
% Two-dimensional overlapping grid for a cross-section through the terrain. Grid $\Gc^{(4)}$ (second-order accuracy) is shown
%    along with a magnified view of the grid near the surface.
}
\label{fig:backStepRefineGrids}
\end{center}
\end{figure}
}

% ======================================================================
\clearpage
\subsection{Grid resolution study}

Figure~\ref{fig:backStepResolutions} shows how the solution depends on the grid resolution.
As the grid is refined, finer scale features of the flow can be resolved.
{\bf Resolving these finer scale features may be important if one want to measure the
pressure forces and buffeting of an object sitting in the wake.}
The maximum vorticity also increases as the grid is refined. 
In this case the true viscosity plays no major role (except in the boundary layer when the
grid is finer) the nonlinear SSLES model provides most of the dissipation.


{
\begin{figure}[hbt]
\newcommand{\figWidth}{11cm}
\newcommand{\trimfig}[2]{\trimFigb{#1}{#2}{0.0}{.0}{.33}{.33}}
\begin{center}\small
% ------------------------------------------------------------------------------------------------
\begin{tikzpicture}[scale=1]
  \useasboundingbox (0,0.25) rectangle (11,13.);  % set the bounding box (so we have less surrounding white space)
  \draw ( 0,0.0) node[anchor=south west,xshift=-4pt,yshift=-4pt] {\trimfig{\backStepDir/fig/backStepRefineG32Vort20}{\figWidth}};
  \draw ( 0,4.5) node[anchor=south west,xshift=-4pt,yshift=-4pt] {\trimfig{\backStepDir/fig/backStepRefineG16Vort20}{\figWidth}};
  \draw ( 0,9.0) node[anchor=south west,xshift=-4pt,yshift=-4pt] {\trimfig{\backStepDir/fig/backStepRefineG8Vort20}{\figWidth}};
%   \draw (10.5, 0) node[anchor=south west,xshift=-4pt,yshift=-4pt] {\trimfiga{\backStepDir/fig/backStepRefineGridZoom}{\figWidtha}};
%  \draw[step=1cm,gray] (0,0) grid (11,13.);
\end{tikzpicture}
% ----------------------------------------------------------------------------------------
\caption{Grid resolution study.
Vorticity at time $t=20$ for grids of different resolutions using the SSLES4 turbulence model.
Solution in the near wake region is shown. The max and min contours of the vorticity are limited to the range $[-20,20]$.
Top: coarse grid $\Gc^{(8)}$. Middle: medium grid $\Gc^{(16)}$. Bottom: fine grid $\Gc^{(32)}$ (approx. $2.3M$ grid pts).  
}
\label{fig:backStepResolutions}
\end{center}
\end{figure}
}


\clearpage
% =================================================================================================================================
% =================================================================================================================================
% =================================================================================================================================
\subsection{Flow past a two-dimensional backward facing step with moving cylinder in the wake}

A moving cylinder is placed in the wake of the backward facing step using the 
moving grid capabilities of the Cgins solver.

Figure~\ref{fig:backStepAndBodyGrids} shows a basic overlapping grid for backward facing step
with a cylinder in the wake. The cylinder oscillates up and down. 

Figure~\ref{fig:backStepAndBodyMoveCylinder} shows the solution at two instants in time
computed on grid $\Gc^{(16})$ (approx. $750,000$ grid points) using AFS24-MG. 
{
\begin{figure}[hbt]
\newcommand{\figWidth}{15cm}
\newcommand{\trimfig}[2]{\trimFigb{#1}{#2}{0.0}{.25}{.27}{.275}}
\begin{center}\small
% ------------------------------------------------------------------------------------------------
\begin{tikzpicture}[scale=1]
  \useasboundingbox (0,0.25) rectangle (16,6.5);  % set the bounding box (so we have less surrounding white space)
  \draw ( 0, 0) node[anchor=south west,xshift=-4pt,yshift=-4pt] {\trimfig{\backStepDir/fig/backStepAndBodyGrid}{\figWidth}};
%   \draw (10.5, 0) node[anchor=south west,xshift=-4pt,yshift=-4pt] {\trimfiga{\backStepDir/fig/backStepRefineGridZoom}{\figWidtha}};
% \draw[step=1cm,gray] (0,0) grid (16,6);
\end{tikzpicture}
% ----------------------------------------------------------------------------------------
\caption{
Overlapping grid for the backward facing step with a cylindrical body in the wake, grid $\Gc^{(4)}$. 
}
\label{fig:backStepAndBodyGrids}
\end{center}
\end{figure}
}

{
\begin{figure}[hbt]
\newcommand{\figWidth}{11cm}
\newcommand{\trimfig}[2]{\trimFigb{#1}{#2}{0.0}{.0}{.27}{.3}}
\begin{center}\small
% ------------------------------------------------------------------------------------------------
\begin{tikzpicture}[scale=1]
  \useasboundingbox (0,0.25) rectangle (11,10.25);  % set the bounding box (so we have less surrounding white space)
  \draw ( 0, 0) node[anchor=south west,xshift=-4pt,yshift=-4pt] {\trimfig{\backStepDir/fig/backStepAndBodyG16-Vor-t31}{\figWidth}};
  \draw ( 0, 5) node[anchor=south west,xshift=-4pt,yshift=-4pt] {\trimfig{\backStepDir/fig/backStepAndBodyG16-Vor-t33}{\figWidth}};
%   \draw (10.5, 0) node[anchor=south west,xshift=-4pt,yshift=-4pt] {\trimfiga{\backStepDir/fig/backStepRefineGridZoom}{\figWidtha}};
% \draw[step=1cm,gray] (0,0) grid (11,10.25);
\end{tikzpicture}
% ----------------------------------------------------------------------------------------
\caption{
Vorticity at two times for flow past a backward facing step and a moving cylinder in the wake. 
The cylinder oscillates up and down. 
Solution computed with AFS24-MG on grid $\Gc^{(16)}$.
}
\label{fig:backStepAndBodyMoveCylinder}
\end{center}
\end{figure}
}


% =================================================================================================================================
% =================================================================================================================================
% =================================================================================================================================
\clearpage
\subsection{Flow past a two-dimensional backward facing step in a channel - comparison to experiment }

The paper by Kim, Kline, and Johnston [1980] presents experiement results for flow past a
backward facing step in a channel in a wind tunnel. 

The {\em reference case} has parameters: $H=3.81 cm (1.5 in)$ (step height), $w_1 = 7.62cm (3 in)$ inlet height,
reference speed $U=18.2 m/s$ , $\nu=1.56 e-5 m^2/s$ (air at 300k). This gives
a Reynolds number based on the step height of
\begin{align*}
 &  Re(H=3.81 cm) = \frac{U H}{\nu} = \frac{18.2 \times .0381}{1.568 \times 10^{-5}} \approx  4.445 \times 10^4, \\
 &  \frac{1}{Re} \approx 2.25 \times 10^{-5}. 
\end{align*}

The problem is non-dimensionalized using the length scale $L=H$ and reference speed $U$. 
The grid for the geometry is shown in figure~\ref{fig:backStepInChannelGrid} where the non-dimensional
step height is $1$ and inflow velocity is $1$ (with a small parbolic correction at the top and bottom
no-slip walls of the inlet. The nondimensional kinematic viscosity is then $\nu= 1/Re = 2.25 \times 10^{-5}$. 

{\bf Smallest scale:} We can estimate how fine the mesh must be to perform a resolved DNS.
The smallest scale of the flow is~\cite{HKR1}
\begin{align}
    \lambda = \sqrt{\frac{\nu}{|\grad\uv|}}.  \label{eq:smallestScale}
\end{align}
If we guess that the non-dimensional vorticity is on the order of $Re^{1/2} \approx 200$ and use $|\grad\uv|=200$ in ~\eqref{eq:smallestScale}
then we arrive at the estimate 
\begin{align*}
    \lambda \approx  \sqrt{\frac{2.25 \times 10^{-5}}{200}} \approx 3.4e-4
\end{align*}
Thus the grid spacing in the boundary layer (where the vorticity is largest) show be roughly $\ds\approx 3.4e-4$.

The overlapping grid for the geometry is shown in Figure~\ref{fig:backStepInChannelGrid} . 
The grid $\Gc^{(j)}$ has nominal spacing $\ds^{(j)}=1/(10 j)$ and boundary layer spacing of $\ds_{BL}=1/(50 j)$. 
For $j=32$ this gives a spacing $\ds_{BL}^{(32)} = 6.25e-4$ while for $j=64$  we have $\ds_{BL}^{(32)} = 3.125e-4$
{\bf We thus guess that $\Gc^{(16)}$ or $\Gc^{(32)}$ will be of sufficient resolution to perform an accurate DNS simulation.}

\bigskip
Figures~\ref{fig:backStepInChannel} and ~\ref{fig:backStepInChannelII} show some results for grid $\Gc^{(16)}$ (approx. 1.1M grid points).
Contours of the vorticity are shown at different times as well as instantaneous streamlines. 
The flow is highly non-steady with a complex wake that contains vorticity that has been ejected 
into the main part of the channel.

Figure~\ref{fig:backStepInChannelTimeAverageI} shows the time-averaged solution when averaged over the time interval $t\in[25,50]$
while Figure~\ref{fig:backStepInChannelTimeAverage} shows the time averaged solution averaged over the longer time 
interval $t\in[50,200]$.
Figure~\ref{fig:backStepInChannelTimeAverageStreamLinesZoom} shows a closeup of the streamlines near the back-step.
The streamlines of the time-average solution shows a small recirculation buuble that re-attaches at approximately $x=2.1$. 
However, the influence of the back-step extends far downstream. The time-averged flow is still showing variation
in the horizontal direction at $x=25$ which is the end of the computational domain. 

Figure~\ref{fig:backStepInChannelTimeAverageVelocityProfiles} plots profiles of the horizontal component of the
velocity along vertical lines at various values of $x$ downstream of the back-step. 
The profile at $x=1.33$ is in the re-circulation bubble and shows a strong backflow near the lower wall and
a line boundary layer. 

{
\begin{figure}[hbt]
\newcommand{\figWidth}{16cm}
\newcommand{\trimfig}[2]{\trimFigb{#1}{#2}{0.01}{.01}{.275}{.275}}
\newcommand{\figWidtha}{5.5cm}
\newcommand{\trimfiga}[2]{\trimFigb{#1}{#2}{0.0}{.0}{.0}{.0}}
\begin{center}\small
% ------------------------------------------------------------------------------------------------
\begin{tikzpicture}[scale=1]
  \useasboundingbox (0,0.25) rectangle (16,10.);  % set the bounding box (so we have less surrounding white space)
  \draw ( 0, 5.6) node[anchor=south west,xshift=-4pt,yshift=-4pt] {\trimfig{\backStepDir/fig/backStepInChannelGrid}{\figWidth}};
  \draw ( 5, 0.0) node[anchor=south west,xshift=-4pt,yshift=-4pt] {\trimfiga{\backStepDir/fig/backStepInChannelGridZoom}{\figWidtha}};
%   \draw (10.5, 0) node[anchor=south west,xshift=-4pt,yshift=-4pt] {\trimfiga{\backStepDir/fig/backStepRefineGrid2Zoom}{\figWidtha}};
%% \draw[step=1cm,gray] (0,0) grid (16,10.);
\end{tikzpicture}
% ----------------------------------------------------------------------------------------
\caption{
Overlapping grid for the backward facing step in a channel (upper wall is no-slip). Very coarse grid $\Gc^{(2)}$ (second accuracy). Bottom: zoomed view near rounded step.
Interpolation points (two layers) are marked as black squares.
% Two-dimensional overlapping grid for a cross-section through the terrain. Grid $\Gc^{(4)}$ (second-order accuracy) is shown
%    along with a magnified view of the grid near the surface.
}
\label{fig:backStepInChannelGrid}
\end{center}
\end{figure}
}

{
\begin{figure}[hbt]
\newcommand{\figWidth}{17cm}
\newcommand{\trimfig}[2]{\trimFigb{#1}{#2}{0.0}{.0}{.29}{.29}}
\begin{center}\small
% ------------------------------------------------------------------------------------------------
\begin{tikzpicture}[scale=1]
  \useasboundingbox (0,0.25) rectangle (16,9.);  % set the bounding box (so we have less surrounding white space)
  \draw ( 0,6.0) node[anchor=south west,xshift=-4pt,yshift=-4pt] {\trimfig{\backStepDir/fig/backStepRefineG16Vort10t50}{\figWidth}};
  \draw ( 0,3.0) node[anchor=south west,xshift=-4pt,yshift=-4pt] {\trimfig{\backStepDir/fig/backStepRefineG16Vort10t60}{\figWidth}};
  \draw ( 0,0.0) node[anchor=south west,xshift=-4pt,yshift=-4pt] {\trimfig{\backStepDir/fig/backStepRefineG16Vort10t70}{\figWidth}};
%   \draw (10.5, 0) node[anchor=south west,xshift=-4pt,yshift=-4pt] {\trimfiga{\backStepDir/fig/backStepRefineGridZoom}{\figWidtha}};
%  \draw[step=1cm,gray] (0,0) grid (11,13.);
\end{tikzpicture}
% ----------------------------------------------------------------------------------------
\caption{
Vorticity at times $t=50$, $60$, $70$ (top to bottom) for grid $\Gc^{(16)}$.
% Solution in the near wake region is shown. The max and min contours of the vorticity are limited to the range $[-20,20]$.
% Top: coarse grid $\Gc^{(8)}$. Middle: medium grid $\Gc^{(16)}$. Bottom: fine grid $\Gc^{(32)}$ (approx. $2.3M$ grid pts).  
}
\label{fig:backStepInChannel}
\end{center}
\end{figure}
}



{
\begin{figure}[hbt]
\newcommand{\figWidth}{17cm}
\newcommand{\trimfig}[2]{\trimFigb{#1}{#2}{0.0}{.0}{.29}{.29}}
\begin{center}\small
% ------------------------------------------------------------------------------------------------
\begin{tikzpicture}[scale=1]
  \useasboundingbox (0,0.25) rectangle (16,6.);  % set the bounding box (so we have less surrounding white space)
  \draw ( 0,3.0) node[anchor=south west,xshift=-4pt,yshift=-4pt] {\trimfig{\backStepDir/fig/backStepRefineG16Vort10t50}{\figWidth}};
  \draw ( 0,0.0) node[anchor=south west,xshift=-4pt,yshift=-4pt] {\trimfig{\backStepDir/fig/backStepRefineG16SLt50}{\figWidth}};
%   \draw (10.5, 0) node[anchor=south west,xshift=-4pt,yshift=-4pt] {\trimfiga{\backStepDir/fig/backStepRefineGridZoom}{\figWidtha}};
%  \draw[step=1cm,gray] (0,0) grid (11,13.);
\end{tikzpicture}
% ----------------------------------------------------------------------------------------
\caption{
Vorticity and streamlines at time $t=50$ for grid $\Gc^{(16)}$.
% Solution in the near wake region is shown. The max and min contours of the vorticity are limited to the range $[-20,20]$.
% Top: coarse grid $\Gc^{(8)}$. Middle: medium grid $\Gc^{(16)}$. Bottom: fine grid $\Gc^{(32)}$ (approx. $2.3M$ grid pts).  
}
\label{fig:backStepInChannelII}
\end{center}
\end{figure}
}


{
\begin{figure}[hbt]
\newcommand{\figWidth}{17cm}
\newcommand{\trimfig}[2]{\trimFigb{#1}{#2}{0.0}{.0}{.26}{.32}}
\begin{center}\small
% ------------------------------------------------------------------------------------------------
\begin{tikzpicture}[scale=1]
  \useasboundingbox (0,0.25) rectangle (16,9.8);  % set the bounding box (so we have less surrounding white space)
  \draw ( 0,6.4) node[anchor=south west,xshift=-4pt,yshift=-4pt] {\trimfig{\backStepDir/fig/backStepInChannelAve25to50SL}{\figWidth}};
  \draw ( 0,3.2) node[anchor=south west,xshift=-4pt,yshift=-4pt] {\trimfig{\backStepDir/fig/backStepInChannelAve25to50u}{\figWidth}};
  \draw ( 0,0.0) node[anchor=south west,xshift=-4pt,yshift=-4pt] {\trimfig{\backStepDir/fig/backStepInChannelAve25to50v}{\figWidth}};
%   \draw (10.5, 0) node[anchor=south west,xshift=-4pt,yshift=-4pt] {\trimfiga{\backStepDir/fig/backStepRefineGridZoom}{\figWidtha}};
%  \draw[step=1cm,gray] (0,0) grid (11,13.);
\end{tikzpicture}
% ----------------------------------------------------------------------------------------
\caption{
Time averaged streamlines and contours of $u$ and $v$ for grid $\Gc^{(16)}$.
Averaged over $t\in[25,50]$.
% Solution in the near wake region is shown. The max and min contours of the vorticity are limited to the range $[-20,20]$.
% Top: coarse grid $\Gc^{(8)}$. Middle: medium grid $\Gc^{(16)}$. Bottom: fine grid $\Gc^{(32)}$ (approx. $2.3M$ grid pts).  
}
\label{fig:backStepInChannelTimeAverageI}
\end{center}
\end{figure}
}


{
\begin{figure}[hbt]
\newcommand{\figWidth}{12cm}
\newcommand{\trimfig}[2]{\trimFigb{#1}{#2}{0.125}{.15}{.378}{.35}}
\begin{center}\small
% ------------------------------------------------------------------------------------------------
\begin{tikzpicture}[scale=1]
  \useasboundingbox (0,0.25) rectangle (14,13);  % set the bounding box (so we have less surrounding white space)
  \draw ( 0,9.6) node[anchor=south west,xshift=-4pt,yshift=-4pt] {\trimfig{\backStepDir/fig/backStepInChannel16Ave50to200sl}{\figWidth}};
  \draw ( 0,6.4) node[anchor=south west,xshift=-4pt,yshift=-4pt] {\trimfig{\backStepDir/fig/backStepInChannel16Ave50to200u}{\figWidth}};
  \draw ( 0,3.2) node[anchor=south west,xshift=-4pt,yshift=-4pt] {\trimfig{\backStepDir/fig/backStepInChannel16Ave50to200v}{\figWidth}};
  \draw ( 0,0.0) node[anchor=south west,xshift=-4pt,yshift=-4pt] {\trimfig{\backStepDir/fig/backStepInChannel16Ave50to200p}{\figWidth}};
%   \draw (10.5, 0) node[anchor=south west,xshift=-4pt,yshift=-4pt] {\trimfiga{\backStepDir/fig/backStepRefineGridZoom}{\figWidtha}};
% \draw[step=1cm,gray] (0,0) grid (14,13.);
\end{tikzpicture}
% ----------------------------------------------------------------------------------------
\caption{
Time averaged streamlines and contours of $u$, $v$ and $p$ for grid $\Gc^{(16)}$.
Averaged over $t\in[50,200]$.
% Solution in the near wake region is shown. The max and min contours of the vorticity are limited to the range $[-20,20]$.
% Top: coarse grid $\Gc^{(8)}$. Middle: medium grid $\Gc^{(16)}$. Bottom: fine grid $\Gc^{(32)}$ (approx. $2.3M$ grid pts).  
}
\label{fig:backStepInChannelTimeAverage}
\end{center}
\end{figure}
}


{
\begin{figure}[hbt]
\newcommand{\figWidth}{12cm}
\newcommand{\trimfig}[2]{\trimFigb{#1}{#2}{0.0}{.0}{.2}{.2}}
\begin{center}\small
% ------------------------------------------------------------------------------------------------
\begin{tikzpicture}[scale=1]
  \useasboundingbox (0,0.25) rectangle (12,8.);  % set the bounding box (so we have less surrounding white space)
  \draw ( 0, 0) node[anchor=south west,xshift=-4pt,yshift=-4pt] {\trimfig{\backStepDir/fig/backStepInChannelAve50to200Streamlines}{\figWidth}};
%   \draw (10.5, 0) node[anchor=south west,xshift=-4pt,yshift=-4pt] {\trimfiga{\backStepDir/fig/backStepRefineGridZoom}{\figWidtha}};
% \draw[step=1cm,gray] (0,0) grid (12,8);
\end{tikzpicture}
% ----------------------------------------------------------------------------------------
\caption{
Time averaged streamlines, $u$ and $v$ for grid $\Gc^{(16)}$.
Averaged over $t\in[50,200]$.
}
\label{fig:backStepInChannelTimeAverageStreamLinesZoom}
\end{center}
\end{figure}
}

{
\begin{figure}[hbt]
\newcommand{\figWidth}{12cm}
\newcommand{\trimfig}[2]{\trimFigb{#1}{#2}{0.0}{.0}{.0}{.0}}
\begin{center}\small
% ------------------------------------------------------------------------------------------------
\begin{tikzpicture}[scale=1]
  \useasboundingbox (0,0.25) rectangle (12,8.);  % set the bounding box (so we have less surrounding white space)
  \draw ( 0, 0) node[anchor=south west,xshift=-4pt,yshift=-4pt] {\trimfig{\backStepDir/fig/backStepInChannel16Ave50to200Profiles}{\figWidth}};
%   \draw (10.5, 0) node[anchor=south west,xshift=-4pt,yshift=-4pt] {\trimfiga{\backStepDir/fig/backStepRefineGridZoom}{\figWidtha}};
% \draw[step=1cm,gray] (0,0) grid (12,8);
\end{tikzpicture}
% ----------------------------------------------------------------------------------------
\caption{
Horizontal velocity preofiles for the time averaged solution, 
for grid $\Gc^{(16)}$ averaged over $t\in[50,200]$.
}
\label{fig:backStepInChannelTimeAverageVelocityProfiles}
\end{center}
\end{figure}
}
