%-----------------------------------------------------------------------
% Flow past a box or half box or quarter box (3D backward facing step)
%-----------------------------------------------------------------------
\documentclass[11pt]{article}
% \usepackage[bookmarks=true]{hyperref}  % this changes the page location !
\usepackage[bookmarks=true,colorlinks=true,linkcolor=blue]{hyperref}

% \input documentationPageSize.tex
\hbadness=10000 
\sloppy \hfuzz=30pt

% \voffset=-.25truein
% \hoffset=-1.25truein
% \setlength{\textwidth}{7in}      % page width
% \setlength{\textheight}{9.5in}    % page height

\usepackage{calc}
\usepackage[lmargin=.75in,rmargin=.75in,tmargin=.75in,bmargin=.75in]{geometry}

\input homeHenshaw

% \input{pstricks}\input{pst-node}
% \input{colours}
\newcommand{\blue}{\color{blue}}
\newcommand{\green}{\color{green}}
\newcommand{\red}{\color{red}}
\newcommand{\black}{\color{black}}


\usepackage{amsmath}
\usepackage{amssymb}

\usepackage{verbatim}
\usepackage{moreverb}

\usepackage{graphics}    
\usepackage{epsfig}    
\usepackage{calc}
\usepackage{ifthen}
\usepackage{float}
% the next one cause the table of contents to disappear!
% * \usepackage{fancybox}

\usepackage{makeidx} % index
\makeindex
\newcommand{\Index}[1]{#1\index{#1}}

\usepackage{tikz}
% \usepackage{pgfplots}
\input trimFig.tex


% ---- we have lemmas and theorems in this paper ----
\newtheorem{assumption}{Assumption}
\newtheorem{definition}{Definition}

% \newcommand{\homeHenshaw}{/home/henshaw.0}

\newcommand{\Overture}{{\bf Over\-ture\ }}
\newcommand{\ogenDir}{\homeHenshaw/Overture/ogen}

\newcommand{\cgDoc}{\homeHenshaw/cgDoc}
\newcommand{\vpDir}{\homeHenshaw/cgDoc/ins/viscoPlastic}

\newcommand{\ovFigures}{\homeHenshaw/OvertureFigures}
\newcommand{\obFigures}{\homeHenshaw/res/OverBlown/docFigures}  % for figures
\newcommand{\convDir}{\homeHenshaw/cgDoc/ins/tables}
\newcommand{\insDocDir}{\homeHenshaw/cgDoc/ins}

% *** See http://www.eng.cam.ac.uk/help/tpl/textprocessing/squeeze.html
% By default, LaTeX doesn't like to fill more than 0.7 of a text page with tables and graphics, nor does it like too many figures per page. This behaviour can be changed by placing lines like the following before \begin{document}

\renewcommand\floatpagefraction{.99}
\renewcommand\topfraction{.99}
\renewcommand\bottomfraction{.99}
\renewcommand\textfraction{.01}   
\setcounter{totalnumber}{50}
\setcounter{topnumber}{50}
\setcounter{bottomnumber}{50}

\begin{document}

\input wdhDefinitions.tex

\def\comma  {~~~,~~}
\newcommand{\uvd}{\mathbf{U}}
\def\ud     {{    U}}
\def\pd     {{    P}}
\def\calo{{\cal O}}

\newcommand{\mbar}{\bar{m}}
\newcommand{\Rbar}{\bar{R}}
\newcommand{\Ru}{R_u}         % universal gas constant
% \newcommand{\Iv}{{\bf I}}
% \newcommand{\qv}{{\bf q}}
\newcommand{\Div}{\grad\cdot}
\newcommand{\tauv}{\boldsymbol{\tau}}
\newcommand{\thetav}{\boldsymbol{\theta}}
% \newcommand{\omegav}{\mathbf{\omega}}
% \newcommand{\Omegav}{\mathbf{\Omega}}

\newcommand{\Omegav}{\boldsymbol{\Omega}}
\newcommand{\omegav}{\boldsymbol{\omega}}
\newcommand{\sigmav}{\boldsymbol{\sigma}}
\newcommand{\cm}{{\rm cm}}

\newcommand{\ds}{\Delta s}
\newcommand{\dsbl}{\ds_{\rm bl}}


\newcommand{\sumi}{\sum_{i=1}^n}
% \newcommand{\half}{{1\over2}}
\newcommand{\dt}{{\Delta t}}

\def\ff {\tt} % font for fortran variables

% define the clipFig commands:
%% \input clipFig.tex

\newcommand{\Bc}{{\mathcal B}}
\newcommand{\Dc}{{\mathcal D}}
\newcommand{\Ec}{{\mathcal E}}
\newcommand{\Fc}{{\mathcal F}}
\newcommand{\Gc}{{\mathcal G}}
\newcommand{\Hc}{{\mathcal H}}
\newcommand{\Ic}{{\mathcal I}}
\newcommand{\Jc}{{\mathcal J}}
\newcommand{\Lc}{{\mathcal L}}
\newcommand{\Nc}{{\mathcal N}}
\newcommand{\Pc}{{\mathcal P}}
\newcommand{\Rc}{{\mathcal R}}
\newcommand{\Sc}{{\mathcal S}}

\newcommand{\bogus}[1]{}  % removes is argument completely

\vspace{5\baselineskip}
\begin{flushleft}
{\LARGE
Simulating Airflow Past a Three-dimensional Backward Facing Step \\
Using Overture's Cgins Solver\\
}
\vspace{2\baselineskip}
Jeffrey W. Banks,  \\
William D. Henshaw, \\
Donald W. Schwendeman \\
% 
\vspace{2\baselineskip}
% 
Department of Mathematical Sciences, \\
Rensselaer Polytechnic Institute (RPI), \\
Troy, NY, USA, 12180. \\
\vspace{\baselineskip}
\today\\

\vspace{4\baselineskip}

\noindent{\bf\large Abstract:}

This document describes results for simulating flow past a 3D backward facing step.
In some cases there are moving bodies situated in the wake.

\end{flushleft}

% \clearpage
\tableofcontents
% \listoffigures

\clearpage


\section{Introduction}

Cgins is an incompressible flow solver built upon the Overture framework.
Cgins solves the incompressible Navier-Stokes equations (with Boussinesq approimation
for temperature dependent buoyant flows). See~\cite{CginsUserGuide} and~\cite{CginsReferenceManual} for further 
details. See~\cite{ICNS}~\cite{splitStep2003} for further details of the basic numerical scheme. 

The steps for simulating flow over some specified geometry are
\begin{enumerate}
  \item Use Ogen, the overlapping grid generator, to generate the initial grid using on of the provided command files (e.g. {\tt loftedQuarterBoxGrid.cmd}).
  \item Use one of the Cgins command files,  ({\tt e.g. boxInAChannel.cmd}) to simulate the flow using Cgins.
\end{enumerate}
% Note the relevant scripts  can be found in 
% the directory {\tt cg/ins/runs/backStep} of the CG distribution. 

The default Cgins scheme being used in these computations is 
\begin{enumerate}
  \item AFS24-MG : Approximate factored scheme, 2nd-order in time, 4th-order in space, multigrid solver for the pressure equation,
    typcially run at a CFL number of $2-4$. 
\end{enumerate}

Notes:
\begin{enumerate}
  \item Cgins uses a non-linear dissipation to stabilize under-resolved flows. For the fourth-order accurate method this is
    a non-linear fourth-order eddy viscosity known as SSLES4 (smallest-scale large eddy simulation model, order 4).
  The effective dissipation term in the momentum equations is thus of the form
  \begin{align*}
     \Dc & =  \nu \Delta \uv  + \Big(\text{ad41} + \text{ad42} \vert \grad\uv\vert \Big) ~h^4~ \Delta^2 \uv , 
  \end{align*}
  where ad41 and ad42 are the coefficients of the SSLES viscosity (e.g. ad41=1, ad42=1) and $h$ is the approximate grid spacing.
\end{enumerate}




% =================================================================================================================================
% =================================================================================================================================
% =================================================================================================================================
\section{Flow past a three-dimensional backward facing step}

The geometry of the three-dimensional backward facing step is shown in~\ref{fig:backStepGrids}. 
An inlet region starting at $x=x_a$ reaches the step at $x=0$. The step has height of $1$ and the bottom
of the step is located at $x=0$, $y=0$. The outlet region extends to $x=x_b$. 

The boundary conditions are chosen as inflow on the left at $x=x_a$. The inflow profile is a parabola
over some small height and then uniform above this. An outflow condition is used on the right.
The top wall is either a slip wall (symmetry) or an outflow or zero-traction condition $p=0$. 
The lower wall is a no-slip wall.


The overlapping grid is denoted by $\Gc^{(j)}$ where $j$ denotes the grid resolution factor: the grid spacing in the 
grids in vicinity of the step have a a grid spacing of approximately $\Delta s = 1/(10 j)$. The grids are stretched
in the direction normal to the boundary by a factor of about $5$ (?)

Figure~\ref{fig:backStepGrids} shows a basic overlapping grid for the backward facing step.
The step is rounded and represented with the SmoothedPolygon mapping.
Two background Cartesian grids fill the inlet and outlet regions.

% Figure~\ref{fig:backStepRefineGrids} shows an {\em improved} versin of the grid including far-field grids
% to treat a larger domain and a stretched boundary layer grid on the bottom wall that follows the step.

{
\begin{figure}[hbt]
\newcommand{\figWidth}{10cm}
\newcommand{\trimfig}[2]{\trimFigb{#1}{#2}{0.025}{.0}{.05}{.3}}
\newcommand{\figWidtha}{5.5cm}
\newcommand{\trimfiga}[2]{\trimFigb{#1}{#2}{0.0}{.0}{.0}{.0}}
\begin{center}\small
% ------------------------------------------------------------------------------------------------
\begin{tikzpicture}[scale=1]
  \useasboundingbox (0,0.25) rectangle (10,8);  % set the bounding box (so we have less surrounding white space)
  \draw ( 0, 0) node[anchor=south west,xshift=-4pt,yshift=-4pt] {\trimfig{fig/loftedQuarterBoxGrid4}{\figWidth}};
%%  \draw (10.5, 0) node[anchor=south west,xshift=-4pt,yshift=-4pt] {\trimfiga{fig/backStepGridZoom}{\figWidtha}};
%% \draw[step=1cm,gray] (0,0) grid (10,8);
\end{tikzpicture}
% ----------------------------------------------------------------------------------------
\caption{
Overlapping grid for the three-dimensional backward facing step. Grid $\Gc^{(4)}$ is shown with only every 2nd grid line plotted.
The red and pink grids cover the step. The green grid covers the
body and near wake. The blue background grid is coarsened by a factor of 2. 
}
\label{fig:backStepGrids}
\end{center}
\end{figure}
}


%- {
%- \begin{figure}[hbt]
%- \newcommand{\figWidth}{10cm}
%- \newcommand{\trimfig}[2]{\trimFigb{#1}{#2}{0.025}{.5}{.375}{.375}}
%- \newcommand{\figWidtha}{5.5cm}
%- \newcommand{\trimfiga}[2]{\trimFigb{#1}{#2}{0.0}{.0}{.0}{.0}}
%- \begin{center}\small
%- % ------------------------------------------------------------------------------------------------
%- \begin{tikzpicture}[scale=1]
%-   \useasboundingbox (0,0.25) rectangle (16,5.5);  % set the bounding box (so we have less surrounding white space)
%-   \draw ( 0, 0) node[anchor=south west,xshift=-4pt,yshift=-4pt] {\trimfig{fig/backStepRefineGrid}{\figWidth}};
%-   \draw (10.5, 0) node[anchor=south west,xshift=-4pt,yshift=-4pt] {\trimfiga{fig/backStepRefineGridZoom}{\figWidtha}};
%- %  \draw[step=1cm,gray] (0,0) grid (16,5);
%- \end{tikzpicture}
%- % ----------------------------------------------------------------------------------------
%- \caption{
%- Overlapping grid for the backward facing step with far-field grids. Grid $\Gc^{(4)}$ (fourth-order accuracy). Right: zoomed view near rounded step.
%- Interpolation points (two layers) are marked as black squares.
%- % Two-dimensional overlapping grid for a cross-section through the terrain. Grid $\Gc^{(4)}$ (second-order accuracy) is shown
%- %    along with a magnified view of the grid near the surface.
%- }
%- \label{fig:backStepRefineGrids}
%- \end{center}
%- \end{figure}
%- }



Figure~\ref{fig:backStepResolutions} shows how the solution depends on the grid resolution.
As the grid is refined, finer scale features of the flow can be resolved.
The maximum vorticity also increases as the grid is refined. 
In this case the true viscosity plays no role as the nonlinear SSLES model provides the dissipation.

{
\begin{figure}[hbt]
\newcommand{\figWidth}{11cm}
\newcommand{\trimfig}[2]{\trimFigb{#1}{#2}{0.0}{.0}{.1}{.4}}
\newcommand{\trimfiga}[2]{\trimFigb{#1}{#2}{0.0}{.0}{.2}{.4}}
\newcommand{\trimfigb}[2]{\trimFigb{#1}{#2}{0.0}{.0}{.05}{.35}}
\begin{center}\small
% ------------------------------------------------------------------------------------------------
\begin{tikzpicture}[scale=1]
  \useasboundingbox (0,0.25) rectangle (11,20.);  % set the bounding box (so we have less surrounding white space)
  \draw ( 0, 0.0) node[anchor=south west,xshift=-4pt,yshift=-4pt] {\trimfig{fig/quarterBox16Vort4p5}{\figWidth}};
  \draw ( 0, 6.5) node[anchor=south west,xshift=-4pt,yshift=-4pt] {\trimfiga{fig/quarterBox16Vort4p5a}{\figWidth}};
  \draw ( 0,12.0) node[anchor=south west,xshift=-4pt,yshift=-4pt] {\trimfigb{fig/quarterBox16Vort4p5b}{\figWidth}};
%   \draw (10.5, 0) node[anchor=south west,xshift=-4pt,yshift=-4pt] {\trimfiga{fig/backStepRefineGridZoom}{\figWidtha}};
% \draw[step=1cm,gray] (0,0) grid (11,20.);
\end{tikzpicture}
% ----------------------------------------------------------------------------------------
\caption{
Enstrophy. Results from fine grid $\Gc^{(16)}$ (32M pts).
}
\label{fig:backStepResults}
\end{center}
\end{figure}
}

{
\begin{figure}[hbt]
\newcommand{\figWidth}{11cm}
\newcommand{\trimfig}[2]{\trimFigb{#1}{#2}{0.0}{.0}{.1}{.4}}
\begin{center}\small
% ------------------------------------------------------------------------------------------------
\begin{tikzpicture}[scale=1]
  \useasboundingbox (0,0.25) rectangle (11,13.);  % set the bounding box (so we have less surrounding white space)
  \draw ( 0,0.0) node[anchor=south west,xshift=-4pt,yshift=-4pt] {\trimfig{fig/quarterBox16Vort4p5}{\figWidth}};
  \draw ( 0,6.5) node[anchor=south west,xshift=-4pt,yshift=-4pt] {\trimfig{fig/quarterBox8Vort10p0}{\figWidth}};
%%  \draw ( 0,9.0) node[anchor=south west,xshift=-4pt,yshift=-4pt] {\trimfig{fig/backStepRefineG8Vort20}{\figWidth}};
%   \draw (10.5, 0) node[anchor=south west,xshift=-4pt,yshift=-4pt] {\trimfiga{fig/backStepRefineGridZoom}{\figWidtha}};
%  \draw[step=1cm,gray] (0,0) grid (11,13.);
\end{tikzpicture}
% ----------------------------------------------------------------------------------------
\caption{
Enstrophy for grids of different resolutions using the SSLES4 turbulence model.
Solution in the near wake region is shown. 
% The max and min contours of the vorticity are limited to the range $[-20,20]$.
Top: coarse grid $\Gc^{(8)}$. Bottom: fine grid $\Gc^{(16)}$ (32M pts).
%  Bottom: fine grid $\Gc^{(32)}$ (approx. $2.3M$ grid pts).  
}
\label{fig:backStepResolutions}
\end{center}
\end{figure}
}








\clearpage
\bibliography{\homeHenshaw/papers/common/henshaw,\homeHenshaw/papers/common/henshawPapers}
\bibliographystyle{siam}



\end{document}
% *************************************************************************************************************************************
% *************************************************************************************************************************************
% *************************************************************************************************************************************
% *************************************************************************************************************************************


