%-----------------------------------------------------------------------
% User's Guide for CGAD -- Advection diffusion solver
% 
%-----------------------------------------------------------------------
\documentclass[11pt]{article}
\usepackage[bookmarks=true,colorlinks=true,linkcolor=blue]{hyperref}


% \input documentationPageSize.tex
\hbadness=10000 
\sloppy \hfuzz=30pt

% \voffset=-.25truein
% \hoffset=-1.25truein
% \setlength{\textwidth}{7in}      % page width
% \setlength{\textheight}{9.5in}    % page height

\usepackage{calc}
\usepackage[lmargin=.75in,rmargin=.75in,tmargin=.75in,bmargin=.75in]{geometry}

\input homeHenshaw

\usepackage{amsmath}
\usepackage{amssymb}

\usepackage{verbatim}
\usepackage{moreverb}

% \usepackage{epsfig}    
% This next section will allow graphics files to be ps or pdf  -- from Jeff via Jeff
% \usepackage{ifpdf}
% \ifpdf
%     \usepackage[pdftex]{graphicx}
%     \usepackage{epstopdf}
%     \pdfcompresslevel=9
%     \pdfpagewidth=8.5 true in
%     \pdfpageheight=11 true in
%     \pdfhorigin=1 true in
%     \pdfvorigin=1.25 true in
% \else
%     \usepackage{graphicx}
% \fi

% \input{pstricks}\input{pst-node}
% \input{colours}

% define the clipFig commands:
\input trimFig.tex

% --------------------------------------------
\usepackage[usenames]{color} % e.g. \color{red}
\newcommand{\red}{\color{red}}
\newcommand{\blue}{\color{blue}}
\newcommand{\green}{\color{green}}
\newcommand{\jwb}[2]{{\color{red}(old: #1) }{\color{green} #2}}

\usepackage{tikz}


\usepackage{makeidx} % index
\makeindex
\newcommand{\Index}[1]{#1\index{#1}}


% ---- we have lemmas and theorems in this paper ----
\newtheorem{assumption}{Assumption}
\newtheorem{definition}{Definition}

% \newcommand{\homeHenshaw}{/home/henshaw.0}

\newcommand{\primer}{/users/henshaw/res/primer}
\newcommand{\GF}{/users/\-henshaw/\-res/\-gf}
\newcommand{\gf}{/users/henshaw/res/gf}
\newcommand{\mapping}{/users/henshaw/res/mapping}

\newcommand{\docFigures}{\homeHenshaw/OvertureFigures}
\newcommand{\figures}{\homeHenshaw/res/OverBlown/docFigures}
\newcommand{\obFigures}{\homeHenshaw/res/OverBlown/docFigures}  % note: local version for OverBlown
\newcommand{\maxDoc}{\homeHenshaw/res/maxwell/doc}

\newcommand{\OVERTUREOVERTURE}{/users/\-henshaw/\-Overture/\-Overture}
\newcommand{\OvertureOverture}{/users/henshaw/Overture/Overture}

\newcommand{\Overture}{{\bf Overture\ }}
\newcommand{\OverBlown}{{\bf OverBlown\ }}
\newcommand{\overBlown}{{\bf overBlown\ }}


% *** See http://www.eng.cam.ac.uk/help/tpl/textprocessing/squeeze.html
% By default, LaTeX doesn't like to fill more than 0.7 of a text page with tables and graphics, nor does it like too many figures per page. This behaviour can be changed by placing lines like the following before \begin{document}

\renewcommand\floatpagefraction{.9}
\renewcommand\topfraction{.9}
\renewcommand\bottomfraction{.9}
\renewcommand\textfraction{.1}   
\setcounter{totalnumber}{50}
\setcounter{topnumber}{50}
\setcounter{bottomnumber}{50}


\begin{document}


% -----definitions-----
\input wdhDefinitions.tex

\def\ud     {{    U}}
\def\pd     {{    P}}

\newcommand{\mbar}{\bar{m}}
\newcommand{\Rbar}{\bar{R}}
\newcommand{\Ru}{R_u}         % universal gas constant
% \newcommand{\Iv}{{\bf I}}
% \newcommand{\qv}{{\bf q}}
\newcommand{\Div}{\grad\cdot}
\newcommand{\tauv}{\boldsymbol{\tau}}
\newcommand{\sumi}{\sum_{i=1}^n}
% \newcommand{\half}{{1\over2}}
\newcommand{\dt}{{\Delta t}}
\newcommand{\eps}{\epsilon}

\vglue 10\baselineskip
\begin{flushleft}
{\Large
Cgad User Guide: An Overture Solver for the Advection Diffusion Equations on Composite Grids \\
}
\vspace{2\baselineskip}
William D. Henshaw,  \\
Department of Mathematical Sciences,  \\
Rensselaer Polytechnic Institute ,    \\
Troy, NY, USA, 12180.  \\
{\tt overtureFramework.org} \\
\vspace{\baselineskip}
\today\\
\vspace{\baselineskip}
% UCRL-MA-134288

\vspace{4\baselineskip}

\noindent{\bf\large Abstract:}

Cgad is a program that can solve the advection diffusion (and reaction) equations on overlapping grids.
Cgad can solve problems on moving and deforming domains.
Cgad can also be used with Cgmp to solve the heat equation in solid regions, when performing conjugate heat
transfer problems in conjuction with the Cgins flow solver.


\end{flushleft}

% \clearpage
\tableofcontents
% \listoffigures


\vfill\eject

\section{Introduction}

   Cgad is a program that can solve the advection diffusion equations on overlapping grids and is based
on the Overture framework\footnote{More information about
{\bf Overture} can be found on the \Overture home page, {\tt www.overtureFramework.org}.}.
Cgad can solve problems on moving and deforming domains.
Cgad can also be used with Cgmp to solve the heat equation in solid regions, when performing conjugate heat
transfer problems in conjuction with the Cgins flow solver.


Cgad can solve for $m$ unknowns dependent variables $u_j=u_j(\xv,t)$, $j=1,2,\ldots,m$ where the {\em components}
satisfy a potentially coupled system of {\em advection diffusion} (and reaction) equations of the form,
\begin{align}
&  \frac{\partial u_j}{\partial t} 
       + a_1(\xv,t,\uv)\frac{\partial u_j}{\partial x}
       + a_2(\xv,t,\uv)\frac{\partial u_j}{\partial y}
       + a_3(\xv,t,\uv)\frac{\partial u_j}{\partial z}
   =  \grad\cdot\big( \kappa_j(\xv,t,\uv) \grad u_j\big)  + f_j(\xv,t,\uv),     
\end{align}
where $a_k(\xv,t,\uv)$, $k=1,2,3$ are the advection coefficients, $\kappa_j=\kappa_j(\xv,t,\uv)$ are the diffusion
coefficients, and $f_j=f_j(\xv,t,\uv)$ is a body forcing. Here $\uv=[u_1, u_2,\ldots, u_m]^T$ 
denotes the vector with components $u_j$. 
Boundary conditions take the form of Dirichlet, Neumann or mixed (Robin).


\vskip\baselineskip
\noindent{\bf Cgad features:}
\begin{itemize}
  \item supports multiple components,
  \item variable diffusivity coefficients $\kappa_j(\xv,t,\uv)$ (that can depend on $\xv$, $t$ and the solution),
  \item variable advection coefficients, $a_k(\xv,t,\uv)$,
  \item 2D axisymmetric option,
  \item user defined body forcing,
  \item user defined moving and deforming surfaces.
\end{itemize}
The variable diffusivity and advection coefficents can be specified through a {\em userDefined} function.

\vskip\baselineskip
\noindent{\bf Installation:} To install cgad you should follow the instructions for installing Overture and cg
from the Overture web page. Note that cgad is NOT built by default when building the cg solvers so that
after installing cg you must go into the {\tt cg/ad} directory and type `make'. 
If you only wish to build cgad and not any of the other cg solvers, then
after building Overture and unpacking cg, go to the {\tt cg/ad} and type make. 

\noindent The cgad solver is found in the {\tt ad} directory in the {\bf cg} distribution and has
sub-directories
\begin{description}
 \item[{\tt bin}] : contains the executable, cgad. You may want to put this directory in your path.
 \item[{\tt check}] : contains regression tests.
 \item[{\tt cmd}] : sample command files for running cgad, see section (\ref{sec:demo}).
 \item[{\tt doc}] : documentation.
 \item[{\tt lib}] : contains the cgad library, {\tt libCgad.a}.
 \item[{\tt src}] : source files 
 \item[{\tt runs}] : sample demonstrations of using Cgad to solve various problems.
\end{description}


\subsection{Basic steps}\index{basic steps}
Here are the basic steps to solve a problem with cgad.
\begin{enumerate}
  \item Generate an overlapping grid with ogen. 
  \item Run cgad (found in the {\tt bin/cgad} directory).
  \item Assign the boundary conditions and initial conditions.
  \item Choose the parameters for the PDE (such as material properties).
  \item Choose run time parameters, time to integrate to, time stepping method etc.
  \item Compute the solution (optionally plotting the results as the code runs).
  \item When the code is finished you can look at the results (provided you saved a
     `show file') using {\tt plotStuff}.
\end{enumerate}
The commands that you enter to run cgad can be saved in a \Index{command file} (by default
they are saved in the file `cgad.cmd'). This command file can be used to re-run
the same problem by typing `cgad file.cmd'. The command file can be edited to change parameters.

To get started you can run one of the examples in the {\tt cmd} directory or
in the {\tt runs} directory.

\clearpage
\section{Sample command files for running cgad} \label{sec:demo}

Command files are supported throughout the Overture. They are files
that contain lists of commands. These commands can initially be saved
when the user is interactively choosing options.  The \Index{command files}
can then be used to re-run the job. Command files can be edited and
changed.

In this section we present a number of command files that can be used
to run cgad.

\subsection{Running a command file} \label{sec:runningCommandFiles} 

Given a \Index{command file} for cgad such as {\tt heatSource.cmd}, found in {\tt
cmd/heatSource.cmd}, one can type `{\tt cgad heatSource.cmd}' to run this command
file . You can also just type `{\tt cgad heatSource}, leaving off the {\tt
.cmd} suffix. Typing `{\tt cgad -noplot heatSource}' will run without
interactive graphics (unless the command file turns on graphics). Note that here
I assume that the {\tt bin} directory is in your path so that the {\tt
cgad} command is found when you type it's name. The Cgad sample
command files will automatically look for an overlapping grid in the {\tt
Overture/sampleGrids} directory, unless the grid is first found in the location
specified in the command file.

When you run a command file a graphics screen will appear and after some
processing the run-time dialog should appear and the initial conditions will be
plotted. The program will also print out some information about the problem
being solved. At this point choose {\tt continue} or {\tt movie
mode}. Section~(\ref{sec:runTimeDialog}) describes the options available in the
run time dialog.

When running in parallel it is convenient to define a shell variable for the parallel
version of cgad.
For example in the tcsh shell you might use
\begin{verbatim}
  set cgadp  = ${CGBUILDPREFIX}/ad/bin/cgad
\end{verbatim} % $
An example of running the parallel version is then
\begin{verbatim}
  mpirun -np 2 $cgadp heatSource
\end{verbatim} % $


%------------------------------------------------------------------------------------------
\clearpage
\subsection{A deforming domain example} \label{sec:deform}


The directory {\tt cg/ad/runs/deform} holds an example {\em run} that demonstrates using Cgad to solve 
a problem on a deforming domain.



\noindent \textbf{Sinusoid Deform Example}

\noindent (1) Generate the grid using the command file {\tt freeSurfaceGrid2d.cmd} in {\tt Overture/sampleGrids}:
\begin{verbatim}
   ogen -noplot freeSurfaceGrid2d -interp=e -factor=4 -ml=1 
\end{verbatim}

\noindent (2) Run cgad: Advection diffusion (AD) of a Gaussian pulse with a sinusoidally deforming surface. 
\begin{verbatim}
  cgad deformingSurface -g=freeSurfaceGrid2de4.order2.ml1 -motion=sinusoid -kappa=.02 ...
         -ampy=.05 -tf=2. -tp=.01 -ic=pulse -tz=none -go=halt
\end{verbatim}
Results from this run are shown in Figure~\ref{fig:pulseDeformSinusoid}.

\noindent{\bf Notes:}
\begin{enumerate}
  \item Note that the command line arguments are processed and used in the .cmd file (they are not directly passed
to cgadMain, but only indirectly passed). 
  \item The motion of the interface is defined in {\tt cg/ad/userDefinedDeformingSurface.C}
  \item To support deforming motion, the grid to be deformed should be constructed using a NurbsMapping to define
    the surface an the hyperbolic grid generator (as found in {\tt freeSurfaceGrid2d}).
\end{enumerate}



\noindent \textbf{Concentration Deform Example}
This example demonstrates a simple case where the motion of the deforming surface depends on the solution.
  Use the same grid as for the {\em  Sinusoid Deform Example}.
\noindent (2) Run cgad: Advection diffusion (AD) of a Gaussian pulse with surface whose motion depends 
  on the concentration.
\begin{verbatim}
cgad deformingSurface -g=freeSurfaceGrid2de4.order2.ml1 -motion=concentration -kappa=.05 ...
         -tf=10. -tp=.05 -tz=none -ic=pulse -a=0 -b=0 -bc=n -ts=im -go=halt
\end{verbatim}
Results from this run are shown in Figure~\ref{fig:pulseDeformConcentration}.

\noindent{\bf Notes:}
\begin{enumerate}
   \item the motion of the interface is defined in {\tt cg/ad/userDefinedDeformingSurface.C} and satisfies the equation
\begin{align*}
   \frac{\partial \eta(s,t)}{\partial t} = \alpha ( u(s,t) - u_e )  
\end{align*}
where $y=\eta(s,t)$ is the height of the interface, $s$ is the arclength variable (for the undeformed interface),
$u(\xv(s),t)$ is the solution to the AD equations and where $\alpha$ and $u_e$ are some specified constants.
For $\alpha>0$, the interface will move downward where the solution $u$ on the interface is less than $u_e$,
and move upward where $u$ is greater than $u_e$. 
  \item Note that Neumann boundary conditions are used (-bc=n). 
  \item The option {\tt -ts=im} causes an implicit time-stepping algortihm to be chosen.
\end{enumerate}

{
\begin{figure}[hbt]
\newcommand{\figWidth}{5.5cm}
\newcommand{\trimfig}[2]{\trimFig{#1}{#2}{0.1}{0.05}{.05}{.05}}
\begin{center}
\begin{tikzpicture}[scale=1]
  \useasboundingbox (0,.5) rectangle (17,5.75);  % set the bounding box (so we have less surrounding white space)
  \draw ( 0.0, 0) node[anchor=south west] {\trimfig{figures/cgadDeform_sinusoid0}{\figWidth}};
  \draw ( 5.7, 0) node[anchor=south west] {\trimfig{figures/cgadDeform_sinusoid1}{\figWidth}};
  \draw (11.4, 0) node[anchor=south west] {\trimfig{figures/cgadDeform_sinusoid2}{\figWidth}};
 % - labels
 %   \draw (\txa,4.75) node[draw,fill=white,anchor=east] {\scriptsize $t=0.5$};
 %   \draw (\txb,4.75) node[draw,fill=white,anchor=east] {\scriptsize $t=1.0$};
 %   \draw (\txc,4.75) node[draw,fill=white,anchor=east] {\scriptsize $t=1.5$};
 %  \draw (current bounding box.south west) rectangle (current bounding box.north east);
% grid:
%  \draw[step=1cm,gray] (0,0) grid (17.0,5);
\end{tikzpicture}
\end{center}
\caption{Results from the {\em  Sinusoid Deform Example}. The top surface deforms over time with a sunusoidal motion as
a pulse advects (upward and to the right) and diffuses.}
\label{fig:pulseDeformSinusoid}
\end{figure}
}

{
\begin{figure}[hbt]
\newcommand{\figWidth}{5.5cm}
\newcommand{\trimfig}[2]{\trimFig{#1}{#2}{0.1}{0.05}{.05}{.05}}
\begin{center}
\begin{tikzpicture}[scale=1]
  \useasboundingbox (0,.5) rectangle (17,5.75);  % set the bounding box (so we have less surrounding white space)
  \draw ( 0.0, 0) node[anchor=south west] {\trimfig{figures/cgadDeform_concentration0}{\figWidth}};
  \draw ( 5.7, 0) node[anchor=south west] {\trimfig{figures/cgadDeform_concentration1}{\figWidth}};
  \draw (11.4, 0) node[anchor=south west] {\trimfig{figures/cgadDeform_concentration2}{\figWidth}};
 % - labels
 %   \draw (\txa,4.75) node[draw,fill=white,anchor=east] {\scriptsize $t=0.5$};
 %   \draw (\txb,4.75) node[draw,fill=white,anchor=east] {\scriptsize $t=1.0$};
 %   \draw (\txc,4.75) node[draw,fill=white,anchor=east] {\scriptsize $t=1.5$};
 %  \draw (current bounding box.south west) rectangle (current bounding box.north east);
% grid:
%  \draw[step=1cm,gray] (0,0) grid (17.0,5);
\end{tikzpicture}
\end{center}
\caption{Results from the {\em  Concentration Deform Example}. The top surface deforms over time as a function of the
current concentration on the interface.}
\label{fig:pulseDeformConcentration}
\end{figure}
}

{
\begin{figure}[hbt]
\newcommand{\figWidth}{8.cm}
\newcommand{\trimfig}[2]{\trimFig{#1}{#2}{0.1}{0.05}{.3}{.4}}
\begin{center}
\begin{tikzpicture}[scale=1]
  \useasboundingbox (0,.5) rectangle (17,11.5);  % set the bounding box (so we have less surrounding white space)
  \draw ( 0.0,4.0) node[anchor=south west] {\trimfig{figures/deformingEye0p0}{\figWidth}};
  \draw ( 8.0,4.0) node[anchor=south west] {\trimfig{figures/deformingEye0p1}{\figWidth}};
  \draw ( 0.0,  0) node[anchor=south west] {\trimfig{figures/deformingEye0p25}{\figWidth}};
  \draw ( 8.0,  0) node[anchor=south west] {\trimfig{figures/deformingEye0p5}{\figWidth}};
 % - labels
 %   \draw (\txa,4.75) node[draw,fill=white,anchor=east] {\scriptsize $t=0.5$};
 %   \draw (\txb,4.75) node[draw,fill=white,anchor=east] {\scriptsize $t=1.0$};
 %   \draw (\txc,4.75) node[draw,fill=white,anchor=east] {\scriptsize $t=1.5$};
 %  \draw (current bounding box.south west) rectangle (current bounding box.north east);
% grid:
%  \draw[step=1cm,gray] (0,0) grid (17.0,5);
\end{tikzpicture}
\end{center}
\caption{Results from the {\em  deforming eye}. The top surface deforms over time by a specified motion.}
\label{fig:deformingEye}
\end{figure}
}



\clearpage
% -----------------------------------------------------------------------
\subsection{User defined variable coefficients} \label{sec:userDefinedVariableCoefficients}

A user can supply variable advection coefficients and variaable diffusion coefficients
by editing the file {\tt cg/ad/src/getCoefficients.C} and using the appropriate
run time options. 


% \clearpage
% -----------------------------------------------------------------------
\subsection{User defined deforming surface} \label{sec:userDefinedDeformingSurface}

To define the motion of a deforming surface, one can add a new option to the
file {\tt cg/ad/src/userDefinedDeformingSurface.C}. 



% \clearpage
% -----------------------------------------------------------------------
\subsection{User defined forcing} \label{sec:userDefinedForcing}


The file {\tt cg/common/src/userDefinedForcing.C} can be used to define new forcing functions
for use with Cgad. Here are the steps to take to add your own forcing:
\begin{enumerate}
  \item Edit the file  {\tt cg/common/src/userDefinedForcing.C}.
  \item Add a new option to the function {\tt setupUserDefinedForcing()}. Follow one of the previous
         examples. 
  \item Implement the forcing option in the function {\tt userDefinedForcing(...)}.
  \item Type {\em make} from the {\tt cg/ad} directory to recompile Cgad.
  \item Run Cgad and choose the {\em userDefinedForcing} option from the {\em forcing options...} menu.
    (see the example command file {\tt ad/cmd/userDefinedForcing.cmd}).
\end{enumerate}

%  {
%  \begin{figure}[hbt]
%  \newcommand{\figWidth}{7.5cm}
%  \newcommand{\trimfig}[2]{\trimFig{#1}{#2}{0.1}{0.05}{.05}{.05}}
%  \begin{center}
%  \begin{tikzpicture}[scale=1]
%    \useasboundingbox (0,.5) rectangle (15,7.75);  % set the bounding box (so we have less surrounding white space)
%    \draw ( 0.0, 0) node[anchor=south west] {\trimfig{figures/userDefinedGaussianSourcesEx}{\figWidth}};
%    \draw ( 7.5, 0) node[anchor=south west] {\trimfig{figures/userDefinedGaussianSourcesEy}{\figWidth}};
%   % - labels
%   %   \draw (\txa,4.75) node[draw,fill=white,anchor=east] {\scriptsize $t=0.5$};
%   %   \draw (\txb,4.75) node[draw,fill=white,anchor=east] {\scriptsize $t=1.0$};
%   %   \draw (\txc,4.75) node[draw,fill=white,anchor=east] {\scriptsize $t=1.5$};
%  %  \draw (current bounding box.south west) rectangle (current bounding box.north east);
%  % grid:
%  %  \draw[step=1cm,gray] (0,0) grid (17.0,5);
%  \end{tikzpicture}
%  \end{center}
%  \caption{User defined forcing example showing results from specifying two Gaussian sources. Left $E_x$ and right $E_y$}
%  \label{fig:userDefinedForcingGuassianSources}
%  \end{figure}
%  }

% =======================================================================================================
% \clearpage
% \section{Variable material properties} \label{sec:varMat}
% 
% Finish me...




% =======================================================================================================
% \clearpage
\section{Options} \label{sec:option}

% ---------------------------------------------------------------------------
\subsection{Options affecting the scheme}

\begin{description}
  \item [\qquad cfl] $value$ : set the CFL number. The schemes are usually stable to CFL=1. The default is CFL=.9.
\end{description}


% ---------------------------------------------------------------------------
\subsection{Run time options}

** FINISH ME **

\begin{description}
  \item [\qquad tFinal] $value$ : solve the equations to this time.
  \item [\qquad tPlot] $value$ : time increment to save results and/or plot the solution.
  \item [\qquad debug] $value$ : an integer bit flag that turns on debugging information. For example, set debug=1 for some info, debug=3 (=1+2) for some
    more, debug=7 (=1=2+4)for even more. 
\end{description}

% ---------------------------------------------------------------------------
\subsection{Plotting options}

** FINISH ME **

\begin{description}
  \item [\qquad plot errors] $[0|1]$ :
  \item [\qquad check errors] $[0|1]$ :
\end{description}

% ---------------------------------------------------------------------------
\subsection{Output options}

** FINISH ME **

\begin{description}
  \item [\qquad specify probes] : specify a list of probe locations as $x$, $y$, $z$ values, one per line, finishing the
         list with 'done'. The solution values at these locations (actually the closest grid point to each location, with these
         values written to the screen) are
    written to a text file whose default name is "probeFile.dat". 
    In the following example we specify two probe locations, $(.2,.3,.1)$ and $(.4,.6,.3)$, 
\begin{verbatim}
specify probes
  .2 .3 .1.
  .4 .6 .3
done
\end{verbatim}
%     Each line of the probe file contains the time followed
%     by the three components of $\Ev$ (or in two-dimensions $E_x$, $E_y$ and $H_z$) for each probe location. For example,
%     with two probes specified the file in three dimensions would contain data of the form 
% \begin{verbatim}
%    t1 Ex11 Ey11 Ez11  Ex12 Ey12 Ez12 
%    t2 Ex21 Ey21 Ez21  Ex22 Ey22 Ez22 
%    ...
% \end{verbatim}
  \item [\qquad probe file:] $name$ : specify the name of the probe file.
  \item [\qquad probe frequency] $value$ : specify the frequency at which values are saved in the probe file. For example,
      if the probe frequency is set to 2 then the solution will be saved every 2nd time step to the probe file. 
\end{description}





\vfill\eject
\bibliography{\homeHenshaw/papers/henshaw}
\bibliographystyle{siam}

\printindex

\end{document}

% ***************************************************************************************************




