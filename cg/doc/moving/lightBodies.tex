\section{Motion of ``Light'' Rigid Bodies}\label{sec:lightBodies}

The coupling of fluid flow with ``light'' moving rigid bodies can cause 
the standard time stepping algorithm to go unstable. In this section
we discuss this issue.

We first derive the expression that relates the acceleration of a point on the boundary
of a rigid body to the force $\Fv$ and torque $\Gv$ on the body. 


The motion of a point $\rv$ on the boundary of a rigid body is given 
given by the sum of the position of the centre of mass, $\xv(t)$ plus a rotation,
\[
    \rv(t) = \xv(t) + R(t) (\rv(0)-\xv(0)) .
\]
The acceleration of this point is
\begin{align*}
   \av= \Fv/M +\sum_i (\dot{\omegav}\times\ev_i+(\omegav\cdot\ev_i)\omegav+|\omegav|^2\ev_i)(r_i(0)-x_i(0)) 
\end{align*}
where $M$ is the mass of the body and $\Fv$ is the force on the center of mass.


Now if $E=[\ev_1~ \ev_2~ \ev_3]$ is the matrix of principal axes of inertia, $\ev_i$ then the angular momentum is
\begin{align*}
   \hv &= E(t) M_I \omegav, \\
    M_I &= diag( I_1, I_2, I_3 ), \qquad \text{$I_i$ are the moments of inertia}, \\
    R(t) &= E(t) E^{-1}(0)\\
   \dot{\hv} &= \dot{E} M_I \omegav + E M_I \dot{\omegav} = \Gv  \\
    \dot{E} &= \omegav^*\Ev , \quad (\text{$\omega^*$ is the matrix such that $\dot{\ev_i} = \omegav\times\ev_i, i=1,2,3$}), 
\end{align*}
and whence
\begin{align*}
   \dot{\omegav} &= M_I^{-1} E^{-1}\Big( \Gv - \omegav^*\Ev M_I \omegav\Big)
\end{align*}
Thus we arrive at the expression relating the acceleration of a point on the boundary to the force and 
torque on the body
\begin{align}
   \av & = \Fv/M +\sum_i (\dot{\omegav}\times\ev_i+(\omegav\cdot\ev_i)\omegav+|\omegav|^2\ev_i)(r_i(0)-x_i(0)) , \label{eq:solidAccel}\\
   \dot{\omegav} &= M_I^{-1} E^{-1}\Big( \Gv - \omegav^*\Ev M_I \omegav\Big).
\end{align}

Consider now a rigid body $B$ immersed in a fluid with density $\rho_f$ and velocity $\uv$ and pressure $p$,
The momentum equation for the fluid in a reference frame moving with the body is 
\begin{align*}
  \rho_f \big( \dot{\uv} + (\uv-\dot{\gv})\cdot\grad\uv \big)+ \grad p &= \grad\cdot\tauv ,
\end{align*}
where $\gv$ is the ``grid'' velocity.
If the fluid-solid boundary is taken as a  no-slip wall then $\uv=\dot{\gv}$ for $\xv \in \partial B$ and
\begin{align}
  \rho_f \dot{\uv} + \grad p = \grad\cdot\tauv ,\quad \xv\in\partial B .
\end{align}
In particular the following condition on the normal
derivative of the pressure is satisfied
\begin{align}
  {\partial p\over\partial n} &= - \rho_f \big( \nv\cdot\dot{\uv} \big) + \nv\cdot\grad\cdot\tauv ,\quad \xv\in\partial B
     \label{eq:pn}
\end{align}
On the boundary $\partial B$ the fluid acceleration $\dot{\uv}$ is equal to the solid acceleration $\av$ from~\eqref{eq:solidAccel}
and thus 
\begin{align}
  {\partial p\over\partial n} &= - \rho_f \big( \nv\cdot\Big( \Fv/M +\sum_i  (M_I^{-1} E^{-1}\Gv)\times\ev_i (r_i(0)-x_i(0))
             + ... \Big)    + \nv\cdot\grad\cdot\tauv ,\quad \xv\in\partial B
     \label{eq:pn2}
\end{align}
Now
\begin{align}
  \Fv &= \int_{\partial B} p \nv ~ds + \text{viscous terms} \label{eq:F}\\
  \Gv &= \int_{\partial B} (\rv-\xv)\times(p \nv) ~ds + \text{viscous terms} \label{eq:G}
\end{align}
Combining~\eqref{eq:pn2} and the expression for $\Fv$ from~\eqref{eq:F} and $\Gv$ from~\eqref{eq:G} we see that
\begin{align}
  {\partial p\over\partial n} &= - {\rho_f\over M} \nv(\xv) \cdot\Big( \int_{\partial B} p(s) \nv(s) ~ds \Big) \\
          &-  \rho_f \nv\cdot\Big( \sum_i  (M_I^{-1} E^{-1}\int_{\partial B} (\rv-\xv)\times(p \nv) ~ds )\times\ev_i (r_i(0)-x_i(0))  + ...
   \label{eq:pCoupled}
\end{align}
This expression gives us a clearer indication of interface condition on the pressure that couples the fluid and solid.

In the FSI time stepping algorithm, the pressure in the fluid is computed using the boundary condition~\eqref{eq:pn}
using a approximate (predicted) value for $\dot{\uv}=\av$. Corrector steps are then applied to the fluid and
solid. This sequence of predictor and corrector steps would correspond roughly to evaluating equation~\eqref{eq:pCoupled}
as 
\begin{align}
  {\partial p^{k}\over\partial n} &= - {\rho_f\over M} \nv(\xv) \cdot\Big( \int_{\partial B} p^{k-1}(s) \nv(s) ~ds \Big) + ...
\end{align}
In the case of an incompressible fluid the above interface BC would be used when solving 
the Possion equation for the pressure, $\Delta p^{k}=...$.  We can see why
``light'' bodies may pose difficulties as the coefficient ${\rho_f\over M}$ on
the right-hand-side may be large and this iteration may not converge.



\subsection{Model problem}

Consider the one-dimensional FSI model problem
\begin{align}
   p_{xx} & = f, \qquad a < x < 0 , \quad \text{(fluid)}, \\
   M_s \ddot{x} &= p(0,t)  A , \qquad \text{(solid)}, \\
   p_x(0,t) &= -\rho_f \ddot{x} = -\rho_f p(0,t) A/M_s,  \qquad \text{(interface)},\\
   p(a,t) &=0 
\end{align}
where the solid has height $A$. 
The pressure in the fluid thus satisfies
\begin{align}
   p_{xx} & = f, \qquad a < x < 0 , \quad \text{(fluid)} ,\\
   p_x(0,t) &= -\rho_f p(0,t) A/M_s , \quad p(a,t) =0 .
\end{align}
We solve this by iteration (to mimic the normal time stepping method for the general FSI problem) 
\begin{align}
   p^{k+1}_{xx} & = f, \qquad a < x < 0 , \quad \text{(fluid)}, \\
   p^{k+1}_x(0,t) &= -\rho_f p^k(0,t) A/M_s , \quad p^{k+1}(a,t) =0 .
\end{align}
The difference $q^{k} = p^{k}-p^{k-1}$  satisfies $q^{k}_{xx}=0$ and thus
\begin{align}
   q^{k+1} = C^{k+1}(x-a)
\end{align}
Whence from the interface condition
\begin{align}
    C^{k+1} & = -\rho_f (-a)  A/M_s C^k , \\
            &= \kappa C^k, \\
   \kappa &\equiv \rho_f a A/M_s ,\\
          &= - M_f/M_s ,
\end{align}
where $M_f = \rho_f (-a) A$ is the mass of the fluid (since $(-a) A = V_f$ is the volume of the fluid region). 
The iteration will converge provided
\begin{align}
   \vert \kappa \vert = \vert M_f/M_s \vert <1 .
\end{align}
Note that $\kappa<0$ implying that $q^k$ will change signs at each iteration. This is consistent with 
what is observed in practice.

Consider the under-relaxed iteration
\begin{align}
   p^{k+1}_{xx} & = f, \qquad a < x < 0 ,  \\
   p^{k+1}_x(0,t) &= (1-\alpha) p^k_x(0,t)  + \alpha  (-\rho_f  A/M_s  p^k(0,t) ) 
\end{align}
Then
\begin{align}
   C^{k+1} & = (1-\alpha) C^{k} - M_f/M_s  \alpha C^k, \\
   C^{k+1} & = \kappa C^k , \\
      \kappa &= 1- (1+M_f/M_s)\alpha 
\end{align}
The under-relaxed iteration converges provided
\begin{align}
   \alpha < {2\over 1+ M_f/M_s} 
\end{align}
