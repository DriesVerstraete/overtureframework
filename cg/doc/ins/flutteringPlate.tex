\subsection{Fluttering plate}\label{sec:flutteringPlate}

This example shows the simulation of a light plate falling under the influence
of gravity. There is an upward flow with speed approximately equal to the
rate at which the plate falls. The plate starts to sway from side to side and
eventually also tumbles.
This example demonstrates the new time stepping scheme that has been developed
to treat the motion of ``light'' rigid bodies (v24). 
% The simulation used the command file {\tt cg/ins/wing2d.cmd}
% and the grid was made with the ogen script {\tt Overture/sampleGrids/joukowsky2d.cmd}. 
% \newcommand{\dropStickDir}{\homeHenshaw/runs/cgins/dropStick}

% 
{
\newcommand{\figWidthd}{6.5cm}
\newcommand{\trimfig}[2]{\trimPlot{#1}{#2}{.0}{.0}{.0}{.0}}
\begin{figure}[hbt]
\begin{center}
\begin{tikzpicture}[scale=1]
  \useasboundingbox (0,.75) rectangle (13.,19.5);  % set the bounding box (so we have less surrounding white space)
%
  \draw ( 0.0,13.) node[anchor=south west,xshift=-4pt,yshift=+0pt] {\trimfig{\cgDoc/ins/fig/dropStickVor2p0}{\figWidthd}};
  \draw ( 0.0,6.5) node[anchor=south west,xshift=-4pt,yshift=+0pt] {\trimfig{\cgDoc/ins/fig/dropStickVor4p0}{\figWidthd}};
  \draw ( 0.0,  0) node[anchor=south west,xshift=-4pt,yshift=+0pt] {\trimfig{\cgDoc/ins/fig/dropStickVor6p0}{\figWidthd}};
  \draw ( 6.5,13.) node[anchor=south west,xshift=-4pt,yshift=+0pt] {\trimfig{\cgDoc/ins/fig/dropStickVor16p0}{\figWidthd}};
  \draw ( 6.5,6.5) node[anchor=south west,xshift=-4pt,yshift=+0pt] {\trimfig{\cgDoc/ins/fig/dropStickVor18p0}{\figWidthd}};
  \draw ( 6.5, .0) node[anchor=south west,xshift=-4pt,yshift=+0pt] {\trimfig{\cgDoc/ins/fig/dropStickVor20p0}{\figWidthd}};
%
 % \draw (current bounding box.south west) rectangle (current bounding box.north east);
% grid:
% \draw[step=1cm,gray] (0,0) grid (13,19);
\end{tikzpicture}
\end{center}
\caption{A fluttering plate computed with Cgins. A light plate falling due to the force of gravity in an incompressible flow.
        Contour plots of the vorticity.}
\end{figure}
}


%- \newcommand{\dropStickDir}{\ovFigures}
%- {
%- \begin{figure}[H]
%- \psset{xunit=1.cm,yunit=1.cm,runit=1.cm}%
%- \newcommand{\figWidthd}{8cm}% 
%- \newcommand{\clipfigd}[2]{\clipFig{#1}{#2}{.0}{1.}{0.2}{1.}}
%- \begin{center}%
%- \begin{pspicture}(0,0)(17.,18)%
%-  %\psgrid[subgriddiv=2]
%-  \rput(4.00,14.0){\clipfigd{\dropStickDir/dropStickVor2p0.ps}{\figWidthd}}
%-  \rput(4.00, 8.0){\clipfigd{\dropStickDir/dropStickVor4p0.ps}{\figWidthd}}
%-  \rput(4.00, 2.0){\clipfigd{\dropStickDir/dropStickVor6p0.ps}{\figWidthd}}
%-  \rput(12.2,14.0){\clipfigd{\dropStickDir/dropStickVor16p0.ps}{\figWidthd}}
%-  \rput(12.2, 8.0){\clipfigd{\dropStickDir/dropStickVor18p0.ps}{\figWidthd}}
%-  \rput(12.2, 2.0){\clipfigd{\dropStickDir/dropStickVor20p0.ps}{\figWidthd}}
%- \end{pspicture}%
%- \end{center}%
%- \caption{A fluttering plate computed with Cgins. A light plate falling due to the force of gravity in an incompressible flow.
%-         Contour plots of the vorticity.}
%- \end{figure}
%- }
% -----------------------------------------------------------------------------------
