%
% ==============================================================================================================
\section{Specification of body forces and boundary forcing} \label{sec:bodyAndBoundaryForcing}


In this section we describe how to define {\em body forces} and {\em boundary forcings}.
{\em Body forces} are volume forcing terms that are added to the right-hand-side of
the PDE's (momentum equations, temperature equation etc.). 
{\em Boundary forcings} are right-hand-sides added to the boundary conditions.

\noindent Examples of body forces are
\begin{enumerate}
   \item add a heat source to a given region of the domain.
   \item add a drag force  to a given region of the domain.
   \item define an immersed boundary over some region. This option provides an approximate way to add 
         new geometry (e.g. small or complicated features) to a problem without having to build a proper overlapping grid 
         for the features.
   \item define a wake model such as an actuator disk (e.g. modeling a fan).
   \item change material parameters such as thermal conductivity, heat capacity or viscosity over a given region (needs to be finished).
\end{enumerate}

\noindent Examples of boundary forcings are:
\begin{enumerate}
   \item specify the temperature or heat flux on a sub-region of a grid face.
   \item add a local inflow region with parabolic profile to a portion of a grid face.
   \item define a temperature boundary condition that varies from Dirichlet (isothermal) to Neumann (adiabatic) or mixed
     over a given grid face.
\end{enumerate}


\noindent The construction of a body/boundary force requires the specification of
\begin{description}
   \item[\quad Region] : specify a region (e.g. box, cylinder, triangulated surface) over which the body force applies.
   \item[\quad Parameters] : specify parameters that define the force (e.g. inflow velocity, temperature, heat flux)
   \item[\quad Profile] : define a spatial profile for the force (e.g. parabolic inflow).
   \item[\quad Time variation] : define the time variation of the force (e.g. define a time varying heat flux on
        a sub-region of a boundary).
\end{description}
In addition, a given body force (e.g. drag force or actuator disk) will have parameters associated with it (e.q. drag law).


{\bf Note:} If multiple forcings are defined for the same volume or boundary, the corresponding 
forces are (by default) added together. 


%
\begin{figure}[hbt]
\newcommand{\figWidth}{7.0cm}
\newcommand{\trimfig}[2]{\trimFig{#1}{#2}{0.23}{.225}{.2}{.22}}
\begin{center}
 \begin{tikzpicture}[scale=1]
 \useasboundingbox (0,.5) rectangle (15.,7.);  % set the bounding box (so we have less surrounding white space)
%
 \draw (0,0) node[anchor=south west] {\trimfig{\insDocDir/fig/bodyForcingT}{\figWidth}};
 \draw (8,0) node[anchor=south west] {\trimfig{\insDocDir/fig/bodyForcingSL}{\figWidth}};
% \draw[step=1cm,gray] (0,0) grid (15,7);
% \draw (current bounding box.south west) rectangle (current bounding box.north east);
 \end{tikzpicture}
\end{center}
\caption{This figure, temperature on left and stream-lines on the right, 
    illustrates uses of various body-forces and boundary-forcings. 
Boundary-forces are
   used to define inflow regions on the left and bottom walls. Immersed-boundary body forces are used
  to define a square obstacle and elliptical obstacle.}
\end{figure}



% ----------------------------------------------------------------------------------------
\subsection{Defining body-force and boundary-force regions} \label{sec:bodyAndBoundaryForcingRegions}


Forcing regions over which body/boundary forces are applied can be defined
using one or more of the following 
\begin{description}
   \item[\quad Box] : a box (square in two-dimensions). 
   \item[\quad Ellipse] : an ellipse (or circle).
   \item[\quad mask from grid function]: the body is defined by a grid function {\em mask} that contains values
        indicating whether a grid point is inside or outside the body. Ideally this should be a signed
        distance function with negative values being inside the body. 
   \item[\quad Mapping] : use this option to define an arbitrary curve in 2D or a surface in 3D (described in more detail below). 
\end{description}

{\bf Note:} when defining regions for boundary forcings, the actual boundary should be enclosed 
in the region. Thus in two-dimensions, to define a boundary forcing on an interval $[x_a,x_b]$ of a horizontal
wall at $y=y_w$ one should define a narrow box (square) to enclose the interval, $B = [x_a,x_b]\times[y_w-\epsilon,y_w+\epsilon]$.

\paragraph{2D regions defined by a curve:} Very general body force regions in 2D can be defined
  from Overture Mappings that can define closed curves in 2D (e.g. {\tt AirfoilMapping}, {\tt NurbsMapping}, {\tt SplineMapping}). 
  The determination of whether a point is inside and outside the region is determined with a fast ray-tracing algorithm.
  Figure~\ref{fig:bodyForceRegions2D} shows examples of 2D body force regions defined from Mappings.

\paragraph{3D regions defined by surface triangulations:} Very general body force regions in 3D can be defined
as an unstructured surface. The {\tt UnstructuredMapping} class can be used to
represent these surfaces. Input formats include the popular STL file format
(STereoLithography or Standard Tessellation Language) and PLY file format
(Polygon File Format or the Stanford Triangle
Format). Figure~\ref{fig:bodyForceRegions} shows examples of some surfaces. The
{\tt UnstructuredMapping} has a fast algorithm for determining whether a given
point in space is inside or outside of the surface (this requires the surface to
be {\em water-tight}. The algorithm uses an alternating-digit-tree (ADT) and
ray-tracing.

\begin{figure}[hbt]
\newcommand{\figWidth}{5.5cm}
\newcommand{\trimfig}[2]{\trimFigb{#1}{#2}{0.0}{.0}{.0}{.0}}
\begin{center}
 \begin{tikzpicture}[scale=1]
 \useasboundingbox (0,.5) rectangle (16.5,6.25);  % set the bounding box (so we have less surrounding white space)
%
 \draw (  0,0) node[anchor=south west] {\trimfig{\insDocDir/fig/blob128T3p0}{\figWidth}};
 \draw (5.5,0) node[anchor=south west] {\trimfig{\insDocDir/fig/starfish256T4p0}{\figWidth}};
 \draw (11.,0) node[anchor=south west] {\trimfig{\insDocDir/fig/immersedHeatedAirfoilT5p0}{\figWidth}};
% \draw[step=1cm,gray] (0,0) grid (16.5,6);
% \draw (current bounding box.south west) rectangle (current bounding box.north east);
 \end{tikzpicture}
\end{center}
\caption{Examples of 2D body force regions defined from a curve used as immersed boundaries and heat sources. 
Left: buoyant incompressible flow past a heated {\em blob} - the curve was defined with the {\tt NurbsMapping}.
Middle: buoyant incompressible flow past a heated {\em starfish} - the curve was defined with the {\tt NurbsMapping}.
Right: buoyant incompressible flow past a heated {\em NACA airfoil} - the curve was defined with the {\tt AirfoilMapping}.
}
\label{fig:bodyForceRegions2D}
\end{figure}

\begin{figure}[hbt]
\newcommand{\figWidth}{5.5cm}
\newcommand{\trimfig}[2]{\trimFigb{#1}{#2}{0.0}{.0}{.0}{.0}}
\begin{center}
 \begin{tikzpicture}[scale=1]
 \useasboundingbox (0,.5) rectangle (16.5,6.25);  % set the bounding box (so we have less surrounding white space)
%
 \draw (  0,0) node[anchor=south west] {\trimfig{\insDocDir/fig/cowPlySurface}{\figWidth}};
 \draw (5.5,0) node[anchor=south west] {\trimfig{\insDocDir/fig/candleStickPly}{\figWidth}};
 \draw (11.,0) node[anchor=south west] {\trimfig{\insDocDir/fig/dolphinsPLY}{\figWidth}};
% \draw[step=1cm,gray] (0,0) grid (16.5,6);
% \draw (current bounding box.south west) rectangle (current bounding box.north east);
 \end{tikzpicture}
\end{center}
\caption{Examples of unstructured surfaces create from PLY files. These can be used as immersed boundary regions and heat
sources in a Cgins computation.}
\label{fig:bodyForceRegionsPLY}
\end{figure}


\begin{figure}[hbt]
\newcommand{\figWidth}{7.0cm}
\newcommand{\trimfiga}[2]{\trimFigb{#1}{#2}{0.0}{.0}{.0}{.0}}
\newcommand{\figWidthb}{7.0cm}
\newcommand{\trimfigb}[2]{\trimFigb{#1}{#2}{0.1}{.25}{.15}{.2}}
\begin{center}
 \begin{tikzpicture}[scale=1]
 \useasboundingbox (0,.5) rectangle (15.,7.25);  % set the bounding box (so we have less surrounding white space)
%
 \draw (0,0) node[anchor=south west] {\trimfiga{\insDocDir/fig/lego1024Enstropy1p0b}{\figWidth}};
 \draw (8,0) node[anchor=south west] {\trimfigb{\insDocDir/fig/bunnyU}{\figWidthb}};
% \draw[step=1cm,gray] (0,0) grid (15,7);
% \draw (current bounding box.south west) rectangle (current bounding box.north east);
 \end{tikzpicture}
\end{center}
\caption{Examples of 3D body force regions defined from a unstructured surface used as immersed boundaries and heat sources.
 The UnstructuredMapping class is
used to define the triangulated surface. Left: incompressible flow past a {\em LEGO} block - the surface was defined
from an STL file. Right: incompressible flow past the the Stanford {\em bunny} - the surface was defined from a PLY file.}
\label{fig:bodyForceRegions}
\end{figure}


% ----------------------------------------------------------------------------------------
\subsection{Defining body forces} \label{sec:bodyForces}

For the incompressible Navier-Stokes (with Boussinesq approximation), the body forcings, $\Fv_{\uv}$ and $F_T$ are
added to the momentum and temperature equations as 
\begin{align}
  \uv_t + \uv\cdot\grad\uv + \grad p = \nu \Delta \uv + \alpha \gv (T-T_0)  + \Fv_{\uv}, \\
  T_t + \uv\cdot\grad T  = \kappa \Delta T  + F_T. 
\end{align}


\noindent {\bf Drag force}: The drag force consists of the sum of a linear and quadratic drag law, 
\begin{align}
  \Fv_{\uv} & =  - \Big\{ \beta_1 ( \uv- \uv_d )  +-\beta_2 |\uv- \uv_d| ( \uv - \uv_d ) \Big\} \Pc(\xv,t), \\
   F_T       & = - \frac{\beta_T}{\dt}\Pc(\xv,t)( T - T_d ), 
\end{align}
where $\Pc(\xv,t)$ is a profile function. The {\em target} velocity is $\uv_d$ and the target temperature is $T_d$. 
In general (to-do) the target functions can be a function of space and time $\uv_d=\uv_d(\xv,t)$, $T_d=T_d(\xv,t)$.

\noindent {\bf Immersed boundary}: The {\em immersed boundary} force takes the form
\begin{align}
  \Fv_{\uv} & =  - \frac{\beta}{\dt}\Pc(\xv,t)( \uv - \uv_d ), \\
  F_T       & = - \frac{\beta_T}{\dt}\Pc(\xv,t)( T - T_d ), 
\end{align}
where $\Pc(\xv,t)$ is a profile function and $\dt$ is the time-step. 

\noindent {\bf Heat source}: For the incompressible Navier-Stokes (with Boussinesq approximation) the heat source 
    $F_T(\xv,t)$ is added to the right-hand-side of the temperature equation 
\begin{align}
  F_T =  \Pc(\xv,t) \frac{1}{C_p} F_T(\xv,t) . 
\end{align}
{\bf Question: should we scale the heat source by $C_p$ ?}


% ----------------------------------------------------------------------------------------
\subsection{Defining boundary forcing} \label{sec:boundaryForcing}

The {\em body-force regions} defined in Section~\ref{sec:bodyAndBoundaryForcingRegions}
can be used to construct boundary conditions that vary over a a given boundary.



% ----------------------------------------------------------------------------------------
\subsection{Defining body force profiles} \label{sec:bodyAndBoundaryForcingProfiles}


% -------------------------------------------------------------------------------------------------------
\subsubsection{Parabolic velocity profile} \label{sec:parabolic}

  A `parabolic' profile can be specified for a body or boundary forcing.
The parabolic profile can be useful, for example,
in specifying the velocity profile at an inflow boundary.
The parabolic profile is zero at the
boundary of the body force region and increases to a specified value $U_{\rm max}$ at 
a distance $d$ from the boundary:
\[
     u(\xv) = \begin{cases}
               U_{\rm max} (2 -s/d ) s/d & \text{if $s\le d$ } \\
               U_{\rm max} &     \text{if $s> d$ }
              \end{cases}
\]
Here $s$ is the shortest distance between a point $\xv$ in the region to the 
boundary of the region, 
and $d$ is the user specified {\it parabolic depth}.

% ----------------------------------------------------------------------------------------
% \subsection{Defining body region material parameters} \label{sec:bodyForcingMaterialParameters}
% 
% 
% The material parameters such as $\kappa$ and $C_p$ can vary in space. ... finish me ...