%
% ==============================================================================================================
\section{Probes: defining output quantities at points and on regions} \label{sec:probes}


Probes can be used to output results during a simulation. 
A {\em point-probe} can be used, for example, to record the value of the solution at a given location over time.
A {\em region-probe} can be used, for example, to record the integrated heat flux over a surface covered by multiple
overlapping grids.
% 
There are five different types of probes: 
\begin{description}
   \item[\quad grid point probe] : output the solution at a given grid point over time.
   \item[\quad location probe] : output the solution at a given physical location over time.
   \item[\quad region probe] : output the value(s) (e.g. average value) of a quantity over a region 
      (e.g. a portion of the domain or boundary contained within a specified box) over time.
   \item[\quad boundary surface probe] : output value(s) (e.g. integrated value) of a quantity over a specified boundary surface (e.g. surface
        of a body that is covered by multiple overlapping grids).
   \item[\quad bounding box probe] : output the values of the solution at all grid points that lie
    on the faces of a {\em index box} (the box being in the index space of a grid),
\end{description}
The values of probes can be output at a specified {\em frequency} (e.g. every 10 time-steps).

A probe is defined by its {\em type} (e.g. grid-point-probe, region-probe, ...) and its
{\em quantity} (e.g. pressure, temperature, heat-flux). In addition,
probes defined over regions or boundary surfaces are characterized 
by a {\em quantity-measure} (i.e. whether to save point-values, the average-value, or the total integrated (or summed) value),
and by a {\em measure-type} (e.g. whether to use a discrete-sum or an integral to compute the average-value or total-value).


The probe output is saved to one or more output files (text files). Results from multiple region probes can
be saved in the same file if desired. The
results in these files can be plotted with the matlab scripts {\em plotProbes.m} for point-probes
and {\em plotRegionProbe.m} for region-probes.


% ----------------------------------------------------------------------------------
\subsection{Point probes} \label{sec:pointProbes}


Point probes are used to save solution values at specified positions within the domain.

The position of a {\em location} probe is specified by a physical location $\xv$.
The values at this location will be interpolated from the nearby values on the grid.

The position of a grid-point point probe can be specified by
\begin{enumerate}
  \item inputing a grid point (i.e. by grid number and $(i_1,i_2,i_3)$).
  \item specifying a grid point as the closest point to a given physical point $\xv$ (the code will determine the nearest grid point).
\end{enumerate}


% ----------------------------------------------------------------------------------
\subsection{Region probes} \label{sec:regionProbes}


Examples of {\em region probes} are
\begin{enumerate}
  \item output the average value of the velocity over a specified box enclosed in the domain, 
  \item output the integral of the temperature over the entire domain,
  \item output the integral of the fluid force over the face of a body covered by multiple overlapping grids,
  \item output the average heat flux over a portion of a boundary that lies within a specified box.
\end{enumerate}
The region probe is defined by a {\em region}, a {\em quantity}, a {\em measure-type} 
and a {\em quantity-measure}.

\paragraph{Regions:} Each {\em region probe} is associated with a region (volume or surface) that can be 
specified as one of the following types:
\begin{description}
   \item[\quad full domain region] : choose the full domain,
   \item[\quad box region] : specify a box in physical space (that intersects the grid volume), 
   \item[\quad boundary region] : choose a surface that is some portion of the boundary, 
   \item[\quad box boundary region] : specify a box in physical space (that intersects grid boundaries), 
\end{description}
The region can be chosen by specifying the corners of a box. It can also
be specified by choosing an existing {\bf body force region} (see Section~\ref{sec:bodyAndBoundaryForcing}),
and using the region associated with that object.

\noindent The {\em surface} for a {\bf boundary region} can be specified as
\begin{enumerate}
  \item a list of grid faces $(side,axis,grid)$,
  \item all grid faces with a specified {\em shared boundary flag} value,
  \item choosing an existing {\bf boundary force region} (see Section~\ref{sec:bodyAndBoundaryForcing}), 
        and using the region associated with that object.
\end{enumerate}

\paragraph{Quantity:} The probe {\em quantity} defines the solution component(s) or derived quantity to be saved, 
\begin{description}
    \item[\quad all-components] : normally all components of the solution are saved for point probes,
   \item[\quad density, velocity, temperature, pressure, ..] : save any component of the solution (the available components depend
                on the equations being solved), 
   \item[\quad heat flux] : $k\partial T/\partial n$ 
   \item[\quad lift, drag, force] : finish me ...
\end{description}

\paragraph{Measure-type:} Region probes generally save an integrated value (sum or integral) over the probe {\em region} and
the {\em measure-type} specifies whether to save the sum of discrete values or the integral,
\begin{description}
   \item[\quad sum] : sum discrete values, 
   \item[\quad volume weighted sum] : (Vsum) sum discrete values weighted by cell {\em volume}. For boundary
    regions the {\em volume} weighting becomes the cell area (3D) or cell arclength (2D).
   \item[\quad integral] : integrate values.
\end{description}


\paragraph{Quantity measure:} The integrated value of a region probe can be normalized by the number of values or by
    the volume (or surface area) of the region:
\begin{description}
   \item[\quad average] : provide the average of the quantity, i.e. sum of values divided by the number of values (if measure-type is
 {\em sum}), or integral of values divide by area (if measure-type is {\em integral}), 
   \item[\quad total] : provide the total (sum or integral). 
\end{description}

Here are mathematical descriptions of some of the probe options (MT=measure-type, QM=quantity-measure, R=region), 
\begin{alignat}{3}
   q_p  &= \sum_{\iv} \chi_R(\xv_\iv)\,q_{\iv}\, \delta_\iv ,                               \qquad && \text{MT=sum/Vsum, QM=total}, \\
   q_p  &=\frac{1}{N} \sum_{\iv} \chi_R(\xv_\iv)~q_{\iv}\, \delta_\iv ,~~ N=\sum_{\iv} \chi_R(\xv_\iv)\, \delta_\iv,  \qquad && \text{MT=sum/Vsum, QM=average}, \\
   q_p  &= \int_{\partial R} q ~dS                           \qquad && \text{MT=integral, QM=total, R=boundary}, \\
   q_p  &= \frac{1}{A} \int_{\partial R} q(\xv) ~dS,~~ A=\int_{\partial R} 1 ~dS     \qquad && \text{MT=integral, QM=average, R=boundary}.\\
   q_p  &= \int_{R} q ~d\xv                          \qquad && \text{MT=integral, QM=total, R=volume}, \\
   q_p  &= \frac{1}{V} \int_{R} q(\xv) ~d\xv,~~ V=\int_{R} 1 ~d\xv     \qquad && \text{MT=integral, QM=average, R=volume}.
\end{alignat}
Here $q_\iv$ or $q(\xv)$ is the quantity to be evaluated (e.g. temperature or heat flux) and $q_p$ is the probe value. 
The sum or integral is over
the portion of the domain defined by the choice of region. $\chi_R(\xv)$ is the {\em characteristic} function
for the region $R$, being $1$ inside the region and $0$ outside,
\begin{align}
   \chi_R(\xv) &= \begin{cases} 1, & \text{$\xv\in R$}, \\
                                0, & \text{$\xv\notin R$} .
                      \end{cases} 
\end{align} 
The value of $\delta_\iv$ is equal to $1$, the cell volume, cell area or cell arclength, depending on the measure-type, 
and number of space dimensions,
\begin{align}
   \delta_\iv &= \begin{cases} 1, & \text{MT=sum}, \\
                               \delta V_\iv, & \text{MT=Vsum, R=volume},  \\
                               \delta S_\iv, & \text{MT=Vsum, R=boundary}.
                      \end{cases}
\end{align} 
The integral is computed with the Integrate class~\cite{OTHERSTUFF}.
% When the region is defined by intersection of a box with the full-domain or the boundary, the sums or integrals only include
% grid points that lie inside the box. In this case the integrals are only first order accurate.

% ----------------------------------------------------------------------------------
\subsection{Bounding box probes} \label{sec:boundingBoxProbes}

The bounding box probe can be used to output the solution on the faces (and ghost points) of a bounding
box (in the index space of a grid). These values can be used, for example, to provide 
time dependent boundary conditions for a refined calculation that only solves the equations within the
bounding box.

The bounding box is specified by 
\begin{enumerate}
  \item a component grid number, $g_{\rm bb}$, 
  \item an index box: [i1a,i1b], [i2a,i2b], [i3a,i3b] (in the index space of grid $g_{\rm bb}$), 
  \item number of layers $n_l$, (i.e. save this many lines of data on each face). For example, on the left
   face save the points $i_1=i1a-n_l+1,\ldots,i1a$.
\end{enumerate}

