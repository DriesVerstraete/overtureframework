\subsection{Flow in a 3D room} \label{sec:heatedRoom3d}

Figure~\ref{fig:room3dFlow} shows some results from a 
simulation of the flow in a small three-dimensional
room with a desk, hot computer, inlet and outlet. 
Cold air enters through an inlet vent on the ceiling into a room with an initially
uniform temperature.  We solve the INS equations with
Boussinesq approximation (i.e. with gravity and buoyancy effects). 
The grid was constructed with the ogen command file {\tt room3d.cmd}. The
command file for cgins was {\tt heatedRoom3d.cmd}.

%
{
\begin{figure}[hbt]
\newcommand{\figWidtha}{7cm}
\newcommand{\trimfig}[2]{\trimFig{#1}{#2}{0.1}{0.25}{.05}{.175}}
\begin{center}
 \begin{tikzpicture}[scale=1]
 \useasboundingbox (0,.75) rectangle (16,8);  % set the bounding box (so we have less surrounding white space)
%
 \draw ( 0,0.0) node[anchor=south west] {\trimfig{\insDocDir/fig/room3dT15p5}{\figWidtha}};
% \draw ( 6,0.0) node[anchor=south west] {\trimfig{\homeHenshaw/funding/gpic/doc/room3dT15p5a.ps}{\figWidth}};
 \draw (7.5,0.0) node[anchor=south west] {\trimfig{\insDocDir/fig/room3dT15p5b}{\figWidtha}};
%
% \draw[step=1cm,gray] (0,0) grid (15,8);
%  \draw (current bounding box.south west) rectangle (current bounding box.north east);
 \end{tikzpicture}
\end{center}
\caption{Results from a computation of flow in a 3D room. The figure shows contours of the temperature on selected
cutting planes at $t=15.5$ from a time dependent simulation. }
\label{fig:room3dFlow}
\end{figure}
}