\subsection{Centrifugal pump}\label{sec:centrifugalPump}

This example shows the simulation of a centrifugal pump.
Fluid flows into the domain through the central core and is accelerated by the
counter-clockwise rotating blades before exiting at the top outlet.
The grid was constructed with the ogen command file {\tt pump2dGrid.cmd}. The
command file for cgins was {\tt pump2d.cmd}. Thanks to Franck Monmont for help
with the problem.


{
\newcommand{\figWidthp}{8.cm}
\newcommand{\trimfig}[2]{\trimPlotb{#1}{#2}{.0}{.0}{.0}{.0}}
\begin{figure}[hbt]
\begin{center}
\begin{tikzpicture}[scale=1]
  \useasboundingbox (0,.5) rectangle (16.,16.5);  % set the bounding box (so we have less surrounding white space)
%
  \draw ( 0.0, 8) node[anchor=south west,xshift=-4pt,yshift=+0pt] {\trimfig{fig/pump8i0p5SLt0p5}{\figWidthp}};
  \draw ( 8.0, 8) node[anchor=south west,xshift=-4pt,yshift=+0pt] {\trimfig{fig/pump8i0p5pt0p5}{\figWidthp}};
  \draw ( 0.0, 0) node[anchor=south west,xshift=-4pt,yshift=+0pt] {\trimfig{fig/pump8i0p5SLt4p0}{\figWidthp}};
  \draw ( 8.0, 0) node[anchor=south west,xshift=-4pt,yshift=+0pt] {\trimfig{fig/pump8i0p5pt4p0}{\figWidthp}};
%%  \draw ( 5.5, 0) node[anchor=south west,xshift=-4pt,yshift=+0pt] {\trimfig{images/ellipseShock32Schlieren10}{\figWidth}};
%%  \draw (11.0, 0) node[anchor=south west,xshift=-4pt,yshift=+0pt] {\trimfig{images/ellipseShockMassZero16l1r4Dirk3Av1Schlieren10}{\figWidth}};
% 
%   \draw ( 0.0, 0) node[anchor=south west,xshift=-4pt,yshift=+0pt] {\trimfig{images/ellipseShock4l1r4Schlieren10}{\figWidth}};
%   \draw ( 5.5, 0) node[anchor=south west,xshift=-4pt,yshift=+0pt] {\trimfig{images/ellipseShock8l1r4Schlieren10}{\figWidth}};
%%  \draw (11.0,5) node[anchor=south west,xshift=-4pt,yshift=+0pt] {\trimfig{images/ellipseShockMassZero64Dirk3Av1Schlieren10}{\figWidth}};
%
 % \draw (current bounding box.south west) rectangle (current bounding box.north east);
% grid:
% \draw[step=1cm,gray] (0,0) grid (16,16);
\end{tikzpicture}
\end{center}
  \caption{Centrifugal pump. Streamlines and pressure at $t=0.5$ (top) and $t=4.0$ (bottom). }
  \label{fig:centrifugalPump}
\end{figure}
}


