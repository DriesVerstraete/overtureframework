%
%  Notes on the two-phase model 
%
\documentclass[11pt]{article} 

\input documentationPageSize.tex
% \addtolength{\oddsidemargin}{-.975in}
% \addtolength {\textwidth} {2.0in}

% \addtolength{\topmargin}{-1.0in}
% \addtolength {\textheight} {1.5in}

% \voffset=-1.25truein
% \hoffset=-1.truein
% \setlength{\textwidth}{6.75in}      % page width
% \setlength{\textheight}{9.5in}    % page height

% --------------------------------------------
\input{pstricks}\input{pst-node}
\input{colours}

% or use the epsfig package if you prefer to use the old commands
\usepackage{epsfig}
\usepackage{calc}
\input clipFig.tex

% The amssymb package provides various useful mathematical symbols
\usepackage{amsmath}
\usepackage{amssymb}

\newcommand{\Largebf}{\sffamily\bfseries\Large}
\newcommand{\largebf}{\sffamily\bfseries\large}
\newcommand{\largess}{\sffamily\large}
\newcommand{\Largess}{\sffamily\Large}
\newcommand{\bfss}{\sffamily\bfseries}
\newcommand{\smallss}{\sffamily\small}

\newcommand{\beq}{\begin{equation}}
\newcommand{\eeq}{\end{equation}}
\newcommand{\Omegav}{\boldsymbol{\Omega}}
\newcommand{\omegav}{\boldsymbol{\omega}}

\input wdhDefinitions.tex
\newcommand{\mbar}{\bar{m}}
\newcommand{\Rbar}{\bar{R}}
\newcommand{\Ru}{R_u}         % universal gas constant
% \newcommand{\grad}{\nabla}
\newcommand{\Div}{\grad\cdot}
\newcommand{\tauv}{\boldsymbol{\tau}}
\newcommand{\sigmav}{\boldsymbol{\sigma}}
\newcommand{\sumi}{\sum_{i=1}^n}


\newcommand{\Pc}{{\mathcal P}}
\newcommand{\Hc}{{\mathcal H}}

% \usepackage{verbatim}
% \usepackage{moreverb}
% \usepackage{graphics}    
% \usepackage{epsfig}    
% \usepackage{fancybox}    


\begin{document}
 
\title{Notes on the Incompressible Two-Phase Flow Model}

\author{
Bill Henshaw \\
% \  \\
%Centre for Applied Scientific Computing, \\
%Lawrence Livermore National Laboratory, \\
%henshaw@llnl.gov 
}
 
\maketitle

% \tableofcontents

\section{The coupled level-set volume-of-fluid method of Sussman et.al.}


Sussman et.al~\cite{Sussman2007} describe a coupled level-set volume-of-fluid method method
for modeling two phase flows with sharp interfaces.
They solve
\begin{align*}
\rho {D\over Dt} \Uv &= -\grad p + \grad\cdot( 2\mu \Dv) + \rho \gv - \sigma \kappa \grad H \\
\grad\cdot\Uv &= 0 \\
{D\over Dt} \phi &= 0 \qquad\text{(level set function)} \\
{D\over Dt} F = F_t + (\Uv\cdot\grad)F &= 0, ~~\text{or}~ F_t + \grad\cdot( F\Uv)=0, ~~ \qquad\text{(VOF function)}
\end{align*}
with 
\begin{align*}
   \Dv &= \half(\grad\Uv + \grad\Uv^T) \\
   \rho &= \rho_L H(\phi) + \rho_G(1-H(\phi)) \\
   \mu &= \mu_L H(\phi) + \mu_G(1-H(\phi)) \\
   \kappa &= \grad\cdot( \grad\phi/\vert \grad\phi \vert) \qquad\text{(curvature term)} \\
   H(\phi) &= sgn(\phi) \qquad\text{(Heavi-side function, $H=1$ in the liquid.)}  \\
   \sigma & = \text{coefficient of surface tension}
\end{align*} 
Here the sub- or super-scripts $L$ and $G$ refer to the liquid and gas phases respectively.

These equations implicitly satisfy the jump conditions 
\begin{align*}
  [ \Uv ] & = 0 \\
  [ \nv\cdot( -p \Iv + 2\mu \Dv )\cdot \nv ] & = \sigma\kappa 
\end{align*} 
at the interface $\phi=0$. 


\subsection{Notes on the implementation:}

\begin{enumerate}
  \item A liquid velocity $\Uv^L$ and total velocity $\Uv$ are stored so that values of $\Uv^L$ can be extrapolated
        into the gas region.
  \item Normals for the reconstruction of the piece-wise linear VOF interface are taken from $\phi$, $\nv=\grad\phi/\vert \grad\phi \vert$.
  \item The interface curvature $\kappa$ is computed from $F$.
  \item After each time step the level set function $\phi$ is re-initialized as the distance to the 
        VOF interface.
\end{enumerate} 


\subsection{Pressure Poisson Formulation}

\newcommand{\rhoi}{\frac{1}{\rho}}
  The pressure-Poisson formulation for this problem is 
\begin{align}
{D\over Dt} \Uv + \rhoi \grad p &=  \rhoi\grad\cdot(2\mu \Dv) + \gv - \rhoi\sigma \kappa \grad H \label{eq:momentumEquation} \\
\grad\cdot( \rhoi \grad p ) &= \grad\cdot( \Pv)  + \alpha_d \grad\cdot\Uv \label{eq:pressureEquation} \\
 \Pv &= - (\Uv\cdot\grad) \Uv + \rhoi \grad\cdot( 2\mu \Dv) + \gv - \rhoi\sigma \kappa \grad H \\
{D\over Dt} \phi &= 0 \label{eq:levelSetEquation} \\
{D\over Dt} F &= 0  \label{eq:VOFEquation}
\end{align}
where $\alpha_d$ is the coefficient of the divergence damping term. 

The pressure boundary condition at a no-slip wall is 
\begin{align*}
  {\partial\over\partial n}p &= \nv\cdot\Big( \grad\cdot( 2\mu \Dv) + \rho \gv - \sigma \kappa \grad H\Big) 
\end{align*}

{\bf Notes:}
\begin{enumerate}
  \item The equations\eqref{eq:levelSetEquation} and \eqref{eq:VOFEquation} 
        for level set function $\phi$ and VOF function $F$ will need special time-marching schemes. 
  \item The pressure equation~\eqref{eq:pressureEquation} changes at each time-step which increases the
     cost of solving. The multigrid solver could potentially be used here but may need improvements to
     handle discontinuous coefficients. 
  \item If we have boundary layers then we will want to use an implicit time stepping method for at least 
        the viscous terms in the momentum equations~\eqref{eq:momentumEquation}.
  \item If we use slip-wall BC's and if the viscosity $\mu(\xv)$ is small then we could instead use
        explicit time-stepping for~\eqref{eq:momentumEquation}.
\end{enumerate} 


\subsection{Preliminary model I}

Here is a preliminary model I in which we solve for a single scalar advected quantity $\psi$ instead of
$\phi$ and $F$, 
\begin{align}
{D\over Dt} \Uv + \rhoi \grad p &=  \rhoi\grad\cdot(2\mu \Dv) + \gv , \label{eq:momentumEquationI} \\
\grad\cdot( \rhoi \grad p ) &= \grad\cdot( \Pv)  + \alpha_d \grad\cdot\Uv , \label{eq:pressureEquationI} \\
 \Pv &= - (\Uv\cdot\grad) \Uv + \rhoi \grad\cdot( 2\mu \Dv) + \gv , \\
{D\over Dt}\psi &= 0 \qquad\text{(scalar advected function, $0\le \psi \le 1$, $\psi=1$ for the pure liquid)}, \\
   \rho &= \rho_L \psi + \rho_G(1-\psi)) , \\
   \mu &= \mu_L \psi + \mu_G(1- \psi) . \\
\end{align}



% -------------------------------------------------------------------------------------------------
\vfill\eject
\bibliography{/home/henshaw.0/papers/henshaw}
\bibliographystyle{siam}


\end{document}
