\noindent The push buttons are
\begin{description}
  \item[\qquad choose grids for implicit] : For use with the {\tt implicit} time stepping option. Choose 
         which grids to integrate implicitly and which to integrate explicitly. Normally one should choose
         those grids with fine grid spacing (such as in boundary layers) to be implicit while a back-ground
       grid could be explicit. 
\end{description}

\noindent The toggle buttons are 
 \begin{description}
  \item[\qquad use local time stepping] : for steady state solvers, use a different $\Delta t$ for each grid point.
  \item[\qquad adjust dt for moving bodies] :
  \item[\qquad use full implicit system] :
  \item[\qquad apply explicit BCs to implicit grids] : 
\end{description}

\noindent The text strings are 
 \begin{description}
  \item[\qquad final time] : Integrate to this time.
  \item[\qquad max iterations] : maximum number of iterations for steady state solvers.
  \item[\qquad cfl] :Set the {\tt cfl} parameter. 
         The maximum time step based on stability is scaled by this factor.
         By default {\tt cfl=.9}.
  \item[\qquad dtMax] : Restrict the time step to be no larger than this value.
  \item[\qquad implicit factor] :This value in $[0.,1.]$ is used with the implicit time-stepping. A value
         of $.5$ will correspond to a 2nd-order Crank-Nicolson approach for the viscous terms,
        a value of $1.$ will be backward-Euler and a value of $0.$ will be forward-Euler. See 
        the the reference manual for more details.
  \item[\qquad recompute dt every] : The time step, dt,  is recomputed every time the solution is plotted/saved.
              In addition you may specify the maximum number of steps that will be taken
              before dt is recomputed. Use this if the solution is not plotted very often. See also `slow start steps' for
             recomputing dt more often during a slow start.
%
  \item[\qquad slow start cfl] : The initial time step for the slow start option is determined by this cfl value, default$=0.25$.
  \item[\qquad slow start steps] : Ramp the time step $\Delta t$ from a small value (determined by slow start cfl)
         to its maximum value (as  determined by the {\tt cfl} parameter) {\em over this many steps} (over-rides `slow start' based
         on the time interval). 
  \item[\qquad slow start] : Ramp the time step $\Delta t$ from a small value (determined by slow start cfl)
         to its maximum value (as  determined by the {\tt cfl} parameter) {\em over this time interval}.
  \item[\qquad slow start recompute dt] : during the slow start interval, recompute the time-step every this many steps.
% 
  \item[\qquad fixup unused frequency] : specifies how often to fixup unused points (which may get very large values
                over time that can eventually lead to the program crashing).
  \item[\qquad cflMin, cflMax] :
  \item[\qquad preconditioner frequency] :
  \item[\qquad number of PC corrections] : specify how many correction iterations should be taken for
             predictor correction time-stepping (default=1). 
\end{description}

