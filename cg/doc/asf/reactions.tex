\documentclass{article} 

\usepackage{amsmath}
\usepackage{amssymb}

\usepackage{multicol}
\usepackage{verbatim}
\usepackage{moreverb}
\usepackage{graphics}    
\usepackage{epsfig}    
\usepackage{fancybox}    

\addtolength{\oddsidemargin}{-.975in}
\addtolength {\textwidth} {2.0in}

\addtolength{\topmargin}{-1.0in}
\addtolength {\textheight} {1.5in}

% \voffset=-1.25truein
% \hoffset=-1.truein
% \setlength{\textwidth}{6.75in}      % page width
% \setlength{\textheight}{9.5in}    % page height


\newcommand{\homeHenshaw}{/home/henshaw.0}

% \newcommand{\obFigures}{../docFigures}  % note: local version for OverBlown
\newcommand{\obFigures}{\homeHenshaw/res/OverBlown/docFigures}  % note: local version for OverBlown

\newcommand{\primer}{\homeHenshaw/res/primer}

\begin{document}
 
\title{Combustion Cheat Sheet}

\author{
Bill Henshaw \\
\  \\
Scientific Computing Group \\
Computing Information and Communications Division\\
Los Alamos, NM, USA, 87545 \\
henshaw@lanl.gov }
 
\maketitle

\input wdhDefinitions.tex
\newcommand{\mbar}{\bar{m}}
\newcommand{\Rbar}{\bar{R}}
\newcommand{\Ru}{R_u}         % universal gas constant
% \newcommand{\grad}{\nabla}
\newcommand{\Div}{\grad\cdot}
\newcommand{\tauv}{\boldsymbol{\tau}}
\newcommand{\sigmav}{\boldsymbol{\sigma}}
\newcommand{\sumi}{\sum_{i=1}^n}

\tableofcontents

\section{General}

\begin{align}
   {\partial \rho  \over \partial t} + \grad\cdot(\rho \vv)&=0 \label{eq:continuity}\\
   {\partial \vv  \over \partial t}  + (\vv\cdot\grad)\vv +{1\over\rho} \grad p 
       &= {1\over\rho}\grad\cdot\tauv +\sum_i Y_i \fv_i\\
       & = {1\over\rho} \grad\cdot( \mu [\grad \vv +(\grad\vv)^T] +(\kappa-{2\over3}\mu) (\grad\cdot\vv)I)
          +\sum_i Y_i \fv_i\\
   {\partial e  \over \partial t}  + (\vv\cdot\grad)e + {p\over\rho}\grad\cdot\vv
           &= -{1\over\rho} \grad\cdot\qv  + {1\over\rho}\Phi    \\
  {\partial Y_i \over \partial t} + (\vv\cdot\grad) Y_i &= {\sigma_i \over \rho}
               -{1\over\rho}\grad\cdot\jv_i \qquad i=1,2,\ldots,n
\end{align}
\begin{align}
  p &= \rho R_u T \sumi {Y_i\over\mbar_i} \qquad = \rho R T ,\qquad R_i = {R_u\over \mbar_i}\\
  R & = \sumi R_i Y_i = R_u \sumi {Y_i\over\mbar_i} = {R_u \over \mbar}, \qquad \mbar=\sumi \mbar_i X_i   \\
  e &= \sumi( h_i Y_i) - {p\over\rho} \quad = h - {p\over\rho} \quad = \sumi( h_i - R_i T )Y_i \\
  \tau &= \mu[ \grad\vv + (\grad\vv)^T ] + (\kappa-{2\over3}\mu)(\grad\cdot\vv)\Iv , \\
     \Phi&=\tau : \grad\vv \qquad \mbox{viscous heating term} \\
  \mu &=\mu(T,Y_i), \qquad \lambda=\lambda(T,Y_i), \qquad D_i = D_i(p,T,Y_i) \\
  h_i(T) &= \Delta h_i^0 + \int_{T^0}^T c_{p,i}(T) dT \qquad i=1,2,...,n \\
  \qv &= - \lambda \grad T + \sumi h_i ~\jv_i  \quad  \\
  \jv_i & = \rho Y_i \Vv_i \qquad\mbox{species flux} \\
 \Vv_i &\approx -D_i \grad ln(X_i) \qquad\mbox{ (diffusion velocity, Fick's law)} \\
  c_p &= \sumi c_{p,i} Y_i , \qquad   c_v = c_p - R = \sumi (c_{p,i}-R_i) Y_i,\qquad   
        \gamma = \gamma(T,Y_i) = {c_p \over c_v} \\
  c_v &= R/(\gamma-1)
\end{align}
\begin{itemize}
\item {\bf Note 1}: the enthalpy appearing here, $h_i$, is the {\bf absolute enthalpy}, which includes 
  the energy of formation $ \Delta h_i^0$
  at the reference temperature, usually taken as $T_0=298.15$.
\item \noindent {\bf Note 2}: The thermodynamic variables $c_{p,i}(T)$, $h_i(T)$ are assumed to be mass based instead
  of mole based (to get the mass based version from the mole based version, divide by $\mbar_i$ ).
\item {\bf Note 3:} The quantities $c_p$, $c_v$ and $\gamma$ are called the frozen specific heats and
     gas constant (the non-frozen $c_p$ is $\partial h /\partial T$).
\end{itemize}

Other forms for the ``energy'' equation: Here is the temperature equation
(derived in more detail in section (\ref{sec:temperature})
\begin{align}
  {D T\over Dt} + (\gamma-1) T \grad\cdot\vv & = 
       {1\over \rho c_v}  \left\{ \sumi  (R_i T-h_i) \sigma_i + \grad\cdot(\lambda\grad T)
      - T \sumi  R_i \grad\cdot\jv_i - \grad T \cdot\sumi c_{p,i} ~ \jv_i
              + \Phi \right\} 
           \label{eq:temperature} 
\end{align}
Taking $RT$ times the continuity equation (\ref{eq:continuity}) plus $\rho R$ 
times the temperature equation (\ref{eq:temperature}) gives the pressure equation
(derived in more detail in section (\ref{sec:pressure}))
\begin{align}
    {D p\over Dt} + \gamma p \grad\cdot\vv &= 
        \sumi  (\gamma R_i T-(\gamma-1)h_i) \sigma_i + (\gamma-1) \grad\cdot(\lambda\grad T) \\
    &  - \gamma T \grad\cdot \sumi R_i\jv_i - (\gamma-1) \grad T \cdot \sumi c_{p,i} ~\jv_i
              + (\gamma-1) \Phi \label{eq:pressure}
\end{align}

Another equation that is often used for low mach number is a combination of the Temperature
and pressure equations:
\begin{align}
  {D T\over Dt} - {1\over \rho c_p} {D p\over Dt} & = 
       {1\over \rho c_p}  \left\{ -\sumi (h_i\sigma_i) + \grad\cdot(\lambda\grad T)
      - \grad T \cdot\sumi c_{p,i} ~ \jv_i
              + \Phi \right\} \label{eq:TP}
\end{align}

\subsection{Equations in conservation form}

The equations in conservation form (see {\sl Numerical Simulation of Reactive Flow} by Oran and Boris)
are
\begin{align*}
   {\partial \rho  \over \partial t} + \grad\cdot(\rho \vv)&=0 \\
   {\partial (\rho\vv)  \over \partial t}  + \grad\cdot(\rho\vv\vv) + \grad p &= \grad\cdot\tauv
              +  \sum_i \rho_i \av_i \\
   {\partial E  \over \partial t}  + \grad\cdot( (E+p)\vv ) &= \grad\cdot( \vv\cdot\tauv )
            - \grad\cdot(\qv+\qv_R) + \vv\cdot\sum m_i\av_i + \sum_i \Vv_i\cdot m_i \av_i    \\
  {\partial (\rho Y_i) \over \partial t} + \grad\cdot(\rho Y_i\vv) &= \sigma_i 
               -\grad\cdot\jv_i \qquad i=1,2,\ldots,n
\end{align*}
where the energy density E is
\begin{align*}
 E &= \half \rho \vv\cdot\vv + \rho e
\end{align*}
and the $\av_i$ are body forces acting on species $i$.

\subsection{Diffusion velocities}

In general the diffusion velocities, $\Vv_i$, are determined from the matrix equation
\[
    \sum_j \left( {X_i X_j \over D_{ij} }\right) (\Vv_j-\Vv_i) = \Gv_i
\]
where
\[
   \Gv_i = \grad X_i - (Y_i-X_i){\grad p \over p} - {\rho\over p} \sum_j Y_i Y_j (\fv_i-\fv_j)
    - \sum_j\left( {X_i X_j \over \rho D_{ij}}\right)\left( {D_{T,j}\over Y_j} - {D_{T,i}\over Y_i}\right) 
      {\grad T\over T}
\]

% ok \end{document} 


\subsection{Entropy}
\begin{align*}
\tilde{S} &= \sumi Y_i \left(\phi_{0,i} (T) -R_i \ln(X_i)\right) - R \ln(p)  \\
\phi_{0,i} (T) &= \int_{T_0}^T {c_{p,i}(\tau) \over \tau} d\tau 
\end{align*}
When the flow is frozen ($Y_i$=constant) and inviscid the entropy will be constant. This can be
seen from
\begin{align*}
{D S\over Dt} &= \left\{ \sumi Y_i {c_{p,i}\over T}{D T\over Dt}
         - R {1\over p} {D p\over Dt} \right\} \\
 &= c_p [ -(\gamma-1)\grad\cdot\uv] - R [-\gamma\grad\cdot\uv] \\
 & = 0
\end{align*}
This is also (supposedly true) in equilibrium flow.


\subsection{enthalpy}
An equation for the enthalpy is obtained by substituting $e=h-p/\rho$ into the energy equation
\[
   {D h  \over D t}  - {1\over \rho}{D p\over D t} + {p\over\rho^2}{D \rho\over D t}+ {p\over\rho}\grad\cdot\vv
           = -{1\over\rho} \grad\cdot\qv  + {1\over\rho}\Phi   
\]
Use of the continuity equation gives
\[
   {D h  \over D t}  - {1\over \rho}{D p\over D t} 
           = -{1\over\rho} \grad\cdot\qv  + {1\over\rho}\Phi  
\]
Replacing $\uv\cdot\grad p/\rho$ from the momentum equation gives
\[
   {D h  \over D t}  - {1\over \rho}{\partial p\over \partial t} 
         - \vv\cdot\left[ - \vv_t -  (\vv\cdot\grad)\vv + \Div \tauv \right]
           = -{1\over\rho} \grad\cdot\qv  + {1\over\rho}\Phi  
\]
or
\newcommand{\halfo}{{1\over2}}
\begin{equation}
   {\partial h  \over \partial t}  
  - {1\over \rho}{\partial p\over \partial t} 
  + \halfo {\partial (\vv\cdot\vv)\over \partial t}
   + \vv\cdot\left[ \grad h  +  (\vv\cdot\grad)\vv \right]
           = -{1\over\rho} \grad\cdot\qv  + {1\over\rho}\Phi  + \vv\cdot(\Div\tauv)
\end{equation}
Thus in steady inviscid 1D flow we have
\[
     h_x + \halfo (u^2)_x = 0 \qquad \implies\qquad  h+\halfo u^2 = \text{constant}
\]

Another equation for the enthalpy is obtained by substituting $E=\rho h+\half\rho v^2-p$ into the conservative
form of the energy equation giving
\[
    {\partial( \rho h+\half\rho v^2 ) \over \partial t} - {\partial p \over \partial t}
 + \grad\cdot( (\rho h+\half\rho v^2)\vv ) =
              \grad\cdot( \vv\cdot\tauv )
            - \grad\cdot(\qv+\qv_R) + \vv\cdot\sum m_i\av_i + \sum_i \Vv_i\cdot m_i \av_i 
\]
Thus in steady inviscid flow 
\begin{align*}
   \grad\cdot(\rho\vv) &=0 \\
   \grad\cdot( (h+\half v^2)\rho \vv ) &=0
\end{align*}
Whence
\[
   \grad(h+\half v^2)\cdot(\rho\vv) =0
\]

\subsection{Reactions:}
\[
    \sum_{i=1}^n \nu_{ij}' A_i \Leftrightarrow \sum_{i=1}^n \nu_{ij}''A_i  \qquad j=1,2,\ldots,m
\]


\begin{alignat}{3}
   Y_i &= \mbox{mass fraction} &\qquad 
      X_i &= \mbox{mole fraction} \\
        [A] &= \mbox{molar concentration} = {\mbox{moles}\over\mbox{volume}}  &\quad R_u&=\mbox{Universal Gas constant} \\
  \mbar_j &= \mbox{molecular weight} = {\mbox{grams}\over\mbox{mole}} &\qquad 
   \rho &= {\mbox{mass}\over\mbox{volume}}
\end{alignat}
\begin{alignat}{2}
   Le &= {\lambda\over \rho D c_p} & \qquad\mbox{Lewis number} 
\end{alignat}

% no \end{document} 

\[
    Y_i = { [A_i] \mbar_i \over \rho } = {\mbar_i X_i \over \sum_j \mbar_j X_j} = {\mbar_i X_i \over \mbar}
    = { \rho_i \over \rho} = { m_i \over m}
\]
\[
    X_j = { (Y_j /\mbar_j) \over \sumi (Y_i / \mbar_i) }
\]
\[
    [A_i] = {\rho Y_i \over \mbar_i} = {p X_i \over R_u T}
\]

% \end{document} % no

\section{Reaction equations}

\begin{align*}
  {\partial Y_i \over \partial t} + (\vv\cdot\grad) Y_i & = {1\over\rho}\sigma_i
              - {1\over\rho}\grad\cdot \jv_i \qquad i=1,2,\ldots,n \\
   \sigma_i \equiv \dot{w}_i &= \sigma_i(\rho,T,\Yv)   \\
      \jv_i & = \rho Y_i \Vv_i \qquad\mbox{species flux} \\
      \Vv_i &= - {1\over X_i} D_{im} \dv_i - {D_k^T\over \rho Y_i T} \grad T \qquad\mbox{diffusion velocity} \\
      \dv_i &= \grad X_i + (X_i-Y_i) {1\over p} \grad p    \\
      D_{im} = \mbox{mixture diffusion coefficient} & \quad D_k^T = \mbox{thermal diffusion coefficient}
\end{align*}
For the species flux we make the approximations
\begin{align*}
  \Vv_i & \approx - {1\over X_i} D_{im} \dv_i  \qquad \mbox{(the rest is expensive to compute)} \\
  \dv_i & \approx \grad X_i        \qquad \mbox{ (valid for low Mach numbers)}           \\
  \mbox{which gives}\quad \jv_i & \approx - \rho {Y_i \over X_i} D_{im}\grad X_i
\end{align*}
and thus the diffusion term on the right hand side of the species equation is
\begin{align*}
-{1\over\rho}\grad\cdot \jv_i &\approx  {1\over\rho}\grad\cdot(\rho {Y_i \over X_i} D_{im}\grad X_i) \\
          &= {1\over\rho}\grad\cdot( \rho D_{im}\grad Y_i + \rho D_{im} Y_i {\grad \mbar \over \mbar_i} )
\end{align*}

The species source terms
\begin{align}
    \sigma_i & = \mbar_i \sum_{j=1}^m {d \over dt} [A_i]_j \\
             & = \mbar_i \sum_{j=1}^m \Delta \nu_{ij} G_j
\end{align}
\begin{align}
    G_i &=  K_{fi}(T)\prod_{k=1}^{m} [A_k]^{\nu_{ki}'} -  K_{bi}(T)\prod_{k=1}^{m} [A_k]^{\nu_{ki}''} \\
        &=  K_{fi}(T)\prod_{k=1}^{m} \left({\rho Y_k \over \mbar_k} \right)^{\nu_{ki}'}
         -  K_{bi}(T)\prod_{k=1}^{m}  \left({\rho Y_k \over \mbar_k} \right)^{\nu_{ki}''}
\end{align}
Reaction rate constants, 
\begin{align}
  K_{fj}(T) &=  B_{fj} T^{\alpha_{fj}} \exp({-E_{fj} \over \Ru T}) \\
  K_{bj}(T) &=  B_{bj} T^{\alpha_{bj}} \exp({-E_{bj} \over \Ru T}) \\
  {K_f(T)\over K_b(T)} &= K_p(T) (\Ru T)^{-\Delta \nu}
\end{align}
This next Jacobian depends on which dependent variables we use:
\begin{align}
    {\partial G_i \over \partial Y_j} &= 
            K_{fi}(T) {\nu_{j,i}' \over Y_j}  \prod_{k=1}^{m} \left({\rho Y_k \over \mbar_k} \right)^{\nu_{ki}'}
         -  K_{bi}(T){\nu_{j,i}''\over Y_j}  \prod_{k=1}^{m}  \left({\rho Y_k \over \mbar_k} \right)^{\nu_{ki}''}
\end{align}
% \begin{align}
%     {\partial G_i \over \partial Y_j} &= 
%             K_{fk}(T) {\nu_{j,i}' \over Y_j}  \prod_{k=1}^{m} \left({\rho Y_k \over \mbar_k} \right)^{\nu_{ki}'}
%          -  K_{bk}(T){\nu_{j,i}''\over Y_j}  \prod_{k=1}^{m}  \left({\rho Y_k \over \mbar_k} \right)^{\nu_{ki}''} \\
%    &  + {\partial K_{fk}(T) \over\partial T}{\partial T \over\partial Y_j} 
%                      \prod_{k=1}^{m} \left({\rho Y_k \over \mbar_k} \right)^{\nu_{ki}'}
%       - {\partial K_{bk}(T) \over\partial T}{\partial T \over\partial Y_j} 
%             \prod_{k=1}^{m}  \left({\rho Y_k \over \mbar_k} \right)^{\nu_{ki}''}
% \end{align}

\section{Scaling}

\newcommand{\rhot}{\tilde{\rho}}
\newcommand{\jvt}{\tilde{\jv}}
\newcommand{\vvt}{\tilde{\vv}}
\newcommand{\sigmat}{\tilde{\sigma}}
\newcommand{\pt}{\tilde{p}}
\newcommand{\ttt}{\tilde{t}}
\newcommand{\Tt}{\tilde{T}}
\newcommand{\gradt}{\tilde{\grad}}
\newcommand{\Deltat}{\tilde{\Delta}}
\newcommand{\lambdat}{\tilde{\lambda}}
\newcommand{\ct}{\tilde{c}}
\newcommand{\hht}{\tilde{h}}
\newcommand{\Rt}{\tilde{R}}
\newcommand{\Dt}{\tilde{D}}
\newcommand{\mut}{\tilde{\mu}}
\newcommand{\Phit}{\tilde{\Phi}}

We nondimensionalize the equations using the scales $L_0$, $u_0$, $\rho_0$, and $T_0$,
\begin{alignat*}{4}
  t_0 &:={L_0 \over U_0} & \qquad \vvt &:={\vv \over U_0}, &
        \qquad    a & = b         &  \qquad \tilde{t} &= {t\over t_0} \\
  \tilde{x} &= {x\over L_0}  &  \qquad  \tilde{T} &= { T \over T_0 } &
       \qquad \tilde{\rho} &= { \rho \over \rho_0 } &    \qquad \tilde{\vv} &= {\vv\over u_0} \\
  \mut &:={\mu \over \rho_o U_0 L_0 } & \qquad p_0 &:=\rho_0 u_0^2, &
        \qquad R_0 & := {P_0 p_0\over \rho_0 T_0} &  \qquad \tilde{t} &= {t\over t_0} \\
  \tilde{p} &= {p \over p_0} - P_0 &   \qquad \Rt_i&= { R_i \over R_0}   &
        \qquad\tilde{\sigma} &= {\sigma \over {\rho_0 U_0/ L_0}} &  
        \qquad \tilde{h}_i &= { h_i \over R_0 T_0} \\
  \tilde{\lambda} &= {\lambda\over \rho_0 U_0 L_0 R_0} &  \qquad\Dt_i &= { D_i \over L_0 U_0} &
      \qquad \ct_{p,i} &= { c_{p,i} \over R_0} & \qquad \ct_v &= { c_v \over R_0} = {\Rt \over \gamma-1}
\end{alignat*}
We have scaled the pressure using a scale for the static pressure and a scale for the dynamic pressure:
\[
   p = [ P_0 + \pt ] p_0 , \qquad P_{static} = P_0 p_0
\]
In general the spatially constant static pressure may vary with time,  $P_0=P_0(t)$. Note that $P_0$ is
proportional to one over the square of the Mach number
\[
 P_0 = { \rho_0 R_0 T_0 \over \rho U_0^2 } = {1\over  \gamma } { \gamma R_0 T_0 \over U_0^2 } = {1\over \gamma M_0^2}
  \qquad \mbox{ where } M_0 = {U_0 \over \sqrt{\gamma R_0 T_0}} = {U_0 \over a_0}.
\]


Here are the scaled equations
\begin{align*}
   {\partial \rhot  \over \partial \ttt} + \gradt\cdot(\rhot \vvt)&=0 \\
   {\partial \vvt  \over \partial \ttt}  + (\vvt\cdot\gradt)\vvt +{1\over\rhot} \gradt \pt  
            & = {1\over\rhot} \gradt\cdot( \mut [\gradt \vvt + (\gradt\vvt)^T] -{2\over3}\mut (\gradt\cdot\vvt)I) \\
%   {\partial e  \over \partial t}  + (\vv\cdot\grad)e + {p\over\rho}\grad\cdot\vv
%           &= -{1\over\rho} \grad\cdot\qv  + {1\over\rho}\Phi    \\
  {\partial Y_i \over \partial \ttt} + (\vvt\cdot\gradt) Y_i &= {\sigmat_i \over \rhot}
               +{1\over\rhot}\gradt\cdot(\rhot \Dt_i\grad Y_i) \qquad i=1,2,\ldots,n
\end{align*}
\begin{align*}
  {D \Tt\over D\ttt} + (\gamma-1)\Tt\gradt\cdot\vvt & = 
       {1\over \rhot \ct_v}  \left\{ \sumi  ( \Rt_i \Tt - \hht_i) \sigmat_i 
+ \gradt\cdot(\lambdat\gradt \Tt)
- \Tt \sumi  \Rt_i \gradt\cdot\jvt_i - \gradt \Tt \cdot\sumi \ct_{p,i} ~\jvt_i
              + {1\over P_0} \Phit \right\} 
\end{align*}
or the alternative form
\begin{align*}
  {D \Tt\over D\ttt} - {1\over P_0} {1\over \rhot \ct_p} {D \pt\over D\ttt} & = 
       {1\over \rhot \ct_p}  \left\{ -\sumi (\hht_i\sigmat_i) + \gradt\cdot(\lambdat\gradt \Tt)
       - \gradt \Tt \cdot\sumi \ct_{p,i}  ~\jvt_i
               + {1\over P_0} \Phit \right\} 
\end{align*}
\begin{align*}
     {D \pt\over D\ttt}+ {\partial P_0 \over \partial \ttt} + \gamma (P_0 + \pt) \gradt\cdot\vvt &= 
    P_0 \left\{ \sumi  ({\gamma \Rt_i} \Tt - (\gamma-1) \hht_i) \sigmat_i 
        + (\gamma-1) \gradt\cdot(\lambdat\gradt \Tt) \right. \\
    &  \left. - \gamma \Tt \gradt\cdot( \sumi \Rt_i~\jvt_i) - (\gamma-1)\gradt \Tt\cdot\sumi \ct_{p,i}~\jvt_i
              + {1\over P_0} (\gamma-1) \Phit \right\}
\end{align*}
where
\begin{align*}
  R_e &= {1\over \mut} = {\rho_0 U_0 L_0 \over \mu}  \qquad \text{Reynolds number} \\
  P_r & =     blah                   \qquad \text{Prandtl number} \\
  Le & = {\lambda \over \rho D c_p } \qquad \text{Lewis number}
\end{align*}


\section{Solution of the Reaction equations}

\newcommand{\dt}{{\Delta t}}
\newcommand{\dY}{{\delta Y}}
\newcommand{\dT}{{\delta T}}
% \newcommand{\half}{{1\over 2}}

The temperature and reactions equations
\begin{align*}
  {D T\over Dt} & = 
       {1\over \rho c_p}  \left\{ -\sumi h_i\sigma_i + \grad\cdot(\lambda\grad T)
      - \grad T \cdot\sumi c_{p,i} ~ \jv_i
              + \Phi \right\} \\
  {\partial Y_i \over \partial t} + (\vv\cdot\grad) Y_i &= {1\over\rho}{\sigma_i(\rho,T,Y_k) }
               -{1\over\rho}\grad\cdot\jv_i \qquad i=1,2,\ldots,n
\end{align*}
are solved with a Backward Euler scheme (here we assume that we are given and $p^{n+1}$ and we must solve
for $\rho^{n+1}$, $T^{n+1}$ and $Y_i^{n+1}$)
\begin{align*}
  p^{n+1} & = \rho^{n+1} R^{n+1} T^{n+1} \\
  {T^{n+1}-T^n \over \dt} &- {1\over \rho^{n+1} c_p(\Yv^{n+1},T^{n+1})}  -\sumi h_i\sigma_i (\Yv^{n+1},T) =
             \mbox{ stuff }\\
  {\Yv^{n+1}-\Yv^n \over \dt} &- {1\over \rho^{n+1}} \sigmav(\Yv^{n+1},T) = \left\{ -\vv\cdot\grad \Yv 
                    -{1\over\rho}\grad\cdot\jv_i \right\}^{n+\half}
\end{align*}
or
\begin{align*}
  p^{n+1} & = \rho^{n+1} R^{n+1} T^{n+1} \\
  T_{n+1}  &- {\dt\over \rho^{n+1} c_p(\Yv^{n+1},T^{n+1})}  -\sumi h_i\sigmav_i (\Yv^{n+1},T) =
      T^n + \dt\left\{ \mbox{ stuff } \right\} \\ 
  \Yv^{n+1} & - {\dt \over \rho^{n+1}} \sigmav(\Yv^{n+1},T) = \Yv^n +\dt\left\{ -\vv\cdot\grad \Yv
        - {1\over\rho}\grad\cdot\jv_i \right\}^{n+\half}
\end{align*}
We solve this nonlinear equation by Newton's method. 
We write the system of equations for $T^{n+1}$ and $\Yv^{n+1}$ as
\begin{align*}
  \Zv^{n+1} - \dt \Sv(\Zv^{n+1}) & = \Gv \\
  \Zv^{n+1} &= \Zv^n + \delta\Zv
\end{align*}
where
\[
  \Zv = \begin{bmatrix} T^{n+1} \\ \Yv^{n+1} \end{bmatrix} \qquad
  \Sv = \begin{bmatrix} -{1\over \rho c_p}  \sum h_i\sigma_i \\
                         {1\over \rho} \sigmav \end{bmatrix}
\]

Then a Newton step is
\begin{align*}
  \left[I - \dt {\partial \Sv \over\partial \Zv}(\Zv^n)\right] \delta\Zv & = -\Zv^n + \dt \Sv(\Zv^n) + \Gv
\end{align*}

% Denoting the Newton iterations by $(T^k,Y^k)$, 
% and $(T^{k+1},Y^{k+1})=(T^k,Y^k)+(\dT^k,\dY^k)$, we have
% \begin{align*}
%   T^{k+1} - {\dt\over \rho^{n+1} c_p(Y^{n+1},T^{n+1})}  -\sumi h_i\sigma_i (Y^k+\dY^k,T^k+\dT^k) &= h \\
%   Y^{k+1} - {\dt \over \rho^{n+1}} \sigma(Y^k + \dY^k,T+\dT^k) & =g 
% \end{align*}
% or
% \begin{align*}
%   T^{k+1} - {\dt\over \rho^{n+1} c_v}  \sumi  (R_i T-h_i) \sigma_i (Y^k+\dY^k,T^k+\dT^k) &= h \\
%   Y^k+ \dY^k - {\dt \over \rho^{n+1}} 
%        \left\{ \sigma(Y^k,T)  + {\partial\sigma \over \partial Y}(Y^k,T) \dY^k 
%                               + {\partial\sigma \over \partial T}(Y^k,T) \dT^k    \right\} &= g 
% \end{align*}
% Thus the Newton iteration is 
% \begin{align*}
%   \left(I - {\dt \over \rho^{n+1}}{\partial\sigma \over \partial Y}(Y^k,T) \right)\dY^k
%      &= -Y^k + {\dt \over \rho^{n+1}}  \sigma(Y^k,T) + g  \\
%     Y^{k+1} &=Y^k+\dY^k \\
%         g&=  Y^n +\dt\left\{ -\vv\cdot\grad Y - {1\over\rho}\grad\cdot\jv_i \right\}^{n+\half}
% \end{align*}

\section{Derivation of Some Equations}

\subsection{Derivation of the Temperature Equation}\label{sec:temperature}

Differentiating
\begin{align*}
  e &= \sumi( h_i - R_i T)Y_i 
\end{align*}
gives
\begin{align*}
  {D e\over Dt} &= \sumi\left\{ (c_{p,i}-R_i){D T\over Dt} Y_i + (h_i-R_i T){D Y_i\over Dt} \right\} \\
                &= c_v {D T\over Dt} + \sumi  (h_i-R_i T){D Y_i\over Dt}  
%                &= c_v {D T\over Dt} + \sumi  (h_i-R_i T)\left( {\sigma_i \over \rho}
%               +{1\over\rho}\grad\cdot(\rho D_i\grad Y_i) \right)
\end{align*}
Substituting this last expression into the energy equation
\begin{align*}
   {\partial e  \over \partial t}  + (\vv\cdot\grad)e + {p\over\rho}\grad\cdot\vv
           &= -{1\over\rho} \grad\cdot\qv  + {1\over\rho}\Phi 
\end{align*}
results in
\begin{align*}
  c_v {D T\over Dt} + {p\over\rho}\grad\cdot\vv & = \sumi  (R_i T-h_i){D Y_i\over Dt}
                 -{1\over\rho} \grad\cdot\qv  + {1\over\rho}\Phi \label{eq:Ta}
\end{align*}
and thus
\begin{align*}
  {D T\over Dt} + {p\over\rho c_v}\grad\cdot\vv & = 
       {1\over \rho c_v}  \left\{ \sumi  (R_i T-h_i)\left( \sigma_i -\grad\cdot \jv_i \right) 
           - \grad\cdot\qv  + \Phi \right\} \\
   & = {1\over \rho c_v}  \left\{ \sumi  (R_i T-h_i) \sigma_i + \grad\cdot(\lambda\grad T)
      -\sumi  (R_i T-h_i) \grad\cdot\jv_i - \grad\cdot( \sumi h_i \jv_i)
              + \Phi \right\} \\
   & = {1\over \rho c_v}  \left\{ \sumi  (R_i T-h_i) \sigma_i + \grad\cdot(\lambda\grad T)
      - T \sumi  R_i \grad\cdot\jv_i - \grad T \cdot\sumi c_{p,i} ~\jv_i
              + \Phi \right\} 
\end{align*}
whence
\begin{align*}
  {D T\over Dt} + (\gamma-1) T \grad\cdot\vv & = 
       {1\over \rho c_v}  \left\{ \sumi  (R_i T-h_i) \sigma_i + \grad\cdot(\lambda\grad T)
      - T \sumi  R_i \grad\cdot\jv_i - \grad T \cdot\sumi c_{p,i} ~ \jv_i
              + \Phi \right\} 
\end{align*}


\noindent {\bf Note:} An alternative form of the temperature equation retains the pressure as a variable in
\begin{align*}
  e &= \sumi( h_i Y_i ) - {p\over \rho}
\end{align*}
Then 
\begin{align*}
  {D e\over Dt} &= c_p {D T\over Dt} + \sumi(h_i{D Y_i\over Dt})
            + {p\over \rho^2} {D \rho\over Dt} - {1\over \rho} {D p\over Dt}  \\
                &= c_p {D T\over Dt} + \sumi(h_i{D Y_i\over Dt}) - {p\over \rho} \grad\cdot\vv - {1\over \rho} {D p\over Dt}
\end{align*}
Substituting this last expression into the energy equation gives
\begin{align}
  {D T\over Dt} - {1\over \rho c_p} {D p\over Dt} & = 
       {1\over \rho c_p}  \left\{ -\sumi (h_i\sigma_i) + \grad\cdot(\lambda\grad T)
      -  \grad T \cdot \sumi c_{p,i} ~\jv_i
              + \Phi \right\} 
\end{align}
This last equation is useful if $p=\mbox{constant}$.

\subsection{Derivation of the pressure equation}\label{sec:pressure}

From $p=\rho R T$ it follows that
\begin{align}
    {D p\over Dt} &=  RT {D \rho\over Dt} + \rho R  {D T\over Dt} + \rho T  {D R\over Dt}
\end{align}
and
\begin{align}
   {D R\over Dt} &=  \sumi R_i {D Y_i \over Dt}
%      &= \sumi R_i \left( {\sigma_i \over \rho}
%               +{1\over\rho}\grad\cdot(\rho D_i\grad Y_i)  \right)
\end{align}
whence
\begin{align}
    {D p\over Dt} + \gamma p \grad\cdot\vv &= 
       {R\over c_v}  \left\{ \sumi  (R_i T-h_i) (\rho{D Y_i \over Dt}) + \grad\cdot(\lambda\grad T)
              + \Phi \right\}   + \rho T \sumi R_i ({D Y_i \over Dt}) \label{eq:pa}
\end{align}
or substituting for $DY_i/Dt$ 
\begin{align}
    {D p\over Dt} + \gamma p \grad\cdot\vv &= 
       {R\over c_v}  \left\{ \sumi  (R_i T-h_i) \sigma_i + \grad\cdot(\lambda\grad T)
      - T \grad\cdot( \sumi  R_i ~\jv_i) - \grad T \cdot ( \sumi c_{p,i} ~\jv_i )
              + \Phi \right\} \\
      & + \rho T \sumi R_i \left( {\sigma_i \over \rho} - {1\over\rho}\grad\cdot\jv_i  \right)
\end{align}
giving the pressure equation 
\begin{align}
    {D p\over Dt} + \gamma p \grad\cdot\vv &= 
        \sumi  ({\gamma R_i} T- (\gamma-1)h_i) \sigma_i + (\gamma-1) \grad\cdot(\lambda\grad T) \\
    &  - \gamma T \grad\cdot( \sumi R_i~\jv_i) - (\gamma-1)\grad T \cdot \sumi c_{p,i}~\jv_i
              + (\gamma-1) \Phi 
\end{align}
where we have used
\begin{align}
   {R\over c_v} &= \gamma -1
\end{align}


\section{Frozen Flow}
 The equations for frozen flow are determined by keeping the mass fractions $Y_i$ constant in space and
time. Then $DY_i/Dt$ in equations (\ref{eq:Ta}) and (\ref{eq:pa}) is zero and thus
\begin{align*}
   {\partial \rho  \over \partial t} + \grad\cdot(\rho \vv)&=0 \\
   {\partial \vv  \over \partial t}  + (\vv\cdot\grad)\vv +{1\over\rho} \grad p &= {1\over\rho}\grad\cdot\tauv \\
  {D T\over Dt} + (\gamma-1) T \grad\cdot\vv & = 
       {1\over \rho c_v}  \left\{ \grad\cdot(\lambda\grad T) + \Phi \right\} \\
    {D p\over Dt} + \gamma p \grad\cdot\vv &=  \grad\cdot(\lambda\grad T) + (\gamma-1) \Phi  \\
   {\partial Y_i  \over \partial t} & = 0 
\end{align*}

\section{Equilibrium Flow}

In equilibrium flow the reactions are assumed to proceed infinitely fast so that $\sigma_i=0$. 
In this case we retain the term $DY_i/Dt$ in equations (\ref{eq:Ta}) and (\ref{eq:pa})
Note that we could not set $\sigma_i=0$ in the temperature and pressure equations as this
would not conserve energy.

% To conserve $h+u^2/2$ in steady flow we must replace $\sigma_i$ in the source terms for $T$ and $p$ with 
% $\rho\vv\cdot\grad Y_i$ 

\begin{align*}
   {\partial \rho  \over \partial t} + \grad\cdot(\rho \vv)&=0 \\
   {\partial \vv  \over \partial t}  + (\vv\cdot\grad)\vv +{1\over\rho} \grad p &= {1\over\rho}\grad\cdot\tauv \\
  {D T\over Dt} + (\gamma-1) T \grad\cdot\vv & = 
       {1\over \rho c_v}  \left\{ \sumi  (R_i T-h_i) (\rho{D Y_i\over Dt}) + \grad\cdot(\lambda\grad T)
              + \Phi \right\}  \\
    {D p\over Dt} + \gamma p \grad\cdot\vv &= 
        \sumi  ({\gamma R_i} T- (\gamma-1)h_i) (\rho{D Y_i\over Dt}) + (\gamma-1) \grad\cdot(\lambda\grad T)
              + (\gamma-1) \Phi \\
   \sigma_i(Y_i,T,\rho) & = 0 
\end{align*}

\section{The Bomb problem: constant pressure combustion}

A standard test for Chemkin is the constant pressure combustion problem for a gas mixture

The equations that are solved are
\begin{align*}
   {\partial T\over \partial t} &= - {1\over \rho c_p} \sumi (h_i\sigma_i)   \\
   {\partial Y_i\over \partial t} &= {1\over \rho }  \sigma_i \qquad i=1,2,\ldots,n \\
   p_0 &= \rho R T
\end{align*}

NOTE that since the pressure is constant, the density will vary and thus $\grad\cdot(\rho\vv)$
will not be zero. This would correspond to a reaction in a box that is allowed to expand to
keep the pressure constant.

{\bf Note:} that the equation for $T$ comes from the temperature/pressure equation (\ref{eq:TP})  rather than
the temperature equation (\ref{eq:temperature}). Use of the latter would result in
a different answer! 

% This illustrates that the equations above for the bomb problem
% are {\bf inconsistent} with the full set of equations. Another way to see this is that
% $\rho$ will not be constant for the bomb problem even though it should be from 
% $\rho_t=0$ (since $\grad\cdot(\rho\vv)=0$).


\section{H-F reactions}

Here is the H-F reaction taken from Zucrow and Hoffaman.

Reaction mechanism:
\begin{align*}
  F_2 + M & \Leftrightarrow 2 F + M \\
  H_2 + M & \Leftrightarrow 2H + M   \\
  HF + M & \Leftrightarrow H + F + M \\
  HF + F & \Leftrightarrow H + F_2   \\
  HF + H & \Leftrightarrow H_2 + F   \\
 2HF     & \Leftrightarrow H_2 + F_2 \\
\end{align*}

Source terms:
\begin{align*}
   \sigma_{H_2} &= \mbar_{H_2} [ -G_1 + G_4 + G_5 ] \\
   \sigma_{F_2} &= \mbar_{F_2} [ -G_0 + G_3 + G_5 ] \\
   \sigma_{HF}  &= \mbar_{HF}  [ -G_2 -G_3-G_4-2G_5 ] \\
   \sigma_{H}   &= \mbar_{H}   [ 2G_1+G_2+G_3-G_4 ] \\
   \sigma_{F}   &= \mbar_{F}   [ 2G_0 + G_2 - G_3 + G_4 ] 
\end{align*}

Reaction step functions:
\begin{alignat*}{2}
  G_0 & = K_{f0}(T) [F_2] [M] &-& K_{b0}(T) [F]^2 [M] \\
  G_1 & = K_{f1}(T) [H_2] [M] &-& K_{b1}(T) [H]^2 [M] \\
  G_2 & = K_{f2}(T) [HF]  [M] &-& K_{b2}(T) [H][F][M] \\
  G_3 & = K_{f3}(T) [HF]  [F] &-& K_{b3}(T) [H] [F_2] \\
  G_4 & = K_{f4}(T) [HF]  [H] &-& K_{b4}(T) [H_2] [F] \\
  G_5 & = K_{f5}(T) [HF]^2    &-& K_{b5}(T) [H_2][F_2] 
\end{alignat*}

Reaction rate constants, cm, Mole, K
\begin{align*}
   K_{b0} &=  1.10 \times 10^{18} T^{-3/2} \\
   K_{b1} &=  7.50 \times 10^{18} T^{-1} \\
   K_{b2} &=  7.50 \times 10^{18} T^{-1} \\
   K_{b3} &=  5.28 \times 10^{12} T^{1/2} exp(-4000/T) \\
   K_{b4} &=  5.00 \times 10^{12} exp(-5700/T) \\
   K_{b5} &=  1.75 \times 10^{10} T^{1/2} exp(-19,997/T)\\
\end{align*}   
The forward rates are related to the backward rates through the equilibrium constants, $K_p$.
\begin{align*}
   K_{f0} &=  1.10 \times 10^{18} T^{-3/2} ~K_{p,F}^2 (A \Ru T)^{-1} \\
   K_{f1} &=  7.50 \times 10^{18} T^{-1}~K_{p,H}^2 (A \Ru T)^{-1}  \\
   K_{f2} &=  7.50 \times 10^{18} T^{-1}~K_{p,H} K_{p,F}/K_{p,HF} (A \Ru T)^{-1}  \\
   K_{f3} &=  5.28 \times 10^{12} T^{1/2} K_{p,H}/(K_{p,HF} K_{p,F})exp(-4000/T)  \\
   K_{f4} &=  5.00 \times 10^{12} K_{p,F}/(K_{p,HF} K_{p,H}) exp(-5700/T)  \\
   K_{f5} &=  1.75 \times 10^{10} T^{1/2} 1/K_{p,HF}^2 exp(-19,997/T)
\end{align*}   
Since the values of $K_P$ from the JANNAF tables are based on a reference pressure of
1 atmosphere, we need to convert some of the $K_p$ by the factor $A= atmospheres/(N m^2)=1./1.01325e5$ if
we want to use MKS units.

Eigenvalue of the jacobian
\begin{align*}
    {\partial G_i \over \partial Y_j} &= 
            K_{fi}(T) {\nu_{j,i}' \over Y_j}  \prod_{k=1}^{m} \left({\rho Y_k \over \mbar_k} \right)^{\nu_{ki}'}
         -  K_{bi}(T){\nu_{j,i}''\over Y_j}  \prod_{k=1}^{m}  \left({\rho Y_k \over \mbar_k} \right)^{\nu_{ki}''} \\
    &=  K_{bi}(T) \left(
            K_{pi}(T)(\Ru T)^{-\Delta \nu_j}
                     {\nu_{j,i}' \over Y_j}  \prod_{k=1}^{m} \left({\rho Y_k \over \mbar_k} \right)^{\nu_{ki}'}
         -           {\nu_{j,i}''\over Y_j}  \prod_{k=1}^{m}  \left({\rho Y_k \over \mbar_k} \right)^{\nu_{ki}''}
                  \right) \\
\end{align*}

\section{Coso: Combustion Solver}

Coso is a program that solves the full time dependent equations with finite difference methods.

Coso can be used to solve one dimensional reacting flow.

\noindent Current features, limitations are
\begin{itemize}
 \item no transport terms yet, i.e. no real viscous, or diffusion terms
 \item simple artificial viscosity
 \item Dependent variables can be $\rho, u, p, Y_i$ or $\rho, u, T, Y_i$.
 \item Species equations are integrated with backward Euler.
 \item Chemistry comes Chemkin or an H-F reaction from from Zucrow and Hoffman
 \item The class Reactions encapsulates the chemistry -- number of species, reaction mechanism etc.
       This allows the reactions to be changed with no changes to Coso.
\end{itemize}


\subsection{Flow in a variable area channel}


The equations of motion for flow in a channel with cross section $A(x)$ are
\begin{align}
{\partial \over \partial t} \begin{bmatrix} \rho \\ \rho u \\ \rho E \\ \rho Y \end{bmatrix}
+{1\over A} {\partial \over \partial x} 
\begin{bmatrix} \rho u A \\ (\rho u^2 + p)A  \\ (\rho E u + p u)A  \\ \rho u Y A \end{bmatrix}
= \begin{bmatrix} 0 \\ \mu \Delta u + ...  \\ -q_x  \\ \sigma + ... \end{bmatrix}
\end{align}
or 
\begin{align}
{\partial \over \partial t} \begin{bmatrix} \rho \\ \rho u \\ \rho E \\ \rho Y \end{bmatrix}
+{\partial \over \partial x} \begin{bmatrix} \rho u \\ \rho u^2 + p  \\ \rho E u + p u  \\ \rho u Y \end{bmatrix}
+{u\over A(x)} {d A(x) \over d x} \begin{bmatrix} \rho \\ \rho u  + p  \\ \rho E + p  \\ \rho  Y \end{bmatrix}
= \begin{bmatrix} 0 \\ \mu \Delta u + ...  \\ -q_x  \\ \sigma + ... \end{bmatrix}
\end{align}
or 
\begin{align}
   \rho_t + (\rho u)_x  + \rho u {A' \over A}&=0 \qquad\mbox{or}\qquad \rho_t+(\rho u A)_x/A =0\\
   u_t  + u u_x +{1\over\rho} p_x &= {1\over\rho}\grad\cdot\tauv \\
   e_t  + u e_x + {p\over\rho} ( u_x + u {A' \over A})
           &= -{1\over\rho} \grad\cdot\qv  + {1\over\rho}\Phi    \\
   T_t  + u T_x + (\gamma-1) T ( u_x + u {A' \over A})
           &= ...    \\
   p_t  + u p_x + \gamma p ( u_x + u {A' \over A})
           &= ...    \\
   Y_t + u Y_x &= {\sigma_i \over \rho} +{1\over\rho}\grad\cdot(\rho D_i\grad Y_i) 
\end{align}


\section{Numerical Experiments}

\subsection{Flow in a De Laval Nozzle with an H-F reaction}

Here we attempt to reproduce the results from the Zucrow and Hoffman.

We have two solution techniques. The first technique follows the method 
proposed in the book, using significant analytical simplifications for steady
flow. The results for frozen and equilibrium flow in a nozzle are shown in
figures \ref{fig:ZH1} to \ref{fig:ZH3}. 

\begin{figure}[htb]
  \begin{center}
   \epsfig{file=\obFigures/ZHfrozenTu.ps}
  \caption{Frozen flow ala Zucrow and Hoffman, temperature and velocity}  \label{fig:ZH1}
  \end{center}
\end{figure}

\begin{figure}[htb]
  \begin{center}
   \epsfig{file=\obFigures/ZHequilibriumTu.ps}
  \caption{Equilibrium flow ala Zucrow and Hoffman, temperature and velocity}  \label{fig:ZH2}
  \end{center}
\end{figure}

\begin{figure}[htb]
  \begin{center}
   \epsfig{file=\obFigures/ZHequilibriumSpecies.ps}
  \caption{Equilibrium flow ala Zucrow and Hoffman, species}  \label{fig:ZH3}
  \end{center}
\end{figure}


The second solution technique is el brute force, solving the full time dependent equations 
(program Coso, COmbustion SOlver).
The results for frozen, equilibrium and non-equilibrium flow are shown in 
figures \ref{fig:frozen1} - \ref{fig:nonEquilibrium4}.
In these figures we solve the pressure equation and get the temperature from the equation of state.

In figure \ref{fig:nonEquilibrium4} the eigenvalues of the scaled species jacobian matrix,
\[
     {1\over \rho} {\partial \sigma_i \over \partial Y_j } {L_0\over U_0} 
\]
are shown. The eigenvalues are all real and less than or equal to zero. These values are
indicative of the stiffness of the chemistry.

\begin{figure}[htb]
  \begin{center}
   \epsfig{file=\obFigures/frozenTu.ps}
  \caption{Frozen flow from Coso, temperature and velocity}  \label{fig:frozen1}
  \end{center}
\end{figure}
\begin{figure}[htb]
  \begin{center}
   \epsfig{file=\obFigures/frozenMEH.ps}
  \caption{Frozen flow from Coso}  \label{fig:frozen2}
  \end{center}
\end{figure}

\begin{figure}[htb]
  \begin{center}
   \epsfig{file=\obFigures/equilibriumTu.ps}
  \caption{Equilibrium flow from Coso, temperature and velocity}  \label{fig:equilibrium1}
  \end{center}
\end{figure}
\begin{figure}[htb]
  \begin{center}
   \epsfig{file=\obFigures/equilibriumSpecies.ps}
  \caption{Equilibrium flow from Coso, species}  \label{fig:equilibrium2}
  \end{center}
\end{figure}
\begin{figure}[htb]
  \begin{center}
   \epsfig{file=\obFigures/equilibriumMEH.ps}
  \caption{Equilibrium flow from Coso}  \label{fig:equilibrium3}
  \end{center}
\end{figure}

\begin{figure}[htb]
  \begin{center}
   \epsfig{file=\obFigures/nonEquilibriumTu.ps}
  \caption{Non-equilibrium flow from Coso, temperature and velocity}  \label{fig:nonEquilibrium1}
  \end{center}
\end{figure}
\begin{figure}[htb]
  \begin{center}
   \epsfig{file=\obFigures/nonEquilibriumSpecies.ps}
  \caption{Non-equilibrium flow from Coso, species}  \label{fig:nonEquilibrium2}
  \end{center}
\end{figure}
\begin{figure}[htb]
  \begin{center}
   \epsfig{file=\obFigures/nonEquilibriumMEH.ps}
  \caption{Non-equilibrium flow from Coso}  \label{fig:nonEquilibrium3}
  \end{center}
\end{figure}
\begin{figure}[htb]
  \begin{center}
   \epsfig{file=\obFigures/nonEquilibriumEigenvalues.ps}
  \caption{Non-equilibrium flow from Coso, scaled eigenvalues of the Species Jacobian}  \label{fig:nonEquilibrium4}
  \end{center}
\end{figure}

\end{document}

