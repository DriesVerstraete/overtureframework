%-----------------------------------------------------------------------
% OverBlown -- the super de duper solver
%
%-----------------------------------------------------------------------
\documentclass{article}

\addtolength{\oddsidemargin}{-.975in}
\addtolength {\textwidth} {2.0in}

\addtolength{\topmargin}{-.25in}
\addtolength {\textheight} {1.5in}

\hbadness=10000 % \tolerance=10000
\sloppy \hfuzz=30pt

\usepackage{amsmath}
\usepackage{amssymb}

\usepackage{verbatim}
\usepackage{moreverb}

\usepackage{graphics}    
\usepackage{epsfig}    
\usepackage{calc}
\usepackage{ifthen}
\usepackage{float}
\usepackage{fancybox}

% ---- we have lemmas and theorems in this paper ----
\newtheorem{assumption}{Assumption}
\newtheorem{definition}{Definition}

\newcommand{\primer}{/users/henshaw/res/primer}
\newcommand{\GF}{/users/\-henshaw/\-res/\-gf}
\newcommand{\gf}{/users/henshaw/res/gf}
\newcommand{\mapping}{/users/henshaw/res/mapping}
\newcommand{\ogshow}{/users/henshaw/res/ogshow}
\newcommand{\oges}{/users/henshaw/res/oges}
\newcommand{\cguser}{/users/henshaw/cgap/cguser}
\newcommand{\figures}{../docFigures}
\newcommand{\obFigures}{./docFigures}  % note: local version for OverBlown

\newcommand{\OVERTUREOVERTURE}{/users/\-henshaw/\-Overture/\-Overture}
\newcommand{\OvertureOverture}{/users/henshaw/Overture/Overture}

\newcommand{\RA}{realArray}
\newcommand{\MGF}{MappedGridFunction}
\newcommand{\RMGF}{realMappedGridFunction}
\newcommand{\RCGF}{realCompositeGridFunction}

\newcommand{\DABO}{Differential\-And\-Boundary\-Operators}

\newcommand{\MG}{Mapped\-Grid}
\newcommand{\GC}{Grid\-Collection}
\newcommand{\CG}{Composite\-Grid}
\newcommand{\MGCG}{Multigrid\-Composite\-Grid}

\newcommand{\MGO}{MappedGridOperators}
\newcommand{\GCO}{Grid\-Collection\-Operators}
\newcommand{\CGO}{Composite\-Grid\-Operators}
\newcommand{\MGCGO}{Multigrid\-Composite\-Grid\-Operators}

\newcommand{\Overture}{{\bf Overture\ }}
\newcommand{\OverBlown}{{\bf OverBlown\ }}
\newcommand{\overBlown}{{\bf overBlown\ }}


\begin{document}


% -----definitions-----
\def\R      {{\bf R}}
\def\Dv     {{\bf D}}
\def\bv     {{\bf b}}
\def\dv     {{\bf d}}
\def\fv     {{\bf f}}
\def\Fv     {{\bf F}}
\def\gv     {{\bf g}}
\def\Gv     {{\bf G}}
\newcommand{\lv}{{\bf l}}
\newcommand{\ev}{{\bf e}}
\newcommand{\hv}{{\bf h}}
\def\Hv     {{\bf H}}
\def\iv     {{\bf i}}
\def\jv     {{\bf j}}
\def\Jv     {{\bf J}}
\def\kv     {{\bf k}}
\def\nv     {{\bf n}}
\def\Pv     {{\bf P}}
\def\rv     {{\bf r}}
\def\tv     {{\bf t}}
\def\uv     {{\bf u}}
\def\Uv     {{\bf U}}
\def\Rv     {{\bf R}}
\def\vv     {{\bf v}}
\def\Vv     {{\bf V}}
\def\xv     {{\bf x}}
\def\yv     {{\bf y}}
\def\zv     {{\bf z}}
\def\lt     {{<}}
\def\grad    {\nabla}
\def\comma  {~~~,~~}
\def\uvd    {{\bf U}}
\def\ud     {{    U}}
\def\pd     {{    P}}
\def\calo{{\cal O}}

\newcommand{\mbar}{\bar{m}}
\newcommand{\Rbar}{\bar{R}}
\newcommand{\Ru}{R_u}         % universal gas constant
\newcommand{\Iv}{{\bf I}}
\newcommand{\qv}{{\bf q}}
\newcommand{\Div}{\grad\cdot}
\newcommand{\tauv}{{\bf \tau}}
\newcommand{\sumi}{\sum_{i=1}^n}
\newcommand{\half}{{1\over2}}
\newcommand{\dt}{{\Delta t}}

\def\ff {\tt} % font for fortran variables

\vglue5\baselineskip
\begin{flushleft}
{ \Large A Solution Adaptive Solver for the Navier-Stokes Equations for All-Speed Flows 
}
\vskip2\baselineskip
Bill Henshaw \\
\vskip\baselineskip
Centre for Applied Scientific Computing, \\
Lawrence Livermore National Laboratory, \\
henshaw@llnl.gov 
 
\vskip4\baselineskip

\noindent{\bf Abstract:}
We describe a numerical approach for solving the Navier-Stokes equations
at all speeds. For medium to large Mach number the scheme becomes a
standard high-resolution shock-capturing method (any one of a number
of possiblities). For small Mach numbers we use an implicit method that
damps sounds waves and uses a time step determined by the convective
velocity. The implicit method converges to a standard pressure-Poisson
method for incompressible flows.


\end{flushleft}

\tableofcontents
% \listoffigures

\section{Introduction}

\subsection{Conservative form} 
The compressible Navier-Stokes equations can be written in
conservation form
$$
   \uv_t + {\partial \Fv\over \partial x_1}
         + {\partial \Gv\over \partial x_2}
         + {\partial \Hv\over \partial x_3}    =  0 .
$$
The vector of conserved variables $\uv$ is
$$
 \uv = \left[ \begin{array}{c}
    \rho \\ E \\ \rho v_1 \\ \rho v_2  \\  \rho v_3
              \end{array}\right]  ~,~
$$
where $\rho$, $E$ and $\uv=[v_1,v_2,v_3]^T$ denote the density, energy and
velocity vector with components parallel to the $x_1$, $x_2$, and $x_3$ axes, respectively.  The fluxes
$$
  \left[ \begin{array}{c}
    \Fv \\ \Gv \\ \Hv
              \end{array}\right]
 =\left[ \begin{array}{c}
    \Fv^C \\ \Gv^C \\ \Hv^C
              \end{array}\right]
 -\left[ \begin{array}{c}
    \Fv^V \\ \Gv^V \\ \Hv^V
              \end{array}\right]
$$
 are a combination of
{\it convective} fluxes $[\Fv^C,\Gv^C,\Hv^C]^T$ and {\it viscous}
fluxes $[\Fv^V,\Gv^V,\Hv^V]^T$.
% $[\Fv,\Gv,\Hv]=[\Fv^C-\Fv^V,\Gv^C-\Gv^V,\Hv^C-\Hv^V]$
The convective (or Euler) fluxes are given by
$$
 \Fv^C = \left[ \begin{array}{c}
 \rho v_1 \\ v_1(E+p) \\ \rho v_1^2 + p \\ \rho v_2 v_1 \\ \rho v_3 v_1
                \end{array}\right]  ~,~
 \Gv^C = \left[ \begin{array}{c}
 \rho v_2 \\ v_2(E+p) \\ \rho v_1 v_2 \\ \rho v_2^2+p \\ \rho v_3 v_2
                \end{array}\right]  ~,~
 \Hv^C = \left[ \begin{array}{c}
 \rho v_3 \\ v_3(E+p) \\ \rho v_1 v_3 \\ \rho v_2 v_3 \\ \rho v_3^2+p
                \end{array}\right]  ~,~
$$
and the viscous fluxes are
$$
 \Fv^V = \left[ \begin{array}{c}
    0 \\ \sum_n v_n\tau_{1n}-q_1
      \\ \tau_{11} \\  \tau_{12} \\ \tau_{13}
                \end{array}\right]  ~,~
 \Gv^V = \left[ \begin{array}{c}
    0 \\ \sum_n v_n\tau_{2n}-q_2
      \\ \tau_{21} \\  \tau_{22} \\ \tau_{23}
                \end{array}\right] ~,~
 \Hv^V = \left[ \begin{array}{c}
    0 \\ \sum_n v_n\tau_{3n}-q_3
      \\ \tau_{31} \\  \tau_{32} \\ \tau_{33}
                \end{array}\right]   ~.
$$
The pressure, p, and temperature, $T$, are given by the relations
\begin{eqnarray*}
   p &=& (\gamma-1)[ E - {1\over2} \rho (v_1^2+v_2^2+v_3^2) ], \\
   T &=& {p\over\rho R_g},
\end{eqnarray*}
where $R_g$ is the gas constant.
The viscous stress terms, $\tau_{mn}$, and heat flux, $q_n$, are given by
\begin{eqnarray*}
  \tau_{mn} &=&  \mu( {\partial v_n \over\partial x_m}
                     +{\partial v_m \over\partial x_n}     )
            - {2\over3}\mu( \grad\cdot\uv ) \delta_{mn}, \\
   q_n &=& -k {\partial T \over \partial x_n} \\
%  =   -{k\over R_g} {\partial \over \partial x_n}
%      \left({p\over\rho}\right)
%  := -{\tilde{k}} {\partial \over \partial x_n}
%      \left({p\over\rho}\right),
\end{eqnarray*}
where $\delta_{mn}$ is the Kronecker delta and $\tilde{k}=k/R_g$
is a scaled thermal conductivity.



\subsection{Non-conservative form} 

The compressible Navier-Stokes equations for an ideal gas can be
written in the the form
\begin{align}
   \rho_t + (\uv\cdot\grad)\rho + \rho( \grad\cdot\uv) &= 0           \label{eq:ncr} \\
   \uv_t +(\uv\cdot\grad)\uv + {1\over \rho} \grad p
  &= {\mu\over \rho}[\Delta \uv + {1\over 3}\grad(\grad\cdot\uv)]  \label{eq:ncu}\\
   T_t + (\uv\cdot\grad) T +(\gamma-1) T \grad\cdot\uv
       & = {1\over\rho c_v} \left\{ \grad\cdot( \lambda \grad T) + \Phi \right\}  \\
       & = {\gamma-1\over \rho R}  \left\{ \grad\cdot( k \grad T) + \Phi \right\}   \label{eq:nct}
\end{align}
where $\rho$ is the density, $\uv=[v_1,v_2,v_3]^T$ is the velocity,
 $p$ the pressure
and $T$ the temperature.
The equation of state for an ideal gas is
$$
     p = \rho R_g T
$$
and
$$
\Phi = \mu
\sum_{ij} (v_i)_{x_j} ((v_j)_{x_i} + (v_i)_{x_j} )
+ \lambda (\grad\cdot\uv)^2
$$
with $\lambda=-{2\over3} \mu$.
Also
$$
\begin{array}{rl}
 \mu  &= \mbox{viscosity coefficient} \\
 R_g  & = p/(\rho T), \mbox{ gas constant} \\
   \gamma & = C_p/C_v, \mbox{ ratio of specific heats} \\
\tilde{k}&=k/R_g,\mbox{ Coefficient of thermal conductivity over } R_g
\end{array}
$$
In non-dimensional units ( with $\rho$,$\uv$,$T$ scaled to one) there are
three nondimensional parameters, $Re$ (Reynolds number),
$M$ (Mach number)
and $Pr$ (Prandtl number). They are related to $\mu$,
$\tilde{k}$, and $R_g$ as:
$$
           \mu    = {1\over Re}   \comma
            R_g   = {1 \over \gamma M^2}   \comma
      \tilde{k}   = {\gamma \over \gamma-1} {1\over Pr} {1\over Re}
$$



\section{The Low-Mach Number Algorithm}


When the mach number is low we solve the Navier-Stokes equations in the following form
\begin{align*}
   \rho_t + \grad\cdot(\rho \uv) &= 0 \\
   \uv_t + (\uv\cdot\grad)\uv + {1\over\rho}\grad p &= {1\over\rho} \mu [\Delta \uv + \grad(\grad\cdot\uv) ] \\
   p_t + (\uv\cdot\grad) p + \gamma p \grad\cdot\uv &= 
           (\gamma-1)\left\{  \grad\cdot( k \grad T) + \Phi \right\} 
\end{align*}
The equation for the energy has been rewritten as an equation for the pressure. This pressure equation can be 
derived by multiplying equation (\ref{eq:ncr}) by $R_gT$ and adding this to equation (\ref{eq:nct}) times
$R_g\rho$.

We write the system as
\[
   \Uv_t = A \Uv + B \Uv 
\]
where $\Uv=[\rho, \Uv, p]^T$ and  $A$ holds the {\sl fast} part of the operator:
\[
  A = \left[ \begin{array}{ccc}
                0 & 0 & 0 \\
                0 & 0 & -{1\over\rho} \grad  \\
                0 &  -\gamma p \grad  & 0 
             \end{array} \right]
\]
The operator $A$ will be treated with an implicit time-stepping approach.  
We use the a semi-implicit time-stepping method of the form
\begin{equation}
(I - \alpha_0 {\Delta t} A) \Uv^{n+1} = \alpha_1 \Uv^n + \alpha_2\Uv^{n-1}
                  + \sum_{j=0}^2 \beta_j B \Uv^{n-j}  \label{eq:ts}
\end{equation}
One choice of method is
\begin{alignat*}{3}
 \alpha_0 &=2/3, &\quad \alpha_1&=4/3,  &\quad \alpha_2&=-1/3, \\
 \beta_0&=16/9,  &\quad \beta_1&=-14/9, &\quad  \beta_2&=4/9 
\end{alignat*}
In this case he operator $A$ has been discretized in time with a 2nd order backward differentiation 
formula while the
operator $B$ is treated with a 3rd order accurate explicit time discretization. 

Another choice is
\begin{alignat*}{3}
 \alpha_0 &=1, , &\quad \alpha_1 &=1, , &\quad \alpha_2&=0, \\
 \beta_0 &=3/2,, &\quad  \beta_1 &=-1/2, , &\quad \beta_2&=0
\end{alignat*}
which is a combination of backward-Euler and Adams-Bashforth.

As the Mach number goes to zero, both these  methods will damp the sound waves. The
backward-Euler approach will damp the sound waves more rapidly than the second-order
backward-differential scheme. (I think BE goes like $e^{-t/M}$ and the 2nd-order-BD like
$e^{-t/\sqrt{M}}$).
The time step $\Delta t$ will not be limited by the speed of sound.
The method will be stable provided $\Delta t$ satisfies a condition of the form
\[
           c_0 |\uv| {\Delta t\over \Delta x} + c_1 \nu {\Delta t \over \Delta x ^2} < 1
\]


\newcommand{\dx}{{\Delta x}}
\newcommand{\pf}{\tilde{p}}   
To prevent the checker-board instability in the pressure we discretize and solve these
equations in a particular manner. The implicit part of the equations look like
\begin{align}
   \uv^{n+1} + \alpha_0 \dt {1\over\rho^{n+1}}\grad p^{n+1} &= \fv_1  \label{eq:ui} \\
   p^{n+1} + \alpha_0 \dt \gamma p^{n+1} \grad\cdot\uv^{n+1} &= f_2   \label{eq:pi}
\end{align}
where
\begin{align}
  \fv_1 & = \alpha_1 \uv^{n} + \alpha_2 \uv^{n-1} + \sum_{j=0}^2 \beta_j \hat{\fv}^{n-j} \\
  \hat{\fv}^{j} &= -(\uv^j \cdot\grad)\uv^j  + {1\over\rho^j } \mu [\Delta \uv^j  + \grad(\grad\cdot\uv^j ) ] \\
  f_2 &= \alpha_1 p^{n} + \alpha_2 p^{n-1} + \sum_{j=0}^2 \beta_j \hat{g}^{n-j} \\
  \hat{g}^j &= -(\uv^j\cdot\grad) p^j + (\gamma-1)\tilde{k} \Delta({p^j\over \rho^j}) +(\gamma-1)\mu \Phi^j
\end{align}
Rather than discretize and solve the system (\ref{eq:ui})-(\ref{eq:pi}), we first eliminate $\uv^{n+1}$ and
form an equation for $p^{n+1}$
\begin{equation} \label{eqn:pressure}
   p^{n+1} - (\alpha_0\dt)^2 \gamma p^{n+1} \grad\cdot( {1\over\rho^{n+1}}\grad p^{n+1} ) 
          = f_2 -   \alpha_0\dt \gamma p^{n+1} \grad\cdot\fv_1
\end{equation}
In principle this is a horrible nonlinear equation for $p^{n+1}$. In practice, since
$p^{n+1} \sim P_0(t)/M^2 + \pf/M$ as the Mach number goes to zero, this is basically
a nice linear equation for the {\sl incompressible} part of the pressure.
**Some problems with round-off errors may occur in computing the incompressible
pressure when the Mach number is very small.

We solve the pressure equation \ref{eqn:pressure} as follows:
\begin{itemize}
  \item Get $\rho^{n+1}$ by an explicit time step.
  \item Make a guess for $p^{n+1}$ of $p^* = 2 p^n - p^{n-1}$
  \item Solve the \underline{linear} system
\begin{equation} \label{eqn:pressure2}
   p^{n+1} - (\alpha_0\dt)^2 \gamma p^* \grad\cdot( {1\over\rho^{n+1}}\grad p^{n+1} ) 
          = f_2 -   \alpha_0\dt \gamma p^{n+1} \grad\cdot\fv_1
\end{equation}
    Note that since $p^*$ is almost constant and appears only as a coefficient (no derivatives
    of $p^*$ are used) then we expect this to give an accurate result.
\end{itemize}


% ------------------------------------------------------------------------------------------------------------------------
\section{Motivation}

Suppose that for low Mach numbers we are interested in computing solutions to
the compressible N-S equations that converge to the solutions of the incompressible
N-S equations. In this case we are not interested in the fast acoustic sound waves.

If we scale the velocity and density to be order one then
for low Mach numbers, $\epsilon=M$, the pressure will take the form
\[
    p = {1\over \epsilon^2} P_0(t) + {1\over\epsilon} \pf
\]
where the mean value of $\pf$ is zero??
(Latter we will see that actually $\pf=O(\epsilon)$ in our case).
The large, spatially constant, part of the pressure, $P_0(t)$ will statisfy the equation
\[
  {d P_0 \over dt} \int_{\Omega} d\xv + \gamma P_0(t) \int_{\partial \Omega} \partial_n \uv ds = 0
\]

Then the system becomes
\begin{eqnarray*}
   \rho_t + \grad\cdot(\rho \uv) &=& 0 \\
   \uv_t + (\uv\cdot\grad)\uv + {1\over\epsilon\rho}\grad \pf &=& 0 \\
   \pf_t + (\uv\cdot\grad) \pf + \gamma ({1\over \epsilon} P_0(t) + \pf) \grad\cdot\uv &=& 0 
\end{eqnarray*}
In this form we see more clearly why the time step for an explicit method must become small when
$\epsilon$ is small.

Solutions of this system converge, as $\epsilon\rightarrow 0$, 
 to solutions of the incompressible NS equations under certain conditions on the initial data.
For example by applying the bounded-derivative principle: assume that $\uv_t$ and $p_t$ are
bounded at $t=0$ then for an order 1 time
$\grad\pf=O(\epsilon)$ and $\grad\cdot\uv=O(\epsilon)$. Thus $\pf=O(\epsilon)$ and we see
that we are almost solving the incompressible NS equations.


\section{Some more details on the algorithm}

\subsection{Subtracting off the mean pressure}

When $M\ll 1$ the pressure becomes large and we can loose accuracy in the non-constant part of
the pressure. For example if $M=.01$ then $p=10^4+\pf$ and if $\pf=O(1)$ then we loose 4 digits
in the representation of $\pf$.
To fix this difficulty we can subtract off a mean value from the pressure
\[
    \hat{p} = p - p_0
\]
and then we can solve for $\hat{p}$ instead of $p$. ***But what happens if $P_0$ depends on time???


\subsection{linearization of the pressure equation}

When the Mach number is small the pressure equation (\ref{eqn:pressure}) is basically constant coefficient.
We can thus linearize the pressure equation so that the matrix does not change over time (this may
be important if we have a sparse solver that factors the matrix or one that does substantial work
to compute preconditionners). 

We can still retain an accurate answer by putting the correction terms into the explicit
part of the equations. Thus when we write the system as
\[
   \Uv_t = A \Uv + B \Uv 
\]
we can take 
\[
  A = \left[ \begin{array}{ccc}
                0 & 0 & 0 \\
                0 & 0 & -{1\over\rho_L}  \grad  \\
                0 &  -\gamma p_L  \grad  & 0 
             \end{array} \right]
\]
where $p_L$, $\rho_L$ are the states that we linearize about. The matrix $B$ will change so that
we still solve the full system. Of course in this case the time step will depend on the fast time scale
but this will be okay as long as $p-p_L$ is not too big. The time step restriction will be something like (??)
$$
      u + \sqrt{ \gamma {(p-p_L){{1\over\rho}- {1\over\rho_L}} } } {\Delta t \over \Delta x}
$$

\section{Choosing the time step}

To determine the time step for the system
\begin{align*}
   \rho_t + \grad\cdot(\rho \uv) &= 0 \\
   \uv_t + (\uv\cdot\grad)\uv + {1\over\rho}\grad p &= {1\over\rho} \mu [\Delta \uv + \grad(\grad\cdot\uv) ] \\
   p_t + (\uv\cdot\grad) p + \gamma p \grad\cdot\uv &= 
           (\gamma-1)\tilde{k} \Delta({p\over \rho}) +(\gamma-1) \Phi
\end{align*}
we first transform the equations to the unit square coordinates,
\[
  \Uv_t + A_i {\partial \over \partial r_i} \Uv 
     + B_{ij} {\partial^2 \over \partial r_i \partial r_j} \Uv = 0
\]
here
\[
A_m =
\begin{bmatrix} 
            U_m & \rho \partial_x r_m & \rho \partial_y r_m & \rho \partial_z r_m & 0 \\
            0   & U_m                 & 0                   & 0                   & (\partial_x r_m)/\rho \\ 
            0   & 0                   & U_m                 & 0                   & (\partial_y r_m)/\rho \\
            0   & 0                   & 0                   & U_m                 & (\partial_z r_m)/\rho \\
            0   & (\gamma p/\rho)\partial_x r_m & (\gamma p/\rho)\partial_y r_m &(\gamma p/\rho)\partial_z r_m  & U_m
\end{bmatrix}
\]
and
\[
    U_m =  \sum_i {\partial r_m \over \partial x_i} u_i 
\]

Next we discretize using central difference operators
\[
  \partial_ t \Vv_\nu + A_i D_{0,r_i} \Vv_\nu
     + \sum_i B_{ii} D_{+,r_i} D_{-,r_i}\Vv_\nu + \sum_{ij,i\ne j} D_{0,r_i} D_{0,r_j} \Vv_\nu = 0
\]
Here $\Vv_\nu(t)\approx \Uv(\xv_\nu,t)$.


\newcommand{\omegav}{{\bf\omega}}
If we freeze coefficients and Fourier transform, $\Vv_\nu(t) = \hat{\Vv}(t)e^{2\pi i \omegav\cdot\xv_\nu}$ 
 then we get a system
\[
   \hat{\Vv}_t + \sum_j A_j {i\sin(\xi_j) \over \Delta r_j} \hat{\Vv}
     - \sum_j B_{jj} {4 \sin^2( \xi_j/2) \over {\Delta r_j}^2 } \hat{\Vv} + ...
\]
Here $\xv_j = j \Delta \rv$, $\xi_j = 2\pi\omega_j \Delta r_j$, with $-\pi \le \xi_j < \pi$, and the
coefficient matrixes are parameterized by the frozen coefficients $\bar{\xv}$, $A_j=A_j(\bar{\xv})$

\newcommand{\omegah}{{\hat{\omega}}}
The eigenvalues of this system are
\begin{align}
  \lambda & = i ( \omegah_1 U + \omegah_2 V + \omegah_3 W ) \equiv i \hat{U} \\
  \lambda & = i \hat{U} \pm i \sqrt{ {\gamma p \over \rho}
         [ (\omegah_1 r_x + \omegah_2 s_x + \omegah_3 t_x)^2
          +(\omegah_1 r_y + \omegah_2 s_y + \omegah_3 t_y)^2 + ... ] }   \\
    &     - \sum_j B_{jj}  {4 \sin^2( \xi_j/2)\over {\Delta r_j}^2 }
\end{align}
where
\[
    \omegah_j = { \sin(\xi_j) \over \Delta r_j }
\]
For stability we require that
\[
     \lambda(\xi_j,\bar{\xv}) \dt \in \mathcal{S}
\]
for all $-\pi \le \xi_j < \pi$ , and all possible values for $\bar{\xv}$,
where $\mathcal{S}$ is the 
the stability region of the time stepping method.



\subsection{Variable time step}

When the time step changes we need to alter the parameters $\alpha_j$ and $\beta_j$
that appear in the time stepping formula (\ref{eq:ts}).


The formula for backward-Euler and Adams-Bashforth,
\begin{alignat*}{3}
 \alpha_0&=1, , &\quad \alpha_1&=1, , &\quad \alpha_2&=0, \\
 \beta_0&=3/2,, &\quad  \beta_1&=-1/2, , &\quad \beta_2&=0
\end{alignat*}
becomes
\begin{alignat}{3}
\alpha_0&=1,  &\quad \alpha_1=1,   &\quad  \alpha_2=0, \\
\beta_0&=1+{1\over 2\gamma},  &\quad \beta_1=-{1\over 2\gamma}, &\quad \beta_2=0
\end{alignat}
where
\[
   \gamma = { \Delta t_{n-1} \over \Delta t_{n} }
\]

The values for the 2nd-order backward-differentiation and 3rd-order Adams-Bashforth,
\begin{alignat*}{3}
 \alpha_0&=2/3, &\quad \alpha_1&=4/3,  &\quad \alpha_2&=-1/3, \\
 \beta_0&=16/9,  &\quad \beta_1&=-14/9, &\quad  \beta_2&=4/9 
\end{alignat*}
become
\begin{alignat}{3}
 \alpha_0&={\gamma+1\over \gamma+2}, &\quad \alpha_1&={(\gamma+1)^2\over \gamma(\gamma+2)},  
                                      &\quad \alpha_2&=-{1\over \gamma(\gamma+2)}, \\
 \beta_0 &=1+\gamma\alpha_2-\beta_1-\beta_2,        &\quad 
 \beta_1 &={ \mu/4 +{1\over 6} + \alpha_2\gamma^2/2(\gamma/3-\mu/2) \over \gamma/2(\gamma-\mu)}, &\quad 
 \beta_2 &={\alpha_2\gamma^2-1\over 2\mu} -{\beta_1\gamma \over \mu}
\end{alignat}
where
\begin{align*}
   \gamma & = { \Delta t_{n-1} \over \Delta t_{n} } \\
   \mu & = { \Delta t_{n-2} \over \Delta t_{n} } 
\end{align*}

\section{Boundary Conditions}


For the compressible NS equations the boundary conditions at a no-slip wall usually are given as
\begin{align}
  \uv & = 0 \\
  \alpha {\partial T \over \partial n} + \beta T &= g
\end{align}

For the incompressible equations (written in standard form with $\grad\cdot\uv=0$) the boundary conditions
are
\begin{align}
  \uv & = 0 
\end{align}
while for the incompressible equations written in velocity-pressure form the boundary conditions
are
\begin{align}
  \uv & = 0 \\
  \grad\cdot\uv & = 0
\end{align}
with a numerical boundary condition obtained from the normal component of the momentum equations
\[
  {\partial p \over \partial n} = \nu \nv\cdot\Delta \uv
\]

For our semi-implicit method we would like ...


Recall that the implicit equations are
\begin{align}
   \uv^{n+1} + \alpha_0 \dt {1\over\rho^{n+1}}\grad p^{n+1} &= \fv_1  \label{eq:ui} \\
   p^{n+1} + \alpha_0 \dt \gamma p^{n+1} \grad\cdot\uv^{n+1} &= f_2   \label{eq:pi}
\end{align}
If we impose the normal component of the first equation and the second equation on the boundary 
then we get equations for $p_n$ and $\grad\cdot\uv$ on
the boundary, which, in the limit of the Mach number going to zero, reduce to the same boundary conditions
used in the incompressible case.

In the (most common) zero mach number limit, the temperature decouples from the other variables and can be
solved independently (given $\rho$ and $\uv$)
\begin{align}
  T_t + \uv\cdot\grad T + ... = {\kappa\over\rho} \Delta T \\
  \alpha {\partial T \over \partial n} + \beta T = g \qquad \mbox{on the boundary}
\end{align}
  

% The equation of state is
% \[
%   p = \rho R T = {\rho T \over \gamma M^2}
% \]
% If the wall BC for $T$ is $T_n=0$ then
% \[
%   p_n = { \rho_n  T \over \gamma M^2}
% \] 






\section{Low Mach Number Asymptotics}


Much can be learned about the behaviour of the solutions for low Mach number by considering 
an asymptotic expansion.

We consider the full equations. We write down the pressure equation, temperature equation
and the equation of state, even though these are linearly dependent. After choosing non-dimensional
variables the equations become
\begin{align*}
   \rho_t + \uv\cdot\grad\rho + \rho\grad\cdot\uv &= 0 \\
   \uv_t + (\uv\cdot\grad)\uv + {1\over\rho}\grad p &= {1\over\rho} {1\over Re} [\Delta \uv + \grad(\grad\cdot\uv) ] \\
   p_t + (\uv\cdot\grad) p + \gamma p \grad\cdot\uv &= 
           {1 \over M^2 Pr Re}  \Delta{T} +{(\gamma-1)\over Re} \tilde{\Phi}    \\
   T_t + (\uv\cdot\grad) T +(\gamma-1) T \grad\cdot\uv
       &= {\gamma \over Pr Re} { 1 \over \rho } \Delta T
       + (\gamma-1) { \gamma M^2 \over Re} {1\over \rho } \tilde{\Phi}               \\
   p &= \rho R_g T
\end{align*}


\newcommand{\eps}{\epsilon}

Now if we let $\eps = M^2$ and make the asymptotic expansion
\begin{align*}
    \rho &= \rho^I + \eps \rho^1 + \eps^2 \rho^2 \\
    \uv  &= \uv^I  + \eps \uv^1  + \eps^2 \uv^2 \\
    p   &= {p_0\over\eps} + p^I  + \eps p^1 \\
    T   &= T^I + \eps T^1 
\end{align*}
where $p_0$ is a scalar which may depend on $t$.

Substituting this exapansion into the equations and equating powers of $\eps$
implies that to leading order
\begin{align}
    \rho^I_t + \uv^I\cdot\grad\rho^I  +\rho^I \grad\cdot\uv^I &= 0 \\
   \uv^I_t + (\uv^I\cdot\grad)\uv^I + {1\over\rho^I}\grad p^I &= {1\over\rho^I} {1\over Re}[\Delta \uv^I + \grad(\grad\cdot\uv^I) ] \\
   p_0 \grad\cdot\uv^I &=  {1 \over Pr Re} \Delta T^I  \label{eq:div} \\
   T^I_t + (\uv^I\cdot\grad) T^I +(\gamma-1) T^I \grad\cdot\uv^I
       &= {\gamma \over Pr Re} { 1 \over \rho^I } \Delta T^I \\
   \gamma p_0 &= \rho^I T^I
\end{align}
Thus
\[
\grad\cdot\uv^I = {\gamma \over Pr Re} { 1 \over \rho^I T^I } \Delta T^I
\]
It is interesting that by replacing this expression for $\grad\cdot\uv^I$ in the first equation 
that
\[
    \rho^I_t + \uv^I\cdot\grad\rho^I  =  {\gamma \rho^I \over Pr Re} \grad\cdot( {1\over (\rho^I)^2} \grad \rho^I )
\]
and thus we see that the density equation is diffusive. By a similar subsitution the temperature eqution becomes
\[
   T^I_t + (\uv^I\cdot\grad) T^I 
       = {\gamma(2-\gamma) \over Pr Re} { 1 \over \rho^I } \Delta T^I
\]

If we can neglect the term ${1 \over p_0 Pr Re} \Delta T^I$ in equation (\ref{eq:div}), 
then the equations reduce to the
so-called variable density incompressible equations
\begin{align*}
    \rho^I_t + \uv^I\cdot\grad\rho^I  &= 0 \\
   \uv^I_t + (\uv^I\cdot\grad)\uv^I + {1\over\rho^I}\grad p^I &= {1\over\rho^I} {1\over Re} \Delta \uv^I  \\
   \grad\cdot\uv^I &=  0  \\
   T^I_t + (\uv^I\cdot\grad) T^I  &= {\gamma \over Pr Re} { 1 \over \rho^I } \Delta T^I \\
   \gamma p_0 &= \rho^I T^I
\end{align*}
The variables $(\rho^I,\uv^I,p^I)$ can be determined from the first three equations without needing $T^I$.
Alternatively the variables $(\uv^I,p^I,T^I)$ can be determined from the first, second and fourth
equations without needing to compute $\rho^I$ (using the fifth equation to eliminate $rho^I$ in terms
of $T^I$). Using the variables $(\uv^I,p^I,T^I)$ may be preferable numerically since physical boundary
conditions at a wall are given for $T$, but no boundary conditions are needed for $\rho$.


\section{Background}

\subsection{ICE method of Harlow and Amsden, 1968}

Reference Harlow and Amsden\cite{Harlow68}.

ICE = Implicit continuous Eulerian, continuous=all-speed?

The method reduces essentially to the MAC method for incompressible flow as the Mach number
goes to zero.

\begin{align}
   \rho_t + \grad\cdot( \rho\uv ) &= 0 \label{ice1} \\
   (\rho\uv)_t + \grad\cdot( \rho \uv\otimes \uv ) + \grad p & = \grad\cdot \tauv \label{ice2} \\
   (\rho E)_t + \grad\cdot( \rho E \uv ) + \grad\cdot( p \uv ) & = ... \label{ice2}
\end{align}

Combine the first equation with the divergence of the second equation to give
\[
    -\rho_{tt} + \Delta p + \grad\cdot\left\{\grad\cdot( \rho \uv\otimes \uv ) \right\} 
                  = \grad\cdot\left\{ \grad\cdot \tauv \right\} 
\]
This equation is thought of as an equation for the density $\rho$ when we assume the the
equation of state can be written as
\[
    p = a_0^2( \rho-\rho_0) + (\gamma-1) \rho I
\]
where $I$ is the specific internal energy, $\rho_0$ is a reference density and $a_0$ is the speed
of sound at the reference density with $a_0 \rightarrow \infty$ as $M\rightarrow 0$. 
Is this really $p-P_0$??

Recall that for an ideal gas,
\begin{align*}
  p &= \rho R T \\
  E &= I + {1\over 2}\uv\cdot\uv \\
  I &= c_v T = {1\over \gamma-1} {p \over \rho } \\
  p &= (\gamma-1) \rho I
\end{align*}

\newcommand{\rp}{\rho^{n+1}}
The time discrete system for $\rp$ would look like  
\[
   - \left( \rp -2\rho^n + \rho^{n-1} \over \Delta t^2 \right)
    + \Delta( a_0^2\rp + (\gamma-1)\rp I^n )  + \grad\cdot\left\{\grad\cdot( \rp \uv^n\otimes \uv^n ) \right\} 
                  = \grad\cdot\left\{ \grad\cdot \tauv^n \right\} 
\]
After this equation is solved for $\rp$ then equation (\ref{ice2}) can be used to get $\uv^{n+1}$ and
then the engery $E$ can be computed from equation (\ref{ice3}).
For low mach number $\rho$ is nearly constant and the iteration has trouble so they replace $\rho$ by
the new variable $\sigma$ with
\[
     \sigma \equiv a_0^2( \rho-\rho_0) + K \rho
\]
where $K$ is a constant, $K=U_0^2$ for some reference velocity $U_0$. This variable is sort of
like the incompressible pressure I think.


\subsection{ICE method of Harlow and Amsden, 1970}

Reference Harlow and Amsden\cite{Harlow70}.


The basic difference is that now the equation of state is written as
\[
    p^{n+1} = p^n + c^n( \rp -\rho^n)
\]
where $c= \partial p / \partial \rho$. ``Thus $p^{n+1}$ singles out from the equation
of state its principle dependence upon the density...''

Then the equation
\[
    -\rho_{tt} + \Delta p + \grad\cdot\left\{\grad\cdot( \rho \uv\otimes \uv ) \right\} 
                  = \grad\cdot\left\{ \grad\cdot \tauv \right\} 
\]
is written in terms of the new time pressure
\[
   - {1\over c^n \Delta t^2} p^{n+1}  + \Delta p^{n+1} 
       = {1\over c^n \Delta t^2} p^{n} - \grad\cdot\left\{\grad\cdot( \rho^n \uv^n\otimes \uv^n ) \right\}  + ...
\]
(There is no $\rho^{n-1}$ either.)

The method is also changed to potentially center the pressure in time in the momentum equation and
to center the continuity equation.


\subsection{Casulli and Greenspan 1984}

Reference \cite{Casulli84}.

Starting from
\begin{align*}
   \uv_t + {1\over\rho}\grad p &= -\uv\cdot\grad\uv \\
   e_t + \gamma e \grad\cdot\uv &= -  \uv\grad\cdot e  \qquad\mbox{(just the pressure equation since $p=(\gamma-1)e$)} \\
   \rho_t + \grad\cdot(\rho\uv) &= 0 
\end{align*}
where  $p = (\gamma-1) \rho I = (\gamma-1)e$ and $e=I/\rho$ is the internal energy per unit volume ($I$ is the
internal energy per unit mass). Thus $e$ is just the pressure times a constant. 


Treating some terms implicitly gives
\begin{align*}
 {\uv^{n+1} - \uv^n \over \Delta t} + {1\over\rho^n}\grad p^{n+1} &= -\uv^n\cdot\grad\uv^n \\
 {e^{n+1} - e^n \over \Delta t} + \gamma e^n \grad\cdot\uv^{n+1} &=  -\uv^n\cdot\grad e^n \\
  {\rho^{n+1} - \rho^n \over \Delta t} &= - \grad\cdot(\rho^n\uv^n)   \\
  p^{n+1} &= (\gamma-1) e^{n+1} 
\end{align*}

And thus we can derive an equation for $p^{n+1}$
\[
   {1\over \Delta t^2} p^{n+1} - \gamma[(\gamma-1)e^n]\grad\cdot( {1\over\rho^n}\grad p^{n+1} ) = ...
\]

\subsection{KIVA discretization, Amsden, O'Rourke, Butler}


From the KIVA-II manual:


\newcommand{\rhoDot}{\dot{\rho}}
\newcommand{\Qdot}{\dot{Q}}
\newcommand{\Wdot}{\dot{W}}
\newcommand{\wDot}{\dot{w}}
\newcommand{\sigmav}{{\bf\sigma}}
\newcommand{\Tt}{\tilde{T}}

Equations:

Species $m$:
\[
   \partial_t \rho_m \grad\cdot( \rho_m \uv) = \grad\cdot\left[ \rho D \grad( {\rho_m \over \rho} )\right]
      + \rhoDot_m^c + \rhoDot^s \delta_{m1}
\]
$\rhoDot_m^c$ : chemistry source. $\rhoDot^s$ : spray source.
Spray droplets are from species one. 
\[
   \partial_t \rho +\grad\cdot( \rho \uv ) = \rhoDot^s.
\]

Momentum equations
\[
   \partial_t ( \rho \uv ) +\grad\cdot( \rho \uv \uv ) = {-1\over a^2 } \grad p 
      -A_0 \grad( {2\over3} \rho k) + \grad\cdot\sigmav + \fv^s + \rho \gv
\]
$A_0=0$ for laminar computations and $1$ for turbulence models.
\[
  \sigmav = \mu \left[ \grad\uv + (\grad\uv)^T \right] + \lambda \grad\cdot\uv \Iv
\]

Specify interal energy equation
\[
  \partial_t (\rho I) + \grad\cdot( \rho\uv I) = -p \grad\cdot\uv + (1-A_0)\sigmav:\grad\uv
                - \grad\cdot\Jv + A_0 \rho \epsilon + \Qdot^c + \Qdot^s
\]
Heat flux vector
\[
   \Jv = -K\grad T - \rho D\sum_m h_m \grad( \rho_m/\rho )
\]

$k-\epsilon$ model:
\[
   \partial_t (\rho k) + \grad\cdot( \rho \uv k) = - {2\over 3} \rho k \grad\cdot\uv 
     + \grad\cdot \left[ ({\mu \over Pr_k }) \grad k\right] - \rho\epsilon + \Wdot^s
\]
\[
  \partial(\rho\epsilon) + \grad\cdot( \rho \uv \epsilon) 
      = -( {2\over 3} c_{epsilon_1} -c_{epsilon_3} ) \rho \epsilon\grad\cdot\uv 
        +\grad\cdot \left[ ({\mu \over Pr_\epsilon }) \grad \epsilon \right] 
       + {\epsilon\over k}\left[ c_{epsilon_1}\sigmav:\grad\uv - c_{epsilon_2}\rho\epsilon + c_s \Wdot^s\right]
\]

When the SGS turbulence model is used the value of $\epsilon$ is constrained to satisfy
\[
   \epsilon \ge \left[ c_\mu \over Pr_\epsilon (c_{epsilon_2} -c_{epsilon_1})\right]^{1/2}
        {k^{3/2}\over L_{\rm SGS}}
\]

In regions where the length scale is $L_{SGS}$ the model reduces to the one-equation SGS model.

State relations
\begin{align*}
   p &= R_0 T \sum_m (\rho_m/W_m) \\
  I(T) &= \sum_m (\rho_m/\rho) I_m(T) \\
  c_p(T) &= \sum_m (\rho_m/\rho) c_{pm}(T) \\
  h_m(T) &= I_m(T) + R_0 T /W_m
\end{align*}

$h_m(T)$ and $c_{pm}$ come from the JANAF tables.

$r$ reactions, some are assumed to be in equilibrium.
\[
  \rhoDot^c_m = W_m \sum_r  (b_{mr} - a_{mr}) \wDot_r 
\]
Chemical heat release term
\[
   \Qdot_c = \sum_r Q_r \wDot_r
\]
where $Q_r$ is the negative of the heat of reaction at absolute zero,
\[
 Q_r = \sum_r (a_{mr} -b_{mr}) (\Delta h_f^0)_m
\]
and $(\Delta h_f^0)_m$ is the heat of formation of species $m$ at absolute zero.

Transport coefficients are
\begin{align*}
  \mu & = (1-A_0) \rho \nu_0 + \mu_{air} + A_0 c_\mu k^2/\epsilon \\
  \lambda &= A_3 \mu \\
  K &= {\mu c_p \over Pr} \\
  D &= {\mu \over \rho Sc} \\
  \mu_{air} &= { A_1 T^{3/2} \over T+A_2}
\end{align*}


\subsubsection{Discretization:}

\begin{align}
    { \rho^B - \rho^n \over \dt } 
         & = \grad\cdot\left(  (\rho D)^n \grad\left[ \phi_D Y_m^B + (1-\phi_D) Y_m^A\right] \right)
              + \rhoDot^c + \rhoDot^s \delta_{m,1}  \\ 
   \phi_D &= \phi_D(C_d) \qquad C_d = {\mu\over \rho} {\dt \over \dx^2} \\
   {\rho^A-\rho^n \over \dt} &= \rhoDot^c( \dot{\omega}_r^A ) + \rhoDot^s \delta_{m,1} 
\end{align}
which implies
\[
    \rho^B \left({ Y_m^B - Y_m^A \over d\dt}\right)
         = \grad\cdot\left(  (\rho D)^n \grad\left[ \phi_D Y_m^b + (1-\phi_D) Y_m^A\right]\right)
\]   


Momentum equations
\begin{align*}
  {\rho^B \uv^B - \rho^n\uv^n \over \dt} &= {-1\over a^2} \grad\left[ \phi_p p^B + (1-\phi_p)p^n\right]
       -A_0 \grad( {2\over 3} \rho^A k^A ) \\
     & + \grad\cdot( \phi_D \sigmav(\uv^B) + (1-\phi)\sigmav(\uv^n) ) \\
     & - {1\over \dt} ( \Rv + S \uv^B ) ~~~~~\mbox{coupling to gas and drop velocities}\\
     & \gv \rho^n - \rho^n {1 \over \dt} (ANC)
\end{align*}
\begin{align*}
   \phi_p & = \phi_p( {c\dt\over\dx} )
\end{align*}

Internal energy equation
\begin{align*}
  {\rho^B I^B - \rho^n I^n \over \dt} &= - ({p^B+p^n \over 2})\grad\cdot\uv^B  \\
          & + (1-A_0)[ \phi_D \sigmav(\uv^B):\grad\uv^B + (1-\phi_D) \sigmav(\uv^n):\grad\uv^n] \\
          & +\grad\cdot( K^n \grad[ \phi_D T^B + (1-\phi_D) \tilde{T} ] ) \\
          & +\grad\dot( (\rho_D)^n \{ \sum_m h_m T^n \grad( \phi_D Y^B + (1-\phi_D)Y^A ) \} ) \\
          & + A_0 \rho^A \epsilon^A + \Qdot^c + \Qdot^s 
\end{align*}

First solve
\[
   {\rho^A I^A - \rho^n I^n \over \dt } = \Qdot^c + \Qdot^s 
\]
Then define
\begin{align*}
  {\rho^B I^t - \rho^n I^A \over \dt} &= 
               A_0 \rho^A \epsilon^A  +\grad\dot( (\rho_D)^n \{ \sum_m h_m T^n \grad( \phi_D Y^B + (1-\phi_D)Y^A ) \} )
\end{align*}
Then 
\[
  \Tt = T^n + {1\over c^n}[ I^t - I^n + p^n( {1\over \rho^A} - {1\over \rho^n} ) ]
\]
giving
\begin{align*}
\rho^B { I^B - I^t \over \dt} &= - ({p^B+p^n \over 2})\grad\cdot\uv^B \\
          & + (1-A_0)[ \phi_D \sigmav(\uv^B):\grad\uv^B + (1-\phi_D) \sigmav(\uv^n):\grad\uv^n] \\
  & +\grad\cdot( K^n \grad[ \phi_D T^B + (1-\phi_D) \tilde{T} ] )
\end{align*}


Turbulence models
\begin{align*}
  {\rho^B k^B - \rho^n k^n \over \dt} &= -{2\over3} \rho^B\grad\cdot(\uv^B)[(1-f)k^n + f k^B) 
       + (VD)   \\
       & + \grad\cdot( {\mu^n\over Pr_k} \grad[ \phi_D k^B + (1-\phi_D) k^A] )  \\
       & - \rho^B {\epsilon^n\over k^n} k^B + \Wdot^s
\end{align*}
\begin{align*}
  { \rho^B\epsilon^B -\rho^n\epsilon^n \over \dt} &=
        -( {2\over 3} c_{epsilon_1} -c_{epsilon_3} )\rho^B \grad\cdot\uv^B ( (1-f)\epsilon^n + f \epsilon^B) \\
      &   +\grad\cdot \left[ ({\mu^n \over Pr_\epsilon }) \grad ( \phi_D \epsilon^B+(1-\phi_D)\epsilon^n) \right] \\
      & + {\epsilon^n\over k^n}\left[ c_{epsilon_1}(VD) - c_{epsilon_2}\rho^B\epsilon^B \right] \\
      &  + c_s \Wdot^s{\epsilon^A \over k^A}
\end{align*}


\subsection{inner iteration}


\subsection{outer iteration}

The equations fo $Y_m^B$, $k^B$ and $\epsilon^B$ are weakly coupled to the flow
field...
Thus at the start of the outer iteration we solve for $Y^B$ from equation (77)


Extrapolate the pressure,
\[
   p^p = p^n + {\dt^n\over \dt^{n-1}}( p^{n-1} - p^{n-2} )
\]
Solve the momentum equations using $p^p$ to give $\uv^p$.

Solve for $T^p$ using the internal energy equation, the equation of state.
Combining... gives an equation for $T^p$,
\begin{align}
T_p &= \big\{ T^t + \dt {p^p+p^n\over 2 c^t} {\grad\cdot\uv^B \over \rho^B}  \label{eq:kiva:te} \\
    & + {\dt\over\rho^B c^t} \Big[\grad\cdot( K^n \grad[ \phi_D T^p + (1-\phi_D) \tilde{T} ] ) 
       + (1-A_0)[ \phi_D \sigmav(\uv^p):\grad\uv^p + (1-\phi_D) \sigmav(\uv^n):\grad\uv^n]  \Big] \big\} \\
 & / ( 1 + {p^p+p^n \over 2 c^t p^p } R_0 \sum_m ( Y_m^B/W_m )
\end{align}

Make consistent with the equation of state and the T equation ??

Now solve for a corrected pressure from
(86):
\[
 \rho^B { \uv^B-\uv^A \over \dt} = - {1\over a^2} \grad( \phi_p p^B + (1-\phi_p)\phi^n )
          + A_0\grad( {2\over 3} \rho^A k^A) +...
\]
and the temperature equation (\label{eq:kiva:te})
and a linearized equation of state ??
\[
   p = \rho R T
\]
and cell volume info ??
\[
  {\rho^B-\rho^n \over \dt} + \grad\cdot( \rho^B \uv^B ) =0   \qquad \mbox{guess??}
\]


\subsection{Projection Method of Pember et. al.}

Reference 
{\sl An Adaptive Projection Method for Unsteady, Low-Mach Number Combustion}
and
{\sl A Higher-Order projection Method for the Simulation of Unsteady Turbulent
Nonpremixed Combustion in an Industrial Burner} \cite{Pember??}.

The low Mach number algorithm involves the solution of the following (redeundant) equations 
\begin{align}
  \rho_t + \grad\cdot( \rho\uv ) &= 0 \\
   P_0 & = \rho \Rbar T \\
   \rho \uv_t + \rho (\uv\cdot\grad)\uv + \grad p & = \rho\gv + \grad\cdot \tauv  \\
 \rho C_p {DT\over Dt} &= \grad\cdot( \lambda \grad T ) + 
       \sum \rho D \grad Y_i\cdot\grad h_i - \sum \sigma_i h_i \\
 {\partial \rho h \over \partial t} + \grad\cdot( \rho \uv h ) &= \grad\cdot( (\lambda/c_p) \grad h ) \\
 {\partial \rho Y_i \over \partial t} + \grad\cdot(\rho\uv Y_i) &= \sigma_i +\grad\cdot(\rho D \grad Y_i ) \\
   \grad\cdot\uv &= -{1\over\rho} {D\rho\over Dt} = {\rho \over P_0} ( {DT\over Dt} \Rbar + ...) \\
          &\equiv S \\
 S &= {1\over \rho c_p T} \left( \grad\cdot(\lambda\grad T) + \sum \rho D\grad Y_i\cdot\grad h_i \right) \\
    & + {W\over \rho} \sum {1\over W_i}\grad\cdot(D\rho\grad Y_i) +
          {1\over \rho} \sum_i( {W\over W_i} - {h_i\over c_p T})\sigma_i \\
 \tilde{S} &= S + c (\tilde{p}-p_0) { c_p - R \over \Delta t c_p \tilde{p} }
\end{align}

\clearpage
\section{The *NEW* low mach number algorithm}


In this approach we solve the following equations:
\begin{align}
   \rho_t + \grad\cdot( \rho\uv ) &= 0 \label{eq:rho} \\
   (\rho\uv)_t + \grad\cdot( \rho \uv\otimes \uv ) + \grad p & = \grad\cdot \tauv \label{eq:u} \\
   T_t + (\uv\cdot\grad) T +(\gamma-1) T \grad\cdot\uv
       - (\gamma-1) { \tilde{k} \over \rho } \Delta T
       - (\gamma-1) { 1 \over R_g \rho } \Phi & = 0                 \label{eq:T}  \\
   p &= \rho R T \label{eq:state}
\end{align}
The difference from the other approaches is that we first solve for the temperature and density
using the temperature equation and the equation of state.


We first solve equations (\ref{eq:T}) and (\ref{eq:state}) for $T^{n+1}$ and $\rho^{n+1}$
given a guess for $p^{n+1}$ (recall that p is nearly constant for $M$ small):
\begin{align*}
{T^{n+1} - T^n \over \Delta t} &+ [(\uv\cdot\grad) T]^{n+\half} + [(\gamma-1)T \grad\cdot\uv]^{n+\half}
       - [(\gamma-1) { \tilde{k} \over \rho } \Delta T]^{n+\half}
       - [(\gamma-1) { 1 \over R_g \rho } \Phi]^{n+\half} = 0                  \\
   p^* &= 2 p^n - p^{n-1} \qquad\mbox{guess for $p^{n+1}$, accurate to $O(M^2 \dt^2)$} \\
   \rho^{n+1} &=  { p^* \over R T^{n+1} }
\end{align*}
Here the superscript $n+\half$ denotes some approximation to the solution at time $t=(n+\half)\dt$. The
exact details are for now left unspecified.
After solving these equations we obtain $T^{n+1}$ and $\rho^{n+1}$. 

Discretize the conservative form of the momemtum equations, treating the pressure term in a fully
implicit manner,
\begin{align}
  {(\rho\uv)^{n+1} - (\rho\uv)^{n} \over \dt }+ \grad p^{n+1} 
         &= - \grad\cdot( \rho \uv\otimes \uv )]^{n+\half} +
              \grad\cdot \tauv^{n+\half}  \\
                          &\equiv -\Fv^{n+\half} \label{eq:ut} \\
\end{align}
where
\begin{align}
       \Fv &\equiv \grad\cdot( \rho \uv\otimes \uv ) -  \grad\cdot \tauv \\
             &= (\rho u \uv )_x + (\rho v \uv)_y + (\rho w \uv )_z - \grad\cdot\tauv
\end{align}

Now we take the divergence of (\ref{eq:ut}) 
\begin{align}
  {\grad\cdot(\rho\uv)^{n+1} - \grad\cdot(\rho\uv)^{n} \over \dt }
       + \Delta p^{n+1}  = -\grad\cdot\Fv^{n+\half}   \label{eq:gut}
\end{align}
This will be turned into a Poisson equation for $p^{n+1}$ by replacing the first 
term $\grad\cdot(\rho\uv)^{n+1}$ from a discretization to (\ref{eq:rho}):
\[
  \rho_t^{n+\half}  + \alpha \grad\cdot( \rho\uv )^{n+1} + (1-\alpha) \grad\cdot( \rho\uv )^{n}= 0
\]
where $\alpha=\half$ will be a second-order Crank-Nicolson approximation. Substituting into (\ref{eq:gut})
and re-arranging gives the pressure equation
\begin{align*}
   \Delta p^{n+1} & = { 1\over\alpha} { \rho_t^{n+\half} + \grad\cdot( \rho\uv )^{n} \over \dt} 
                      - \grad\cdot\Fv ^{n+\half}   \label{eq:newp}
\end{align*}
The term $\grad\cdot( \rho\uv )^{n}$ 
acts as the divergence damping term in the incompressible limit (see ??).

Using a centred approximation for $\rho_t^{n+\half}$ gives 
\begin{align}
   \Delta p^{n+1} & = { 1\over \alpha  \dt} \left\{ {\rho^{n+1}-\rho^n \over \dt} + \grad\cdot( \rho\uv )^{n} \right\} 
                      - \grad\cdot\Fv ^{n+\half}  
\end{align}


With $\alpha=1$ (backward-Euler on the density equation) 
we could replace both of the first two terms in (\ref{eq:gut}) to give the alternative pressure
equation
\begin{align*}
   \Delta p^{n+1} &= { \rho^{n+1} -2\rho^n + \rho^{n-1} \over \dt^2 }
                         -  \grad\cdot\Fv^{n+\half}         
\end{align*}
This no longer has a divergence damping term. For $\rho$ nearly constant we can always
add a multiple of the continuity equation to create this damping term,
\begin{align*}
   \Delta p^{n+1} & = { \rho^{n+1} -2\rho^n + \rho^{n-1} \over \dt^2 } -  \grad\cdot\Fv^{n+\half}  
                + \delta[ \rho_t^{n} + \grad\cdot(\rho\uv)^n ]  \label{eq:newp2}
\end{align*}
We usually like to take $\delta\sim \mu/(\Delta x ^2)$.


Now we can solve for $p^{n+1}$ from (\ref{eq:newp}) or (\ref{eq:newp2}).
Given $p^{n+1}$ we can get $\uv^{n+1}$ from a discretization of (\ref{eq:u}) 
\begin{align*}
  {(\rho\uv)^{n+1} - (\rho\uv)^n \over \dt } + \grad p^{n+1} + \Fv^{n+\half} &= 0
\end{align*}
or of the momentum
equation in non-conservative form:
\begin{align*}
  {\uv^{n+1} - \uv^n \over \dt } + [(\uv\cdot\grad)\uv]^{n+\half} + {1\over\rho^{n+1}} \grad p^{n+1}
         &=  \grad\cdot \tauv^{n+\half} 
\end{align*}
We could also treat the viscous terms implicitly in this last equation.



\bibliography{/home/henshaw/papers/henshaw}
\bibliographystyle{siam}
\end{document}
