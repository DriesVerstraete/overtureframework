% ===================================================================================================
\mysection{Enforcing time dependent boundary conditions with particular functions} \label{sec:particularFunctions}


In this section we discuss the use of {\em particular functions} (PF) to enforce time-dependent boundary
conditions in a ROM. There seem to be a number of ways to enforce boundary conditions~\cite{Gunzburger}
and here is one option.

Let us assume that we are solving a PDE, $L\uv=\fv$ on some domain $\Omega$ and suppose that we 
apply some boundary conditions $\Bc_m(\uv;\xv)=\gv_m(\xv,t)$, for $\xv\in\Gamma_m$ where $\Gamma_m \subset \partial\Omega$ denotes
a portion of the boundary and $m=1,2,\ldots,N_b$ denotes the different boundary conditions. 


We look for a ROM solution of the form 
\begin{align}
   \uv^K(\xv,t) = \sum_{i=1}^K q_i(t) \phiv_i  +  \sum_{m=1}^{N_p} g_m(t) \phiv_m^p , 
\end{align}
where $\phiv_i$ are the POD vectors (that satisfy homogeneous boundary conditions) 
and $\phiv_m^p$ are the {\em particular functions}. The coefficients $g_m(t)$ are assumed
to be {\em given} functions of time, chosen so that $\uv^K(\xv,t)$ satisfies the time-dependent boundary conditions.
The particular functions are used to enforce the boundary conditions.


Here is one way to choose the particular functions. Let $\phiv_i^p$ be a function that 
satisfies time-invariant inhomogeneous boundary conditions on $\Gamma_i$ but homogeneous boundary 
conditions on $\Gamma_j$ for $j\ne i$. For example, $\phiv_i^p$, may correspond to some steady 
state solution.

{\bf Question:} What does this mean when $\uv(\xv,t)$ is a vector of unknowns? We may need a different PF for each component.

{\bf Question:} What does this mean when the boundary condition may have different spatial variation (e.g. uniform or parabolic profile?).
   We may needed to have multiple PF's to span the space of spatial variations.


Suppose that we are given a set snapshot vectors $\Wc = \big\{ \wv_n\big\}_{n=1}^N$, each $\wv_n \in \Real^M$.
These snapshots may be taken from a set of simulations in which inhomogeneous and time-dependent
boundary conditions are applied. Before constructing the POD vectors we first adjust the snapshots
so that they satisfy homogeneous boundary conditions by subtracting off the appropriate combination
of PF's, 
\begin{align*}
    \widehat{\wv}_n &= \wv_n - \sum_{m=1}^{N_p} \beta_m \phiv_m^p.
\end{align*}
For example if $\xv_m\in\Gamma_m$ is a point on one of the boundaries,
in some cases we could choose $\beta_m=u(\xv_m,t^n)/\phi^p(\xv_m)$ (in the case when we are solving a PDE for a scalar unknown). 

The POD vectors $\phiv_i$ are then determined as the left singular vectors from the snapshot matrix formed from 
the adjusted snapshots $\widehat{\wv}_n$.

The ROM will be formed from the Galerkin projection of the PDE, $L\uv=\fv$, 
\begin{align}
  <\phiv_i, L\uv^K(\xv,t)>  = <\phiv_i, L \sum_{j=1}^K q_j(t) \phiv_j>  +  <\phiv_i,L \sum_{m=1}^{N_p} g_m(t) \phiv_m^p > = <\phiv_i,\fv >, 
\end{align}
or 
\begin{align}
  <\phiv_i, L \sum_{j=1}^K q_j(t) \phiv_j>  = <\phiv_i,\fv > - <\phiv_i, L \sum_{m=1}^{N_p} g_m(t) \phiv_m^p >.
\end{align}

