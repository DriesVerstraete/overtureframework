%
%  Flow Control Notes
%
\documentclass[11pt]{article} 
\usepackage[bookmarks=true]{hyperref}  

% \input documentationPageSize.tex
\hbadness=10000 
\sloppy \hfuzz=30pt
\usepackage{calc}

% set the page width and height for the paper 
\setlength{\textwidth}{7in}  
% \setlength{\textwidth}{6.5in}  
% \setlength{\textwidth}{6.25in}  
\setlength{\textheight}{9.5in} 
% here we automatically compute the offsets in order to centre the page
\setlength{\oddsidemargin}{(\paperwidth-\textwidth)/2 - 1.in}
% \setlength{\topmargin}{(\paperheight-\textheight -\headheight-\headsep-\footskip)/2 - 1in + .5in }
\setlength{\topmargin}{(\paperheight-\textheight -\headheight-\headsep-\footskip)/2 - 1.25in  }


\input homeHenshaw

% --------------------------------------------
% NOTE: trouble with tikz and program package ??
\usepackage{tikz}

\input trimFig.tex

% The amssymb package provides various useful mathematical symbols
\usepackage{amsmath}
\usepackage{amssymb}

\newcommand{\Largebf}{\sffamily\bfseries\Large}
\newcommand{\largebf}{\sffamily\bfseries\large}
\newcommand{\largess}{\sffamily\large}
\newcommand{\Largess}{\sffamily\Large}
\newcommand{\bfss}{\sffamily\bfseries}
\newcommand{\smallss}{\sffamily\small}

\newcommand{\beq}{\begin{equation}}
\newcommand{\eeq}{\end{equation}}
\newcommand{\Omegav}{\boldsymbol{\Omega}}
\newcommand{\omegav}{\boldsymbol{\omega}}

\input wdhDefinitions.tex
\newcommand{\mbar}{\bar{m}}
\newcommand{\Rbar}{\bar{R}}
\newcommand{\Ru}{R_u}         % universal gas constant
% \newcommand{\grad}{\nabla}
\newcommand{\Div}{\grad\cdot}
\newcommand{\tauv}{\boldsymbol{\tau}}
\newcommand{\sigmav}{\boldsymbol{\sigma}}
\newcommand{\thetav}{\boldsymbol{\theta}}
\newcommand{\kappav}{\boldsymbol{\kappa}}
\newcommand{\lambdav}{\boldsymbol{\lambda}}
\newcommand{\xiv}{\boldsymbol{\xi}}
\newcommand{\sumi}{\sum_{i=1}^n}


\newcommand{\Pc}{{\mathcal P}}
\newcommand{\Hc}{{\mathcal H}}
\newcommand{\Ec}{{\mathcal E}}

\newcommand{\mw}{W}  % molecular weight
\newcommand{\mwBar}{\overline{W}}  % molecular weight of the mixture
\newcommand{\Dc}{\mathcal{D}}

\newcommand{\rate}{{\rm rate}}
\newcommand{\tableFont}{\footnotesize}% font size for tables

% \usepackage{verbatim}
% \usepackage{moreverb}
% \usepackage{graphics}    
% \usepackage{epsfig}    
% \usepackage{fancybox}    

% tell TeX that is ok to have more floats/tables at the top, bottom and total
\setcounter{bottomnumber}{5} % default 2
\setcounter{topnumber}{5}    % default 1 
\setcounter{totalnumber}{10}  % default 3
\renewcommand{\textfraction}{.001}  % default .2

\begin{document}
 
\title{Notes on Flow Control}

\author{
Bill Henshaw \\
\  \\
Centre for Applied Scientific Computing, \\
Lawrence Livermore National Laboratory, \\
henshaw@llnl.gov }
 
\maketitle

\tableofcontents


% ------------------------------------------------------------------------
\clearpage 
\newcommand{\Tinflow}{T_{\rm inflow}}
\newcommand{\Tset}{T_{\rm set}}
\newcommand{\Tave}{T_{\rm ave}}
\section{Control of heated flow past a cylinder}

In this example we consider the flow of an incompressible fluid past a cylinder. 
An incompressible fluid flows from left to right past a cylinder in a square channel. The inflow velocity is $u=1$ and the
inflow temperature is $T=\Tinflow(t)$. The channel walls on top and bottom are no-slip with fixed $T=0$. 
The temperature
of the inflow is controlled in order that the average temperature, $\Tave$ in the domain 
obtains
some specified (constant) set point temperature $\Tset$. 
We use a PID controller~\ref{sec:PIDcontroller}. 
% 
Figure~\ref{fig:flowCyl} shows the steady state solution for $T$ and $u$.
Figure~\ref{fig:flowCylControlVariables} shows the control variables $\Tinflow(t)$, $\Tave(t)$ and $\Tset$ as a function
of time for various values of the PID parameters.


% -----------------------------------------------------------------------------------------------------
\begin{figure}[hbt]
\newcommand{\figWidth}{7.25cm}
\newcommand{\trimfig}[2]{\trimFig{#1}{#2}{0.0}{0.0}{.2}{.0}}
\begin{center}
\begin{tikzpicture}[scale=1]
  \useasboundingbox (0,.9) rectangle (16.,6.5);  % set the bounding box (so we have less surrounding white space)
  \draw (  0, 0) node[anchor=south west,xshift=-4pt,yshift=+0pt] {\trimfig{figures/flowCicT}{\figWidth}};
  \draw (8.0, 0) node[anchor=south west,xshift=-4pt,yshift=+0pt] {\trimfig{figures/flowCicU}{\figWidth}};
%
% \draw (current bounding box.south west) rectangle (current bounding box.north east);
% grid:
%  \draw[step=1cm,gray] (0,0) grid (16,7);
\end{tikzpicture}
\end{center}
  \caption{Flow past a cylinder steady-state solution, temperature (left) and $u$ right. }
  \label{fig:flowCyl}
\end{figure}
%---------------------------------------------------------------------------------------------------------

% -----------------------------------------------------------------------------------------------------
\begin{figure}[hbt]
\newcommand{\figWidth}{6.5cm}
\newcommand{\trimfig}[2]{\trimFig{#1}{#2}{0.0}{0.0}{.0}{.0}}
\begin{center}
\begin{tikzpicture}[scale=1]
  \useasboundingbox (0,1.2) rectangle (17.,11.0);  % set the bounding box (so we have less surrounding white space)
  \draw (  0,5.70) node[anchor=south west,xshift=-4pt,yshift=+0pt] {\trimfig{figures/FlowCylGain1}{\figWidth}};
  \draw (8.0,5.70) node[anchor=south west,xshift=-4pt,yshift=+0pt] {\trimfig{figures/FlowCylGain2}{\figWidth}};
  \draw (0.0, 0.0) node[anchor=south west,xshift=-4pt,yshift=+0pt] {\trimfig{figures/FlowCylGainKp1Ki1}{\figWidth}};
%
% \draw (current bounding box.south west) rectangle (current bounding box.north east);
% grid:
%\draw[step=1cm,gray] (0,0) grid (16,11);
\end{tikzpicture}
\end{center}
  \caption{Flow past a cylinder results. Top left: $K_p=0$, $K_i=1$, $K_d=0$. Top right:  $K_p=0$, $K_i=2$, $K_d=0$.
Bottom left:  $K_p=1$, $K_i=1$, $K_d=0$. }
  \label{fig:flowCylControlVariables}
\end{figure}
%---------------------------------------------------------------------------------------------------------


% ------------------------------------------------------------------------
\clearpage 
\section{Control of heated flow into a two-dimensional room}

In this example we consider the flow in a two-dimensional model of a room containing
a desk, hot computer and ceiling partition. Air flows from a vent on the ceiling and exits
another vent on the ceiling. 




{\bf Case 1:} We control the temperature of the air at the inlet. The room is
initially at a temperature of $T=5$ and the control is chosen to drive the
average temperature in the room to $\bar{T}=0$. The speed of the inlet flow is $.5$ m/s.
We use a PID controller~\ref{sec:PIDcontroller}.
Figure~\ref{fig:heatedRoomT} shows the solution at two
times. Figure~\ref{fig:heatedRoomControlVariables} shows the control variables
over time.


% -----------------------------------------------------------------------------------------------------
\begin{figure}[hbt]
\newcommand{\figWidth}{7.25cm}
\newcommand{\trimfig}[2]{\trimFig{#1}{#2}{0.0}{0.0}{.2}{.2}}
\begin{center}
\begin{tikzpicture}[scale=1]
  \useasboundingbox (0,.9) rectangle (16.,6.0);  % set the bounding box (so we have less surrounding white space)
  \draw (  0, 0) node[anchor=south west,xshift=-4pt,yshift=+0pt] {\trimfig{figures/heatedRoom2d4T10p0}{\figWidth}};
  \draw (8.0, 0) node[anchor=south west,xshift=-4pt,yshift=+0pt] {\trimfig{figures/heatedRoom2d4T300p0}{\figWidth}};
%
% \draw (current bounding box.south west) rectangle (current bounding box.north east);
% grid:
% \draw[step=1cm,gray] (0,0) grid (16,6);
\end{tikzpicture}
\end{center}
  \caption{Case 1: controlling a heated room.
  Left: temperature at time $t=10$ when cool air is entering the initially warm room.
  Right: temperature at $t=300$ which is near the steady-state with an average room temperature of $\bar{T}\approx 0$. }
  \label{fig:heatedRoomT}
\end{figure}
%---------------------------------------------------------------------------------------------------------



% HeatedRoom2d4.pdf
% -----------------------------------------------------------------------------------------------------
\begin{figure}[hbt]
\newcommand{\figWidth}{7.25cm}
\newcommand{\trimfig}[2]{\trimFig{#1}{#2}{0.0}{0.0}{.0}{.0}}
\begin{center}
\begin{tikzpicture}[scale=1]
  \useasboundingbox (0,.75) rectangle (16.,6.75);  % set the bounding box (so we have less surrounding white space)
  \draw (  0, 0) node[anchor=south west,xshift=-4pt,yshift=+0pt] {\trimfig{figures/HeatedRoom2d4}{\figWidth}};
%%  \draw (8.0, 0) node[anchor=south west,xshift=-4pt,yshift=+0pt] {\trimfig{figures/FlowCylGain2}{\figWidth}};
%
% \draw (current bounding box.south west) rectangle (current bounding box.north east);
% grid:
% \draw[step=1cm,gray] (0,0) grid (16,7);
\end{tikzpicture}
\end{center}
  \caption{Case 1: controlling a heated room. Left: $K_p=.5$, $K_i=.1$, $K_d=0$. These results correspond to the solution shown
in Figure~\ref{fig:heatedRoomT}.}
  \label{fig:heatedRoomControlVariables}
\end{figure}
%---------------------------------------------------------------------------------------------------------


% -------------------------------------------------------------------------------------------------------
\clearpage
\input Controller


% -------------------------------------------------------------------------------------------------
% \vfill\eject
% \bibliography{\homeHenshaw/papers/henshaw}
% \bibliographystyle{siam}


\end{document}
