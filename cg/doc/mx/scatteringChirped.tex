%------------------------------------------------------------------------------------------
\clearpage
\subsection{Scattering of a chirped plane wave from a two-dimensional dielectric cylinder} \label{sec:cyl2dDielectricScatChirped}

The command file {\tt runs/emebeddedBody/chirped.cmd} can be used to compute the scattering
of a {\em chirped} plane wave from a PEC. 
A chirp wave has a wavelength that changes over time: 
the definition the chirped wave is given in Section~\ref{sec:chirpedPlaneWaveBoundaryForcing}.

% \noindent Example:

\noindent (1) Generate the grid using the command file {\tt cicArg.cmd} in {\tt Overture/sampleGrids}:
{\small
\begin{verbatim}
  ogen -noplot cicArg -order=4 -interp=e -xa=-5. -xb=5. -ya=-5. -yb=5. -prefix=cicBig -factor=8
\end{verbatim}
}
\noindent (2) Run cgmx: 
{\small
\begin{verbatim}
  cgmx chirped -g=cicBige8.order4.hdf -option=cyl -x0=-.5 -ta=.5 -tb=6 -kx=2 -bandWidth=2 -tf=7 -tp=.5
\end{verbatim}
}

{
\begin{figure}[hbt]
\newcommand{\figWidth}{5.5cm}
\newcommand{\trimfig}[2]{\trimFig{#1}{#2}{0.1}{0.05}{.05}{.05}}
\begin{center}
\begin{tikzpicture}[scale=1]
  \useasboundingbox (0,.5) rectangle (17,5.5);  % set the bounding box (so we have less surrounding white space)
  \draw ( 0.0, 0) node[anchor=south west] {\trimfig{figures/chirpCylExt7p0}{\figWidth}};
  \draw ( 5.7, 0) node[anchor=south west] {\trimfig{figures/chirpCylEyt7p0}{\figWidth}};
  \draw (11.4, 0) node[anchor=south west] {\trimfig{figures/chirpCylHzt7p0}{\figWidth}};
 % - labels
 %   \draw (\txa,4.75) node[draw,fill=white,anchor=east] {\scriptsize $t=0.5$};
 %   \draw (\txb,4.75) node[draw,fill=white,anchor=east] {\scriptsize $t=1.0$};
 %   \draw (\txc,4.75) node[draw,fill=white,anchor=east] {\scriptsize $t=1.5$};
 %  \draw (current bounding box.south west) rectangle (current bounding box.north east);
% grid:
%  \draw[step=1cm,gray] (0,0) grid (17.0,5);
\end{tikzpicture}
\end{center}
\caption{Scattering of a chirped plane wave from a PEC. Note that the wavelength is decreasing over time (i.e. nearer to the cylinder).
The contours show the computed scattered field compoents $E_x$, $E_y$ and $H_z$. at $t=7.0$.}
\label{fig:cyl2dDielectricScatChirped}
\end{figure}
}

\noindent{\bf Notes:}
\begin{enumerate}
  \item The scattered field is computed directly; this is accomplished by adding an inhomogenous PEC boundary condition.
  \item Far field boundary conditions were taken as abcEM2.
  \item See the comments at the top of the command file for command line arguments and further examples.
  \item This case was run with fourth-order accuracy.
\end{enumerate}
