% -------------------------------------------------------------------
\subsection{Simulation of flow past a blood clot filter}\label{sec:bloodFlow}

Figure~\ref{fig:bloodFlow} shows results from computations of the incompressible
flow in a three-dimensional cylindrical pipe (``vein'') with an embedded wire frame ``filter'' and 
embedded blood clots. 
The computations were performed with the pseudo steady-state solution algorithm. 

Further details on these simulations can be found in 
M.A. Singer, WDH, S.L. Wang, {\em Computational Modeling of Blood Flow in the Trapease Inferior Vena Cava Filter},
       Journal of Vascular and Interventional Radiology, {\bf 20}, 2009.

{
\newcommand{\figWidthp}{7.5cm}
\newcommand{\trimfig}[2]{\trimPlotb{#1}{#2}{.0}{.0}{.0}{.0}}
\newcommand{\figWidtha}{7.5cm}
\newcommand{\trimfiga}[2]{\trimPlotb{#1}{#2}{.0}{.35}{.0}{.0}}
\newcommand{\figWidthb}{4.5cm}
\newcommand{\trimfigb}[2]{\trimPlotb{#1}{#2}{.940}{.03}{.09}{.09}}
\begin{figure}[hbt]
\begin{center}
\begin{tikzpicture}[scale=1]
  \useasboundingbox (0,.5) rectangle (16.,13.9);  % set the bounding box (so we have less surrounding white space)
%
  \draw ( 0.0, 0) node[anchor=south west,xshift=-4pt,yshift=+0pt] {\trimfig{fig/clotConeFinal}{\figWidthp}};
  \draw ( 8.0, 0) node[anchor=south west,xshift=-4pt,yshift=+0pt] {\trimfig{fig/clot050infFinal}{\figWidthp}};
  \draw ( 8.0, 7) node[anchor=south west,xshift=-4pt,yshift=+0pt] {\trimfig{fig/clot200Final}{\figWidthp}};
%
  \draw ( 0.0,10.2) node[anchor=south west,xshift=-4pt,yshift=+0pt] {\trimfiga{fig/trapeaseCompare02}{\figWidtha}};
  \draw ( 3.0, 6.7) node[anchor=south west,xshift=-4pt,yshift=+0pt] {\trimfigb{fig/trapeaseCompare02}{\figWidthb}};
%
% \draw (current bounding box.south west) rectangle (current bounding box.north east);
% grid:
%\draw[step=1cm,gray] (0,0) grid (16,13);
\end{tikzpicture}
\end{center}
 \caption{Flow past a blood-clot filter using cgins.}\label{fig:bloodFlow}
\end{figure}
}


\bogus{
\newcommand{\blood}{\homeHenshaw/people/singer/bloodFlowFigures}
\newcommand{\labelsize}{\footnotesize}
% 
\psset{xunit=1.0cm,yunit=1.0cm,runit=1.0cm,linewidth=1.pt}
\renewcommand{\figWidth}{7.3333cm}
\newcommand{\figWidthb}{16.0cm}
%\\psset{xunit=0.75cm,yunit=0.75cm,runit=0.75cm,linewidth=1.pt}
%\\newcommand{\figWidth}{5.5cm}
%\\newcommand{\figWidthb}{12.0cm}
%\psset{xunit=0.65cm,yunit=0.65cm,runit=0.65cm,linewidth=1.pt}
%\newcommand{\figWidth}{4.7666cm}
%\newcommand{\figWidthb}{10.4cm}
% 
\newcommand{\clipfig}[2]{\clipFig{#1}{#2}{-.05}{1.05}{0.07}{.80}}
\newcommand{\clipfiga}[2]{\clipFig{#1}{#2}{.0}{.72}{-.04}{.38}}
\newcommand{\clipfigb}[2]{\clipFig{#1}{#2}{.745}{.975}{.0775}{.25}}
% 
% \begin{frame}[label=blood]{Flow past a blood-clot filter using cgins}%{Mike Singer, LLNL}
\begin{figure}[hbt]
\begin{center}
\begin{pspicture}(0,.25)(16.5,10.4)
\rput(12.5,8.){\clipfig{\blood/clot200Final.eps}{\figWidth}}
\rput( 3.5,2.5){\clipfig{\blood/clotConeFinal.eps}{\figWidth}}
\rput(12.5,2.5){\clipfig{\blood/clot050infFinal.eps}{\figWidth}}
% 
\rput( 3.3,9.1){\clipfiga{\blood/trapeaseCompare02.eps}{\figWidth}}
\rput(-0.1,6.7){\clipfigb{\blood/trapeaseCompare02.eps}{\figWidthb}}
% 
% \rput( 2.0,10){\psframebox*[fillstyle=solid,fillcolor=mediumgoldenrod]{\labelsize $\qsGrid\sp{(b=2,l=0,r=2)}$, $h=\frac{1}{20}$}}
% \rput( 5.0,9.25){\tiny $h={\blue \frac{1}{40}}, {\green \frac{1}{80}}, {\red \frac{1}{160}}, {\frac{1}{320}}$}
% 
%
\rput(2.75,10.5){\labelsize Overlapping grid for the filter}
\rput(6.25,8.25){\labelsize Trap-ease wire filter}
%
%\rput(10.,10){\psframebox*[fillstyle=solid,fillcolor=mediumgoldenrod]{\labelsize Spherical clot}}
%\rput( 2.0,4.5){\psframebox*[fillstyle=solid,fillcolor=mediumgoldenrod]{\labelsize Cone shaped clot}}
%\rput(10.,4.5){\psframebox*[fillstyle=solid,fillcolor=mediumgoldenrod]{\labelsize Spherical clot}}
\rput(11.5,10.2){\labelsize Spherical clot trapped in the filter}
\rput( 2.0,4.6){\labelsize Cone shaped clot}
\rput(11.45,4.7){\labelsize Spherical clot trapped near the front}
% 
% \rput(7.5,11.55){\psframebox*[fillstyle=solid,fillcolor=mediumgoldenrod]{\normalss\blue Flow past a blood-clot filter using cgins (Mike Singer, LLNL)}}
% turn on the grid for placement
% \psgrid[subgriddiv=2]
%
\end{pspicture}
\end{center}
\caption{Flow past a blood-clot filter using cgins.}\label{fig:bloodFlow}
% \vskip.5\baselineskip
\end{figure}


}