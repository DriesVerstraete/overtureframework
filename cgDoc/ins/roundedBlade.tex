% ========================================================================================================
\subsection{Flow past a rounded blade}\label{sec:roundedBlade}


{%%%
% 
{
\newcommand{\figWithCaption}[5]{
\begin{scope}[yshift=#1cm]
  \draw ( 0.0,0.0) node[anchor=south west,xshift=-4pt,yshift=+0pt] {\trimfiga{#2}{\figWidtha}};
  \draw ( 8.0,.0) node[anchor=south west,xshift=-4pt,yshift=+0pt] {\trimfiga{#3}{\figWidtha}};
  \draw ( 0.0,0.3 ) node[draw,fill=white,anchor=south west,xshift=+1pt,yshift=-4pt] {\scriptsize #4};
  \draw ( 8.0,0.3 ) node[draw,fill=white,anchor=south west,xshift=+1pt,yshift=-4pt] {\scriptsize #5};
\end{scope}
}% end figWithCaption
\newcommand{\figWidtha}{6.0cm}
\newcommand{\trimfiga}[2]{\trimPlotb{#1}{#2}{.0}{.0}{.05}{.1}}
% 
\newcommand{\figWidthb}{5.5cm}
\newcommand{\trimfigb}[2]{\trimPlotb{#1}{#2}{.05}{.05}{.05}{.1}}
\newcommand{\figWidthc}{6cm}
\newcommand{\trimfigc}[2]{\trimPlotb{#1}{#2}{.025}{.025}{.095}{.095}}
% % -----------------------------------------------------------------------------------------------------------------------------------------
\begin{figure}[hbt]
\begin{center}
\begin{tikzpicture}[scale=1]
  \useasboundingbox (0,.75) rectangle (16.,15.5);  % set the bounding box (so we have less surrounding white space)
%
\draw (0.0,10.5)  node[anchor=south west,xshift=-4pt,yshift=+0pt] {\trimfigc{\cgDoc/ins/fig/roundedBladeGrid}{\figWidthc}};
\draw (8.0,10.5)  node[anchor=south west,xshift=-4pt,yshift=+0pt] {\trimfigb{\cgDoc/ins/fig/roundedBladeGridTip}{\figWidthb}};
%
\figWithCaption{5.25}{\cgDoc/ins/fig/roundedBlade4Enstrophyt7p0}{\cgDoc/ins/fig/roundedBlade4Enstrophyt8p0}{$t=7$}{$t=8$}
\figWithCaption{0}{\cgDoc/ins/fig/roundedBlade4Enstrophyt9p0}{\cgDoc/ins/fig/roundedBlade4Enstrophyt10p0}{$t=9$}{$t=10$}
%
 % \draw (current bounding box.south west) rectangle (current bounding box.north east);
% grid:
%  \draw[step=1cm,gray] (0,0) grid (16,16.);
\end{tikzpicture}
\end{center}
 \caption{Flow past a {\em rounded blade} in channel. Top: Overlapping grid for the blade and channel. 
   Contour plots of the enstrophy on grid $\Gc^{(4)}$ (7M pts), using scheme AFS42. }
  \label{fig:roundedBladeFlow}
\end{figure}
% -----------------------------------------------------------------------------------------------------------------------------------------------
%
}%%%


We simulate the flow past a stationary and moving {\em rounded blade}.

The grid for this problem was generated from the ogen command file {\tt roundedBladeGrid.cmd}.
The solution was computed with the Cgins command file {\tt cg/ins/cmd/roundedBlade.cmd}.

The geometry for the problem, as shown in Figure~\ref{fig:roundedBladeFlow},
consists of flattened cylinder with rounded ends.
Let $\Gc^{(j)}$ denote the composite grid for this geometry. The target grid spacing is $\ds=1/(10 j)$.
The grid spacing is stretched in the normal direction to the box so that the boundary layer
spacing is $\dsbl$. 


The incoming flow is in the $y$-direction with $v=1$.


Figure~\ref{fig:roundedBladeFlow} shows the solution for a non-rotating blade while
Figure~\ref{fig:rotatingRoundedBladeFlow} shows results for a rotating blade (one full rotation takes $4$ seconds).
Contours of the enstrophy $\xi$, (magnitude of the vorticity vector, $\xi=\| \grad\times \uv\|$) are shown.
The solution was computed with the scheme AFS4 and the SSLES4 turbulence model ($\nu=10^{??}$). 



% scp henshaw@hera:/p/lscratchd/henshaw/runs/cgins/roundedBlade/roundedBladeMove4Enstrophyt{7p0,7p25,7p5,7p75,8p0,8p25}.ps .

{%%%
% 
{
\newcommand{\figWithCaption}[5]{
\begin{scope}[yshift=#1cm]
  \draw ( 0.0,0.0) node[anchor=south west,xshift=-4pt,yshift=+0pt] {\trimfiga{#2}{\figWidtha}};
  \draw ( 8.0,.0) node[anchor=south west,xshift=-4pt,yshift=+0pt] {\trimfiga{#3}{\figWidtha}};
  \draw ( 0.0,0.3 ) node[draw,fill=white,anchor=south west,xshift=+1pt,yshift=-4pt] {\scriptsize #4};
  \draw ( 8.0,0.3 ) node[draw,fill=white,anchor=south west,xshift=+1pt,yshift=-4pt] {\scriptsize #5};
\end{scope}
}% end figWithCaption
\newcommand{\figWidtha}{6.0cm}
\newcommand{\trimfiga}[2]{\trimPlotb{#1}{#2}{.0}{.0}{.05}{.1}}
% 
\newcommand{\figWidthb}{5.5cm}
\newcommand{\trimfigb}[2]{\trimPlotb{#1}{#2}{.05}{.05}{.05}{.1}}
\newcommand{\figWidthc}{6cm}
\newcommand{\trimfigc}[2]{\trimPlotb{#1}{#2}{.025}{.025}{.095}{.095}}
% % -----------------------------------------------------------------------------------------------------------------------------------------
\begin{figure}[hbt]
\begin{center}
\begin{tikzpicture}[scale=1]
  \useasboundingbox (0,.75) rectangle (16.,10.5);  % set the bounding box (so we have less surrounding white space)
%
\figWithCaption{5.25}{\cgDoc/ins/fig/roundedBladeMove4Enstrophyt7p5}{\cgDoc/ins/fig/roundedBladeMove4Enstrophyt7p75}{$t=7.5$}{$t=7.75$}
\figWithCaption{0}{\cgDoc/ins/fig/roundedBladeMove4Enstrophyt8p0}{\cgDoc/ins/fig/roundedBladeMove4Enstrophyt8p25}{$t=8.0$}{$t=8.25$}
%
 % \draw (current bounding box.south west) rectangle (current bounding box.north east);
% grid:
%  \draw[step=1cm,gray] (0,0) grid (16,10.);
\end{tikzpicture}
\end{center}
 \caption{Flow past a rotating {\em rounded blade} in channel.
   Contour plots of the enstrophy on grid $\Gc^{(4)}$ (7M pts), using scheme AFS42. }
  \label{fig:rotatingRoundedBladeFlow}
\end{figure}
% -----------------------------------------------------------------------------------------------------------------------------------------------
%
}%%%
