% ========================================================================================================
\clearpage
\subsection{Flow past a moving 3D cylinder at different Reynolds numbers}\label{sec:oscillatingCyl3d}


We simulate the flow past a moving three-dimensional cylinder using the second- and
fourth-order accurate versions of the AFS scheme.  The second-order and
fourth-order SSLES turbulence models are used. The results indicate how the flow
character changes as the effective Reynolds number increases.

The grid for this problem was generated from the ogen command file {\tt cylinderInAChannel.cmd}.
The solution was computed with the Cgins command file {\tt cg/ins/cmd/cyl3d.cmd}.

The geometry for the problem consists of a cylinder of radius $r=0.25$ in a channel $[-1,3]\times[-1,1]\times[-1,1]$. 
The axis of the cylinder is
in the $z$-direction and the center of the cylinder is initially located at $(x,y)=(0,0)$. 
Let $\Gc^{(j)}$ denote the composite grid for this geometry. The target grid spacing is $\ds=1/(10 j)$.
The grid spacing is stretched in the normal direction to the cylinder so that the boundary layer
spacing is $\dsbl$. 


The incoming flow is in the $x$-direction with $u=1$.


Figure~\ref{fig:cyl3d2Oscillate} shows the solution for a cylinder that oscillates up and down
in the y-direction with a sinusoidal motion given by 
\begin{align*}
   y(t) &= a_0 \sin( 2\pi f_0 t) ,
\end{align*}
where the amplitude is taken as $a_0=0.25$ and the frequency is $f_0=0.25$.
The grid is $\Gc^{(2)}$ and the boundary layer spacing is 5 times smaller than the target grid spacing, with $\dsbl/\ds=1/5$.
Contours of the enstrophy $\xi$, (magnitude of the vorticity vector, $\xi=\| \grad\times \uv\|$) are shown.
The solution was computed with the scheme AFS4 and the SSLES4 turbulence model ($\nu=10^{-3}$). 
The results show that the flow develops into a fully three-dimensional wake with vortices being shed.


{%%%
% 
{
\newcommand{\figWithCaption}[5]{
\begin{scope}[yshift=#1cm]
  \draw ( 0.0,0.0) node[anchor=south west,xshift=-4pt,yshift=+0pt] {\trimfiga{#2}{\figWidtha}};
  \draw ( 8.0,.0) node[anchor=south west,xshift=-4pt,yshift=+0pt] {\trimfiga{#3}{\figWidtha}};
  \draw ( 0.0,0.3 ) node[draw,fill=white,anchor=south west,xshift=+1pt,yshift=-4pt] {\scriptsize #4};
  \draw ( 8.0,0.3 ) node[draw,fill=white,anchor=south west,xshift=+1pt,yshift=-4pt] {\scriptsize #5};
\end{scope}
}% end figWithCaption
\newcommand{\figWidtha}{7.5cm}
\newcommand{\trimfiga}[2]{\trimPlotb{#1}{#2}{.05}{.13}{.25}{.25}}
% -----------------------------------------------------------------------------------------------------------------------------------------
\begin{figure}[hbt]
\begin{center}
\begin{tikzpicture}[scale=1]
  \useasboundingbox (0,.75) rectangle (16.,10);  % set the bounding box (so we have less surrounding white space)
%
\figWithCaption{5.25}{\cgDoc/ins/fig/cyl3dMove2Vort3p5}{\cgDoc/ins/fig/cyl3dMove2Vort4p0}{$t=3.5$}{$t=4.0$}
%
\figWithCaption{0}{\cgDoc/ins/fig/cyl3dMove2Vort4p5}{\cgDoc/ins/fig/cyl3dMove2Vort5p0}{$t=4.5$}{$t=5.0$}
%
 % \draw (current bounding box.south west) rectangle (current bounding box.north east);
% grid:
% \draw[step=1cm,gray] (0,0) grid (16,10.);
\end{tikzpicture}
\end{center}
 \caption{Flow past a 3D oscillating cylinder using Scheme AFS4 on grid $\Gc^{(2)}$. Contour plots of the enstrophy times $t=3.5$, $4.0$, $4.5$ and $5.0$.
  The contour levels are scaled to $[0,20]$. The cylinder is moving the $y-direction$. }
  \label{fig:cyl3d2Oscillate}
\end{figure}
% -----------------------------------------------------------------------------------------------------------------------------------------------
%
}%%%
