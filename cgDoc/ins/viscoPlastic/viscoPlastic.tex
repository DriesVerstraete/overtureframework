% \subsection{A Visco-Plastic Flow Model}

A model for visco-plastic flows (with possible temperature and buoyancy effects) is
\begin{align}
  \uv_t + (\uv\cdot\grad)\uv + \grad p - \grad\cdot \sigmav' +\alpha \gv T  -\fv &=0 , \label{eq:vpu} \\
  \Delta p - \Jc(\grad\uv) -\grad\cdot(\grad\cdot \sigmav') 
                -\alpha (\gv\cdot\grad) T - \grad\cdot\fv &=0, \label{eq:vpp} \\
  T_t + (\uv\cdot\grad) T - \kappa \Delta T - f_T &=0 , \\
  \Jc(\grad\uv) \equiv (\grad u\cdot\uv_x + \grad v\cdot\uv_y + \grad w\cdot \uv_z) &~.
\end{align}
Here $T$ represents a temperature perturbation, $\gv$ is the acceleration due to gravity,
$\alpha$ is the coefficient of thermal expansivity and $\kappa$ is a coefficient of
thermal diffusivity, $\kappa=\lambda/(\rho C_p)$. 

% \newcommand{\esr}{\| \overline{ {\dot \ev} } \|}
\newcommand{\esr}{\overline{ {\dot \ev} }}

The quantity $\sigmav'$ is the stress deviator. One model for visco-plastic models with
a yield stress $\sigma_y$ is given by
\begin{align}
   \sigmav' &= 2 \eta~ {\dot \ev}, \\
   \eta &=  \eta_p(T) + {\sigma_y(T) \over \esr }( 1 - \exp(-m\esr) , \label{VP:etaModel} \\
    {\dot \ev}_{ij} &= \half( {\partial u_i\over\partial x_j} + {\partial u_j\over\partial x_i}), \\
    \esr &= \sqrt{ {2\over 3} {\dot \ev}_{ij}{\dot \ev}_{ij} } . 
\end{align}


Here $\eta=\eta(\esr,T)$ is the effective coefficient of viscosity that depends on
the temperature and the effective strain rate, $\esr$. $\eta_p$ is the coefficient
of viscosity in the plastic regime. 


The limiting behaviours of the viscosity coefficient $\eta$ for high and low strain rates are
\begin{alignat*}{3}
\eta &\sim  \eta_p(T), &\qquad& \text{for $\esr \gg \sigma_y$}, \\
\eta &\sim  \eta_p(T) + m \sigma_y(T) , &\qquad& \text{for $\esr \ll \sigma_y$ and $m\esr \ll 1$} . 
\end{alignat*}
The value of $m$ is choosen to be large and positive  % ($m \approx 100-1000$)
so that $\eta$ is a large value when the strain rates are small (corresponding to a nearly solid state
with almost no flow).
When the strain rates become sufficiently large the fluid can flow with an effective
coefficient of viscosity of $\eta_p$. 

{\bf Note:} $m$ has dimensions of inverse time, $[1/T]$. Thus the statement that {\em $m$ is large} depends on how
the problem is non-dimensionalized or not. 
% It might be better to introduce a reference scale for $\esr$, say $\esr_0$
% and instead use $\exp(-m~\esr/\esr_0)$ so that $m$ is non-dimensional (?).

The full implicit time-stepping method and implicit line-solver time-stepping method
discretize $\grad\cdot \sigmav'$ in ~\eqref{eq:vpu} using a conservative approximation, i.e. 
a discrete form of 
\begin{align*}
  (\grad\cdot \sigmav')_{i} &= 
     \partial_{x_j} \big( \eta ( \partial_{x_j} u_i + \partial_{x_i} u_j ) \big) 
\end{align*}


The term $\grad\cdot(\grad\cdot\sigmav')$ in the pressure equation~\eqref{eq:vpp} is not discretized
in conservation form. By expanding $\grad\cdot(\grad\cdot\sigmav')$, the terms involving the highest
derivatives of $\uv$ drop out since $\grad\cdot\uv=0$. This reduces the stencil size when discretizing
this term. 
Note that 
\begin{align*}
  (\grad\cdot \sigmav')_{i} &= 
     \partial_{x_j} \big( \eta ( \partial_{x_j} u_i + \partial_{x_i} u_j ) \big) \\
          &= \eta \partial_{x_j}\partial_{x_j} u_i + 
                  (\partial_{x_j}\eta)(\partial_{x_j} u_i + \partial_{x_i} u_j )
\end{align*}
where the incompressibility condition, $\partial_{x_j} u_j = 0$, has been used.
It also follows that
\begin{align*}
\grad\cdot(\grad\cdot \sigmav') &= 
    \partial_{x_i}\Big( \eta \partial_{x_j}\partial_{x_j} u_i + 
                  (\partial_{x_j}\eta)(\partial_{x_j} u_i + \partial_{x_i} u_j ) \Big) \\
   &= (\partial_{x_i}\eta)\partial_{x_j}\partial_{x_j} u_i +
         (\partial_{x_i}\partial_{x_j}\eta)(\partial_{x_j} u_i + \partial_{x_i} u_j )  +
         (\partial_{x_j}\eta)(\partial_{x_i}\partial_{x_i} u_j ) \\ 
   &= 2 (\partial_{x_i}\eta)\partial_{x_j}\partial_{x_j} u_i +
         (\partial_{x_i}\partial_{x_j}\eta)(\partial_{x_j} u_i + \partial_{x_i} u_j )
\end{align*}

In two dimensions
\begin{align*}
 \esr^2 &=  {2\over 3} \Big( u_x^2 +  \half( u_y + v_x )^2 + v_y^2 \Big)  \\
 (\grad\cdot \sigmav')_1 &= \eta\Delta u 
         + \eta_x(2u_x) + \eta_y(u_y+v_x) \\
 (\grad\cdot \sigmav')_2 &= \eta\Delta v 
         + \eta_x(u_y+v_x) + \eta_y(2v_y) \\
 \grad\cdot(\grad\cdot \sigmav') &= 2\left[ \eta_x\Delta u + \eta_y\Delta v +
     \eta_{xx}u_x+ \eta_{xy}(u_y+v_x) + \eta_{yy}v_y \right]
\end{align*}

In three dimensions
\begin{align*}
 \esr^2 &=  {2\over 3} \Big( u_x^2 +  \half( u_y + v_x )^2 +  \half( u_z + w_x )^2 + v_y^2 +  \half( v_z + w_y )^2 + w_z^2 \Big)  \\
\end{align*}


\subsubsection{\bf Solution algorithms:}

The visco-plastic equations can be solved in different ways:
\begin{enumerate}
  \item {\bf full implicit time-stepping} that treats the full system for $(\uv,T)$ as a coupled (linearized)
        implicit system. The implicit matrix is refactored every some number of steps (user specified).
        The pressure equation is currently solved as a separate implicit system, which seems to work reasonably
        well. 
        For steady state problems, the time-stepping parameters can be chosen so that the 
        approach is basically backward Euler in time.
        This is currently the suggested way to solve the equations. 
  \item {\bf implicit line-solver} that solves scalar tridiagonal systems (still under-development). This could be
        a more memory efficient solver than the full implicit method. 
  \item {\bf explicit time stepping} : this requires a very small time step. 
\end{enumerate}

\subsubsection{\bf NOTES for developers:}

\begin{enumerate}
  \item The definition of the visco-plastic viscosity $\eta$ is defined in viscoPlastic.C (using viscoPlasticMacrosCpp.h).
        This is the only file where the exact form for $\eta$ (equation~eqref{VP:etaModel}) is used. 
  \item addForcing.C : defines the twilight-zone (manufactured solution) 
        forcing to the momentum equations.
  \item insImpVP.f (generated from insImpVP.bf and insImp.h) : defines the VP implicit equations for the ``full''-implicit method, along with the
         right-hand-side and residual.
  \item addForcingToPressureEquation in pressureEquation.C defines the TZ forcing for the pressure in the VP model. 
  \item inspf.bf : defines the forcing for the pressure equation. Called from assignPressureRHS in pressureEquation.C
  \item getDerivedFunction in InsParameters.C defines the extra VP variables that we plot: $\esr$
    and the yield-surface.
  \item lineSolveVP.h defines the VP equations for the line solver. This file is included
        in insLineSolveNew.bf.
  \item Support for moving grids has been mostly added for the full implicit method, but a few things remain.
\end{enumerate}




% Currently, the approximation in insdt.f (which was done earlier)
% approximates this term using a non-conservative approximation
% to the expanded form:
% \begin{align*}
%   (\grad\cdot \sigmav')_{i} &= 
%            \eta \partial_{x_j}\partial_{x_j} u_i + 
%                   (\partial_{x_j}\eta)(\partial_{x_j} u_i + \partial_{x_i} u_j )
% \end{align*}
