%-----------------------------------------------------------------------
% vp: notes on a visco plastic model
%-----------------------------------------------------------------------
\documentclass[11pt]{article}
\usepackage{times}  % for embeddable fonts, Also use: dvips -P pdf -G0

\input documentationPageSize.tex

\input homeHenshaw

\newcommand{\url}[1]{}

\input{pstricks}\input{pst-node}
\input{colours}

\usepackage{amsmath}
\usepackage{amssymb}

\usepackage{verbatim}
\usepackage{moreverb}

\usepackage{graphics}    
\usepackage{epsfig}    
\usepackage{calc}
\usepackage{ifthen}
\usepackage{float}
% the next one cause the table of contents to disappear!
% * \usepackage{fancybox}

\usepackage{makeidx} % index
\makeindex
\newcommand{\Index}[1]{#1\index{#1}}



% ---- we have lemmas and theorems in this paper ----
\newtheorem{assumption}{Assumption}
\newtheorem{definition}{Definition}

\newcommand{\Overture}{{\bf Over\-ture\ }}
\newcommand{\ogenDir}{\homeHenshaw/overtureFramework/Overture/ogen}

\newcommand{\cgDoc}{\homeHenshaw/overtureFramework/cgDoc}
\newcommand{\vpDir}{\homeHenshaw/overtureFramework/cgDoc/ins/viscoPlastic}

\newcommand{\obFigures}{\homeHenshaw/res/OverBlown/docFigures}  % for figures
\newcommand{\convDir}{.}

\begin{document}

\input wdhDefinitions.tex

\def\comma  {~~~,~~}
\newcommand{\uvd}{\mathbf{U}}
\def\ud     {{    U}}
\def\pd     {{    P}}
\def\calo{{\cal O}}

\newcommand{\mbar}{\bar{m}}
\newcommand{\Rbar}{\bar{R}}
\newcommand{\Ru}{R_u}         % universal gas constant
% \newcommand{\Iv}{{\bf I}}
% \newcommand{\qv}{{\bf q}}
\newcommand{\Div}{\grad\cdot}
\newcommand{\tauv}{\boldsymbol{\tau}}
\newcommand{\thetav}{\boldsymbol{\theta}}
% \newcommand{\omegav}{\mathbf{\omega}}
% \newcommand{\Omegav}{\mathbf{\Omega}}

\newcommand{\Omegav}{\boldsymbol{\Omega}}
\newcommand{\omegav}{\boldsymbol{\omega}}
\newcommand{\sigmav}{\boldsymbol{\sigma}}
\newcommand{\cm}{{\rm cm}}
\newcommand{\Jc}{{\mathcal J}}

\newcommand{\sumi}{\sum_{i=1}^n}
% \newcommand{\half}{{1\over2}}
\newcommand{\dt}{{\Delta t}}

\def\ff {\tt} % font for fortran variables

% define the clipFig commands:
\input clipFig.tex

\newcommand{\figWidth}{}
\newcommand{\clipfig}{}

\newcommand{\bogus}[1]{}  % removes is argument completely

\vspace{5\baselineskip}
\begin{flushleft}
{\Large
A Visco-Plastic Flow Model in Cgins \\
}
\vspace{2\baselineskip}
William D. Henshaw,\\
Department of Mathematical Sciences, \\
Rensselaer Polytechnic Institute, \\
Troy, NY, USA, 12180.
%- \vspace{2\baselineskip}
%- William D. Henshaw  \\
%- Centre for Applied Scientific Computing  \\
%- Lawrence Livermore National Laboratory      \\
%- Livermore, CA, 94551.  \\
%- henshaw@llnl.gov \\
% http://www.llnl.gov/casc/people/henshaw \\
% http://www.llnl.gov/casc/Overture\\
\vspace{\baselineskip}
\today\\
\vspace{\baselineskip}
% UCRL-MA-134289

\vspace{4\baselineskip}

\noindent{\bf\large Abstract:}

This document describes the visco-plastic flow model that is available with 
{\bf Cgins}, a solver written using the \Overture framework
to solve the incompressible Navier-Stokes (INS).  

\end{flushleft}

\tableofcontents
% \listoffigures

\vfill\eject


\section{Introduction}

This document describes the visco-plastic model that is available with the {\bf Cgins} solver~\cite{CginsUserGuide}.
The model described here is taken from the Ph.D. thesis of Graeme Thorn~\cite{Thorn2004}.

\input \vpDir/viscoPlastic.tex

\input vpDiscretization

\input vpMoving

\input vpAxisymmetric

\input \vpDir/viscoPlasticResults.tex


% -------------------------------------------------------------------------------------------------
\vfill\eject
\bibliography{\homeHenshaw/papers/common/journalISI,\homeHenshaw/papers/common/henshaw,\homeHenshaw/papers/common/henshawPapers}
\bibliographystyle{siam}

\end{document}
