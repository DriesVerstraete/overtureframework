\subsection{Two falling drops in an incompressible flow}\label{sec:fallingDrops}

Figure (\ref{fig:twoDrop}) shows two rigid bodies (``drops'') failing under the influence
of gravity in an incompressible flow. 
% 
{
% \newcommand{\dropsDir}{\ovFigures}
\begin{figure}[H]
% \psset{xunit=1.cm,yunit=1.cm,runit=1.cm}%
\newcommand{\figWidthd}{3cm}% 
%% \newcommand{\clipfigd}[2]{\clipFig{#1}{#2}{.35}{.65}{0.0}{.95}}
\newcommand{\trimfig}[2]{\trimPlot{#1}{#2}{.35}{.35}{.0}{.08}}
% 
\begin{center}%
\begin{tikzpicture}[scale=1]
  \useasboundingbox (0,.5) rectangle (16.,9.5);  % set the bounding box (so we have less surrounding white space)
% \begin{pspicture}(0,0)(15.,8)%
 % \psgrid[subgriddiv=2]
%  \rput( 1.5, 4.0){\clipfigd{\dropsDir/twoDropVor1p0.ps}{\figWidthd}}
%  \rput( 4.5, 4.0){\clipfigd{\dropsDir/twoDropVor2p0.ps}{\figWidthd}}
%  \rput( 7.5, 4.0){\clipfigd{\dropsDir/twoDropVor3p0.ps}{\figWidthd}}
%  \rput(10.5, 4.0){\clipfigd{\dropsDir/twoDropVor4p0.ps}{\figWidthd}}
%  \rput(13.5, 4.0){\clipfigd{\dropsDir/twoDropVor5p0.ps}{\figWidthd}}
% 
 \draw ( 0.0, 0.0) node[anchor=south west,xshift=-4pt,yshift=+0pt] {\trimfig{./fig/twoDropVor1p0}{\figWidthd}};
 \draw ( 3.2, 0.0) node[anchor=south west,xshift=-4pt,yshift=+0pt] {\trimfig{./fig/twoDropVor2p0}{\figWidthd}};
 \draw ( 6.4, 0.0) node[anchor=south west,xshift=-4pt,yshift=+0pt] {\trimfig{./fig/twoDropVor3p0}{\figWidthd}};
 \draw ( 9.6, 0.0) node[anchor=south west,xshift=-4pt,yshift=+0pt] {\trimfig{./fig/twoDropVor4p0}{\figWidthd}};
 \draw (12.8, 0.0) node[anchor=south west,xshift=-4pt,yshift=+0pt] {\trimfig{./fig/twoDropVor5p0}{\figWidthd}};
% \draw[step=1cm,gray] (0,0) grid (16,9);
\end{tikzpicture}
\end{center}%
\caption{Two drops falling in an incompressible flow, contour plots of the vorticity. The upper drop wants to ``draft'' in behind the
lower drop where the pressure is lower.}
\label{fig:twoDrop}
\end{figure}
}

This computation used the command file {\tt cg/ins/cmd/twoDrop.cmd}. 
The grid can be created with
{\tt Overture/sampleGrids/twoDrop.cmd} (finer grids with {\tt Overture/sampleGrids/twoDropArg.cmd}). The initial conditions for
the drops include their initial position, velocity, and angular velocity. The
mass and moments of inertia must be specified for each drop. There can be problems
for the grid generator if the drops get too close together since there will not
be enough grid points in the gap between the drops. To avoid this problem 
there is an option ``detect collisions''
that has been turned on that will detect when the drops get close and perform an elastic
collision. This collision detection currently only works for circular drops. 