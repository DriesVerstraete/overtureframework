%=======================================================================================================
% Moving Bodies in Overture and CG
%=======================================================================================================

% -- article: 
\documentclass[11pt]{article}
% \usepackage[bookmarks=true]{hyperref}
\usepackage[bookmarks=true,colorlinks=true,linkcolor=blue]{hyperref}

% \input documentationPageSize.tex
\hbadness=10000 
\sloppy \hfuzz=30pt

\usepackage{calc}
\usepackage[lmargin=1.in,rmargin=1.in,tmargin=1.in,bmargin=1.in]{geometry}

\input homeHenshaw


\input wdhDefinitions.tex

% \input{pstricks}\input{pst-node}
% \input{colours}

\newcommand{\bogus}[1]{}  % begin a section that will not be printed

\usepackage{color}
\usepackage{amsmath}
\usepackage{amssymb}

\usepackage{graphicx}

% --------------------------------------------
% NOTE: trouble with tikz and program package ??
\usepackage{tikz}

\input trimFig.tex


% \usepackage{verbatim}
% \usepackage{moreverb}
% \usepackage{graphics}    
% \usepackage{epsfig}    
% \usepackage{calc}
% \usepackage{ifthen}
% \usepackage{float}
% \usepackage{fancybox}

% define the clipFig commands:
% \input clipFig.tex

\newcommand{\obDir}{\homeHenshaw/res/OverBlown}
\newcommand{\ogenDir}{\homeHenshaw/Overture/ogen}

% \newcommand{\grad}{\nabla}
% \newcommand{\half}{\frac{1}{2}}}

\newcommand{\eps}{\epsilon}
\newcommand{\zerov}{\mathbf{0}}

\newcommand{\bc}[1]{\mbox{\bf#1}}   % bold name
\newcommand{\cc}[1]{\mbox{  : #1}}  % comment

\newcommand{\Overture}{{Overture}}

\newcommand{\Largebf}{\sffamily\bfseries\Large}
\newcommand{\largebf}{\sffamily\bfseries\large}
\newcommand{\largess}{\sffamily\large}
\newcommand{\Largess}{\sffamily\Large}
\newcommand{\bfss}{\sffamily\bfseries}
\newcommand{\smallss}{\sffamily\small}
\newcommand{\normalss}{\sffamily}
\newcommand{\scriptsizess}{\sffamily\scriptsize}

\newcommand{\Div}{\grad\cdot}
\newcommand{\tauv}{\boldsymbol{\tau}}
\newcommand{\thetav}{\boldsymbol{\theta}}

\newcommand{\Omegav}{\boldsymbol{\Omega}}
\newcommand{\omegav}{\boldsymbol{\omega}}
\newcommand{\cm}{{\rm cm}}

\newcommand{\sumi}{\sum_{i=1}^n}
% \newcommand{\half}{{1\over2}}
\newcommand{\dt}{{\Delta t}}


\newcommand{\Gc}{\mathcal{G}}
\newcommand{\Fc}{\mathcal{F}}
\newcommand{\sgn}{\operatorname{sgn}}

\renewcommand{\url}[1]{Available from www.OvertureFramework.org}


% \psset{xunit=1.cm,yunit=1.cm,runit=1.cm}


% *** See http://www.eng.cam.ac.uk/help/tpl/textprocessing/squeeze.html
% By default, LaTeX doesn't like to fill more than 0.7 of a text page with tables and graphics, nor does it like too many figures per page. This behaviour can be changed by placing lines like the following before \begin{document}

\renewcommand\floatpagefraction{.9}
\renewcommand\topfraction{.9}
\renewcommand\bottomfraction{.9}
\renewcommand\textfraction{.1}   
\setcounter{totalnumber}{50}
\setcounter{topnumber}{50}
\setcounter{bottomnumber}{50}


\begin{document}

\vglue 5\baselineskip
\begin{flushleft}
{\Large Moving Bodies with Overture and CG} \\
\vspace{2\baselineskip}
William D. Henshaw\footnote{This work was performed under the auspices of the U.S. Department of Energy (DOE) by
Lawrence Livermore National Laboratory under Contract DE-AC52-07NA27344 and by 
DOE contracts from the ASCR Applied Math Program.}  \\
Centre for Applied Scientific Computing  \\
Lawrence Livermore National Laboratory      \\
Livermore, CA, 94551.  \\
% henshaw@llnl.gov \\
% http://www.llnl.gov/casc/people/henshaw \\
% http://www.llnl.gov/casc/Overture\\
\vspace{\baselineskip}
\today\\
\vspace{\baselineskip}
LLNL-SM-455792

\vspace{4\baselineskip}

\noindent{\bf Abstract:} This article provides background and documentation for
the use of moving bodies with Overture and the CG suite of partial differential
equation solvers. The topics covered include
\begin{description}
  \item[rigid body motion] : a description of the equations governing the motion of rigid bodies (i.e. bodies that move
     under the influence of external forces such as fluid forces and gravity) and documentation
      for the {\tt RigidBodyMotion} class.
  \item[matrix motion] : a description of the {\tt MatrixMotion} and {\tt TimeFunction} classes that can be used to
         define complex specified motions by composing together elemenray motions such as rotations and translations. For example
         one can define the motion of an airfoil that pitches (i.e. rotates) and plunges (i.e. translates
       up and down). 
  \item[light bodies] : a discussion of issues related to the coupling of fluid motion with ``light bodies''. 
  \item[deforming bodies] : a description of the deforming body equations.
\end{description}

\end{flushleft}

% ----------------------------------------------------
\clearpage
\tableofcontents


\section{Introduction}\label{sec:intro}

  This article provides background and documentation for the use of moving bodies with Overture and the CG suite
of partial differential equation solvers. The topics covered include
\begin{description}
  \item[rigid body motion] : a description of the equations governing the motion of rigid bodies (i.e. bodies that move
     under the influence of external forces such as fluid forces and gravity) and documentation
      for the {\tt RigidBodyMotion} class.
  \item[matrix motion] : a description of the {\tt MatrixMotion} and {\tt TimeFunction} classes that can be used to
         define complex specified motions by composing together elemenray motions such as rotations and translations. For example
         one can define the motion of an airfoil that pitches (i.e. rotates) and plunges (i.e. translates
       up and down). 
  \item[light bodies] : a discussion of issues related to the coupling of fluid motion with ``light bodies''. 
  \item[deforming bodies] : a description of the deforming body equations.
\end{description}

Other documents of interest are the Cgins User Guide~\cite{CginsUserGuide}, the Cgins Reference Manual~\cite{CginsReferenceManual},
as well as~\cite{mog2006}. 



% --------------  rigid body motion -----------------
\clearpage
\input RigidBodyMotion


% -------------- matrix motion -----------------
\clearpage
\input matrixMotion

% -------------- light bodies -----------------
\clearpage
\input lightBodies

% -------------- deforming bodies -----------------
\clearpage
\input deformingBodies

%\clearpage
\bibliography{\homeHenshaw/papers/henshaw}
\bibliographystyle{plain}
\end{document}
