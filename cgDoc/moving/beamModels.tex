%=======================================================================================================
% Beam Models for Overture and cg 
%=======================================================================================================

% -- article: 
\documentclass[11pt]{article}
% \usepackage[bookmarks=true]{hyperref}
\usepackage[bookmarks=true,colorlinks=true,linkcolor=blue]{hyperref}

% \input documentationPageSize.tex
\hbadness=10000 
\sloppy \hfuzz=30pt

\usepackage{calc}
\usepackage[lmargin=1.in,rmargin=1.in,tmargin=1.in,bmargin=1.in]{geometry}

\input homeHenshaw


\input wdhDefinitions.tex

\newcommand{\tableFont}{\footnotesize}
\newcommand{\num}[2]{#1e#2} % Use this macro to define the format of the numbers in the table
\newcommand{\eem}{e^{(j)}}
\newcommand{\rateLabel}{rate}



% \input{pstricks}\input{pst-node}
% \input{colours}

\newcommand{\bogus}[1]{}  % begin a section that will not be printed

\usepackage{color}
\usepackage{amsmath}
\usepackage{amssymb}

\usepackage{graphicx}

% --------------------------------------------
% NOTE: trouble with tikz and program package ??
\usepackage{tikz}

\input trimFig.tex


% \usepackage{verbatim}
% \usepackage{moreverb}
% \usepackage{graphics}    
% \usepackage{epsfig}    
% \usepackage{calc}
% \usepackage{ifthen}
% \usepackage{float}
% \usepackage{fancybox}

% define the clipFig commands:
% \input clipFig.tex

\newcommand{\obDir}{\homeHenshaw/res/OverBlown}
\newcommand{\ogenDir}{\homeHenshaw/Overture/ogen}

% \newcommand{\grad}{\nabla}
% \newcommand{\half}{\frac{1}{2}}}

\newcommand{\eps}{\epsilon}
\newcommand{\zerov}{\mathbf{0}}

\newcommand{\bc}[1]{\mbox{\bf#1}}   % bold name
\newcommand{\cc}[1]{\mbox{  : #1}}  % comment

\newcommand{\Overture}{{Overture}}

\newcommand{\Largebf}{\sffamily\bfseries\Large}
\newcommand{\largebf}{\sffamily\bfseries\large}
\newcommand{\largess}{\sffamily\large}
\newcommand{\Largess}{\sffamily\Large}
\newcommand{\bfss}{\sffamily\bfseries}
\newcommand{\smallss}{\sffamily\small}
\newcommand{\normalss}{\sffamily}
\newcommand{\scriptsizess}{\sffamily\scriptsize}

\newcommand{\Div}{\grad\cdot}
\newcommand{\tauv}{\boldsymbol{\tau}}
\newcommand{\thetav}{\boldsymbol{\theta}}

\newcommand{\Omegav}{\boldsymbol{\Omega}}
\newcommand{\omegav}{\boldsymbol{\omega}}
\newcommand{\cm}{{\rm cm}}

\newcommand{\sumi}{\sum_{i=1}^n}
% \newcommand{\half}{{1\over2}}
\newcommand{\dt}{{\Delta t}}


\newcommand{\Gc}{\mathcal{G}}
\newcommand{\Fc}{\mathcal{F}}
\newcommand{\sgn}{\operatorname{sgn}}

\renewcommand{\url}[1]{Available from www.OvertureFramework.org}


% \psset{xunit=1.cm,yunit=1.cm,runit=1.cm}


% *** See http://www.eng.cam.ac.uk/help/tpl/textprocessing/squeeze.html
% By default, LaTeX doesn't like to fill more than 0.7 of a text page with tables and graphics, nor does it like too many figures per page. This behaviour can be changed by placing lines like the following before \begin{document}

\renewcommand\floatpagefraction{.9}
\renewcommand\topfraction{.9}
\renewcommand\bottomfraction{.9}
\renewcommand\textfraction{.1}   
\setcounter{totalnumber}{50}
\setcounter{topnumber}{50}
\setcounter{bottomnumber}{50}


\begin{document}

\vglue 5\baselineskip
\begin{flushleft}
{\LARGE Beam Models for Overture and CG} \\
\vspace{2\baselineskip}
William D. Henshaw, \\
Department of Mathematical Sciences, \\
Rensselaer Polytechnic Institute,     \\
Troy, NY, 12180.  \\
www.overtureFramework.org \\
\vspace{2\baselineskip}
\today
\vspace{4\baselineskip}
% 
\end{flushleft}
%
%
\noindent{\bf Abstract:} This article provides background and documentation for
the beam models that have been developed for use with Overture and the CG suite of partial differential
equation solvers. The topics covered include
\begin{description}
  \item[BeamModel] : a class that implements an Euler-Bernoulli beam.
  \item[NonLinearBeamModel] : a non-linear beam model.
\end{description}


% ----------------------------------------------------
\clearpage
\tableofcontents
\section{Acknowledgements}

Thanks to Dr. Alex Main for developing the first version of the BeamModel and NonLinearBeamModel classes while
a summer student at LLNL in 2013.

% --------------------------------------------------------------------------------------------------------------------
% --------------------------------------------------------------------------------------------------------------------
% --------------------------------------------------------------------------------------------------------------------
\section{Class BeamModel}\label{sec:BeamModel}

The BeamModel class defines a generalized Euler-Bernoulli (EB) beam,

\newcommand{\rhos}{\bar{\rho}}
\newcommand{\hs}{\bar{h}}
\newcommand{\bs}{\bar{b}}
\newcommand{\As}{\bar{A}}
\newcommand{\Ts}{\bar{T}}
\newcommand{\Ls}{\bar{L}}
\begin{align}
  \rhos \As \frac{\partial^2 w}{\partial t^2} = 
                   \frac{\partial}{\partial x}\left( T \frac{\partial w}{\partial x}\right)
                   - \frac{\partial^2}{\partial x^2}\left( E I \frac{\partial^2 w}{\partial x^2}\right)
                   + f(x,t), \qquad x\in[0,\Ls], \quad t\ge 0,
\end{align}
where $y=w(x,t)$ is the displacement of the beam, 
$\As$ is the cross-sectional area of the beam ($\As=\hs\bs$ for a beam of constant thickness, $\hs$ and constant breadth $\bs$ in the z-direction), 
$\rhos$ is the beam density, $\Ls$ is the length of the beam, 
$E$ is Young's modulus, $I$ is the area-moment of inertial (e.g. $I=\frac{1}{12} h^3 b$ for a rectangular beam),
and $T$ is the {\em tension} coefficient. 
% 
The EB beam is generally consider valid for thin beams and small slopes,  $|w_x| \ll 1$.


A variational form of the problem is found by multiplying by $v$ and integrating over the domain with inner product,
\begin{align}
   (v,w)_B \equiv \int_0^{\Ls} v \, w \, dx. 
\end{align}
After integration by parts the result is 
\begin{align}
 \left( v,  \rhos \As \frac{\partial^2 w}{\partial t^2} \right)_B  &= 
                  - \left( \frac{\partial v}{\partial x} , T \frac{\partial w}{\partial x} \right)_B
                  -\left(\frac{\partial^2 v}{\partial x^2}, E I \frac{\partial^2 w}{\partial x^2}\right)_B  + ( v, f(x,t))_B \\
% boundary terms
             &\quad  +   \left[ v T \frac{\partial w}{\partial x} ~ -v \frac{\partial}{\partial x}\Big( E I \frac{\partial^2 w}{\partial x^2}\Big) 
                      ~ + \frac{\partial v}{\partial x}E I \frac{\partial^2 w}{\partial x^2}\Big)  
                      \right]_0^{\Ls}   \label{eq:beamBoundaryTerms}
\end{align}

The boundary terms in~\eqref{eq:beamBoundaryTerms} indicate some valid combinations of boundary conditions that will given an energy estimate
when $v=w$ (e.g. if all the boundary terms vanish when the forcing functions $g_m$, $h_m$, $M$ and $S$, given below, are all zero). 
For example, the following three options are common boundary conditions (BC's), 
\begin{align}
   \text{clamped BC:}\quad & \quad w(x_m,t)=g_m(t),\quad  w_x(x_m,t)=h_m(t),  \\
   \text{pinned BC:} \quad & \quad w(x_m,t)=g_m(t), \quad E I w_{xx}(x_m,t)=M(t), \\
   \text{free BC:}   \quad & \quad  E I w_{xx}(x_m,t)=M(t), \quad \partial_x( E I w_{xx})(x_m,t)=S(t),
\end{align}
where $x_0=0$ and $x_1=\Ls$ for $m=0,1$, and 
where $M(t)$ is the imposed moment and $S(t)$ the imposed shear force. 
We also see there is a possible fourth choice (not so common ?) which we call a {\em slope} BC, since
the end slope and shear are specified, 
\begin{align}
   \text{slope BC:}\quad & \quad w_x(x_m,t)=g_m(t),\quad  \partial_x( E I w_{xx})(x_m,t)=S(t). 
\end{align}


% --------------------------------------------------------------------------------------------------------------------
\newcommand{\Ne}{N_e}% number of elements 
\newcommand{\Nn}{N_n}% number of nodes
\newcommand{\dx}{\Delta x}
\newcommand{\Mt}{\tilde{M}}
\newcommand{\Kt}{\tilde{K}}
\newcommand{\Bc}{\mathcal{B}}
\subsection{FEM approximation}\label{sec:BeamModelFEM}

We define a FEM approximation for the EB-beam using Hermite polynomials and the Newmark-beta time stepping scheme.
Let $\Ne$ denote the number of elements and $\Nn=\Ne+1$ the number of nodes. 

The FEM representation is taken as 
\begin{align}
  w^h(x,t) = \sum_{j=1}^{\Nn} w_j(t) \phi_j(x) + w_j'(t) \psi_j(x)  = \sum_{j=1}^{\Ne} [ \phi_j(x)\, \psi_j(x)] \begin{bmatrix} w_j(t)\\w_j'(t)\end{bmatrix}
\end{align}
where the degrees of freedom are nodal displacement $w_j(t)$ and the nodal slope $w'_j(t)$, and where $\phi_j(x)$ and $\psi_j(x)$ are 
the cubic Hermite polynomials 
that are non-zero only on $[x_{j-1},x_{j+1}]$,  and
that satisfy,
\begin{align}
  \phi_j(x_i)=\delta_{ij}, ~ \phi_i'(x_j)=0, \\
  \psi_j(x_i)=0, ~ \psi_i'(x_j)=\delta_{ij}.
\end{align}
% 
The Galerkin FEM approximation is defined by the $2\Nn$ equations
\begin{align}
 \left( \chi_i,  \rhos \As \frac{\partial^2 w^h}{\partial t^2} \right)_B  &= 
                  - \left( \frac{\partial v^h_i}{\partial x} , T \frac{\partial w^h}{\partial x} \right)_B
                  -\left(\frac{\partial^2 \chi_i}{\partial x^2}, E I \frac{\partial^2 w^h}{\partial x^2}\right)_B  + ( \chi_i, f(x,t))_B \\
% boundary terms
             &\quad  +   \left[ \chi_i T \frac{\partial w^h}{\partial x} ~ -\chi_i \frac{\partial}{\partial x}\Big( E I \frac{\partial^2 w^h}{\partial x^2}\Big) 
                      ~ + \frac{\partial \chi_i}{\partial x}E I \frac{\partial^2 w^h}{\partial x^2}\Big)  
                      \right]_0^{\Ls} ,  \label{eq:beamFEM_I}
\end{align}
with $\chi_i=\phi_i$ or $\chi_i=\psi$, $i=1,2,\ldots,\Nn$. Written out,
\begin{align}
\sum_{j=1}^{\Nn} \left\{ M_{ij} \partial_t^2 w_j 
                + \Mt_{ij} \partial_t^2 w'_j \right\}  &= 
     - \sum_{j=1}^{\Nn} \left\{  K_{ij} w_j + \Kt_{ij} w'_j \right\} 
                  ~+ ( \phi_i, f(x,t))_B \\
% boundary terms
             &\quad  +   \left[ \chi_i T \frac{\partial w^h}{\partial x} ~ - \chi_i \frac{\partial}{\partial x}\Big( E I \frac{\partial^2 w^h}{\partial x^2}\Big) 
                      ~ + \frac{\partial \chi_i}{\partial x}E I \frac{\partial^2 w^h}{\partial x^2}\Big)  
                      \right]_0^{\Ls}  ,  \label{eq:beamFEM_II}
\end{align}
where 
\begin{align}
  M_{ij} &= \left( \chi_i,  \rhos \As \frac{\partial^2 \phi_j}{\partial t^2} \right)_B , \\
  \Mt_{ij} &= \left( \chi_i,  \rhos \As \frac{\partial^2 \psi_j}{\partial t^2} \right)_B , \\
  K_{ij} &= \left( \frac{\partial \chi_i}{\partial x} , T \frac{\partial \phi_j}{\partial x} \right)_B
               + \left(\frac{\partial^2 \chi_i}{\partial x^2}, E I \frac{\partial^2 \phi_j}{\partial x^2}\right)_B \\
  \Kt{ij} &= \left( \frac{\partial \chi_i}{\partial x} , T \frac{\partial \phi_j}{\partial x} \right)_B
                + \left(\frac{\partial^2 \chi_i}{\partial x^2}, E I \frac{\partial^2 \psi_j}{\partial x^2}\right)_B 
\end{align}
with $\chi_i = \phi_i$ or $\chi_i=\psi_i$.
This can be written as the matrix differential equation
\begin{align}
   M \ddot\uv = - K \uv + \fv , \label{eq:beamODE}
\end{align}
where $\uv$ is the vector of the degrees of freedom and 
$M$ and $K$ are $2\times2$-block tridiagonal matrices , 
\begin{align}
   \uv = \begin{bmatrix} w_1 \\ w'_1 \\ w_2 \\ w'_2 \\ \vdots \\ w_{\Nn} \\ w'_{\Nn} \end{bmatrix} , 
\quad 
   M = \begin{bmatrix} 
          B^M_1 & C^M_1 &     &      &     &     \\ 
          A^M_2 & B^M_2 & C^M_2 &      &     &     \\
              & A^M_3 & B^M_3 & C^M_3  &     &     \\
              &     & \ddots & \ddots  & \ddots  & \\
              &     &     &     & A^M_{\Nn} & B^M_{\Nn}
           \end{bmatrix} 
\quad               
   K = \begin{bmatrix} 
          B^K_1 & C^K_1 &     &      &     &     \\ 
          A^K_2 & B^K_2 & C^K_2 &      &     &     \\
              & A^K_3 & B^K_3 & C^K_3  &     &     \\
              &     & \ddots & \ddots  & \ddots  & \\
              &     &     &     & A^K_{\Nn} & B^K_{\Nn}
           \end{bmatrix} 
\end{align}
The matrices $A^M_i$, $B^M_i$, $C^M_i$,  $A^K_i$, $B^K_i$, and $C^K_i$ are two-by-two matricies, $\Real^{2\times2}$.

{\bf Note:} The FEM approximation using Hermite polynomials also solves for $\ddot w'$ and thus is effectively
an Hermite scheme (i.e. solving the PDE and the x-derivative of the PDE).

% -----------------------------------------------------------------------------------------
\subsubsection{Element matrices}

The global mass matrix, $M$, and stiffness matrix $K$, can be computed by summing the contributions
from the element mass matrices and stiffness matrices, respectively. The entries in the global matrices are
integrals over the entire domain while the element matrices are just integrals of the same quantities but over a single element.

The solution $w$ on an element $\Omega_i=[x_i,x_{i+1}]$ can be written as
\begin{align}
&  w(x(\xi)) =  \wv^T \Nv(\xi), \qquad \xi\in[-1,1], \\
&  \wv_i = \begin{bmatrix} 
               w_i \\  w'_i \\ w_{i+1} \\ w'_{i+1}
              \end{bmatrix},
\qquad 
\Nv(\xi) = \begin{bmatrix} 
               \frac{1}{4} (1-\xi)^2 (2+\xi) \\
               \frac{\dx}{8} (1-\xi)^2 (1+\xi) \\
               \frac{1}{4} (1+\xi)^2 (2-\xi) \\
               \frac{\dx}{8} (1+\xi)^2 (\xi-1) 
              \end{bmatrix}
  ~= \begin{bmatrix} \phi(\xi) \\ \psi(\xi) \\ \phi(-\xi) \\ -\psi(-\xi) \end{bmatrix}, \\
&  x(\xi) = x_i (1-\xi)/2 + x_{i+1}( \xi+1)/2 = x_i + \half(\xi+1)\dx , 
\end{align} 
where $\dx=x_{i+1}-x_i$ is the element length. 
% 
The $4\times 4$ element mass matrix is symmetric and given by (assuming $\rhos\As$ is constant)
\begin{align}
   M_e = \int_0^{\dx} \rhos \As \Nv \Nv^T \, dx = 
% 
\begin{bmatrix}
13\dx/35 & 11\dx^2/210 & 9\dx/70      & -13\dx^2/420 \\
         & \dx^3/105   & 13\dx^2/420  & -\dx^3/420   \\
         &             & 13\dx/35     & -11\dx^2/210 \\
         &             &              & \dx^3/105
\end{bmatrix}.
%
\end{align} 
The element stiffness matrix is also symmetric and given by  (assuming $EI$ is constant)
\begin{align}
   K_e = \int_0^{\dx} EI \, \partial_x^2\Nv \, \partial_x^2\Nv^T \, dx = 
\begin{bmatrix}
12/\dx^3 & 6/\dx^2 & -12/\dx^3 & 6/\dx^2 \\
         & 4/\dx   & -6/\dx^2  & 2/\dx   \\
         &         & 12/\dx^3  &-6/\dx^2 \\
         &         &           &4/\dx
\end{bmatrix}.
\end{align} 
% 
The element force vector is 
\begin{align}
 \fv_e = \int_0^{\dx} f(x,t) \Nv dx
\end{align} 

% --------------------------------------------------------------------------------------------------------------------
\subsection{Force computation} \label{eq:BeamModel_ForceComputation}

The integrals defining the contribution of the external force are
\begin{align}
& F_i^\phi  = ( \phi_i, f(x,t) )_B  = \int_0^{\Ls} \phi_i(x) f(x,t) \, dx, \\
& F_i^\psi =  ( \psi_i, f(x,t) )_B  = \int_0^{\Ls} \psi_i(x) f(x,t) \, dx.
\end{align}
or in element form
\begin{align}
 \fv_e = \int_0^{\dx} f(x(\xi),t) \Nv(\xi) (\half \dx) \, d\xi
\end{align} 
These can be approximated to second-order accuracy by representing $f(x,t)$ as a linear
function of $x$ over the element $\Omega_i$,
\begin{align}
   f(x,t) = f_i\frac{x_{i+1}-x}{\dx}  + f_{i+1} \frac{x-x_i}{\dx} ,
\end{align}
and performing the resulting integrals exactly. 

For an FSI computation the external force (e.g. pressure) is defined on a different grid. In this case we need
to sum the contributions from partial segments of the element,
\begin{align}
 \fv_e(a,b) = \int_a^{b} f(x(\xi),t) \Nv(\xi) (\half \dx) \, d\xi, \\
   f(x(\xi),t) = f_a\frac{b-\xi}{b-a}  + f_b \frac{\xi-a}{b-a} .
\end{align} 
Here $-1\le a \le b \le 1$. 

\paragraph{More accurate force computation.} If we had approximations to $f$ and $\partial_x f$ at the nodes we
could represent $f$ as an Hermite interpolant,
\begin{align}
&  f(x(\xi)) =  \fv^T \Nv(\xi) = \Nv^T\fv , \\
&  \fv_i = \begin{bmatrix} 
 f_i \\  f'_i \\ f_{i+1} \\ f'_{i+1}
              \end{bmatrix},
\end{align} 
In this case 
\begin{align}
 \fv_e &= \int_0^{\dx} \Nv(\xi) \Nv^T(\xi)  (\half \dx) \, d\xi ~\fv, \\
       &= \frac{1}{\rhos\As} \, M_e \, \fv
\end{align} 
The formulae are a bit more complicated when $f$ is defined on a sub-interval $[a,b]$. 

% --------------------------------------------------------------------------------------------------------------------
\subsection{Time-stepping}
The Newmark-beta scheme for the matrix ODE 
\begin{align*}
   M \ddot\uv = - K \uv + \fv ,
\end{align*}
is an implicit scheme defined by
\begin{align*}
 &  \uv^{n+1} = \uv^n + \dt\vv^n + \frac{\dt^2}{2}\Big[ (1-2\beta) \av^n + 2\beta \av^{n+1} \Big],  \\
 &  \vv^{n+1} = \vv^n + \dt\Big[  (1-\gamma) \av^n + \gamma \av^{n+1} \Big]    , \\
 &  M \av^{n+1} =  -K \uv^{n+1} + \fv^{n+1} .
\end{align*}
where $\uv^n$ is the displacement, $\vv^n$ the velocity and $\av^n$ the acceleration at time $t^n$. 
Each time step requires the solution of the implicit system
\begin{align}
 &  \Big[ M +\beta\dt^2 K \Big] \av^{n+1} = \fv^{n+1} - K\Big[ \uv^n + \dt\vv^n + \frac{\dt^2}{2} (1-2\beta) \av^n \Big] 
   \label{eq:NewmarkMatrix}
\end{align}
to determine the acceleration $\av^{n+1}$. 
The scheme is second-order accurate and unconditionally stable for $2\beta=\half$ and $\gamma=\half$. 

% --------------------------------------------------------------------------------------------------------------------
\subsection{Implementing the boundary conditions} \label{eq:BeamModel_FEM_BC}

We describe some of the details involved in implementing the BC's. We consider the 
boundary at $x=0$, the case of $x=L$ is similar. 

\paragraph{Clamped boundary condition.} For clamped BC's at $x=0$ we set the two degrees of
freedom at $j=1$ to be given
\begin{align}
  w_1 = g_0(t), \qquad w'_1(t)=h_0(t).
\end{align}
The {\em second-time derivative} of these equations replace the first two equations in the system~\eqref{eq:beamODE}. 
Thus we set 
\begin{align*}
&  B_1^M=I, \quad C_1^M=0, \quad B^F_1=0, \quad C^F_1=0 , \\
&  [f_1, ~f_2]=[\partial_t^2g_0(t), ~\partial_t^2 h_0(t)].
\end{align*}
The boundary terms in~\eqref{eq:beamFEM_II} at $x=0$ will vanish for equations $i=2,3,\ldots$ 
since $\chi_i(0)=0$ and $\partial_x\chi_i(0)=0$
for $i\ne 1$.

\paragraph{Pinned boundary condition.} For pinned BC's at $x=0$ we set one degree of
freedom at $j=1$ to be given
\begin{align}
  w_1 = g_0(t) .
\end{align}
The {\em second-time derivative} of this equation will replace the first equation in the system~\eqref{eq:beamODE} (as for
the clamped BC).  
We also need to impose
\begin{align}
  EI \partial_x^2 w_1 = M(t). 
\end{align}
This condition is a natural boundary condition and appears in the boundary terms in~\eqref{eq:beamFEM_II}, 
\begin{align}
   \Bc(\chi_i) = -\left[ \chi_i T \frac{\partial w^h}{\partial x} ~ - \chi_i \frac{\partial}{\partial x}\Big( E I \frac{\partial^2 w^h}{\partial x^2}\Big) 
                      ~ + \frac{\partial \chi_i}{\partial x}E I \frac{\partial^2 w^h}{\partial x^2}\Big)  
                      \right]_0 ~.
\end{align}
The boundary term will contribute to the second equation, 
\begin{align}
   \Bc(\psi_1) &= -\left[ \psi_1 T \frac{\partial w^h}{\partial x} ~ - \psi_1 \frac{\partial}{\partial x}\Big( E I \frac{\partial^2 w^h}{\partial x^2}\Big) 
                      ~ + \frac{\partial \psi_1}{\partial x} M(t)  
                      \right]_0, \\
        &= -M(t) 
\end{align}
since $\psi_1(0)=0$, and $\partial_x\psi_1(0)=1$. 

\paragraph{Free boundary condition.} For a free BC we impose
\begin{align}
&    E I w_{xx}(0,t)=M(t), \\
&   \partial_x( E I w_{xx})(0,t)=S(t) .
\end{align}
These two natural BC's contribute to the boundary terms.
The first equation will have the contribution (using $\phi_1(0)=1$, $\partial_x\phi_1(0)=0$)
\begin{align}
   \Bc(\phi_1) &= -\left[ \phi_1 T \frac{\partial w^h}{\partial x} ~ - \phi_1 \frac{\partial}{\partial x}\Big( E I \frac{\partial^2 w^h}{\partial x^2}\Big) 
                      ~ + \frac{\partial \phi_1}{\partial x}E I \frac{\partial^2 w^h}{\partial x^2}\Big)  
                      \right]_0  \\
     &=  -T w'_1   + S(t) . \label{eq:beam:pinned}
\end{align}
since $\phi_1(0)=1$, $\partial_x\phi_1(0)=0$.
Note that the term $T w'_1$ in equation~\eqref{eq:beam:pinned} will contribute an adjustment to the stiffness matrix
and will thus change the implicit matrix~\eqref{eq:NewmarkMatrix} in the Newmark scheme.


The second equation in our system will have the contribution 
\begin{align}
   \Bc(\psi_1) &= -\left[\psi_1 T \frac{\partial w^h}{\partial x} ~ - \psi_1 \frac{\partial}{\partial x}\Big( E I \frac{\partial^2 w^h}{\partial x^2}\Big) 
                      ~ + \frac{\partial \psi_1}{\partial x}E I \frac{\partial^2 w^h}{\partial x^2}\Big)  
                      \right]_0 \\
        &= - M(t) , 
\end{align}
since $\psi_1(0)=0$, $\partial_x\psi_1(0)=1$.

{\bf Note:} The case of $E I= 0$ (i.e. the wave equaiton for a {\em string}) 
is a special case since the order of the PDE in space goes from four to two. In this
case there is only one BC applied at each end -- currently only the Dirichlet case
is implemented usng the {\em pinned} BC. 



% --------------------------------------------------------------------------------------------------------------------
% --------------------------------------------------------------------------------------------------------------------
\clearpage
\subsection{BeamModel Numerical results} \label{eq:BeamModel_Results}


Here are some numerical results for the BeamModel.


% --------------------------------------------------------------------------------------------------------------------
\subsubsection{Twilight-zone} 

Tables ?? show results for a trigonometric TZ solution for different boundary conditions, c=clamped, p=pinned, f=free.
The results show that the scheme is second-order accurate. 
The scheme is 4th-order (?) in space but the force integration is only done to second-order.

{
\input results/tbm.linearBeam.c.trig.ConvTable.tex 
\input results/tbm.linearBeam.p.trig.ConvTable.tex 
\input results/tbm.linearBeam.f.trig.ConvTable.tex 
}

\clearpage
% --------------------------------------------------------------------------------------------------------------------
\subsubsection{Standing wave} 

An exact standing wave solution is given by
\begin{align*}
   w(x,t) = \alpha \sin(k x)\cos(\omega t) .
\end{align*}
Given $k$, $\omega$ is 
\begin{align*}
   \omega =\sqrt{ \frac{E I k^4 + T k^2}{\rhos\As} } .
\end{align*}
Convergence results are given in Table~\ref{table:tbm.linearBeam.p.sw} for $k=2\pi$, $E I=1$, $\rhos\As=10$.
The errors converge close to fourth-order accuracy. The time-stepping scheme is second-order but 
we choose $\dt \propto \dx^2$, which leads to the fourth-order convergence.

\input results/tbm.linearBeam.p.sw.ConvTable.tex



%\clearpage
%\bibliography{\homeHenshaw/papers/henshaw}
%\bibliographystyle{plain}
\end{document}
