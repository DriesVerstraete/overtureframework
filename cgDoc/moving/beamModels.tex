%=======================================================================================================
% Beam Models for Overture and cg 
%=======================================================================================================

% -- article: 
\documentclass[11pt]{article}
% \usepackage[bookmarks=true]{hyperref}
\usepackage[bookmarks=true,colorlinks=true,linkcolor=blue]{hyperref}

% \input documentationPageSize.tex
\hbadness=10000 
\sloppy \hfuzz=30pt

\usepackage{calc}
\usepackage[lmargin=1.in,rmargin=1.in,tmargin=1.in,bmargin=1.in]{geometry}

\input homeHenshaw


\input wdhDefinitions.tex

% \input{pstricks}\input{pst-node}
% \input{colours}

\newcommand{\bogus}[1]{}  % begin a section that will not be printed

\usepackage{color}
\usepackage{amsmath}
\usepackage{amssymb}

\usepackage{graphicx}

% --------------------------------------------
% NOTE: trouble with tikz and program package ??
\usepackage{tikz}

\input trimFig.tex


% \usepackage{verbatim}
% \usepackage{moreverb}
% \usepackage{graphics}    
% \usepackage{epsfig}    
% \usepackage{calc}
% \usepackage{ifthen}
% \usepackage{float}
% \usepackage{fancybox}

% define the clipFig commands:
% \input clipFig.tex

\newcommand{\obDir}{\homeHenshaw/res/OverBlown}
\newcommand{\ogenDir}{\homeHenshaw/Overture/ogen}

% \newcommand{\grad}{\nabla}
% \newcommand{\half}{\frac{1}{2}}}

\newcommand{\eps}{\epsilon}
\newcommand{\zerov}{\mathbf{0}}

\newcommand{\bc}[1]{\mbox{\bf#1}}   % bold name
\newcommand{\cc}[1]{\mbox{  : #1}}  % comment

\newcommand{\Overture}{{Overture}}

\newcommand{\Largebf}{\sffamily\bfseries\Large}
\newcommand{\largebf}{\sffamily\bfseries\large}
\newcommand{\largess}{\sffamily\large}
\newcommand{\Largess}{\sffamily\Large}
\newcommand{\bfss}{\sffamily\bfseries}
\newcommand{\smallss}{\sffamily\small}
\newcommand{\normalss}{\sffamily}
\newcommand{\scriptsizess}{\sffamily\scriptsize}

\newcommand{\Div}{\grad\cdot}
\newcommand{\tauv}{\boldsymbol{\tau}}
\newcommand{\thetav}{\boldsymbol{\theta}}

\newcommand{\Omegav}{\boldsymbol{\Omega}}
\newcommand{\omegav}{\boldsymbol{\omega}}
\newcommand{\cm}{{\rm cm}}

\newcommand{\sumi}{\sum_{i=1}^n}
% \newcommand{\half}{{1\over2}}
\newcommand{\dt}{{\Delta t}}


\newcommand{\Gc}{\mathcal{G}}
\newcommand{\Fc}{\mathcal{F}}
\newcommand{\sgn}{\operatorname{sgn}}

\renewcommand{\url}[1]{Available from www.OvertureFramework.org}


% \psset{xunit=1.cm,yunit=1.cm,runit=1.cm}


% *** See http://www.eng.cam.ac.uk/help/tpl/textprocessing/squeeze.html
% By default, LaTeX doesn't like to fill more than 0.7 of a text page with tables and graphics, nor does it like too many figures per page. This behaviour can be changed by placing lines like the following before \begin{document}

\renewcommand\floatpagefraction{.9}
\renewcommand\topfraction{.9}
\renewcommand\bottomfraction{.9}
\renewcommand\textfraction{.1}   
\setcounter{totalnumber}{50}
\setcounter{topnumber}{50}
\setcounter{bottomnumber}{50}


\begin{document}

\vglue 5\baselineskip
\begin{flushleft}
{\LARGE Beam Models for Overture and CG} \\
\vspace{2\baselineskip}
William D. Henshaw, \\
Department of Mathematical Sciences, \\
Rensselaer Polytechnic Institute,     \\
Troy, NY, 12180.  \\

www.overtureFramework.org \\
\vspace{2\baselineskip}
\today
\vspace{4\baselineskip}
% 
\end{flushleft}
%
%
\noindent{\bf Abstract:} This article provides background and documentation for
the beam models that have been developed for use with Overture and the CG suite of partial differential
equation solvers. The topics covered include
\begin{description}
  \item[BeamModel] : a class that implements an Euler-Bernoulli beam.
  \item[NonLinearBeamModel] : a non-linear beam model.
\end{description}


% ----------------------------------------------------
\clearpage
\tableofcontents
\section{Acknowledgements}

Thanks to Dr. Alex Main for developing the first version of the BeamModel and NonLinearBeamModel classes while
a summer student at LLNL in 2013.

% --------------------------------------------------------------------------------------------------------------------
% --------------------------------------------------------------------------------------------------------------------
% --------------------------------------------------------------------------------------------------------------------
\section{Class BeamModel}\label{sec:BeamModel}

The BeamModel class defines a generalized Euler-Bernoulli (EB) beam,

\newcommand{\rhos}{\bar{\rho}}
\newcommand{\hs}{\bar{h}}
\newcommand{\bs}{\bar{b}}
\newcommand{\As}{\bar{A}}
\newcommand{\Ts}{\bar{T}}
\newcommand{\Ls}{\bar{L}}
\begin{align}
  \rhos \As \frac{\partial^2 w}{\partial t^2} = 
                   \frac{\partial}{\partial x}\left( T \frac{\partial w}{\partial x}\right)
                   - \frac{\partial^2}{\partial x^2}\left( E I \frac{\partial^2 w}{\partial x^2}\right)
                   + f(x,t), \qquad x\in[0,\Ls], \quad t\ge 0,
\end{align}
where $y=w(x,t)$ is the displacement of the beam, 
$\As$ is the cross-sectional area of the beam ($\As=\hs\bs$ for a beam of constant thickness, $\hs$ and constant breadth $\bs$ in the z-direction), 
$\rhos$ is the beam density, $\Ls$ is the length of the beam, 
$E$ is Young's modulus, $I$ is the area-moment of inertial (e.g. $I=\frac{1}{12} h^3 b$ for a rectangular beam),
and $\Ts$ is the {\em tension} coefficient. 
% 
The EB beam is generally consider valid for thin beams and small slopes,  $|w_x| \ll 1$.


A variational form of the problem is found by multiplying by $v$ and integrating over the domain with inner product,
\begin{align}
   (v,w)_B = \int_0^{\Ls} v \, w \, dx. 
\end{align}
After integration by parts the result is 
\begin{align}
 \left( v,  \rhos \As \frac{\partial^2 w}{\partial t^2} \right)_B  &= 
                  - \left( \frac{\partial v}{\partial x} , T \frac{\partial w}{\partial x} \right)_B
                  -\left(\frac{\partial^2 v}{\partial x^2}, E I \frac{\partial^2 w}{\partial x^2}\right)_B  + ( v, f(x,t))_B \\
% boundary terms
             &\quad  +   \left[ v T \frac{\partial w}{\partial x} ~ -v \frac{\partial}{\partial x}\Big( E I \frac{\partial^2 w}{\partial x^2}\Big) 
                      ~ + \frac{\partial v}{\partial x}E I \frac{\partial^2 w}{\partial x^2}\Big)  
                      \right]_0^{\Ls}   \label{eq:beamBoundaryTerms}
\end{align}

The boundary terms in~\eqref{eq:beamBoundaryTerms} indicate some valid combinations of boundary conditions that will given an energy estimate
when $v=w$ (e.g. if all the boundary terms vanish when the forcing functions $g_m$, $h_m$, $M$ and $S$, given below, are all zero). 
For example, the following three options are common boundary conditions (BC's), 
\begin{align}
   \text{clamped BC:}\quad & \quad w(x_m,t)=g_m(t),\quad  w_x(x_m,t)=h_m(t),  \\
   \text{pinned BC:} \quad & \quad w(x_m,t)=g_m(t), \quad E I w_{xx}(x_m,t)=M(t), \\
   \text{free BC:}   \quad & \quad  E I w_{xx}(x_m,t)=M(t), \quad \partial_x( E I w_{xx})(x_m,t)=S(t),
\end{align}
where $x_0=0$ and $x_1=\Ls$ for $m=0,1$, and 
where $M(t)$ is the imposed moment and $S(t)$ the imposed shear force. 
We also see there is a possible fourth choice (not so common ?) which we call a {\em slope} BC, since
the end slope and shear are specified, 
\begin{align}
   \text{slope BC:}\quad & \quad w_x(x_m,t)=g_m(t),\quad  \partial_x( E I w_{xx})(x_m,t)=S(t). 
\end{align}


% --------------------------------------------------------------------------------------------------------------------
\newcommand{\Ne}{N_e}
\section{FEM approximation}\label{sec:BeamModelFEM}

We define a FEM approximation for the EB-beam using Hermite polynomials and the Newmark-beta time stepping scheme.

The FEM representation is of the form
\begin{align}
  w(x,t) = \sum_{j=1}^{\Ne} w_j(t) \phi_j(x) + w_j'(t) \psi_j(x)  = \sum_{j=1}^{\Ne} [ \phi_j(x)\, \psi_j(x)] \begin{bmatrix} w_j(t)\\w_j'(t)\end{bmatrix}
\end{align}
where the degrees of freedom are nodal displacement $w_j(t)$ and the nodal slope $w'_j(t)$, and where $\phi_j(x)$ and $\psi_j(x)$ are 
the cubic Hermite polynomials 
that are non-zero only on $[x_{j-1},x_{j+1}]$,  and
that satisfy,
\begin{align}
  \phi_j(x_i)=\delta_{ij}, ~ \phi_i'(x_j)=0, \\
  \psi_j(x_i)=0, ~ \psi_i'(x_j)=\delta_{ij}.
\end{align}





%\clearpage
%\bibliography{\homeHenshaw/papers/henshaw}
%\bibliographystyle{plain}
\end{document}
