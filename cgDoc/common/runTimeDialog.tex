\subsection{\Solver\ Run time dialog}\label{sec:runTimeDialog}\index{run time dialog}

After the equations have been specified, parameters set and initial conditions chosen, 
the run time dialog window will appear, see figure(\ref{fig:runTimeScreen}).
Note that cgins is in the process of converting from popup menus (right mouse button) to
dialog windows so sometimes you will need to look for the command in the popup menu
if it is not in the dialog. 

Normally one would choose {\bf continue} to integrate to the next output time or {\bf movie mode}
to integrate until the final time.

\noindent The {\em plot component} option menu allows one to choose the solution
component to plot.


\noindent The push button commands are 
\begin{description}
  \item[\qquad break] : If running in movie mode this command will cause the program to halt at the next
                   time to plot.
  \item[\qquad continue] : compute the solution to the next time to plot.
  \item[\qquad movie mode] : compute the solution to the final time without waiting. The solution will be
       plotted at each output time interval. 
  \item[\qquad movie and save] : movie mode plus save each frame as a ppm file.
  \item[\qquad contour] : enter the contour plotting function in {\tt PlotStuff}. Here you will more options
     to change the plot.  
  \item[\qquad streamlines] : enter the streamlines plotting function from {\tt PlotStuff}.
  \item[\qquad grid] : enter the grid plotting function from {\tt PlotStuff}. If you don't first erase
      the contour plot then both the contours and the grid will be shown.
  \item[\qquad plot parallel dist.] plot the grid showing the parallel distribution. 
  \item[\qquad erase] : erase the screen.
  \item[\qquad change the grid] : add, remove or change existing grids. (poor man's adaptive mesh refinement).
  \item[\qquad adaptive grids...] : open up a new dialog to show parameters adaptive grids.
    \begin{description}
       \item[\qquad use adaptive grids] : turn adaptive grids on or off.
       \item[\qquad error threshold] : specify the error threshold.
    \end{description}
  \item[\qquad show file options...] : choose show file options; e.g. open or close a show file, see section~\ref{sec:showfileOptions}.
  \item[\qquad file output...] : specify options for saving solutions to an ascii file (for plotting with matlab for example).
      There are a number of options available as to what data should be saved. See also the userDefinedOutput routine
      where you can customize output.
    \begin{description}
       \item[\qquad output periodically to a file] : Open a file for output; specify how often to save data in the
        file (every step, every second step...); specify what data to save in the file (only grid 1, only
         values on some boundaries etc).  Each time this menu item is selected a new file is opened, allowing
         one, for example, to save certain information every step and other information every tenth step.
      \item[\qquad close an output file] : Close a file opened by the command `output periodically to a file'.
     \item[\qquad save a restart file] : save the current solution as a restart file; usually I just use the
       show file for restarts.
  \end{description}
  \item[\qquad pde parameters...] change PDE parameters at run time, see section~\ref{sec:pdeOptions}.
  \item[\qquad general options...] open the general options dialog (see section~\ref{sec:generalOptions}).
\end{description}

\noindent The text commands are 
\begin{description}
  \item[\qquad final time] : change the value for the final time to integrate to.
  \item[\qquad times to plot] : change the time interval between plotting (and output).
  \item[\qquad debug] : enter an integer to turn on debugging info. This is a bit flag with debug=1 turning on just
     a bit of info, debug=3 (1+2) showing more, debug=7 (1+2+4) even more etc.
\end{description}

\noindent The {\bf finish} button means do not compute any further, exit and save the show files etc.
