\subsection{Boundary Conditions} \label{sec:bcMenu}\index{boundary conditions!assigning}

In order to compute the correct flow the user must choose the correct
boundary conditions. Each physical boundary of each grid must be
given a boundary condition.  

The names of the available boundary conditions are 
%
\input boundaryConditionsEnum.tex
%
Not all boundary conditions can be used with all PDEs.
Boundary conditions are specified interactively (or in a command file)
by choosing the {\tt `boundary condition'} option from the main parameters menu
and then typing a string that takes one of the following forms
\begin{flushleft}
\bf $<$grid name$>$(side,axis)=$<$ boundary condition name$>$ [,option] [,option] ...
\end{flushleft}
to change the boundary condition on a given side of a given grid, or  
\begin{flushleft}
\bf $<$grid name$>$=$<$boundary condition name$>$  [,option] [,option] ...  
\end{flushleft}
to change all boundaries on a given grid, or
  \begin{flushleft}
\bf bcNumber$<$num$>$=$<$boundary condition name$>$  [,option] [,option] ...  
\end{flushleft}                             
to change all boundaries that currently have a boundary condition value equal to 
the integer `num'.
Here {\bf $<$grid name$>$} is the name of the grid, side=0,1 and axis=0,1,2.  
{\bf $<$grid name$>$} can also be `all'.
The optional arguments specify data for the boundary conditions:      
\begin{description}
  \item[option = `uniform(p=1.,u=1.,...)'] : to specify a uniform inflow profile   
  \item[option = `parabolic(d=2,p=1.,...)']     : to specify a parabolic inflow profile 
  \item[option = `jet(r=1.,x=0.,y=0,z=0.,d=.1,p=1.,u=$U_{\rm max}$,v=$V_{\rm max}$,...)'] : 
     specify a jet inflow profile.
  \item[ option = `pressure(.1*p+1.*p.n=0.)']     : pressure boundary condition at outflow
  \item[ option = `oscillate(t0=.5,omega=1.,a0=.5,a1=.5,u0=0.,v0=0.,w0=0.)']   : oscillating inflow parameters       
  \item[ option = `ramp(ta=0.,tb=1.,ua=0.,ub=1.,...)']   : ramped inflow parameters       
  \item[ option = `userDefinedBoundaryData'] : use a user defined boundary value option.
  \item[ option = `mixedDerivative(1.*t+2.*t.n=3.)'] : Mixed derivative on the Temperature.
\end{description}
Examples: 
\begin{flushleft}
square(0,0)=inflowWithVelocityGiven , uniform(p=1.,u=1.) \\
square(1,0)=outflow \\
annulus=noSlipWall \\
all=slipWall          \\             
bcNumber1=noSlipWall \\
square(0,1)=outflow , pressure(.1*p+1.*p.n=0.) \\
square(0,0)=inflowWithVelocityGiven , parabolic(d=.25,p=1.,u=1.) , oscillate(t0=0.,omega=1.,a0=.5,a1=.5) \\
square(0,0)=inflowWithVelocityGiven , userDefinedBoundaryData \\
square(0,0)=inflowWithVelocityGiven , parabolic(d=.25,p=1.,u=1.) , userDefinedBoundaryData \\
square(0,0)=noSlipWall,  uniform(u=2,v=0,T=1.), mixedDerivative(0.*t+1.*t.n=0)
\end{flushleft}

The first example, {\tt square(0,0)=inflowWithVelocityGiven}, will set the left edge of the square 
to be an inflow BC, while {\tt square(1,0)=outflow} will set the right edge to be an outflow boundary.
The line, {\tt annulus=noSlipWall}, will set all physical boundaries of the annulus to be no-slip walls.
Note that an annulus will normally have a branch cut and possibly an interpolation boundary. The
boundary conditions on these non-physical boundaries are never changed. The command, {\tt all=slipWall},
will make all physical boundaries slip-walls (and thus over-ride any previous changes to boundary conditions).
  

% ----------------------------------------------------------------------------------------------------------------
\subsection{Data for Boundary Conditions}\label{sec:bcDataMenu}\index{boundary conditions!optional data}

Some boundary conditions require `data', such as an inflow
boundary that requires values for certain quantities such as the velocity. 
These data values are optionally specified when
the boundary condition is given. Here are some examples:
{\footnotesize
\begin{verbatim}
    square(0,0)=inflowWithVelocityGiven , uniform(p=1.,u=1.)
    square(0,1)=outflow , pressure(.1*p+1.*p.n=0.)
    square(0,0)=inflowWithVelocityGiven , parabolic(d=.2,p=1.,u=1.), oscillate(t0=.3,omega=2.5)
\end{verbatim}
}
The available options are
\begin{description}
  \item[]{\bf uniform(component=value [,component=value]...)} Specify a uniform inflow profile and supply
      values for some of the components (components not specified will have a value of zero). Here
     {\tt component0} is the name of a component such as `p' or `u'.
  \item[]{\bf parabolic([d=boundary layer width][,component=value]...)} Specify a parabolic inflow profile
     with a given width. See section (\ref{sec:parabolic}) for more details.
  \item[]{\bf pressure(a*p+b*p.n=c)} Specify the parameters a,b,c for a pressure outflow boundary condition.
	   Here p=pressure and p.n=normal derivative of p. 
	   {\bf Note that a and b should have the same sign or else the condition is unstable.}
  \item[]{\bf mixedDerivative(a*t+b*t.n=c)} Specify the parameters a,b,c for a mixed boundary condition
      on the Temperature. 
  \item[]{\bf oscillate([t0=value][,omega=value])} Specify parameters for an oscillating inflow boundary condition.
    See section (\ref{sec:oscillate}) for more details.
  \item[]{\bf ramp([ta=value][,tb=value][,...])}  : specify values for a {\sl ramped} inflow. 
    See section (\ref{sec:rampedInflow}).
  \item[]{\bf userDefinedBoundaryData} : choose from the currently available user defined options. See
      section (\ref{sec:userDefinedFunctions}) for how to define your own boundary conditions.
\end{description}
Note that not all options can be used with all boundary conditions. 

% -------------------------------------------------------------------------------------------------------
\subsubsection{Parabolic velocity profile} \label{sec:parabolic}

  A `parabolic' profile can be specified as a Dirichlet type
boundary condition. The parabolic profile is zero at the
boundary and increases to a specified value $U_{\rm max}$ at 
a distance $d$ from the boundary:
\[
     u(\xv) = \begin{cases}
               U_{\rm max} (2 -s/d ) s/d & \text{if $s\le d$ } \\
               U_{\rm max} &     \text{if $s> d$ }
              \end{cases}
\]
Here $s$ is the shortest distance between the point $\xv$ on the inflow face to the 
next nearest adjacent boundary.
and $d$ is the user specified {\it boundary layer width}.
The algorithm is quite smart at correctly determining the distance $s$ even if the 
inflow boundary is covered by one or more overlapping grids (such as the
pipe flow example or inlet-outlet grid).

The parabolic profile can be useful, for example,
in specifying the velocity profile at an inflow boundary that
is adjacent to a no-slip wall. A uniform profile would have a
discontinuity at the wall.

% -------------------------------------------------------------------------------------------------------
\subsubsection{Jet velocity profile} \label{sec:jet}

The jet option is {\tt `jet(r=1.,x=0.,y=0,z=0.,d=.1,p=1.,u=$U_{\rm max}$,v=$V_{\rm max}$,w=$W_{\rm max}$,...)'}.

A `jet' profile can be used to define inflow over
a portion of a boundary. The jet has a a center, $(x_0,y_0,z_0)$, a radius $r$, 
and a maximum value of $U_{\rm max}$ for $u$ (or $V_{\rm max}$ for $v$ or $W_{\rm max}$ for $w$) at $r=0$:
\[
     u(\xv) = \begin{cases}
               U_{\rm max} & \text{if $|\xv-\xv_0| \le r$  } \\
               0           & \text{if $|\xv-\xv_0| > r$ }
              \end{cases}
\]
In 3D the jet will have a cylindrical cross section.
The jet can also be defined to go to zero at it's boundary using the parameter
$d$ which defines the width of the transition layer,
\[
     u(\xv) = \begin{cases}
               U_{\rm max} & \text{if $|\xv-\xv_0| \le r-d $ } \\
               U_{\rm max}[1-(\xi/d)^2]   & \text{if $r-d \le |\xv-\xv_0| < r$ }\\
               0           & \text{if $|\xv-\xv_0| > r$ }
              \end{cases}
\]
Here $\xi = |\xv-\xv_0| - (r-d)$.

% -------------------------------------------------------------------------------------------------------
\subsubsection{Oscillating values} \label{sec:oscillate}

An inflow boundary condition, {\tt uniformInflow} or {\tt parabolicInflow}, 
can be given an oscillating time dependence of the form
\[
   \{ a_0 + a_1 \cos[ 2 \pi \omega (t-t_0) ] \} \times \{ \mbox{uniform/parabolic profile} \} + \uv_0
\]
The parameters {\tt omega,t0,a0,a1,u0,v0,w0} are specified with the {\tt oscillate} option. Here
$\uv_0=(u0,v0,w0)$.


% -------------------------------------------------------------------------------------------------------
\subsubsection{Ramped Inflow} \label{sec:rampedInflow}

An inflow boundary condition can be ramped from one value (usually zero) to another value.
The ramp function is a cubic polynomial on the interval $(t_a,t_b)$. The polynomial is
monotone increasing on this interval with slope zero at the ends.
 The variables $(u,v,w)$ vary from $(u_a,v_a,w_a)$ to $(u_b,v_b,w_b)$. Thus the $u$ boundary
condition ramp function would be:
\[
     u(t) = \begin{cases} 
             u_a  & \text{for $t\le t_a$} \\
            (t-t_a)^2( -(t-t-a)/3+(t_b-t_a)/2 ) 6 {(u_b-u_a) \over (t_b-t_a)^3}  +ua  & \text{for $t_a<t<t_b$} \\
            u_b & \text{for $t\ge t_b$}
            \end{cases}
\]

The ramped inflow can also be combined with the parabolic profile as in 
\begin{verbatim}
square(0,0)=inflowWithVelocityGiven , parabolic(d=.25,p=1.,u=1.) , ramp(ta=0.,tb=1.,ua=0.,ub=2.)
\end{verbatim}
to give a ramped parabolic profile.
