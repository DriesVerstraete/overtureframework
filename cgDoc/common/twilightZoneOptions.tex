\subsubsection{Twilight Zone Options Dialog (twilight zone options...)}\label{sec:twilightZoneOptions}\index{twilight zone options}

Here is a description of the {\em Twilight Zone Options Dialog}.

\noindent The twilight zone {\em type} options are
\begin{description}
  \item[\qquad polynomial] : a polynomial function.
  \item[\qquad trigonometric] : a trigonometric function.
  \item[\qquad pulse] : a generalized Gaussian pulse.
\end{description}

\noindent The {\em Error Norm} options are (use this norm when reporting errors)
\begin{description}
  \item[\qquad maximum norm] : 
  \item[\qquad l1 norm] : 
  \item[\qquad l2 norm] : 
\end{description}

\noindent The push button options are
\begin{description}
  \item[\qquad assign polynomial coefficients] : specify the coefficients in the polynomial function.
\end{description}

\noindent The toggle button options are
\begin{description}
  \item[\qquad twilight zone flow] : turn on the twilight zone flow.
  \item[\qquad use 2D function in 3D] : use the 2D twilight zone function even in 3D. 
  \item[\qquad compare 3D run to 2D] : this option will adjust the equations and forcing so
         that a 3D run on an extruded 2D grid can be compared to the 2D computation. This includes setting
         the twilight-zone function to be 2D and changing the divergence damping (INS) to be two-dimensional 
         (otherwise it is scaled in the wrong way).
  \item[\qquad assign TZ initial conditions] : initial conditions are assigned the twilight zone solution.
\end{description}

\noindent The text commands are
\begin{description}
  \item[\qquad degree in space ] : set the degree in space of the polynomial function.
  \item[\qquad degree in time ] : set the degree in time of the polynomial function.
  \item[\qquad frequencies (x,y,z,t)] : set the frequencies in the trigonometric function.
\end{description}
