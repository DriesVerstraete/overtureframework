\section{Cgcns Boundary Conditions} \label{sec:cgcnsBC}\index{boundary conditions}

Here is a description of the boundary conditions for Cgcns.

{\bf Note:} Normally the boundary conditions assign the primitive variables:
\begin{alignat}{3}
  {\tt r} &= \rho , \qquad&&\text{(density)} \\
  {\tt u} &= u,    &&\text{(x-component of the velocity)} \\
  {\tt v} &= v ,   &&\text{(y-component of the velocity)}\\
  {\tt w} &= w ,   &&\text{(z-component of the velocity)}\\
  {\tt T} &= T,    &&\text{({\em Temperature})} \\
  {\tt s} &= s,    &&\text{(species component)}
\end{alignat}

{\bf NOTE:} For the Godunov scheme ${\tt T}$ is defined to be $p/\rho = R_g T$, i.e. the Godunov scheme
works with $R_g T$ instead of $T$.


%      noSlipWall,
%      slipWall,
%      superSonicOutflow,
%      superSonicInflow,
%      subSonicInflow,
%      symmetry,
%      dirichletBoundaryCondition,
%      neumannBoundaryCondition,
%      axisymmetric,
%      farField

% -------------------------------------------------------------
\subsection{superSonicInflow} \label{sec:bc:superSonicInflow}

At a supersonic inflow all variables are given (since all charcteristics are entering the domain):
\begin{align}
   \begin{bmatrix}
       \rho \\
       \uv  \\
       T    \\
       \sv  
   \end{bmatrix}
  = 
   \begin{bmatrix}
       \rho_0(\xv,t) \\
       \uv_0(\xv,t)  \\
       T_0 (\xv,t)   \\
       \sv_0(\xv,t)  
   \end{bmatrix}.
\end{align}

Examples:
\begin{enumerate}
  \item Set the density (r), u component of velocity and total energy, e, (${\tt e} =\rho e + \half \rho |\uv|^2$) to a uniform
       value on inflow:
  \begin{flushleft}\tt
    backGround(0,0)=superSonicInflow uniform(r=2.6667,u=1.25,e=10.119)
  \end{flushleft}
  \item Set the density (r), u component of velocity and total energy, temperature, and a species component (s) to a uniform
       value on inflow:
  \begin{flushleft}\tt
    channel(1,0)=superSonicInflow uniform(r=1.0,u=-3.2,T=9.325417e-01,s=2.033854e-02)
  \end{flushleft}
\end{enumerate}

% ------------------------------------------------------------------------------------------------
\subsection{superSonicOutflow} \label{sec:bc:superSonicOutflow}

At a supersonic outflow all variables are extrapolated (since all characteristics are exiting the domain)
and thus nothing needs to be set.

Example:
\begin{enumerate}
  \item Assign superSonicOutflow:
  \begin{flushleft}\tt
    backGround(0,1)=superSonicOutflow
  \end{flushleft}
\end{enumerate}

% ------------------------------------------------------------------------------------------------
\subsection{slipWall} \label{sec:bc:slipWall}

At a slip wall the normal component of the velocity is specified (normally zero
for a non-moving wall),
\begin{align}
   \nv\cdot\uv & = g(\xv,t) 
\end{align}

Example:
\begin{enumerate}
  \item Assign a slipWall boundary condition,
  \begin{flushleft}\tt
    annulus(0,1)=slipWall
  \end{flushleft}
\end{enumerate}

There are different implementations of the slip wall BC. These are
chosen using the following option:
\begin{flushleft}\tt
  OBPDE:slip wall boundary condition option [0|1|2|3|4]
\end{flushleft}
(** finish me**)

% ------------------------------------------------------------------------------------------------
\subsection{subSonicInflow} \label{sec:bc:subSonicInflow}

At a subsonic inflow there is one characteristic leaving the domain so
that all variables except one are specified. There are different possible options
that one could apply for subsonic inflow. 
Here we set the incoming velocity and apply mixed conditions on
the density and temperature (**check me**)
\begin{align}
  \text{subSonicInflow:}\quad
   \begin{bmatrix}
   \alpha_\rho \rho + \beta_\rho \rho_n  = g_{\rho}(\xv,t) , \\
       \uv  \\
   \alpha_T T + \beta_T T_n  = g_T(\xv,t), \\
       \sv  
   \end{bmatrix}
  = 
   \begin{bmatrix}
       \rho_0(\xv,t) \\
       \uv_0(\xv,t)  \\
       T_0(\xv,t) \\
       \sv_0(\xv,t)  
   \end{bmatrix}.
\end{align}
By default $(\alpha_\rho,\beta_\rho)=(1,1)$ and $(\alpha_T,\beta_T)=(1,1)$.
The mixed derivative condition ($\beta>0$) is a softer condition that allows waves to exit
the domain with fewer reflections. Note that if the flow is nearly uniform at inflow then
$\rho_n\approx 0$ and $T_n \approx 0$ 
and the condition reduces to a dirichlet condition. 

Examples:
\begin{enumerate}
  \item Set the velocity $(u,v,w)=(1,0,0)$, $\rho+\rho_n=1$ and $T+T_n=300$:
  \begin{flushleft}\tt
    square(0,0)=subSonicInflow uniform(r=1.,u=1,v=0,w=0,T=300.)
  \end{flushleft}
\end{enumerate}


% ------------------------------------------------------------------------------------------------
\subsection{subSonicOutflow} \label{sec:bc:subSonicOutflow}


At a subsonic outflow there is one characteristic entering the domain and thus one variable
should be set and all others can be extrapolated. At a subsonic outflow we set the 
temperature (i.e. $p/\rho$ for the Godunov scheme) at outflow or a mixed derivative on the
temperature 
\begin{align}
   \alpha T + \beta T_n  = g(\xv,t).
\end{align}
The mixed derivative condition ($\beta>0$) is a softer condition that allows waves to exit
the domain with fewer reflections. Note that if the flow is nearly uniform at outflow then
$T_n \approx 0$ and the condition reduces to a dirichlet condition. Normally one might set
$\alpha=1$ and $\beta$ a small value, e.g. $\beta=.1$ (but note that $\beta$ dimensional quantity
and thus should scale with the domain size)
\begin{align}
   T + \beta T_n  = T_0(\xv,t). \label{eq:mixedOutflow}
\end{align}
Equation~\eqref{eq:mixedOutflow} will thus approximate $T \approx T_0(\xv,t)$ if $\beta T_n$ is small.

Example:
\begin{enumerate}
  \item Set $T\approx 300$ and outflow:
  \begin{flushleft}\tt
    square(1,0)=subSonicOutflow mixedDerivative(1.*t+.1*t.n=300.)
  \end{flushleft}
\end{enumerate}


% ------------------------------------------------------------------------------------------------
\subsection{Axisymmetric} \label{sec:bc:axisymmetric}

The axisymmetric boundary condition is used to specify an axisymmetric boundary
for two-dimensional axisymmetric computations (see Section~\ref{sec:axisymmetricEquations}),
The conditions on an axisymmetric boundary $y=0$ are (**CHECK ME**)
\begin{align}
   \rho_y(x,0) &=0 , \\
   u_y(x,0) &=0 , \\
   v(x,0) &=0, \\
   T_y(x,0) &=0
\end{align}


Example:
\begin{enumerate}
  \item Assign superSonicOutflow:
  \begin{flushleft}\tt
    bcNumber3=axisymmetric
  \end{flushleft}
\end{enumerate}
