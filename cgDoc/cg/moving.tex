\section{Moving Grids: Notes}\label{sec:movingGrids}


Here are notes on how moving grids are implemented in cg.
{
\small
\begin{verbatim}
advanceEuler.bC: 
  DomainSolver::eulerStep 
    moveGrids( t1,t2,t3,dt0,cgf1,cgf2,cgf3 );

move.C:
  DomainSolver::moveGrids(const real & t1, ... )
    if( t3<t2 ) setInterfacesAtPastTimes( t1,t2,t3,dt0,cgf1,cgf2,cgf3 );
    parameters.dbase.get<MovingGrids >("movingGrids").moveGrids(t1,t2,t3,dt0,cgf1,cgf2,cgf3 );
    gridGenerator->updateOverlap(cgf3.cg,cgf1.cg,hasMoved, ... )
  setInterfacesAtPastTimes( t1,t2,t3,dt0,cgf1,cgf2,cgf3 )
    bd(I1,I2,I3,Rx+uc) = x(I1,I2,I3,Rx) + gridVelocityLocal(I1,I2,I3,Rx)*dt;

MovingGrids.C
  MovingGrids::moveGrids(const real & t1, ... )
    detectCollisions(cgf1);
    rigidBodyMotion( t1,t2,t3,dt0,cgf1,cgf2,cgf3 );
    moveDeformingBodies( t1,t2,t3,dt0,cgf1,cgf2,cgf3 );
    userDefinedMotion( t1,t2,t3,dt0,cgf1,cgf2,cgf3 );
    // rigid bodies: replace Mapping with a MatrixTransform.
    // move rigid bodies (shift, rotate, scale, rigidBody, userDefinedMotion ...).
    deforming bodies: (deformingBodyList[b] == DeformingBodyMotion)
      deformingBodyList[b]->regenerateComponentGrids(  newT, cgf3.cg );

  MovingGrids::rigidBodyMotion(const real & t1, ... )
    for( b )
      parameters.getNormalForce( cgf2.u,stress[grid],ipar,rpar );
      integrate->surfaceIntegral(stress,C,f,b);
      body[b]->integrate( t2,f,g, t3,x,r  );

  MovingGrids::moveDeformingBodies(const real & t1, ...)
    for(b) deformingBodyList[b]->integrate( t1,t2,t3,cgf1,cgf2,cgf3, stress);
\end{verbatim}
}
% ------------------------------
{\small
\begin{verbatim}
DeformingBodyMotion.C
  initialize( cg,t )
    // get surface of the deforming grid
    initializeGrid( cg,t )
  initializeGrid( cg,t )
    // build an HyperbolicMapping for the deforming grid if necessary
    gridEvolution[face]->addGrid(dpm.getDataPoints(),t); // time history of grids
  integrate( real t1, real t2, real t3, ...) // update the startCurve 
    if( userDefinedDeformingBodyMotionOption==interfaceDeform )
      RealArray & bd = parameters.getBoundaryData(sideToMove,axisToMove,gridToMove,cg[gridToMove]);
      startCurve.interpolate(x0,...) 
  regenerateComponentGrids( const real newT, CompositeGrid & cg)
    hyp.generate();
    gridEvolution[face]->addGrid(dpm.getDataPoints(),newT);// time history of grids
  getVelocity( const real time0, ..)
    if( time0<=0 )
     getInitialState( ... )
    else
      gridEvolution[face]->getVelocity(time0,gridVelocity,I1,I2,I3);
  getInitialState( InitialStateOptionEnum stateOption, ...)
\end{verbatim}
}
% ------------------------------
%{\small
%\begin{verbatim}
%   
%\end{verbatim}
%}
