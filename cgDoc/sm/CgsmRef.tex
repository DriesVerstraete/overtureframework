%-----------------------------------------------------------------------
% Cgsm: Solid Mechanics Solver
%        REFERENCE MANUAL
%-----------------------------------------------------------------------
\documentclass[11pt]{article}
% \usepackage[bookmarks=true]{hyperref}  % this changes the page location !
\usepackage[bookmarks=true,colorlinks=true,linkcolor=blue]{hyperref}

% \input documentationPageSize.tex
\hbadness=10000 
\sloppy \hfuzz=30pt

% \voffset=-.25truein
% \hoffset=-1.25truein
% \setlength{\textwidth}{7in}      % page width
% \setlength{\textheight}{9.5in}    % page height

\usepackage{calc}
\usepackage[lmargin=.75in,rmargin=.75in,tmargin=.75in,bmargin=.75in]{geometry}

\input homeHenshaw

% \input{pstricks}\input{pst-node}
% \input{colours}

\usepackage{amsmath}
\usepackage{amssymb}

\usepackage{verbatim}
\usepackage{moreverb}

\usepackage{graphics}    
\usepackage{epsfig}    
\usepackage{calc}
\usepackage{ifthen}
\usepackage{float}
% the next one cause the table of contents to disappear!
% * \usepackage{fancybox}

\usepackage{makeidx} % index
\makeindex
\newcommand{\Index}[1]{#1\index{#1}}

\usepackage{tikz}
% \usepackage{pgfplots}
\input trimFig.tex
% define the clipFig commands:
%% \input clipFig.tex


\input defs


\begin{document}



\vspace{5\baselineskip}
\begin{flushleft}
{\Large
Cgsm Reference Manual: An Overture Solver for the Solving the Equations of Solid Mechanics, \\
}
\vspace{2\baselineskip}
William D. Henshaw \\
\  \\
Department of Mathematical Sciences, \\
Rensselaer Polytechnic Institute, \\
Troy, NY, 12180,\\
www.overtureFramework.org \\
~~ \\
\today\\

\vspace{4\baselineskip}

\noindent{\bf\large Abstract:}

This is the reference guide for {\bf Cgsm}. Cgsm is a program that can be
used to solve the elastic wave equation and other equations of solid mechanics in two and 
three dimensions using composite overlapping grids. It is built upon the
\Overture object-oriented framework.  
This reference guide describes in some detail the equations being solved,
the discrete approximations, time-stepping methods, boundary conditions and convergence results. 
The reference guide also contains various notes related to different aspects of the equations. 
The reference guide concludes by providing a collection of interesting computations that
have been performed with Cgsm. 


\end{flushleft}

\clearpage
\tableofcontents
% \listoffigures

\vfill\eject


\section{Introduction}

This is the reference guide for {\bf Cgsm}. Cgsm is a program that can be
used to solve the elastic wave equation and other equations of solid mechanics
in two and three dimensions using composite overlapping grids. It is built upon the
\Overture object-oriented framework~\cite{Brown97},\cite{Henshaw96a},\cite{iscope97}. 
This reference guide describes in some detail the equations being solved,
the discrete approximations, time-stepping methods, boundary conditions and convergence results. 
The reference guide also contains various notes related to different aspects of the equations. 
The reference guide concludes by providing a collection of interesting computations that
have been performed with Cgsm.

\section{Governing equations} \label{sec:governing}


% {\bf Suggested Notation: }
% \begin{itemize}
%   \item Continuous functions: $\uv(\xv,t)$ has components $u_i(\xv,t)$, $\sigmav$ has components $\sigma_{ij}$. 
%   \item Grid functions: $\uv_\iv^n$, components $u_{k,\iv}^n$ 
% \end{itemize}
% \vskip\baselineskip

% Consider an elastic solid that occupies the space $\xv\in\Omega$ at time $t=0$.  
Consider an elastic solid that at time $t=0$ occupies the domain $\Omega \subset \Real^{\nd}$ in $n_d=2$ or
$n_d=3$ space dimensions. 
Let $\uv(\xv,t)$, with components $u_i(\xv,t)$,
denote the displacement of a material particle originally located at position $\xv\in\Real^\nd$,
and let $\sigmav(\xv,t)$ denote the 
Cauchy stress tensor with components $\sigma_{ij}(\xv,t)$. 
It is assumed that the solid is a homogeneous isotropic material, and
that the evolution of the displacement is governed by the equations of linear elasticity given
by (with Einstein summation convention), 
\begin{equation}
\rho{\partial\sp2u_i\over\partial t\sp2}={\partial\s_{ij}\over\partial x_j}+\rho f_i,
           \qquad \xv\in\Omega,\quad t>0,~~ i=1,2,\ldots,\nd,
\label{eq:sm}
\end{equation}
where $\rho$ is the density of the material (taken to be constant), ${\bf f}$ is an acceleration due
to an applied body force, and the components of stress are given by
\begin{equation}
\s_{ij}=\lambda\left(\epsilon_{kk}\right)\delta_{ij}+2\mu\epsilon_{ij},\qquad \epsilon_{ij}={1\over2}\left({\partial u_i\over\partial x_j}+{\partial u_j\over\partial x_i}\right).
\label{eq:stress}
\end{equation}
Here, $\epsilon_{ij}$ and $\delta_{ij}$ are the components of the (linear) strain tensor and the identity tensor, respectively, $\epsilon_{kk}=\sum_k \epsilon_{kk}=\grad\cdot\uv$ is the divergence of the displacement, 
and $\lambda$ and $\mu$ are Lam\'e parameters.  The latter are related to
Young's modulus $E$ and Poisson's ratio $\nu$ by $\mu={E/(2(1+\nu))}$, and $\lambda={\nu E/((1+\nu)(1-2\nu))}$.
%* \begin{equation}
%* \mu={E\over 2(1+\nu)},\qquad \lambda={\nu E\over(1+\nu)(1-2\nu)}.
%* \label{eq:lameConstants}
%* \end{equation}
% In general $\lambda$ and $\mu$ are functions of $\xv$ but for the purposes of this paper they
% will be taken as constants. In this case the equations become
% \begin{align}
%   \rho {\partial^2 \uv \over \partial t^2} &= (\lambda+\mu) \grad(\grad\cdot\uv) + \mu \Delta \uv + \rho \fv \label{eq:sos}
% \end{align}
% 
Initial conditions for the second-order system in (\ref{eq:sm}) are
\begin{equation}
\uv(\xv,0)=\uv_0(\xv),\qquad {\partial\uv\over\partial t}(\xv,0)=\vv_0(\xv),\qquad \xv\in\Omega,
\label{eq:smICs}
\end{equation}
where $\uv_0(\xv)$ and $\vv_0(\xv)$ are the initial displacement and velocity of
the solid, respectively.  Boundary conditions for (\ref{eq:sm}) are applied
for $\xv\in\partial\Omega$ and take various forms.  The available boundary conditions
are
\begin{alignat}{3}
& \uv=\gv_d(\xv,t),                  \quad && \hbox{displacement boundary condition}, \label{eq:dbc} \\
& \nv\cdot\sigmav =\gv_t(\xv,t),     \quad && \hbox{traction boundary condition},  \label{eq:tbc}  \\
& \!\!\!
    \left.\begin{array}{l}
        \nv\cdot\uv=g_s(\xv,t)  \\
           \nv\cdot\s\cdot\tnv_\alpha=g_{s,\alpha}(\xv,t) 
        \end{array}\right\}                       \qquad&& \hbox{slip-wall boundary conditions}.  \label{eq:sbc} 
  % \label{eq:smBCs}
\end{alignat}
% {\bf Note: I removed the slip BC for now, but we may need to add it back.}
Here, $\nv$ is the unit outward normal on the boundary and
$\tnv_\alpha$, $\alpha=1,\ldots,\nd-1$, are unit tangent vectors (assumed to be mutually orthogonal).  The functions
$\gv_d(\xv,t)$ and $\gv_t(\xv,t)$ give the displacement and traction at the
boundary, respectively, while $g_s(\xv,t)$ and $g_{s,\alpha}(\xv,t)$ define the slip wall motion.
The elastic wave equation~\eqref{eq:sm}-\eqref{eq:stress} with initial conditions~\eqref{eq:smICs} and
boundary conditions~\eqref{eq:dbc}-\eqref{eq:sbc} is a well-posed problem, see, for example~\cite{Graff1991}.

% \newcommand{\w}{{{\bf w}}}

We also consider the equations in (\ref{eq:sm}) and (\ref{eq:stress}) written as a first-order system
\begin{equation}
\left.
\begin{array}{l}
\displaystyle{
{\partial u_i\over\partial t}=v_i,
}\smallskip\\
\displaystyle{
{\partial v_i\over\partial t}={1\over\rho}{\partial\s_{ij}\over\partial x_j}+f_i,
}\smallskip\\
\displaystyle{
{\partial\s_{ij}\over\partial t}=\lambda\left(\dot\epsilon_{kk}\right)\delta_{ij}+2\mu\dot\epsilon_{ij},
}
\end{array}
\right\}\qquad \xv\in\Omega,\quad t>0,~~ i=1,2,\ldots,\nd ~,
\label{eq:smFOS}
\end{equation}
%\begin{align}
%{\partial u_i\over\partial t}&=v_i,  \nonumber \\ 
% {\partial v_i\over\partial t}&={1\over\rho_0}{\partial\s_{ij}\over\partial x_j}+f_i, \label{eq:smFOS} \\
%{\partial\s_{ij}\over\partial t}&=\lambda\left(\dot\epsilon_{kk}\right)\delta_{ij}+2\mu\dot\epsilon_{ij},\nonumber
%\end{align}
where $\vv(\xv,t)$, with components $v_i(\xv,t)$, is the velocity and the components $\dot\epsilon_{ij}$ of the rate of strain tensor are given by
\[
\dot\epsilon_{ij}={1\over2}\left({\partial v_i\over\partial x_j}+{\partial v_j\over\partial x_i}\right).
\]
%
Initial conditions for displacement and velocity are given by $\uv_0(\xv)$ and $\vv_0(\xv)$ as before, and initial conditions for the components of stress may be derived from (\ref{eq:stress}) applied at $t=0$.  Boundary conditions for the first-order system may be taken directly from those described in (\ref{eq:dbc}), (\ref{eq:tbc}) and (\ref{eq:sbc}) for the second-order system.
Note that contrary to what is typically done, we retain the displacements in our formulation of the first order system. 
The displacements are coupled with the velocity and stress through the boundary conditions as discussed in Section~\ref{bc:fos}.
% 
Retaining the displacements in the formulation allows the stress-strain relationship~\eqref{eq:stress} to be explicitly imposed
at the boundary. In addition it will be useful to have the displacement field when solving fluid-structure interaction problems (to define the 
fluid-solid interface for grid generation, for example).
% -----------------------------------
% -------------------------------

The governing equations, whether written as a second-order or first-order system, are
hyperbolic and represent the motion of elastic waves in the solid. For the second-order system, the characteristic 
wave speeds
are $\pm c_p$ and $\pm c_s$, where the pressure and shear wave speeds are given by
\begin{equation}
c_p=\sqrt{\lambda+2\mu\over\rho},\qquad c_s=\sqrt{\mu\over\rho}.
\label{eq:pswaves}
\end{equation}
The first-order system has the wave speeds above as well as characteristics speeds equal to zero.  




% ==============================================================================================================
\clearpage
\input sampleSimulations


% -------------------------------------------------------------------------------------------------
\clearpage
\bibliography{henshaw,henshawPapers}
\bibliographystyle{siam}


\printindex


\end{document}
