\newcommand{\Gd}{\Gc_{CS}} 
\subsection{Diffraction of a p-wave ``shock'' by a circular cavity}\label{sec:cylDiffract}

% --------------------------------------------------------------------------
% ----------------------- Diffraction --------------------------------------
% --------------------------------------------------------------------------

To illustrate the use of adaptive mesh refinement we consider the diffraction
of a p-wave ``shock'' by a circular cavity.  The domain for this problem is taken to be the
two-dimensional region interior to the square $[-3,3]\sp2$ and exterior to the circle of
radius $R=0.5$.  The initial conditions are taken from 
the exact traveling-wave solution for a planar p-wave with a piecewise constant velocity profile.
The displacement and velocity for this solution are given by
\[
 \uv (\xi)  = \begin{cases}  -\xi(v_0/c_p)\kappav  & \mbox{for $\xi<0$}, \\
                                0  & \mbox{for $\xi>0$,}  \end{cases} \qquad
 \vv (\xi)  = \begin{cases}  v_0\,\kappav  & \mbox{for $\xi<0$}, \\
                                0  & \mbox{for $\xi>0$},\end{cases}
\]
where
\[
\xi = \kappav\cdot(\xv-\xv_0) - c_p t ,\qquad c_p = \sqrt{{(\lambda+2\mu)}/{\rho}}.
\]
Here, $\kappav$ defines the direction of propagation of the planar wave and
$\xv_0$ defines its position at $t=0$. For the computations presented, we take
$\kappav=(1,\,0)$, $v_0=c_p$ and $\xv_0=(-1.25,\,0)$, and we assume that
$\rho=\lambda=\mu=1$.  The boundary conditions on the
bottom and top sides of the square are slip-wall conditions, and exact data from the planar p-wave
solution is used as Dirichlet conditions on the left and right sides of the square.  A homogeneous
traction condition is applied on the boundary of the circular cavity.

Calculations are performed using the SOS and FOS schemes with one level of AMR grids
using a refinement factor of $n_r=2$ or $4$.  The base-level composite grid for the two-dimensional domain
is composed of a rectangular Cartesian grid defined previously in~(\ref{eq:rectGrid}) and an annular
grid defined in~(\ref{eq:annulusGrid}).  This circle-in-a-square grid is defined by
\[
  \Gd^{(\gm)} = \Rc([-3,3]^2 ,N_{x}(\gm),N_{x}(\gm)) ~\union~ \AnnulusGrid([R,R+7\dsm], N_{\theta}(\gm),N_{r}),
\]
where $\dsm=1/(10 \gm)$ gives the approximate grid spacing, and the number of grid cells in the
various coordinate directions are given by
\[
N_{x}(j)=\lfloor 6/\dsm+1.5\rfloor, \quad
N_{\theta}(\gm)=\lfloor 2\pi(R+3.5\dsm)/\dsm+1.5\rfloor,\quad 
N_{r}=7.
\]
We note that the composite grid uses a boundary-fitted annular grid with a fixed number of grid
cells in the radial direction which is similar to the \lq\lq narrow'' grid, $\Gsic^{(\gm)}$, used
in Section~\ref{sec:stabilityResults}.
%This narrow grid construction is more difficult to handle in terms of stability, but is more efficient in terms of number of Cartesian grid cells.

Figure~\ref{fig:cylDiffractEvolution} shows the elastic response of the planar p-wave as it
is diffracted by the circular cavity.  The numerical solution is computed using the FOS scheme
with the base-level composite grid given by $\Gd^{(8)}$ and the addition of one refinement level with $n_r=4$.  
The plots show shaded contours of the magnitude of velocity
at times $t=0$, $1.0$ and $1.6$.  When the p-wave meets the cavity, the boundary of the cavity is
deformed and a reflected wave is created.  The reflected wave consists of both pressure and shear waves
which travel at different velocities as seen clearly in the plot at $t=1.0$.  The cavity continues
to deform as the diffracted waves travels around it, and ultimately the waves collide near the
back of the cavity as seen in the plot at $t=1.6$.  The behavior of the refinement grids are shown
in the figure, and these grids are well-positioned during the calculation to increase the grid resolution
of the various waves as they move in time throughout the domain.

\input cylDiffractFig

%\medskip
%\textcolor{red}{(I wonder if the shaded contours are saturated with pink at $t=0.8$?  Also, it might
%be better to show the middle solution at a slightly later time so that the reflected waves are seen
%more clearly.)}
%\medskip

% Figure~\ref{fig:cylDiffractAMR} compares the results at time $t=1.6$ for four
% cases: coarse grid using the FOS scheme and no AMR (top left), fine grid using
% the FOS scheme and no AMR (top right), SOS with AMR (bottom left) and FOS with
% AMR (bottom right).  The results from the computations all generally agree with
% the finer grids providing a a sharper representation of the discontinuous parts
% of the solution.  The FOS results for the magnitude of the velocity are not as
% noisy as the SOS results. Note, however, that the velocity is directly computed
% by FOS while for the SOS scheme it is computed in a post-processing step by a
% finite difference approximation in time, $\vv_\iv^n =
% (\uv_\iv^{n}-\uv_\iv^{n-1})/\dt$. \textcolor{red}{(Do we now say that this
% latter step may add a bit to the noise for SOS?)}
% % This computed velocity for SOS-C 

Figure~\ref{fig:cylDiffractAMR} compares the results for the SOS and FOS schemes for an AMR computation.
For this comparison, we use the base grid $\Gc^{(8)}$ together with one
level of $n_r=4$ refinement.  The magnitudes of the displacement
and velocity at time $t=1.6$ are shown. 
From the figures it can be seen that the results from both schemes are generally
in good agreement. The FOS results are less noisy than the SOS results, particularly
in $\vert \vv \vert$, 
which might be expected from the FOS approach as it is an upwinding scheme. 
Note, however, that the velocity is directly computed
by FOS while for the SOS scheme it is computed in a post-processing step
by a finite difference approximation in time, 
$\vv_\iv^n = (\uv_\iv^{n}-\uv_\iv^{n-1})/\dt$, and this may contribute to some of the noise.  


\input cylDiffractCompareFig

The accuracy of the AMR computations for this example can be made more quantitative.
Given a sequence of three grids of increasing resolution, a posteriori estimates
of the errors and convergence rates can be computed using the procedure
described in~\cite{pog2008a}. These self-convergence estimates assume that the
numerical results are converging to some limiting solution.
A posteriori estimates computed in this way are given in
Table~\ref{tab:cylDiffractConvergence} for three grids of increasing resolution.
The coarse grid computation used the base grid $\Gd^{(2)}$ together with one refinement level of factor 2.
The medium resolution computation used the base grid $\Gd^{(2)}$ together with one refinement level of factor 4,
while the finest resolution used the grid $\Gd^{(64)}$ with no AMR. 
The parameters in the AMR error
estimate~\eqref{eq:componentError} were chosen as $c_1=0$, $c_2=1$ and
$s_k=1$.  The error tolerances for the SOS scheme were taken as $4\times
10^{-3}$ and $1\times 10^{-3}$ for the coarse and medium resolutions
respectively. The corresponding error tolerances for the FOS scheme were
$10^{-2}$ and $2.5\times 10^{-3}$ respectively.

The table provides estimates of the $L_1$-norm errors. Since the exact solution for the displacement 
has discontinuous first derivatives, one cannot expect second-order accurate convergence. 
We expect that the $L_1$-norm error in the displacement would converge at rate of $1$ in the limit of small~$h$, while the
errors in velocity and stress would converge at a rate of $2/3$, see~\cite{BanksAslamRider2008}. 
The results in the table indicate that the computed convergence rates are close to the expected rates.
%
{
\renewcommand{\arraystretch}{\tablearraystretch}% increase the height of rows
\newcommand{\rateLabelb}{\multicolumn{4}{|c|}{rate}}
\begin{table}[hbt]\tableFontSize
\begin{center}
\begin{tabular}{|c|c|c|c|c|c|c|c|c|c|c|c|} \hline
\multicolumn{4}{|c|}{ } & \multicolumn{2}{|c|}{SOS} & \multicolumn{6}{|c|}{FOS} \\ \hline 
Grid &~levels~&~$n_r$~ &~~$h_j$~~& $\eem_u$ & $r$ & $\eem_u$ & $r$ & $\eem_v$ & $r$  & $\eem_\sigma$  & $r$ \\ \hline 
~$\Gd^{(2)}$~ & 2  & 2  &~1/40  ~& ~$8.9\times10^{ -4}$~&         & ~$2.1\times10^{ -3}$~ &        & ~$1.0\times10^{ -2}$~ &        & ~$1.2\times10^{ -2}$~ &        \\ \hline 
~$\Gd^{(2)}$~ & 2  & 4  &~1/80  ~& ~$4.7\times10^{ -4}$~&~$ 1.9$~ & ~$1.0\times10^{ -3}$~ &~$ 2.0$~& ~$6.4\times10^{ -3}$~ &~$ 1.6$~& ~$7.2\times10^{ -3}$~ &~$ 1.6$~\\ \hline 
~$\Gd^{(64)}$~& -- & -- &~1/640 ~& ~$7.1\times10^{ -5}$~&~$ 6.7$~ & ~$1.2\times10^{ -4}$~ &~$ 8.5$~& ~$1.5\times10^{ -3}$~ &~$ 4.3$~& ~$1.7\times10^{ -3}$~ &~$ 4.3$~ \\ \hline 
\rateLabelb                      &  $ 0.91$             &        &  $ 1.03$   &        &  $ 0.70$   &        &  $ 0.71$   &    \\ \hline                                        
\end{tabular}
\caption{A posteriori estimated errors ($L_1$-norm) and convergence rates at $t=1.0$ for diffraction of a p-wave ``shock''
         by a circular cavity using AMR. Note that the finest grid is a factor $8$ times finer than the previous resolution. }
\label{tab:cylDiffractConvergence}
\end{center}
\end{table}
}


