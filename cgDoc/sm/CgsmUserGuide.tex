%
%  User's Guide for cgsm : Solid Mechanics
%
\documentclass{article}
% \usepackage[bookmarks=true]{hyperref} 
\usepackage[bookmarks=true,colorlinks=true,linkcolor=blue]{hyperref}

% \input documentationPageSize.tex
\hbadness=10000 
\sloppy \hfuzz=30pt

% \voffset=-.25truein
% \hoffset=-1.25truein
% \setlength{\textwidth}{7in}      % page width
% \setlength{\textheight}{9.5in}    % page height

\usepackage{calc}
\usepackage[margin=1.in]{geometry}

\input homeHenshaw

% --------------------------------------------
% \input{pstricks}\input{pst-node}
% \input{colours}

% define the clipFig commands:
% \input clipFig.tex
\usepackage{tikz}
\input trimFig.tex

% The amssymb package provides various useful mathematical symbols
\usepackage{amsmath}
\usepackage{amssymb}

\newcommand{\Largebf}{\sffamily\bfseries\Large}
\newcommand{\largebf}{\sffamily\bfseries\large}
\newcommand{\largess}{\sffamily\large}
\newcommand{\Largess}{\sffamily\Large}
\newcommand{\bfss}{\sffamily\bfseries}
\newcommand{\smallss}{\sffamily\small}

\newcommand{\beq}{\begin{equation}}
\newcommand{\eeq}{\end{equation}}
\newcommand{\Omegav}{\boldsymbol{\Omega}}
\newcommand{\omegav}{\boldsymbol{\omega}}

\input wdhDefinitions.tex
\newcommand{\mbar}{\bar{m}}
\newcommand{\Rbar}{\bar{R}}
\newcommand{\Ru}{R_u}         % universal gas constant
% \newcommand{\grad}{\nabla}
\newcommand{\Div}{\grad\cdot}
\newcommand{\tauv}{\boldsymbol{\tau}}
\newcommand{\sigmav}{\boldsymbol{\sigma}}
\newcommand{\sumi}{\sum_{i=1}^n}


\newcommand{\Pc}{{\mathcal P}}
\newcommand{\Hc}{{\mathcal H}}

\newcommand{\mw}{W}  % molecular weight
\newcommand{\mwBar}{\overline{W}}  % molecular weight of the mixture
\newcommand{\Dc}{\mathcal{D}}

% \usepackage{verbatim}
% \usepackage{moreverb}
% \usepackage{graphics}    
% \usepackage{epsfig}    
% \usepackage{fancybox}    


\begin{document}
 
\title{Cgsm User's Guide : An Overture Solver for the Solving the Equations of Solid Mechanics,}

\author{
William D. Henshaw  \\
\  \\
Department of Mathematical Sciences, \\
Rensselaer Polytechnic Institute, \\
Troy, NY, 12180,\\
www.overtureFramework.org \\
~~ \\
\today\\
}
 
\maketitle

\tableofcontents

\section{Nomenclature}
\begin{align}
  \rho & \qquad \mbox{density} \\
  u_i & \qquad \mbox{displacement vector} \\
  \epsilon_{ij} & \qquad \mbox{strain tensor} \\
%   \omega_{ij} & \qquad \mbox{rotation tensor} \\
  \tau_{ij} & \qquad \mbox{stress tensor} \\
  \lambda & \qquad \mbox{shear modulus, Lam\'e constant} \\
  \mu & \qquad \mbox{Lam\'e constant}
\end{align}

% -----------------------------------------------------------------------------------------------
\section{Introduction}  \label{sec:intro}

\subsection{Basic steps}

% -----------------------------------------------------------------------------------------------
\section{Sample command files for running cgsm} \label{sec:sampleCommand}


% -----------------------------------------------------------------------------------------------
\section{Options and parameters} \label{sec:options}


% -----------------------------------------------------------------------------------------------
\section{User defined functions} \label{sec:userDefined}




% -----------------------------------------------------------------------------------------------
\section{Hints for running} \label{sec:hints}


% -----------------------------------------------------------------------------------------------
\section{Frequently ask questions} \label{sec:FAQ}



% -----------------------------------------------------------------------------------------------
\section{Governing Equations}
See~\cite{smog2012}.

The equations of linear elasticity for a homogeneous isotropic material are governed by
\begin{align}
  \rho \partial_t^2 u_i &= \partial_{x_j} \tau_{ij} + \rho f_i \\
  \tau_{ij} &= \lambda \partial_{x_k} u_k \delta_{ij} + 2 \mu \epsilon_{ij} \\
  \epsilon_{ij} &= \half( \partial_{x_j} u_i + \partial_{x_i} u_j )
\end{align}
or
\begin{align}
  \rho \partial_t^2 u_i &= (\lambda+\mu) \partial_{x_i} \partial_{x_k} u_k + \mu \partial_{x_k}^2 u_i  +\rho f_i \\
  \rho \uv_{tt} &= (\lambda+\mu) \grad(\grad\cdot\uv) + \mu \Delta \uv + \rho \fv 
\end{align}


\subsection{Boundary conditions}


%% \Appendix
% --------------------------------------------------------------------------------------------------------
%% \section{Implementation of Boundary Conditions} \label{sec:implementationOfBoundaryConditions}





% -------------------------------------------------------------------------------------------------
\clearpage
\bibliography{henshawPapers,henshaw}
\bibliographystyle{siam}


\end{document}
