\subsection{Chirped plane wave results} \label{sec:chirpedPlaneWaveResults}


Here are some convergence results for verifying the chirped plane wave boundary forcing.
To test the chirped plane wave boundary forcing we consider a simple geometry of a flat
PEC boundary at $x=0$. The scattered field due to an incident chirped plane-wave in the x-direction will just be
another chirped wave. 

Tables~\ref{table:chirpedNFDTDOrder2max}-\ref{table:chirpedNFDTDOrder4max} show convergence results for
scattering of a chirped plane wave off a planar PEC boundary at $x=0$. 

% \newcommand{\tableFont}{\footnotesize}
\newcommand{\tableFont}{\small}
{
\newcommand{\convTitle}{Chirped wave, flat PEC boundary, 2D, FD22}% define the multicolumn title.
\newcommand{\strutt}{\rule{0pt}{9pt}}% strutt to make table column height bigger.
\newcommand{\num}[2]{#1e{#2}}% use this command to set the format of numbers in the table.
\newcommand{\erruA}{$E_j^{E_y}$}% defines column header - note [1234567890]->[ABCDEFGHI]
\newcommand{\erruB}{$E_j^{H_z}$}% defines column header - note [1234567890]->[ABCDEFGHI]
%
% Table generated by processCheckFiles.p, Sat Aug 13 19:03:40 2016
\begin{table}[hbt]\tableFont % you should set \tableFont to \footnotesize or other size
\begin{center}
\begin{tabular}{|l|c|c|c|c|c|} \hline 
  \multicolumn{6}{|c|}{\convTitle} \\ \hline 
    grid      &  N   &     \erruA     &  r   &     \erruB     &  r    \\ \hline 
  nonSquarenp1 &   1  & \num{9.0}{-2} &      & \num{3.0}{-1} &     \\ \hline
  nonSquarenp2 &   2  & \num{4.1}{-2} & 2.2  & \num{8.1}{-2} & 3.7 \\ \hline
  nonSquarenp4 &   4  & \num{1.4}{-2} & 3.0  & \num{2.0}{-2} & 4.1 \\ \hline
  nonSquarenp8 &   8  & \num{3.3}{-3} & 4.2  & \num{4.8}{-3} & 4.1 \\ \hline
  nonSquarenp16 &   16  & \num{7.8}{-4} & 4.2  & \num{1.2}{-3} & 4.0 \\ \hline
  rate        &      &    1.73       &      &    2.00       &      \\ \hline
\end{tabular}
\caption{Cgmx, chirped, method=CGFD2, max norm, order=$2$, $t=1.5$, m=2,  n=2, cfl=$0.9$, diss=$0$, dissOrder=$-1$, Sat Aug 13 19:03:01 2016}\label{table:chirpedNFDTDOrder2max}
\end{center}
\end{table}
}

{
\newcommand{\convTitle}{Chirped wave, flat PEC boundary, 2D, FD44}% define the multicolumn title.
\newcommand{\strutt}{\rule{0pt}{9pt}}% strutt to make table column height bigger.
\newcommand{\num}[2]{#1e{#2}}% use this command to set the format of numbers in the table.
\newcommand{\erruA}{$E_j^{E_y}$}% defines column header - note [1234567890]->[ABCDEFGHI]
\newcommand{\erruB}{$E_j^{H_z}$}% defines column header - note [1234567890]->[ABCDEFGHI]
% Table generated by processCheckFiles.p, Sat Aug 13 19:05:11 2016
\begin{table}[hbt]\tableFont % you should set \tableFont to \footnotesize or other size
\begin{center}
\begin{tabular}{|l|c|c|c|c|c|} \hline 
  \multicolumn{6}{|c|}{\convTitle} \\ \hline 
    grid      &  N   &     \erruA     &  r   &     \erruB     &  r    \\ \hline 
  nonSquarenp1 &   1  & \num{3.1}{-2} &      & \num{4.9}{-2} &     \\ \hline
  nonSquarenp2 &   2  & \num{8.9}{-3} & 3.4  & \num{9.5}{-3} & 5.2 \\ \hline
  nonSquarenp4 &   4  & \num{1.1}{-3} & 8.4  & \num{1.1}{-3} & 8.4 \\ \hline
  nonSquarenp8 &   8  & \num{6.8}{-5} & 15.7  & \num{7.0}{-5} & 16.1 \\ \hline
  nonSquarenp16 &   16  & \num{4.1}{-6} & 16.6  & \num{4.2}{-6} & 16.8 \\ \hline
  rate        &      &    3.28       &      &    3.41       &      \\ \hline
\end{tabular}
\caption{Cgmx, chirped, method=CGFD4, max norm, order=$4$, $t=1.5$, m=2,  n=2, cfl=$0.9$, diss=$0$, dissOrder=$-1$, Sat Aug 13 19:04:55 2016}\label{table:chirpedNFDTDOrder4max}
\end{center}
\end{table}
}

{
% Table generated by processCheckFiles.p, Sun Aug 14 07:14:39 2016
\begin{table}[hbt]\tableFont % you should set \tableFont to \footnotesize or other size
 \newcommand{\convTitle}{Chirped wave, flat PEC boundary, 2D, SOSUP22}% define the multicolumn title.
 \newcommand{\strutt}{\rule{0pt}{9pt}}% strutt to make table column height bigger.
 \newcommand{\num}[2]{#1e{#2}}% use this command to set the format of numbers in the table.
 \newcommand{\erruA}{$E_j^{E_y}$}% defines column header - note [1234567890]->[ABCDEFGHI]
 \newcommand{\erruB}{$E_j^{H_z}$}% defines column header - note [1234567890]->[ABCDEFGHI]
 \newcommand{\erruD}{$E_j^{\dot E_y}$}% defines column header - note [1234567890]->[ABCDEFGHI]
 \newcommand{\erruE}{$E_j^{\dot H_z}$}% defines column header - note [1234567890]->[ABCDEFGHI]
\begin{center}
\begin{tabular}{|l|c|c|c|c|c|c|c|c|c|} \hline 
  \multicolumn{10}{|c|}{\convTitle} \\ \hline 
    grid      &  N   &     \erruA     &  r   &     \erruB     &  r   &     \erruD     &  r   &     \erruE     &  r    \\ \hline 
  nonSquarenp1 &   1  & \num{1.2}{-1} &      & \num{3.6}{-1} &      & \num{2.0}{+} &      & \num{1.8}{+} &     \\ \hline
  nonSquarenp2 &   2  & \num{7.1}{-2} & 1.7  & \num{6.2}{-2} & 5.8  & \num{1.1}{+} & 1.8  & \num{2.1}{+} & 0.9 \\ \hline
  nonSquarenp4 &   4  & \num{2.9}{-2} & 2.5  & \num{2.6}{-2} & 2.4  & \num{4.8}{-1} & 2.4  & \num{6.3}{-1} & 3.3 \\ \hline
  nonSquarenp8 &   8  & \num{8.7}{-3} & 3.3  & \num{8.1}{-3} & 3.2  & \num{1.7}{-1} & 2.9  & \num{1.6}{-1} & 4.0 \\ \hline
  nonSquarenp16 &   16  & \num{2.3}{-3} & 3.8  & \num{2.2}{-3} & 3.7  & \num{4.6}{-2} & 3.7  & \num{4.2}{-2} & 3.8 \\ \hline
  nonSquarenp32 &   32  & \num{5.8}{-4} & 4.0  & \num{5.6}{-4} & 3.9  & \num{1.2}{-2} & 3.9  & \num{1.1}{-2} & 3.8 \\ \hline
  rate        &      &    1.58       &      &    1.80       &      &    1.50       &      &    1.60       &      \\ \hline
\end{tabular}
\caption{Cgmx, chirped, method=sosup, max norm, order=$2$, $t=1.5$, cfl=$0.9$, diss=$0$, dissOrder=$-1$, Sun Aug 14 07:13:47 2016}\label{table:chirpedsosupOrder2max}
\end{center}
\end{table}
}

{
\newcommand{\convTitle}{Chirped wave, flat PEC boundary, 2D, SOSUP44}% define the multicolumn title.
\newcommand{\strutt}{\rule{0pt}{9pt}}% strutt to make table column height bigger.
\newcommand{\num}[2]{#1e{#2}}% use this command to set the format of numbers in the table.
\newcommand{\erruA}{$E_j^{E_y}$}% defines column header - note [1234567890]->[ABCDEFGHI]
\newcommand{\erruB}{$E_j^{H_z}$}% defines column header - note [1234567890]->[ABCDEFGHI]
\newcommand{\erruD}{$E_j^{\dot E_y}$}% defines column header - note [1234567890]->[ABCDEFGHI]
\newcommand{\erruE}{$E_j^{\dot H_z}$}% defines column header - note [1234567890]->[ABCDEFGHI]
% Table generated by processCheckFiles.p, Sat Aug 13 19:18:28 2016
\begin{table}[hbt]\tableFont % you should set \tableFont to \footnotesize or other size
% \newcommand{\convTitle}{Title goes here}% define the multicolumn title.
% \newcommand{\strutt}{\rule{0pt}{9pt}}% strutt to make table column height bigger.
% \newcommand{\num}[2]{#1e{#2}}% use this command to set the format of numbers in the table.
% \newcommand{\erruA}{$E_j^{uA}$}% defines column header - note [1234567890]->[ABCDEFGHI]
% \newcommand{\erruB}{$E_j^{uB}$}% defines column header - note [1234567890]->[ABCDEFGHI]
% \newcommand{\erruD}{$E_j^{uD}$}% defines column header - note [1234567890]->[ABCDEFGHI]
% \newcommand{\erruE}{$E_j^{uE}$}% defines column header - note [1234567890]->[ABCDEFGHI]
\begin{center}
\begin{tabular}{|l|c|c|c|c|c|c|c|c|c|} \hline 
  \multicolumn{10}{|c|}{\convTitle} \\ \hline 
    grid      &  N   &     \erruA     &  r   &     \erruB     &  r   &     \erruD     &  r   &     \erruE     &  r    \\ \hline 
  nonSquarenp1 &   1  & \num{4.4}{-2} &      & \num{4.7}{-2} &      & \num{1.3}{+} &      & \num{1.4}{+} &     \\ \hline
  nonSquarenp2 &   2  & \num{1.8}{-2} & 2.5  & \num{1.9}{-2} & 2.4  & \num{4.7}{-1} & 2.8  & \num{5.0}{-1} & 2.7 \\ \hline
  nonSquarenp4 &   4  & \num{3.3}{-3} & 5.4  & \num{3.5}{-3} & 5.4  & \num{9.9}{-2} & 4.8  & \num{1.0}{-1} & 4.8 \\ \hline
  nonSquarenp8 &   8  & \num{3.1}{-4} & 10.6  & \num{3.2}{-4} & 10.9  & \num{1.0}{-2} & 9.7  & \num{1.1}{-2} & 9.7 \\ \hline
  nonSquarenp16 &   16  & \num{2.2}{-5} & 13.8  & \num{2.3}{-5} & 14.1  & \num{7.7}{-4} & 13.3  & \num{7.9}{-4} & 13.6 \\ \hline
  nonSquarenp32 &   32  & \num{1.4}{-6} & 15.4  & \num{1.5}{-6} & 15.6  & \num{5.0}{-5} & 15.3  & \num{5.1}{-5} & 15.5 \\ \hline
  rate        &      &    3.05       &      &    3.07       &      &    2.99       &      &    2.99       &      \\ \hline
\end{tabular}
\caption{Cgmx, chirped, method=sosup, max norm, order=$4$, $t=1.5$, m=2,  n=2, cfl=$0.9$, diss=$0$, dissOrder=$-1$, Sat Aug 13 19:11:08 2016}\label{table:chirpedsosupOrder4max}
\end{center}
\end{table}
}

{
% Table generated by processCheckFiles.p, Sun Aug 14 07:01:02 2016
\begin{table}[hbt]\tableFont % you should set \tableFont to \footnotesize or other size
 \newcommand{\convTitle}{Chirped wave, flat PEC boundary, 3D, FD22}% define the multicolumn title.
 \newcommand{\strutt}{\rule{0pt}{9pt}}% strutt to make table column height bigger.
 \newcommand{\num}[2]{#1e{#2}}% use this command to set the format of numbers in the table.
 \newcommand{\erruA}{$E_j^{E_y}$}% defines column header - note [1234567890]->[ABCDEFGHI]
\begin{center}
\begin{tabular}{|l|c|c|c|} \hline 
  \multicolumn{4}{|c|}{\convTitle} \\ \hline 
    grid      &  N   &     \erruA     &  r    \\ \hline 
  nonBoxnpp1 &   1  & \num{3.1}{-2} &     \\ \hline
  nonBoxnpp2 &   2  & \num{9.5}{-3} & 3.3 \\ \hline
  nonBoxnpp4 &   4  & \num{2.4}{-3} & 3.9 \\ \hline
  nonBoxnpp8 &   8  & \num{5.8}{-4} & 4.1 \\ \hline
  rate        &      &    1.92       &      \\ \hline
\end{tabular}
\caption{Cgmx, chirped, method=CGFD2, max norm, order=$2$, $t=1.5$, cfl=$0.9$, diss=$0$, dissOrder=$-1$, Sun Aug 14 07:01:43 2016}\label{table:chirpedNFDTDOrder2max}
\end{center}
\end{table}
}

{
% Table generated by processCheckFiles.p, Sun Aug 14 07:16:19 2016
\begin{table}[hbt]\tableFont % you should set \tableFont to \footnotesize or other size
 \newcommand{\convTitle}{Chirped wave, flat PEC boundary, 3D, FD44}% define the multicolumn title.
 \newcommand{\strutt}{\rule{0pt}{9pt}}% strutt to make table column height bigger.
 \newcommand{\num}[2]{#1e{#2}}% use this command to set the format of numbers in the table.
 \newcommand{\erruA}{$E_j^{E_y}$}% defines column header - note [1234567890]->[ABCDEFGHI]
\begin{center}
\begin{tabular}{|l|c|c|c|} \hline 
  \multicolumn{4}{|c|}{\convTitle} \\ \hline 
    grid      &  N   &     \erruA     &  r    \\ \hline 
  nonBoxnpp1 &   1  & \num{5.1}{-3} &     \\ \hline
  nonBoxnpp2 &   2  & \num{4.9}{-4} & 10.5 \\ \hline
  nonBoxnpp4 &   4  & \num{3.2}{-5} & 15.0 \\ \hline
  nonBoxnpp8 &   8  & \num{2.0}{-6} & 16.0 \\ \hline
  rate        &      &    3.78       &      \\ \hline
\end{tabular}
\caption{Cgmx, chirped, method=CGFD4, max norm, order=$4$, $t=1.5$, cfl=$0.9$, diss=$0$, dissOrder=$-1$, Sun Aug 14 07:17:09 2016}\label{table:chirpedNFDTDOrder4max}
\end{center}
\end{table}
}


{
% Table generated by processCheckFiles.p, Sun Aug 14 08:55:03 2016
\begin{table}[hbt]\tableFont % you should set \tableFont to \footnotesize or other size
\newcommand{\convTitle}{Chirped wave, flat PEC boundary, 3D, SOSUP22}% define the multicolumn title.
\newcommand{\strutt}{\rule{0pt}{9pt}}% strutt to make table column height bigger.
\newcommand{\num}[2]{#1e{#2}}% use this command to set the format of numbers in the table.
\newcommand{\erruA}{$E_j^{E_y}$}% defines column header - note [1234567890]->[ABCDEFGHI]
\newcommand{\erruD}{$E_j^{\dot E_y}$}% defines column header - note [1234567890]->[ABCDEFGHI]
\begin{center}
\begin{tabular}{|l|c|c|c|c|c|} \hline 
  \multicolumn{6}{|c|}{\convTitle} \\ \hline 
    grid      &  N   &     \erruA     &  r   &     \erruD     &  r    \\ \hline 
  nonBoxnpp1 &   1  & \num{5.1}{-2} &      & \num{8.6}{-1} &     \\ \hline
  nonBoxnpp2 &   2  & \num{1.8}{-2} & 2.8  & \num{3.4}{-1} & 2.5 \\ \hline
  nonBoxnpp4 &   4  & \num{5.1}{-3} & 3.6  & \num{1.0}{-1} & 3.4 \\ \hline
  rate        &      &    1.65       &      &    1.54       &      \\ \hline
\end{tabular}
\caption{Cgmx, chirped, method=sosup, max norm, order=$2$, $t=1.5$, cfl=$0.9$, diss=$0$, dissOrder=$-1$, Sun Aug 14 08:20:38 2016}\label{table:chirpedsosupOrder2max}
\end{center}
\end{table}
}

{
% Table generated by processCheckFiles.p, Sun Aug 14 11:03:11 2016
\begin{table}[hbt]\tableFont % you should set \tableFont to \footnotesize or other size
\newcommand{\convTitle}{Chirped wave, flat PEC boundary, 3D, SOSUP44}% define the multicolumn title.
\newcommand{\strutt}{\rule{0pt}{9pt}}% strutt to make table column height bigger.
\newcommand{\num}[2]{#1e{#2}}% use this command to set the format of numbers in the table.
\newcommand{\erruA}{$E_j^{E_y}$}% defines column header - note [1234567890]->[ABCDEFGHI]
\newcommand{\erruD}{$E_j^{\dot E_y}$}% defines column header - note [1234567890]->[ABCDEFGHI]
\begin{center}
\begin{tabular}{|l|c|c|c|c|c|} \hline 
  \multicolumn{6}{|c|}{\convTitle} \\ \hline 
    grid      &  N   &     \erruA     &  r   &     \erruD     &  r    \\ \hline 
  nonBoxnpp1 &   1  & \num{1.1}{-2} &      & \num{2.8}{-1} &     \\ \hline
  nonBoxnpp2 &   2  & \num{1.4}{-3} & 7.2  & \num{4.6}{-2} & 6.1 \\ \hline
  nonBoxnpp4 &   4  & \num{1.2}{-4} & 12.2  & \num{4.2}{-3} & 11.1 \\ \hline
  rate        &      &    3.23       &      &    3.04       &      \\ \hline
\end{tabular}
\caption{Cgmx, chirped, method=sosup, max norm, order=$4$, $t=1.5$, cfl=$0.9$, diss=$0$, dissOrder=$-1$, Sun Aug 14 07:46:45 2016}\label{table:chirpedsosupOrder4max}
\end{center}
\end{table}
}
