\section{Boundary forcing functions for direct computation of the scattered field} \label{sec:boundaryForcings}



In some cases on can directly compute the scattered field due to a known incident field.
The total field is written as a sum of the incident field and scattered field,
\[
     \Ev_T(\xv,t) = \Ev_I(\xv,t) + \Ev_s(\xv,t),
\]
and thus the scattered field is given by
\[
     \Ev_s(\xv,t) = \Ev_T(\xv,t) -\Ev_I(\xv,t). 
\]
To compute the scattered field directly we must adjust the PEC boundary conditions $\tv_m\cdot\Ev_T=0$ to
become inhomogeneous conditions, 
\[
    \tv_m\cdot \Ev_s(\xv,t) =  - \tv_m\cdot\Ev_I(\xv,t), \quad\text{(PEC BC for the scattered field)}. 
\]

\noindent The following boundary forcing options are currently available with cgmx,
\begin{description}
  \item[noBoundaryForcing]: (default).
  \item[planeWaveBoundaryForcing]: Incident field is a plane wave, see Section~\ref{sec:planeWaveBoundaryForcing}.
  \item[chirpedPlaneWaveBoundaryForcing]: Incident field is a chirped plane wave, see Section~\ref{sec:chirpedPlaneWaveBoundaryForcing}.
\end{description}


% ----------------------------------------------------------------------------
\subsection{Plane wave boundary forcing} \label{sec:planeWaveBoundaryForcing}

Choose the boundary forcing option to be {\tt planeWaveBoundaryForcing} to directly compute the 
scattered field due to plane wave. The parameters defining the plane wave are described
in Section~\ref{sec:planeWaveIC}.


% ----------------------------------------------------------------------------
\subsection{Chirped plane wave boundary forcing}\label{sec:chirpedPlaneWaveBoundaryForcing}


Choose the boundary forcing option to be {\tt chirpedPlaneWaveBoundaryForcing} to directly compute the 
scattered field due to {\em chirped} plane wave. A {\em chirped} wave has wavelength that
varies over time. The parameters defining the plane wave (defining the carrier frequency below) 
are described in Section~\ref{sec:planeWaveIC}. 
The chirp is defined by the parameters
\begin{align*}
   [t_a,t_b] \quad & \text{: interval of chirp}, \\
   \kv      \quad & \text{: incident wave vector}, \\
   \omega_0 = c \, \kv \quad & \text{: main carrier frequency}, \\
   B         \quad & \text{: band-width}, \\
   \alpha    \quad & \text{: amplitude}, \\
   \beta     \quad & \text{: parameter in $\tanh$ function that ramps the chirp on and off.}
\end{align*}

\input chirpFunctionFigure

The {\em chirped} plane wave with wave vector $\kv$ moving in direction $\av$ is defined by
\begin{align*}
  & \Ev_c(x,t) = \alpha \, \hat\av~ P(\xi), \\
  & \xi \defeq t - \frac{\hat{\kv}}{c}\cdot\xv -\xi_0, \quad
   \hat{\kv} \defeq \frac{\kv}{\| \kv \|}, \quad \kv\cdot\hat\av=0, \quad
  \xi_0 \defeq \frac{t_a+t_b}{2}, \\
\end{align*}
where 
\begin{align*}
  & P(\xi) \defeq \chi(\xi,\tau) e^{-i \phi(\xi) } , \\
  & \phi(\xi) \defeq \omega_0 \xi +  \frac{B}{2\tau} \xi^2,  \\
  & \chi(\xi) = \frac{1}{2}\Big[ \tanh(\beta(\xi+\tau/2) -\tanh(\beta(\xi-\tau/2) \Big], \\
  & \tau \eqdef \frac{t_b-t_a}{2} \quad \omega_0 = c \, \hat{\kv} .
\end{align*}

Note that since $\Ev_c(\xv,t) \propto P(\xi)$ is a function only of $\xi$ it follows that $\Ev_c(\xv,t)$ will be a solution to
the wave equation for any choice of $P(\xi)$.

The instantaneous frequency is
\begin{align*}
&  \phi'(\xi) = \omega_0 + \frac{B}{\tau} \xi \qquad \text{ for $\xi\in[-\tau/2,\tau/2]$}, \\
&  \phi'(\xi) \in [\omega_0 - B/2, \omega_0 + B/2], 
\end{align*}
and varies linearly along the chirp (hence a {\em linear chirp}). 


