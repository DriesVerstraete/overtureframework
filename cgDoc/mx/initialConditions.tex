\section{Initial conditions} \label{sec:ic}

\noindent The following initial conditions are currently available with cgmx,
\begin{description}
  \item[planeWaveInitialCondition]: See Section~\ref{sec:planeWaveIC}. 
  \item[gaussianPlaneWave]:
  \item[gaussianPulseInitialCondition]:
  \item[squareEigenfunctionInitialCondition]:  
  \item[annulusEigenfunctionInitialCondition]:
  \item[zeroInitialCondition]:
  \item[planeWaveScatteredFieldInitialCondition]:
  \item[planeMaterialInterfaceInitialCondition]:
  \item[gaussianIntegralInitialCondition]:   from Tom Hagstrom
  \item[twilightZoneInitialCondition]: for use with twilight-zone forcing. 
  \item[userDefinedInitialConditions] : see Section~\ref{sec:userDefinedInitialConditions}
\end{description}

{\bf To do:} The file {\tt userDefinedInitialConditions.C} can be used to define new initial conditions
for use with cgmx. 

% ------------------------------------------------------------------
\subsection{The plane-wave initial condition} \label{sec:planeWaveIC}

The plane wave initial condition is defined from the plane wave solution
\begin{align}
  \Ev &= \sin( 2\pi (\kv\cdot\xv - \omega t ))~\av, \label{eq:planeWaveE} \\
  \Hv &= \sin( 2\pi (\kv\cdot\xv - \omega t ))~\bv, \label{eq:planeWaveH} \\
  \omega & = c_m \vert \kv \vert, \\
  c_m &= { 1 \over \sqrt{ \epsilon \mu}}   \qquad \text{(speed of light in the material)} ~.
\end{align}
where $\kv=(k_x,k_y,k_z)$, $\vert \kv \vert = \sqrt{ k_x^2 + k_y^2 + k_z^2 }$,
$\av=(a_x,a_y,a_z)$, and $\bv=(b_x,b_y,b_z)$ satisfy 
\begin{align}
  \kv\cdot\av &=0, ~~\kv\cdot\bv=0, ~~\mbox{(from $\grad\cdot\Ev=0$ and $\grad\cdot\Hv=0$)}, \\
  \bv &= \sqrt{\frac{\epsilon}{\mu}}~{\kv\times\av \over \vert \kv \vert}, 
       ~~\mbox{(from $\mu \Hv_t = -\grad\times\Ev$)} .  \label{eq:bvFromav} 
\end{align}
Thus given $\av$ with $\kv\cdot\av =0$, $\bv$ is determined by~\eqref{eq:bvFromav}.
% 
% The plane wave initial condition is defined from the plane wave solution
% \begin{align}
%     \Ev_{\rm pw}(\xv,t) &= \av \sin( \kv\cdot\xv - c_{\rm pw} t ),  \label{eq:planeWaveE} \\
%     \Hv_{\rm pw}(\xv,t) &= \bv \sin( \kv\cdot\xv - c_{\rm pw} t )   \label{eq:planeWaveH}
% \end{align}
% The coefficient $\av$ should satisfy $\kv\cdot\av=0$. Given $\av$, $\bv$ can be determined from
% Maxwell's equations.
The parameters in the plane wave initial condition can be set with the commands:
\begin{verbatim}
  kx,ky,kz $kx $ky $kz
  plane wave coefficients $ax $ay $az $epsPW $muPW
\end{verbatim}
Here {\tt epsPW} and {\tt muPW} (which could be different from the values of $\eps$ and $\mu$ used
for the simulation) are used to define $c_{\rm pw}=1/\sqrt{ \eps_{\rm pw}\cdot \eps_{\rm pw}}$.

% -------------------------------------------------------------------
\subsection{The initial condition bounding box}

The initial condition bounding box can be used to restrict the region where the initial 
condition is assigned (see the example in Section~\ref{sec:knifeEdge2d}).
The command to define the box is 
\begin{verbatim}
  initial condition bounding box $xa $xb $ya $yb $za $zb
\end{verbatim}
This bounding box may only currently work with the plane wave initial condition.

The plane-wave initial conditions are turned off in a smooth way at the boundary of the box using the
function
\begin{align}
  g(\xv) = \half(1-\tanh(\beta 2\pi(k_x(x-x_0)+k_y(y-y_0)))).
\end{align}
where $x_0$ or $y_0$ a point on the edge of the box (usually the transition only happens on one face of the box and
it is assumed here that the  plane wave is parallel to this face). 
The coefficient $\beta$ in this function can be set with the command
\begin{verbatim}
  bounding box decay exponent $beta
\end{verbatim} // $

% -----------------------------------------------------------------------
\subsection{User defined initial conditions} \label{sec:userDefinedInitialConditions}

The file {\tt userDefinedInitialConditions.C} can be used to define new initial conditions
for use with Cgmx. Here are the steps to take to add your own initial conditions:
\begin{enumerate}
  \item Edit the file  {\tt cg/mx/src/userDefinedInitialConditions.C}.
  \item Add a new option to the function {\tt setupUserDefinedInitialConditions()}. Follow one of the previous
         examples. 
  \item Implement the initial condition option in the function {\tt userDefinedInitialConditions(...)}.
  \item Type {\em make} from the {\tt cg/mx} directory to recompile Cgmx.
  \item Run Cgmx and choose the {\em userDefinedInitalConditions} option from the {\em forcing options...} menu.
    (see the example command file {\tt mx/cmd/userDefinedInitialConditions.cmd}).
\end{enumerate}

{
\begin{figure}[hbt]
\newcommand{\figWidth}{5.5cm}
\newcommand{\trimfig}[2]{\trimFig{#1}{#2}{0.1}{0.05}{.05}{.05}}
\begin{center}
\begin{tikzpicture}[scale=1]
  \useasboundingbox (0,.5) rectangle (17,5.75);  % set the bounding box (so we have less surrounding white space)
  \draw ( 0.0, 0) node[anchor=south west] {\trimfig{figures/userDefinedInitialConditionGaussianPulsesBoxEx}{\figWidth}};
  \draw ( 5.7, 0) node[anchor=south west] {\trimfig{figures/userDefinedInitialConditionGaussianPulsesBoxEy}{\figWidth}};
  \draw (11.4, 0) node[anchor=south west] {\trimfig{figures/userDefinedInitialConditionGaussianPulsesBoxEz}{\figWidth}};
 % - labels
 %   \draw (\txa,4.75) node[draw,fill=white,anchor=east] {\scriptsize $t=0.5$};
 %   \draw (\txb,4.75) node[draw,fill=white,anchor=east] {\scriptsize $t=1.0$};
 %   \draw (\txc,4.75) node[draw,fill=white,anchor=east] {\scriptsize $t=1.5$};
 %  \draw (current bounding box.south west) rectangle (current bounding box.north east);
% grid:
%  \draw[step=1cm,gray] (0,0) grid (17.0,5);
\end{tikzpicture}
\end{center}
\caption{User defined initial conditions. Results for two Gaussian pulses in a box.}
\label{fig:userDefinedInitialConditionsGaussianPulses}
\end{figure}
}
