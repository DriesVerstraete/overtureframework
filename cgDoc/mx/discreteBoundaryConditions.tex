\section{Discrete Boundary Conditions}

\newcommand{\pra}{\partial_{r_1}}
\newcommand{\prb}{\partial_{r_2}}
\newcommand{\prc}{\partial_{r_3}}

For perfect electrical conductors (PEC) the boundary conditions for the electric field are
\begin{align}
  \nv\times\Ev & = 0
\end{align}
If we let $\tauv_m$, $m=1,d-1$ denote the tangent vectors at the boundary then
\[
  \tauv_m\cdot\Ev=0
\]
Since $\grad\cdot\Ev=0$ we can also use the condition
\[
  {1\over J}\left\{ \pra( \av_1\cdot\Ev) + \prb( \av_2\cdot\Ev) + \prc( \av_3\cdot\Ev) \right\} =0
\]
Recall that
\begin{align*}
   \av_m &= J \grad_{\xv} r_m
\end{align*}
Now $\av_1$ is always proportional to the normal to the boundary $r=\mbox{constant}$. 
On an orthogonal grid $\av_2$ and $\av_3$ lie in the tangent plane and thus
$\av_2\cdot\Ev$ and $\av_3\cdot\Ev$ can be written as linear combinations of the
tangential tangential components of $\Ev$, (on a boundary $r=\mbox{constant}$) then
the divergence condition becomes
\[
  \pra( \av_1\cdot\Ev ) =0 \qquad \mbox{at $r=0$}.
\]

\subsection{Boundary conditions for the magnetic field}

In two-dimensions we have the equations
\begin{align*}
\partial_t E_x - {1\over \eps} \partial_y H_z &=0 \\
\partial_t E_y + {1\over \eps} \partial_x H_z &=0 \\
\partial_t H_z + {1\over \mu } \left[ \partial_x E_y - \partial_y E_x \right] &=0 
\end{align*}
Taking the dot product of the tangent with the first two equations gives
\begin{align*}
  \eps (\tau\cdot(E_x,E_y))_t & = ( \tau\cdot(\partial_y H_z,-\partial_x H_z) ) \\
                              & = \partial_n H_z
\end{align*}
Thus we have the PEC boundary condition for the magnetic field
\[
   \partial_n H_z = 0 
\]

In the general three-dimensional case we have
\begin{align*}
  \eps (\nv\times\Ev)_t &= \nv\times(\grad\times\Hv) \\
                   &= \nv\cdot(\grad\Hv) - (\nv\cdot\grad)\Hv \\
                   &= n_j \partial_i H_j - n_j \partial_j H_i \qquad\mbox{(component form)}
                   &= \tau_m\cdot(\grad\times\Hv) \qquad\mbox{($\tau_m$ is a tangent vector)}
\end{align*}
We also have
\begin{align*}
 \mu \nv\cdot\Hv_t & = -\nv\cdot(\grad\times\Ev) \\
                   &= \grad\cdot(\nv\times\Ev) - \Ev\cdot(\grad\times\nv) \\
                   &= {1\over J}\sum_m \partial_{r_m}\big\{ \av_m\cdot(\nv\times\Ev)\big\}
                                                  - \Ev\cdot(\grad\times\nv) \\
\end{align*}

Thus in 3D the boundary conditions for $\Hv$ are
\begin{align*} 
  \nv\cdot\Hv &= 0 \\
  \nv\cdot(\grad\Hv) - \partial_n \Hv &= 0
\end{align*}
or 
\begin{align*} 
  n_j H_j  &= 0 \\
  n_j \partial_i H_j - n_j \partial_j H_i &= 0 \\
\end{align*}

\subsection{Second-order accurate boundary conditions}

For a second-order accurate scheme we can use (at the boundary $i=0$)
\begin{align*}
    \tauv_m\cdot\Ev_{0,j}  &=0  \\
     (\av_1\cdot\Ev)_{-1,j} &= (\av_1\cdot\Ev)_{+1,j} \\
    \tauv_m\cdot\Ev_{-1,j} &= \tauv_m \cdot (I-D_+^2)\Ev_{-1,j} \equiv g_m   \qquad\mbox{(extrapolate)}\\
     w_{-1,j} &= w_{+1,j}
\end{align*}
We use the extrapolation, $\tauv_m\cdot D_+^2\Ev_{-1,j}=$, since this is consistent with the
tangential component being an odd function and thus the second-derivative (second-difference)
is expected to be zero.

With our assumption of an orthogonal grid is thus follows that electric field a the 
ghost point is given by
\begin{align*}
\Ev_{-1,j} &= (\av_1\cdot\Ev)_{+1,j} ~{\av_1\over \| \av_1 \|^2} + g_m \tauv_m
\end{align*}


\subsection{Fourth-order accurate boundary conditions}

Since $\tauv_m\cdot \Ev=0$ along the boundary it follows that $\tauv_m\cdot \Ev_{tt}=0$ and thus
\[
   \tauv_m \cdot\Delta \Ev = 0 \quad\mbox{on the boundary}
\]
Taking two time derivatives of the divergence equation
\[
  \pra\big(\av_1\cdot \Ev_{tt} \big) + \prb\big(\av_2\cdot \Ev_{tt}\big) + \prc\big(\av_3\cdot \Ev_{tt}  \big) =0 
\]
And using the fact that $\prb\big(\av_2\cdot \Ev_{tt}\big)$ and 
$\prc\big(\av_3\cdot \Ev_{tt}\big)$ are zero on the boundary
(If $\av_m$, $m=2,3$ are in the tangent plane, which is true on an orthogonal grid) it follows that
the following condition holds on the boundary,
\begin{equation}
  \partial_r\big(\av_1\cdot \Delta\Ev \big) (0,r_2,r_3) =0   \label{eq:divtt}
\end{equation}
Now on an orthogonal grid the Laplacian can be written as
\begin{align}
   \Delta  &= \sum_{m,n=1}^d  c_{mn} \partial_{r_m}\partial_{r_n} + \sum c_m \partial_{r_m} \\
           &= c_{11} \pra^2 + c_{22}\prb^2 + c_{33}\prc  + 
              c_{12} \pra\prb + c_{13}\pra\prc + c_{23}\prb\prc  +
              c_{1} \pra + c_{2} \prb+ c_{3} \prc
\end{align}
(where we have neglected the term $c_{12}\partial_r\partial_s$).
Whence equation~(\ref{eq:divtt}) can be written out in detail as
\begin{align}
   \partial_r\big(\av_1\cdot \Delta\Ev \big) &= 
  \av_1\cdot\Big( \sum_{m,n=1}^d  c_{mn} \partial_r\partial_{r_m}\partial_{r_n} 
                   + \sum c_m \partial_r\partial_{r_m} \label{eq:DrADotuv} \\
 ~~~&+ \sum_{m,n=1}^d  \partial_r(c_{mn}) \partial_{r_m}\partial_{r_n} + \sum \partial_r(c_m) \partial_{r_m} \Big) 
   +  \partial_r\av_1\cdot \Delta\uv \\
=  \av_1\cdot\Big( &c_{11} \uv_{rrr} + c_{22}\uv_{rss} + c_{33}\uv_{r\rc\rc} 
             + c_{1}\uv_{rr} + c_{2}\uv_{rs} + c_{3}\uv_{r\rc}  \label{eq:divtte}\\
                  &\partial_r c_{11} \uv_{rr} + \partial_r c_{22}\uv_{ss} + \partial_r c_{33}\uv_{\rc\rc} 
                   + \partial_r c_{10}\uv_{r} + \partial_r c_{01}\uv_{s}+ \partial_r c_{3}\uv_{\rc} \Big) \\
   &  + \partial_r\av_1\cdot \Delta\uv = 0 \qquad\mbox{(on the boundary, $r=0$)}
\end{align}
We wish to avoid a numerical boundary that includes mixed derivative terms such as those involving
$\uv_{rs}$ or $\uv_{rss}$ in the above expression. These mixed derivatives would make the 
boundary condition non-local by coupling adjacent points. We can, however, find expressions
for these mixed derivative expressions in terms of non-mixed derivatives.

\subsubsection{Two dimensions}

First consider the two-dimensional case.
Taking the $r$- and $s$-derivatives of the divergence equation gives
\begin{align*}
   \partial_r^2(\av_1\cdot\uv) + \partial_r\partial_s(\av_2\cdot\uv) &=0 \\
   \partial_r\partial_s(\av_1\cdot\uv) + \partial_s^2(\av_2\cdot\uv) &=0
\end{align*}
or
\begin{align*}
 \av_2\cdot\uv_{rs} &= -\big\{ \partial_r^2(\av_1\cdot\uv) + \partial_r\partial_s(\av_2)\uv 
                  + \partial_r(\av_2)\uv_s + \partial_s(\av_2)\uv_r\big\}   \\
 \av_1\cdot\uv_{rs} &= -\big\{ \partial_s^2(\av_2\cdot\uv) + \partial_r\partial_s(\av_1)\uv 
                     + \partial_r(\av_1)\uv_s + \partial_s(\av_1)\uv_r  \big\} 
\end{align*}
Since $\av_1\cdot\av_2\ne 0$, we can solve these expressions for $\uv_{rs}$,
\begin{equation}
   \uv_{rs} = \Fv_1(\uv_r,\uv_s,\uv_{rr},\uv_{ss})  \label{eq:urs}
\end{equation}

Now $\partial_s(\grad\cdot\uv)=0$ implies
\[
   \av_1 \cdot \uv_{rs} = -\Big( \partial_s\av_1\cdot\uv_r +
       \partial_s( \partial_r\av_1\cdot\uv) + \partial_s^2(\av_2\cdot\uv)+ \partial_s\prc(\av_3\cdot\uv) \Big)
\]
This last expression can be differentiated with respect to $s$ to give an equation
for 
\begin{equation}
\av_1 \cdot \uv_{rss} =  -\Big( \partial_s\av_1\cdot\uv_{rs} + \partial_s^2\av_1\cdot\uv_{r} +
       \partial_s^2( \partial_r\av_1\cdot\uv) + \partial_s^3(\av_2\cdot\uv)
           + \partial_s^2\prc(\av_3\cdot\uv) \Big)  \label{eq:urss}
\end{equation}

Combining equations~(\ref{eq:divtte}), (\ref{eq:urs}), and (\ref{eq:urss}) allows us to rewrite
the boundary condition~(\ref{eq:divtte}) without any mixed derivatives,
\begin{equation}
  \partial_r\big(\av_1\cdot \Delta\Ev \big) =
    \bv_3\cdot\uv_{rrr} + \bv_2\cdot\uv_{rr} + \bv_1\cdot\uv_{r} = G(\uv_s,\uv_{ss},\uv_{sss})
\end{equation}

\noindent{\bf Magnetic field in two dimensions}:

In two-dimensions the boundary condition for the magnetic field, $w$, is
\begin{align*}
    \partial_n w & = \nv\cdot(\grad w) \\
                 & = (n_1,n_2)\cdot(r_x w_r + s_x w_s, r_y w_r + s_y w_s ) \\
                 & = (n_1 r_x + n_2 r_y ) w_r + (n_1 s_x + n_2 s_y ) w_s 
\end{align*}
On a boundary $r=0$, $\nv=\grad_\xv r/\| \grad_\xv r \|$ and then
\begin{align*}
  \partial_n w & = {(r_x^2 + r_y^2) \over \| \grad_\xv r \|} w_r + {(r_x s_x + r_y s_y )\over \| \grad_\xv r \|}  w_s 
\end{align*}
Thus the Neumann boundary condition for $w$ can be written as 
\begin{align*}
  w_r &= - {\grad_\xv r\cdot\grad_\xv s \over   \| \grad_\xv r \|^2 } w_s 
\end{align*}
On an orthogonal grid the right-hand-side is zero.
Assume now that the grid is orthogonal and $w_r(0,s)=0$, then 
\begin{align*}
  \partial_t^2 w_r &= \partial_r\Delta w \\
   &= \partial_r\big\{ c_{11} \partial_r^2 w+ c_{22}\partial_s^2 w+ c_{10} \partial_r w+ c_{01} \partial_s w\big\}\\
   &= c_{11} \partial_r^3 w+ c_{22}\partial_r\partial_s^2 w+ c_{10} \partial_r^2 w+ c_{01} \partial_r\partial_s w \\
   &~~+ (\partial_r c_{11}) \partial_r^2 w+ (\partial_r c_{22})\partial_s^2 w
      + (\partial_r c_{10}) \partial_r w+ (\partial_r c_{01}) \partial_s w
\end{align*}
giving the BC
\begin{align*}
c_{11} \partial_r^3 w + (c_{10}+\partial_r c_{11})\partial_r^2 w &=
   -\big\{ (\partial_r c_{22})\partial_s^2 w + (\partial_r c_{01}) \partial_s w \big\} \qquad\mbox{at $r=0$}
\end{align*}



We are thus led to the discrete boundary conditions for the fourth-order accurate scheme
\begin{align*}
    \big(\tauv_m\cdot\Ev\big)_{0,j}  &=0  \\
 \tauv_m\cdot\Delta_{2h} \Ev_{0,j}  &=0  \\
 \tauv_m\cdot\Ev_{-2,j} &= \tauv_m \cdot (I-D_+^4)\Ev_{-2,j} \equiv g_m   \quad\mbox{(extrapolate)} \\
  D_{0r}\big(1- (\Delta r)^2/6 D_{+r}D_{-r}\big) \big(\av_1\cdot\Ev\big)_{0,j} &= 0 \\
%  (D_{0r}D_{+r}D_{-r}~\av_1\cdot \Ev)_{0,j} + (\bv\cdot D_{0r}\Ev)_{0,j} &= G_1(\uv_s,\uv_{ss},\uv_{sss})
  \bv_3\cdot(D_{0r}D_{+r}D_{-r}~\Ev) + \bv_2\cdot\uv_{rr} + \bv_1\cdot\uv_{r} &= G(\uv_s,\uv_{ss},\uv_{sss})\\
%  D_{0r} (\av_1\cdot \Delta_{2h}\Ev )_{0,j} &= 0 
  D_{0r}\big(1- (\Delta r)^2/6 D_{+r}D_{-r}\big) w_{0,j} &= 0 \\
  c_{11}~D_{0r}D_{+r}D_{-r}~w + (c_{10}+\partial_r c_{11})D_{+r}D_{-r}~w &=
        -\big\{ (\partial_r c_{22})D_{+s}D_{-s}~w + (\partial_r c_{01})D_{0s}~w \big\}       
\end{align*}
% where $\Delta_{2h}$ is a second order accurate approximation to the Laplacian operator.


\subsubsection{Three dimensions}

Now consider the three-dimensional situation which is more complicated.
We need expressions for the mixed derivatives taht appear in the expression for $\pra(\av_1\cdot\Delta\uv)=0$,
equation~(\ref{eq:DrADotuv}),
\begin{align}
   &\av_1\cdot\uv_{rs}, \\
   &\av_1\cdot\uv_{r\rc}, \\
   &\av_1\cdot\uv_{rss}, \\
   &\av_1\cdot\uv_{r\rc\rc},
\end{align}

The divergence equation implies
\[
  \av_1\cdot\uv_r = -\big\{ (\partial_r\av_1)\cdot\uv + \prb(\av_2\cdot\uv)+\prc(\av_3\cdot\uv) \big\}
\]

Taking the $r$, $s$, and $\rc$ -derivatives of the divergence equation gives
\begin{align*}
   \partial_r^2(\av_1\cdot\uv) + \partial_r\partial_s(\av_2\cdot\uv)+ \partial_r\partial_\rc(\av_3\cdot\uv) &=0 \\
   \partial_r\partial_s(\av_1\cdot\uv) + \partial_s^2(\av_2\cdot\uv)+ \partial_s\prc(\av_3\cdot\uv) &=0 \\
   \partial_r\prc(\av_1\cdot\uv) + \partial_s\prc(\av_2\cdot\uv)+ \prc^2(\av_3\cdot\uv) &=0
\end{align*}
In this 3d case the first equation only gives a relation for 
\begin{align*}
  \av_2\cdot\uv_{rs} + \av_3\cdot\uv_{r\rc} &= -\big\{ \partial_r^2(\av_1\cdot\uv) + ... \big\} 
\end{align*}
as opposed to the 2d case where this equation determined the tangential component of $\uv_{rs}$. This means
the 3d case is more involved.

From $\prb(\grad\cdot\uv)=0$ and $\prc(\grad\cdot\uv)=0$ we get the normal components of the
mixed derivatives,
\begin{align*}
  \av_1\cdot\uv_{rs} &= -\big\{ \prb\av_1\cdot\uv_r + \prb( (\partial_r\av_1)\cdot\uv )
                + \prb^2(\av_2\cdot\uv)+\prb\prc(\av_3\cdot\uv)\big\} \\
  \av_1\cdot\uv_{r\rc} &= -\big\{ \prc\av_1\cdot\uv_r + \prc( (\partial_r\av_1)\cdot\uv )
               + \prb\prc(\av_2\cdot\uv)+\prc^2(\av_3\cdot\uv)\big\} \\
\end{align*}
and thus we have $\av_1\cdot\uv_{rs}$ and $\av_1\cdot\uv_{r\rc}$ defined in terms of tangential derivatives.

From $\prb^2(\grad\cdot\uv)=0$ and $\prc^2(\grad\cdot\uv)=0$ we also get 
\begin{align*}
  \av_1\cdot\uv_{rss} &= -\big\{ \prb\av_1\cdot\uv_{rs} +
            \prb\big( \prb\av_1\cdot\uv_r + \prb( (\partial_r\av_1)\cdot\uv )
                  + \prb^2(\av_2\cdot\uv)+\prb\prc(\av_3\cdot\uv) \big) \big\} \\
  \av_1\cdot\uv_{r\rc\rc} &= -\big\{ \prc\av_1\cdot\uv_{r\rc} +
                \prc\big(  \prc\av_1\cdot\uv_r + \prc( (\partial_r\av_1)\cdot\uv )
                   + \prb\prc(\av_2\cdot\uv)+\prc^2(\av_3\cdot\uv) \big) \big\} \\
\end{align*}
The only difficult terms on the RHS of these
 last two expressions are $\prb\av_1\cdot\uv_{rs}$ and $\prc\av_1\cdot\uv_{r\rc}$.

In two-dimensions we could solve explicitly for $\uv_{rs}$ just from the divergence equation and it's derivatives.
In three-dimensions we need more information. Since we know $\av_1\cdot\uv_{rs}$ we can compute
$\uv_{rs}$ if we know $\tauv_m\cdot\uv_{rs}$ for $m=1,2$.

Now letting $\tauv=\tauv_m$ then
\begin{align*}
   \tauv\cdot\uv_{rs} &= (\tauv\cdot\uv_r)_s - \tauv_s\cdot\uv_r
\end{align*}
The boundary conditions for $\tauv\cdot\uv$ will allow us to compute $\tauv\cdot\uv_r(r,s,\rc)$
on the boundary and at all ghost points  and thus we
can determine an approximation to $(\tauv\cdot\uv_r)_s$. For example a second-order approximation
should be sufficient,
\begin{align*}
  (\tauv\cdot\uv_r)_s(0,\rb,\rc) &:= G^{rs}(s,\rc) \\
                                &\approx D_{0s}\big( \tauv_\iv\cdot (D_{0r}\Uv_\iv) \big)
\end{align*}
% For example, a second-order approximation 
% \begin{align*}
%   \tauv\cdot\uv(\dra,s,\rc) &= \tauv\cdot\uv(0,s,\rc) + \dra \tauv\cdot\uv_r(0,s,\rc) 
%                      + {\dra^2\over2} \tauv\cdot\uv_{rr}(0,s,\rc) + O(\dra^3)
% \end{align*}
% implies
% \begin{align*}
% \tauv\cdot\uv_r(0,s,\rc) &= {\tauv\cdot\uv(\dra,s,\rc) - \tauv\cdot\uv(0,s,\rc) \over \dra} 
%                - {\dra\over2} \tauv\cdot\uv_{rr}(0,s,\rc)  + O(\dra^2)
% \end{align*}
Thus by first computing the tangential components we then can determine $(\tauv\cdot\uv_r)_s(0,s,\rc)$ and thus
we can assume that we know $\tauv\cdot\uv_{rs}$ in terms of $\tauv_s\cdot\uv_r$ and a known function:
\begin{align*}
\tauv\cdot\uv_{rs} &= - \tauv_s\cdot\uv_r + G^{rs} \\
                 G^{rs}(s,\rc) &= (\tauv\cdot\uv_r)_s(0,s,\rc) \quad\mbox{(known) from tangential components}
\end{align*}
Similiary we get
\begin{align*}
\tauv\cdot\uv_{r\rc} &= - \tauv_\rc\cdot\uv_r + G^{r\rc} \\
                 G^{r\rc}(s,\rc) &= (\tauv\cdot\uv_r)_\rc(0,s,\rc) \quad\mbox{(known) from tangential components}
\end{align*}

Combining these results allows us to rewrite
the boundary condition~(\ref{eq:divtte}) in the form 
\begin{equation}
  \partial_r\big(\av_1\cdot \Delta\Ev \big) =
    \bv_3\cdot\uv_{rrr} + \bv_2\cdot\uv_{rr} + \bv_1\cdot\uv_{r} 
        = G(\uv_s,\uv_{ss},\uv_\rc,\uv_{\rc\rc},\uv_{s\rc}, (\tauv\cdot\uv_r)_s, (\tauv\cdot\uv_r)_\rc)
\end{equation}


We are thus led to the discrete boundary conditions for the fourth-order accurate scheme in three-dimensions
\begin{align}
    \big(\tauv_m\cdot\Ev\big)_{0,j}  &=0 \quad m=1,2 \\
 \tauv_m\cdot\Delta_{2h} \Ev_{0,j}  &=0 \quad m=1,2 \\
 \tauv_m\cdot\Ev_{-2,j} &= \tauv_m \cdot (I-D_+^4)\Ev_{-2,j} \equiv g_m , \quad m=1,2  \quad\mbox{(extrapolate)} \\
  D_{0r}\big(1- (\Delta r)^2/6 D_{+r}D_{-r}\big) \big(\av_1\cdot\Ev\big)_{0,j} &= 0 \\
  \bv_3\cdot(D_{0r}D_{+r}D_{-r}~\Ev) + \bv_2\cdot\uv_{rr} + \bv_1\cdot\uv_{r} 
                &= G(\uv_s,\uv_{ss},\uv_{\rc},\uv_{\rc\rc},\uv_{s\rc},(\tauv\cdot\uv_r)_s, (\tauv\cdot\uv_r)_\rc)
\end{align}
To summarize, the first three equations can be used to determine the tangential components
 on the boundary and two ghost lines,
\[
    \tauv_m\cdot\Ev_{i_1,i_2,i_3} \qquad \i_1=0,-1,-2
\]
The last two equations then determine the normal components on the two ghost lines,
\[
    \av_1\cdot\Ev_{i_1,i_2,i_3} \qquad \i_1=-1,-2
\]



% This last expression can be differentiated with respect to $s$ to give
% \begin{equation}
%    \uv_{rss} = \Fv_2(\uv_r,\uv_s,\uv_{rr},\uv_{ss})  \label{eq:urs}
% \end{equation}

% 
% For fourth order accuracy we 
% combine the boundary conditions with $\Ev_{tt} = c^2 \Delta \Ev$ to give the addition conditions.
% Differentiation of the boundary conditions twice with respect to time gives
% \begin{align*}
%    \big(\tauv_m\cdot \Ev_{tt}\big)_{0,j} &=0 \\
%    \partial_r\big(\av_1\cdot \Ev_{tt} \big)_{0,j} &= 0
% \end{align*}
% which implies
% \begin{align*}
%   \tauv_m\cdot\Delta \Ev_{0,j}  &=0  \\
%   \partial_r(\av_1\cdot \Delta\Ev )_{0,j} &= 0
% \end{align*}
% 
% 
% These conditions can be simplified to make them more tractable for numerical computations.
% \begin{align*}
%     \big(\tauv_m\cdot\Ev\big)_{0,j}  &=0  \\
%  \tauv_m\cdot\Delta_{2h} \Ev_{0,j}  &=0  \\
%  \tauv_m\cdot\Ev_{-2,j} &= \tauv_m \cdot (I-D_+^4)\Ev_{-2,j} \equiv g_m   \quad\mbox{(extrapolate)} \\
%   D_{0r}\big(1- (\Delta r)^2/6 D_{+r}D_{-r}\big) \big(\av_1\cdot\Ev\big)_{0,j} &= 0 \\
%   D_{0r} (\av_1\cdot \Delta_{2h}Ev )_{0,j} &= 0 
% \end{align*}
% 
% 
% 
% 
% Rather than discretize directly this last equation we can instead take two $r-$derivatives
% of the divergenec equation,
% \[
%    \partial_r^3( \av_1\cdot\uv) + \partial_r^2\partial_s( \av_2\cdot\uv ) =0
% \]
% Then 
% \begin{align}
% \partial_r^3( \av_1\cdot\uv) &= - \partial_r^2\partial_s( \av_2\cdot\uv ) \\
%                              &:= - G \\
%   G & := \av_2\cdot\uv_{rrs} + ... + \partial_r^2\partial_s(\av_2)\cdot\uv
% \end{align}
% We can determine $G$ using the interior equation. 
% 
% Now,
% \begin{align}
%   \uv_{rr}  &= -\big\{ c_1\uv_r +  c_{22}\uv_{ss} + c_2 \uv_{s} + \uv_{tt} \big\} c_{11}^{-1}
% \end{align}
% and thus 
% \begin{align*}
%   \uv_{rrs} &= -\big\{ (\partial_s c_1)\uv_r + c_1\uv_{rs} 
%                      + (\partial_s c_{22}) \uv_{ss} + c_{22}\uv_{sss}
%                      + (\partial_s c_2) \uv_{s} + c_2 \uv_{ss} + \uv_{stt}
%                     + \big\} c_{11}^{-1}   \\
%             &\quad + \big\{ c_1\uv_r +  c_{22}\uv_{ss} + c_2 \uv_{s} + \uv_{tt} \big\} (\partial_s c_{11}) c_{11}^{-2}
% \end{align*}
% In this expression we need to determine $\uv_{rs}$. We get these mixed derivatives by
% taking derivative of the divergence equation,
% \begin{align*}
%    \partial_r^2(\av_1\cdot\uv) + \partial_r\partial_s(\av_2\cdot\uv) &=0 \\
%    \partial_r\partial_s(\av_1\cdot\uv) + \partial_s^2(\av_2\cdot\uv) &=0 \\
% \end{align*}
% We solve these last two equations to give the mixed derivatives in terms of the unmixed derivatives.
% \begin{equation}
%   \uv_{rs} = F_3(\uv_r,\uv_s,\uv_{rr},\uv_{ss}) \label{eq:mixedDerivatives}
% \end{equation}
% % In particular, in 2D we have
% % \begin{align*}
% %   u_{rs}& ={\frac {{{\it a12}}^{2}{\it vrr}-{{\it a22}}^{2}{\it vss}-{\it a11rs}\,u{\it a22}-{\it a11r}\,{\it us}\,{\it a22}-{\it a12s}\,{\it vr
% % }\,{\it a22}-{\it a22ss}\,v{\it a22}-{\it a21ss}\,u{\it a22}-2\,{\it a21s}\,{\it us}\,{\it a22}-{\it a21}\,{\it uss}\,{\it a22}-2\,{
% % \it a22s}\,{\it vs}\,{\it a22}-{\it a12rs}\,v{\it a22}-{\it a12r}\,{\it vs}\,{\it a22}+2\,{\it a12}\,{\it a11r}\,{\it ur}+{\it a12}\,
% % {\it a11rr}\,u+{\it a12}\,{\it a11}\,{\it urr}+2\,{\it a12}\,{\it a12r}\,{\it vr}+{\it a12}\,{\it a12rr}\,v+{\it a12}\,{\it a22r}\,{
% % \it vs}+{\it a12}\,{\it a22s}\,{\it vr}+{\it a12}\,{\it a22rs}\,v+{\it a12}\,{\it a21r}\,{\it us}+{\it a12}\,{\it a21s}\,{\it ur}+{
% % \it a12}\,{\it a21rs}\,u-{\it a11s}\,{\it ur}\,{\it a22}}{-{\it a21}\,{\it a12}+{\it a11}\,{\it a22}}} \\
% %   v_{rs} &=-{\frac {{\it a11}\,{\it a12}\,{\it vrr}+{\it a11}\,{\it a22r}\,{\it vs}+{\it a11}\,{\it a22s}\,{\it vr}+{\it a11}\,{\it a22rs}\,v+{
% % \it a11}\,{\it a21r}\,{\it us}+{\it a11}\,{\it a21s}\,{\it ur}+{\it a11}\,{\it a21rs}\,u+2\,{\it a11}\,{\it a12r}\,{\it vr}-{\it a21}
% % \,{\it a11s}\,{\it ur}-{\it a21}\,{\it a11r}\,{\it us}-{\it a21}\,{\it a11rs}\,u-{\it a21}\,{\it a22}\,{\it vss}-2\,{\it a21}\,{\it
% % a22s}\,{\it vs}-2\,{\it a21}\,{\it a21s}\,{\it us}+{\it a11}\,{\it a12rr}\,v-{\it a21}\,{\it a21ss}\,u-{\it a21}\,{\it a22ss}\,v-{
% % \it a21}\,{\it a12s}\,{\it vr}-{\it a21}\,{\it a12r}\,{\it vs}-{\it a21}\,{\it a12rs}\,v+{{\it a11}}^{2}{\it urr}+2\,{\it a11}\,{\it
% % a11r}\,{\it ur}-{{\it a21}}^{2}{\it uss}+{\it a11}\,{\it a11rr}\,u}{-{\it a21}\,{\it a12}+{\it a11}\,{\it a22}}}
% % \end{align*}
% 
% 
% We thus have expressions
% \begin{align*}
%   \uv_{rrs} &= F_1(\uv_r,\uv_s,\uv_{rr},\uv_{rs},\uv_{ss},\uv_{rrs},\uv_{sss}) \\
%   \uv_{rr} &= F_2(\uv_r,\uv_s,\uv_{ss}) \\
%   \uv_{rs} &= F_3(\uv_r,\uv_s,\uv_{rr},\uv_{ss})
% \end{align*}
% which allows us to convert $\uv_{rrs}$, $\uv_{rr}$ and $\uv_{rs}$ to depend on $\uv_r$ and $s-$derivatives.
% 
%  Therefore we can write $G$ as
% \begin{align*}
%    G &= -\bv\cdot\uv_r + G_1(\uv_s,\uv_{ss},\uv_{sss})
% \end{align*}
% whence
% \begin{align*}
% \partial_r^3( \av_1\cdot\uv) + \bv\cdot\uv_r &= G_1(\uv_s,\uv_{ss},\uv_{sss})
% \end{align*}
% is a boundary condition we can use to determine $\av_1\cdot\uv$ for $r<0$.



\subsection{Sixth- and higher-order accurate boundary conditions}

For sixth and higher order accurate boundary conditions we proceed in the same fashion
as for the fourth order case. 

By differentiating the boundary conditions $2m$ times with respect to time and using the 
interior PDE, 
is follows that for $m=0,1,2,3\ldots$,
\begin{align*}
  \tauv\cdot\Delta^m \Ev_{0,j}  &=0  \\
  \partial_r(\av_1\cdot \Delta^m\Ev )_{0,j} &= 0
\end{align*}


The Laplacian squared is
\begin{align*}
   \Delta^2  &= \Big(c_{11} \partial_r^2 + c_{22}\partial_s^2  + c_{10} \partial_r + c_{01} \partial_s \Big)^2 \\
          &= d_{1111}\partial_r^4 + d_{2222}\partial_s^4
             + d_{1122} \partial_r^2\partial_s^2
             + d_{1112} \partial_r^3\partial_s + d_{1222} \partial_r\partial_s^3 \\
        &   + d_{111}\partial_r^3 + d_{222}\partial_s^3
             + d_{112} \partial_r^2\partial_s + d_{122} \partial_r\partial_s^2 \\
    &  +d_{11} \partial_r^2 +d_{12} \partial_r\partial_s + d_{22}\partial_s^2  + d_{10} \partial_r + d_{01} \partial_s 
 + c_{10} \partial_r + c_{01} \partial_s 
\end{align*}
We would like to determine expressions at $r=0$ for the mixed-derivatives that appear in this equation.
We already know $\uv_{rs}$.

We make use of 
\[
 \partial_s\big(\tauv\cdot\uv_{tt}\big)(0,s) = 0  
\]
to give
\begin{align*}
  \partial_s\big(\tauv\cdot\uv_{tt}\big)(0,s) &= \partial_s\big(\tauv\cdot\Delta\uv\big)(0,s) \\
     &= \tauv\cdot\Big\{ c_{11} \uv_{rrs} +  c_{10} \uv_{rs}  + c_{22}\partial_s^3\uv  + c_{01} \partial_s^2\uv  \\
     &~~+  \partial_s(c_{11})\partial_r^2\uv + \partial_s(c_{22})\partial_s^2\uv  + 
         \partial_s(c_{10})\partial_r\uv + \partial_s(c_{01}) \partial_s \uv \Big\}
\end{align*}
Whence we know
\begin{equation}
  \tauv\cdot\uv_{rrs} = F_{112}( \uv_{rr}, \uv_{r}, \uv_{sss}, \uv_{ss}, \uv_s) \label{eq:rrs}
\end{equation}
In the same manner, from $\partial_s^2\big(\tauv\cdot\uv_{tt}\big)(0,s) = 0 $
we can determine
 \begin{equation}
  \tauv\cdot\uv_{rrss} = F_{1122}( \uv_{rr}, \uv_{r}, \uv_{ssss}, \uv_{sss}, \uv_{ss}, \uv_s)  \label{eq:rrss}
\end{equation}
We also need $\tauv\cdot\partial_r^3\partial_s\uv$ and $\tauv\cdot\partial_r\partial_s^2\uv$

To get $\tauv\cdot\partial_r^3\partial_s\uv$ we use the divergence condition (cf. $v_{xxxy}=-u_{xxxx}$),
\begin{align*}
  \partial_r^3(\delta) &= \partial_r^4(\av_1\cdot\uv) +\partial_r^3\partial_s(\av_2\cdot\uv) \\
   \partial_r^2\partial_s(\delta) &= \partial_r^3\partial_s(\av_1\cdot\uv) +\partial_r^2\partial_s^2(\av_2\cdot\uv)
\end{align*}
gives
\begin{align*}
\av_2\cdot(\partial_r^3\partial_s\uv) &= -\big\{ \partial_r^4(\av_1\cdot\uv) +\partial_r^3\partial_s(\av_2)\uv
                         +\partial_s(\av_2)\partial_r^3\uv + \ldots \big\} \\
\av_1\cdot(\partial_r^3\partial_s\uv) &= -\big\{ \partial_r^2\partial_s^2(\av_2\cdot\uv)
               \partial_r^3\partial_s(\av_1)\uv + \partial_r^3 (\av_1)\partial_s\uv + \ldots    \big\}
\end{align*}
giving
\begin{align*}
\partial_r^3\partial_s\uv &= \Fv_{1112}(\partial_r\uv,\partial_s\uv, \ldots)
\end{align*}





Discrete boundary conditions for the sixth-order accurate scheme are
\begin{align*}
    \big(\tauv_m\cdot\Ev\big)_{0,j}  &=0  \\
 \tauv_m\cdot\Delta_{4h} \Ev_{0,j}  &=0  \\
 \tauv_m\cdot\Delta_{2h}^2 \Ev_{0,j}  &=0  \\
  \tauv_m\cdot\Ev_{-3,j} &= \tauv_m \cdot (I-D_+^7)\Ev_{-3,j} \equiv g_m   \quad\mbox{(extrapolate)}\\
  D_{0r}\big(1- {(\Delta r)^2\over 6} D_{+r}D_{-r}  
                  \pm {(\Delta r)^4\over 120} D_{+r}^2D_{-r}^2 \big)
                          \big(\av_1\cdot\Ev\big)_{0,j} &= 0 \\
  D_{0r}\big(\av_1\cdot \Delta_{4h}Ev \big)_{0,j}
    -{(\Delta r)^2\over 6} D_{+r}D_{-r}\big( \av_1\cdot \Delta_{2h}Ev\big)_{0,j} &= 0 \\
  D_{0r} (\av_1\cdot \Delta_{2h}^2 Ev )_{0,j} &= 0 
\end{align*}
where $\Delta_{2h}$ is a second order accurate approximation to the Laplacian operator.

\subsection{Boundary conditions for the magnetic field}
For the magnetic field we have
\begin{align*}
  \nv\cdot\Hv &= 0 \\
  \grad\cdot\Hv &=0 \\
\end{align*}

\clearpage
\subsection{Extended boundaries in two-dimensions}

Consider a corner on a two-dimensional orthogonal curvilinear grid at $\rv=0$.
We need to determine values along the extended boundaries (i.e. $s=0$, $r<0$ and $r=0$, $s<0$).


We have
\begin{align*}
  \av_1\cdot\uv(r,0,t) &= 0,  \quad\mbox{(tangential component on $s=0$)}\\
  \av_2\cdot\uv(0,s,t) &= 0,  \quad\mbox{(tangential component on $r=0$)}\\
  \uv(0,0,t)=\partial_t^m\uv(0,0,t) &= 0, \quad\mbox{(all time derivatives are zero at $\rv=0$, $m=0,1,2,\ldots$)} \\
  \Delta\uv(0,0,t) & =0
\end{align*}
where as usual $\av_1=J \grad_\xv r$ and $\av_2=J \grad_\xv s$.

For second-order we can thus solve the four equations 
\begin{align*}
  c_{11} D_{+r}D_{-r} \Uv_\iv + c_{22} D_{+s}D_{-s} \Uv_\iv + c_{1} D_{0r} \Uv_\iv + c_{2} D_{0s} \Uv_\iv &= 0 \\
  (\av_1\cdot\Uv)_{-1,0} &= 0 \\
  (\av_2\cdot\Uv)_{0,-1} &= 0 
\end{align*}
to determine the four unknown components of the vectors $\Uv_{-1,0}$ and $\Uv_{0,-1}$.

Note that for the second-order approximation, since $\Uv_{0,0}=0$ the extended boundary points are
not even used. The question is whether the above second-order approximation is good enough
for the fourth-order accurate method.

\clearpage
\subsection{Corners in two-dimensions}


Given the values on the extended boundaries we can then determine the 
values at the ghost points that lie outside corners.
\newcommand{\trunc}{O(|\rv|^6)}%
\newcommand{\truncb}{O(|\rv|^4)}%
\newcommand{\trunca}{O(|\rv|^2)}%
By Taylor series
\begin{align*}
  u(\ra,\rb)&= u(0,0) + \Ds_1(\ra,\rb) + \Ds_2(\ra,\rb) + \Ds_3(\ra,\rb) + \Ds_4(\ra,\rb) + \trunc
\end{align*}
where
\begin{align*}
  \Ds_1(\ra,\rb) &= (\ra \partial_\ra + \rb \partial_\rb ) u(0,0) \\
  \Ds_2(\ra,\rb) &= {1\over2}( \ra^2\partial_\ra^2+\rb^2\partial_\rb^2+ 2\ra\rb\partial_\ra\partial_\rb  )u(0,0)  \\
  \Ds_3(\ra,\rb) &= {1\over3!}( \ra^3\partial_r^3 + \rb^3\partial_\rb^3 
                   +3\ra^2\rb\partial_\ra^2\partial_\rb 
                   + 3\ra\rb^2\partial_\ra\partial_\rb^2 
                   + \rb^3\partial_\rb^3  )u(0,0)                    \\
  \Ds_4(\ra,\rb) &= {1\over4!}( \ra^4\partial_r^4
                   +4\ra^3\rb\partial_\ra^3\partial_\rb 
                   +6\ra^2\rb^2\partial_\ra^2\partial_\rb^2 
                   +4\ra\rb^3\partial_\ra\partial_\rb^3 
                   + \rb^4\partial_\rb^4   )u(0,0)  
\end{align*}
Whence
\begin{align}
 u(-\ra,-\rb) &= 2 u(0,0) - u(\ra,\rb) + 2\Ds_2(\ra,\rb) + 2\Ds_4(\ra,\rb)+ \trunc \label{taylor1} \\
\end{align}

For second order at the corner we use the approximation
\begin{align}
   u(-\ra,-\rb) &\approx 2 u(0,0) - u(\ra,\rb) + 2\Ds_2(\ra,\rb) + \truncb \\
\Ds_2(\ra,\rb) &= {1\over2}( \ra^2\partial_\ra^2+\rb^2\partial_\rb^2+ 2\ra\rb\partial_\ra\partial_\rb )u(0,0)
\end{align}
where the mixed derivatives, $\partial_\ra\partial_\rb u(0,0) $,
are obtained from equation~(\ref{eq:mixedDerivatives}). 


For fourth-order accuracy we use
\begin{align}
   u(-\ra,-\rb) &\approx 2 u(0,0) - u(\ra,\rb) + 2\Ds_2(\ra,\rb) + 2\Ds_4(\ra,\rb) + \trunc \\
\end{align}
All non-mixed derivatives, $\partial_r^3 u$, $\partial_s^3 u$ etc. can be computed from the boundary
values whcih are assumed known at this point.

The mixed derivatives $\partial_\ra\partial_\rb u(0,0) $,
are obtained from equation~(\ref{eq:mixedDerivatives}). 

We also need approximations for $\partial_\ra^3\partial_\rb u$,
 $\partial_\ra^2\partial_\rb^2 u$ and $\partial_\ra\partial_\rb^3 u$.



\clearpage
\subsection{Extended boundaries in three-dimensions}

Consider the edge $\rv=(0,0,\rc)$, $\rc\in[0.1]$ of an orthogonal grid in three dimensions. 

We need to determine values of the ghost points along the 
extended boundaries (i.e. $\rb=0$, $\ra<0$ and $\ra=0$, $\rb<0$).


The following conditions hold on a PEC boundary
\begin{align*}
  \av_1\cdot\uv(r,0,0,\tc)=\av_3\cdot\uv(r,0,0,\tc) &= 0,  \quad\mbox{(tangential component on $\rb=0$)}\\
  \av_2\cdot\uv(0,s,0,\tc)=\av_3\cdot\uv(0,s,0,\tc) &= 0,  \quad\mbox{(tangential component on $\ra=0$)}\\
  \av_1\cdot\partial_t^m\uv(0,0,t) &= 0, \quad\mbox{($m=0,1,2,\ldots$)} \\
  \av_2\cdot\partial_t^m\uv(0,0,t) &= 0, \quad\mbox{($m=0,1,2,\ldots$)} 
\end{align*}

For second-order accuracy we can thus solve the six equations 
\begin{align*}
  \av_1\cdot L_h^{(2)} \Uv_{0,0,i_3} &= 0 \\
  \av_2\cdot L_h^{(2)} \Uv_{0,0,i_3} &=0 \\
  (\av_1\cdot\Uv)_{-1,0,i_3} &= 0 \quad,\quad (\av_3\cdot\Uv)_{-1,0,i_3} = 0 \\
  (\av_2\cdot\Uv)_{0,-1,i_3} &= 0 \quad,\quad (\av_3\cdot\Uv)_{0,-1,i_3} = 0 
\end{align*}
to determine the six unknown components of the vectors $\Uv_{-1,0,i_3}$ and $\Uv_{0,-1,i_3}$.
Here $L_h^{(2)}$ is a second-order accurate approximation to the Laplacian,
\begin{align*}
  L_h^{(2)} \Uv_\iv &= \big(c_{11} D_{+r}D_{-r} \Uv_\iv + c_{22} D_{+s}D_{-s} \Uv_\iv + c_{33} D_{+t}D_{-t} \Uv_\iv \\
          &~~ + c_{1} D_{0r} \Uv_\iv + c_{2} D_{0s}\big) + c_{3} D_{0t}\Uv_\iv \big) 
\end{align*}

For fourth-order accuracy we use a fourth-order approximation to the equation and extrapolate 
\begin{align*}
  \av_1\cdot L_h^{(4)} \Uv_{0,0,i_3} &=0 \\
  \av_2\cdot L_h^{(4)} \Uv_{0,0,i_3} &=0 \\
  (\av_1\cdot\Uv)_{i_1,0,i_3} &= (\av_3\cdot\Uv)_{i_1,0,i_3} = 0 \quad i_1=-1,-2\\
  \Delta_{+r}^4(\av_2\cdot\Uv)_{-2,0,i_3} &= 0  \\
  (\av_2\cdot\Uv)_{0,i_2,i_3} &= (\av_3\cdot\Uv)_{0,i_2,i_3} = 0  \quad i_2=-1,-2\\
  \Delta_{+s}^4(\av_1\cdot\Uv)_{0,-2,i_3} &= 0  \\
\end{align*}
These 12 equations determine the 12 unknown components of the 
vectors $\Uv_{-1,0,i_3}$, $\Uv_{-2,0,i_3}$, $\Uv_{0,-1,i_3}$ and $\Uv_{0,-2,i_3}$.
Since the ``tangential components'' are known, these 12 equations can be reduced to
the solution of 4 equations for the unknown ``normal'' components.


\clearpage
\subsection{Ghost points outside edges in three-dimensions}
By Taylor series in $\ra$ and $\rb$ (leaving $\rc$ fixed) gives
\begin{align*}
  u(\ra,\rb,\rc)&= u(0,0,\rc) + \Ds_1(u,\ra,\rb,\rc) + \Ds_2(u,\ra,\rb,\rc) 
             + \Ds_3(u,\ra,\rb,\rc) + \Ds_4(u,\ra,\rb,\rc) + \trunc
\end{align*}
where
\begin{align*}
  \Ds_1(u,\ra,\rb,\rc) &= (\ra \partial_\ra + \rb \partial_\rb ) u(0,0,\rc) \\
  \Ds_2(u,\ra,\rb,\rc) &= {1\over2}( \ra^2\partial_\ra^2
                            + 2\ra\rb\partial_\ra\partial_\rb +\rb^2\partial_\rb^2 )u(0,0,\rc)  \\
  \Ds_3(u,\ra,\rb,\rc) &= {1\over3!}( \ra^3\partial_r^3 + \rb^3\partial_\rb^3 
                   +3\ra^2\rb\partial_\ra^2\partial_\rb 
                   + 3\ra\rb^2\partial_\ra\partial_\rb^2 
                   + \rb^3\partial_\rb^3  )u(0,0)                    \\
  \Ds_4(u,\ra,\rb,\rc) &= {1\over4!}( \ra^4\partial_r^4
                   +4\ra^3\rb\partial_\ra^3\partial_\rb 
                   +6\ra^2\rb^2\partial_\ra^2\partial_\rb^2 
                   +4\ra\rb^3\partial_\ra\partial_\rb^3 
                   + \rb^4\partial_\rb^4   )u(0,0,\rc)  
\end{align*}

We can thus derive the following two formulae for the ghost points,
\begin{align}
 u(-\ra,-\rb,\rc) &= 2 u(0,0,\rc) - u(\ra,\rb,\rc) + 2\Ds_2(u,\ra,\rb,\rc) 
                         + 2\Ds_4(u,\ra,\rb,\rc)+ \trunc \label{eq:taylorOdd} \\
 u(-\ra,-\rb,\rc) &= u(\ra,\rb,\rc) - 2\Ds_1(u,\ra,\rb,\rc) - 2\Ds_3(u,\ra,\rb,\rc) 
                       + {O(|\rv|^5)} \label{eq:taylorEven}
\end{align}
The first formula, equation~(\ref{eq:taylorOdd}) is appropriate for odd functions and the
second is appropriate for even functions.

Consider an edge $\rv=(0,0,\rc)$. On a rectangular grid (perhaps rotated) the components 
$\av_1\cdot\uv$ and $\av_2\cdot\uv$ will be odd functions while the component
$\av_3\cdot\uv$ will be even. 
It is thus appropriate to use equation~(\ref{eq:taylorOdd}) for $\av_1\cdot\uv$ and $\av_2\cdot\uv$ but
use equation~(\ref{eq:taylorEven}) for $\av_3\cdot\uv$.

% At the edge $\rc=0$, the components $\av_1\cdot\uv$ and $\av_2\cdot\uv$ will behave like odd functions
% while the component $\av_3\cdot\uv$ will behave like an even function in the sense that on a rectangular
% grid $\av_1\cdot\uv$ and $\av_2\cdot\uv$ will be odd while $\av_3\cdot\uv$ will be even.

For second order at the ghost points outside the edge parallel to $\rc$ we use the approximation
\begin{align}
   \av_m\cdot\uv(-\ra,-\rb,\rc) &\approx 2 \av_m\cdot\uv(0,0,\rc) - \av_m\cdot\uv(\ra,\rb,\rc)  \quad m=1,2 \\
   \av_3\cdot\uv(-\ra,-\rb,\rc) &\approx \av_3\cdot\uv(\ra,\rb,\rc) - \av_3\cdot\Ds_1(\uv,\ra,\rb,\rc) 
\end{align}
Along the edge $\rv=(0,0,\rc)$ all the derivatives of $\av_3\cdot\uv$ are zero and thus
$\av_3\cdot\uv_r = (\av_3\cdot\uv)_r - (\partial_r\av_3)\cdot\uv = 0 $.
This implies $\av_3\cdot\Ds_1(\uv,\ra,\rb,\rc)$ is zero on the edge and thus 
we may more simply use
\begin{align}
   \av_3\cdot\uv(-\ra,-\rb,\rc) &\approx \av_3\cdot\uv(\ra,\rb,\rc) 
\end{align}

For fourth-order accuracy  we use the approximations
\begin{align}
  \av_m\cdot\uv(-\ra,-\rb,\rc) &\approx 2 \av_m\cdot\uv(0,0,\rc) - \av_m\cdot\uv(\ra,\rb,\rc) 
                          + 2\av_m\cdot\Ds_2(\uv,\ra,\rb,\rc) \quad m=1,2 \\
  \av_3\cdot\uv(-\ra,-\rb,\rc) &\approx \av_3\cdot\uv(\ra,\rb,\rc) - 2\av_3\cdot\Ds_1(\uv,\ra,\rb,\rc) 
                     - 2\av_3\cdot\Ds_3(\uv,\ra,\rb,\rc) 
% \Ds_2(\ra,\rb,\rc) &= {1\over2}( \ra^2\partial_\ra^2+ 2\ra\rb\partial_\ra\partial_\rb+\rb^2\partial_\rb^2 )u(0,0,\rc)
\end{align}


The mixed derivatives $\av_1\cdot\uv_{rs}$ and $\av_2\cdot\uv_{rs}$ can be obtained from the
divergence equation
\begin{align}
  \av_1\cdot\uv_{rs} &= -\big\{ \big\} \\
  \av_2\cdot\uv_{rs} &= -\big\{ \big\} 
\end{align}

The mixed derivatives $\av_3\cdot\uv_{rrs}$ and $\av_3\cdot\uv_{rss}$ are also needed.
From the equation we have
\begin{align}
    c_{11} \uv_{rrs} & = - \partial_s\big\{ c_{22} \uv_{ss} + c_{33} \uv_{tt} 
                       + c_{1} \uv_{r} + c_{2} \uv_{s} + c_{3} \uv_{t} \big\} \\
                     &= -\{ ... c_{1} \uv_{rs} + ... \} \\
  c_{22} \uv_{rss} &  = - \partial_r\big\{ c_{11} \uv_{rr} + c_{33} \uv_{tt} 
                       + c_{1} \uv_{r} + c_{2} \uv_{s} + c_{3} \uv_{t} \big\} \\
                     &= -\{ ... c_{2} \uv_{rs} + ... \} \\
\end{align}
This gives $\av_3\cdot\uv_{rrs}$ but requires $\av_3\cdot\uv_{rs}$.
We *could* use a first order approximation for $\av_3\cdot\uv_{rs}$:
\begin{align}
  \av_3\cdot\uv_{rs} &\approx \av_3\cdot D_{+r}D_{+s} \Uv_{0,0,i_3}
\end{align}
% or use a centered approximation with extrapolation for $\av_3\cdot\Uv_{-1,0,i_3}$, $\av_3\cdot\Uv_{0,-1,i_3}$
% and $\av_3\cdot\Uv_{-1,-1,i_3}$
OR we could use the second order approximation
\begin{align*}
  u_{rs}(r,s) &= \big\{ 8 u(r,s) - u(2r,2s) - 7 u(0,0) - 6( ru_r+ su_s) -2( r^2 u_{rr} + s^s u_{ss} )
                     \big\} /(4 r s) + O( r^2 )
\end{align*}



ANOTHER way: First compute $\av_3\cdot\Uv_{-1,0,i_3}$ from a second order
approximation to
\begin{align}
  \av_3\cdot\big\{ c_{11} \uv_{rr} + c_1 \uv_r\big\}  &= - av_3\cdot\big\{ c_{22} \uv_{ss} + c_{33} \uv_{tt} 
                        + c_{2} \uv_{s} + c_{3} \uv_{t} \big\} \\
\end{align}
and compute $\av_3\cdot\Uv_{0,-1,i_3}$ from a second order
approximation to
\begin{align}
  \av_3\cdot\big\{ c_{22} \uv_{ss} + c_2 \uv_s\big\}  &= - av_3\cdot\big\{ c_{11} \uv_{ss} + c_{33} \uv_{tt} 
                       + c_{1} \uv_{r}  + c_{3} \uv_{t} \big\} \\
\end{align}
Then we can approximate $\av_3\cdot\uv_{rrs}$ and $\av_3\cdot\uv_{rss}$ to second order
\begin{align}
   \av_3\cdot\uv_{rrs} &\approx \av_3\cdot D_{+r}^2D_{+s} \Uv_{0,0,i_3}
\end{align}
This last equation contains the unknown value $\av_3\cdot\Uv_{-1,-1,i_3}$ which we need to move to
the left hand side...

% where the mixed derivatives, $\partial_\ra\partial_\rb u(0,0,\rc) $,
% are obtained ??  % from equation~(\ref{eq:mixedDerivatives}). 

\clearpage
\subsection{Corners in three-dimensions}


Given the values on the extended boundaries we can then determine the 
values at the ghost points that lie outside corners (i.e. the vertices of the unit cube).
By Taylor series
\begin{align*}
  u(\ra,\rb,\rc)&= u(0,0) + \Ds_1(\ra,\rb,\rc) + \Ds_2(\ra,\rb,\rc) + \Ds_3(\ra,\rb,\rc) + \Ds_4(\ra,\rb,\rc) + \trunc
\end{align*}
where
\begin{align*}
  \Ds_1(\ra,\rb,\rc) &= (\ra \partial_\ra + \rb \partial_\rb + \rc \partial_\rc ) u(0,0) \\
  \Ds_2(\ra,\rb,\rc) &= {1\over2!}( \ra^2\partial_\ra^2+\rb^2\partial_\rb^2 +\rc^2\partial_\rc^2
                         + 2\ra\rb\partial_\ra\partial_\rb 
                         + 2\ra\rc\partial_\ra\partial_\rc + 2\rb\rc\partial_\rb\partial_\rc 
                     )u(0,0)  \\
\end{align*}
Whence we derive the expression
\begin{align}
 u(-\ra,-\rb,-\rc) &= 2 u(0,0,0) - u(\ra,\rb,\rc) + 2\Ds_2(\ra,\rb\rc) + 2\Ds_4(\ra,\rb\rc)+ \trunc \label{taylor1} \\
\end{align}
On a rectangular grid, all components of $\uv$ are odd functions about the corner and thus
the above formula is the appropriate one to use since it reduces to the condition
$u(-\ra,-\rb,-\rc) = 2 u(0,0,0) - u(\ra,\rb,\rc)$ on a rectangular grid.

For second-order accuracy at the corner we use the approximation
\begin{align}
   u(-\ra,-\rb,-\rc) &\approx 2 u(0,0,0) - u(\ra,\rb,\rc) + \trunca \\
\end{align}
% where the mixed derivatives, $\partial_\ra\partial_\rb u(0,0) $,

For fourth-order accuracy  at the corner we use the approximation
\begin{align}
   u(-\ra,-\rb,-\rc) &\approx 2 u(0,0,0) - u(\ra,\rb,\rc) + 2\Ds_2(\ra,\rb,\rc) + \truncb \\
\end{align}
where all the second derivatives in $\Ds_2$ can be computed from the values on the extended boundaries.
