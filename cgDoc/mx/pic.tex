%-----------------------------------------------------------------------
% Maxwell Solver
%
%-----------------------------------------------------------------------
\documentclass[12pt]{article}
\usepackage{times}  % for embeddable fonts, Also use: dvips -P pdf -G0

\input documentationPageSize.tex

% \voffset=-1.25truein
% \hoffset=-1.truein
% \setlength{\textwidth}{7in}      % page width
% \setlength{\textheight}{9.5in}    % page height
% \renewcommand{\baselinestretch}{1.5}    % "double" spaced

\hbadness=10000 % \tolerance=10000
\sloppy \hfuzz=30pt

\input{pstricks}\input{pst-node}
\input{colours}


\usepackage{amsmath}
\usepackage{amssymb}

% \usepackage{verbatim}
% \usepackage{moreverb}

\usepackage{graphics}    
\usepackage{epsfig}    
% \usepackage{calc}
% \usepackage{ifthen}
% \usepackage{float}
% the next one cause the table of contents to disappear!
% * \usepackage{fancybox}


\usepackage{makeidx} % index
\makeindex
\newcommand{\Index}[1]{#1\index{#1}}


% ---- we have lemmas and theorems in this paper ----
\newtheorem{assumption}{Assumption}
\newtheorem{definition}{Definition}

\newcommand{\grad}{\nabla}

\newcommand{\Overture}{{\bf Overture\ }}
\newcommand{\OverBlown}{{\bf OverBlown\ }}
\newcommand{\overBlown}{{\bf overBlown\ }}
\newcommand{\maxDoc}{/home/henshaw/res/maxwell/doc}
\newcommand{\maxDir}{/home/henshaw/papers/max}

\newcommand{\figWidth}{\linewdith}

\newcommand{\Ds}{{\mathcal D}}%
\newcommand{\ra}{{r_1}}%
\newcommand{\rb}{{r_2}}%
\newcommand{\rc}{{r_3}}%
\newcommand{\tc}{{t}}% time
\newcommand{\dra}{{\Delta r_1}}%
\newcommand{\drb}{{\Delta r_2}}%
\newcommand{\drc}{{\Delta r_3}}%

\begin{document}
% \Large

% -----definitions-----
\input wdhDefinitions.tex

\newcommand{\Div}{\grad\cdot}
\newcommand{\tauv}{\boldsymbol{\tau}}
\newcommand{\kappav}{\boldsymbol{\kappa}}
\newcommand{\betav}{\boldsymbol{\beta}}

\newcommand{\sumi}{\sum_{i=1}^n}
% \newcommand{\half}{{1\over2}}
\newcommand{\deltaT}{{\Delta t}}
\newcommand{\dt}{{\Delta t}}
\newcommand{\eps}{\epsilon}

\section{Maxwell's Equations}

The time dependent Maxwell's equations for linear, isotropic and non-dispersive materials are
\begin{align}
  \eps\partial_t \Ev &=  \grad\times\Hv - \Jv , \label{eq:FOS-Et}  \\
  \mu\partial_t \Hv &= - \grad\times\Ev ,  \label{eq:FOS-Ht} \\
  \grad\cdot(\eps\Ev) &=\rho , ~~ \grad\cdot(\mu\Hv) = 0 , \label{eq:FOS-div}
\end{align}
Here $\Ev=\Ev(\xv,t)$ is the electric field, 
$\Hv=\Hv(\xv,t)$ is the magnetic field, $\rho=\rho(\xv,t)$ is the electric charge density,
$\Jv=\Jv(\xv,t)$ is the electric current density,
$\eps=\eps(\xv)$ is the electric permittivity, and $\mu=\mu(\xv)$ is the magnetic permeability.
This first-order system for Maxwell's equations can also be written in a
second-order form. By taking the time derivatives of~(\ref{eq:FOS-Ht}) and
(\ref{eq:FOS-Et}) and using (\ref{eq:FOS-div}) it follows that 
% (see for example~\cite{BornAndWolf})
% It follows that the electric and magnetic fields each satisfy a vector wave equation
% with lower order terms,
\begin{align}
 \eps\mu~\partial_t^2 \Ev &= \Delta \Ev + \grad\Big( \grad \ln\eps~\cdot\Ev \Big)
        +\grad\ln\mu\times\Big(\grad\times\Ev\Big) -\grad(\frac{1}{\epsilon}\rho)- \mu \partial_t\Jv , \\
 \eps\mu~\partial_t^2 \Hv &= \Delta \Hv + \grad\Big( \grad \ln\mu~\cdot\Hv \Big)
                               +\grad\ln\eps\times\Big(\grad\times\Hv\Big) 
                     + \eps\grad\times(\frac{1}{\epsilon}\Jv ) .
\end{align}
For constant $\epsilon$ and $\mu$ these equations simplify to
\begin{align}
 \eps\mu~\partial_t^2 \Ev &= \Delta \Ev  
                 -\grad(\frac{1}{\epsilon}\rho)- \mu \partial_t\Jv , \label{eq:Ett} \\
 \eps\mu~\partial_t^2 \Hv &= \Delta \Hv + \grad\times( \Jv ) .
\end{align}


\subsection{Modified Equation time stepping and charge forcing}


For the forced wave equation,
\begin{align*}
 \uv_{tt} &= c^2 ( \Delta \uv + \fv ), \\
  \fv &= -\grad(\frac{1}{\epsilon}\rho)- \mu \partial_t\Jv 
\end{align*}
the fourth-order accurate modified equation approach is
\begin{align*}
  \Uv_\iv^{n+1} -2 \Uv_\iv ^n + \Uv_\iv^{n-1} &=
    \textstyle (c\dt)^2 \big( \Delta_{4h} \Uv_\iv^n + \fv \big) % \nonumber \\
          + {(c\dt)^4\over 12} \big( (\Delta^2)_{2h} \Uv_\iv^n + \Delta_{2h} \fv  + c^{-2}\fv_{tt} \big),
\end{align*}
Therefore
\begin{align*}
  \Uv_\iv^{n+1} -2 \Uv_\iv ^n + \Uv_\iv^{n-1} &=
    \textstyle (c\dt)^2 \big( \Delta_{4h} \Uv_\iv^n 
                -\grad(\frac{1}{\epsilon}\rho)- \mu \partial_t\Jv \big) \\
    & + {(c\dt)^4\over 12} \big( (\Delta^2)_{2h} \Uv_\iv^n 
          -\grad(\frac{1}{\epsilon} \Delta\rho)- \mu \partial_t \Delta\Jv 
           + c^{-2}\big[ -\grad(\frac{1}{\epsilon}\rho_{tt})- \mu \partial_t^3\Jv \big] \big).
\end{align*}

% \begin{align}
%   U_\iv^{n+1} -2 U_\iv ^n + U_\iv^{n-1} &=\textstyle \dt^2 \big(c^2 \Delta_{4h} U_\iv^n + f \big) % \nonumber \\
%                    + {\dt^4\over 12} \big( c^4 (\Delta^2)_{2h} U_\iv^n + c^2 \Delta_{2h} f  + f_{tt} \big),
% \end{align}


% ===================================================================================
\section{Charge Sources and Particle in Cell Methods}

From~\eqref{eq:FOS-Et} it follows that
\begin{align}
  \partial_t(\grad\cdot(\eps\Ev)) &=   - \grad\cdot\Jv .
\end{align}
Since $\grad\cdot(\eps\Ev)=\rho$ then $\rho$  will satisfy the charge conservation law
\begin{align}
  \partial_t\rho + \grad\cdot\Jv &=0 \label{eq:chargeConservation}
\end{align}
Thus if we solve Maxwell's equations with initial conditions satisfying $\grad\cdot(\eps\Ev)=\rho$
then~\eqref{eq:chargeConservation} will hold for all time.


\newcommand{\uvc}{\uv^\rho}
Suppose that we are given a charge density $\rho$ that moves through the domain 
with a constant velocity $\uvc$, 
\begin{align}
  \rho(\xv,t) &= f(\xv-\uvc t)
\end{align}
For example $\rho$ might be a Gaussian pulse, 
$\rho(\xv,t)=\rho_0 \exp(-\alpha \vert \xv-\uvc t \vert|^2 )$
or a point charge $\rho(\xv,t)=\rho_0 \delta(\xv-\uvc t)$.

In this case $\rho$ satisfies
\begin{align}
  \rho_t + \uvc\cdot\grad f &=0 \\
  \rho_t + \grad\cdot( \uvc f ) &=0 
\end{align}
and we see that $\Jv=\rho\uvc$. 


If we solve Maxwell's equations with a numerical method then we need to choose an
approximation to $\Jv$ as it appears as a forcing function. Ideally we should choose
$\Jv$ so that $\grad_h(\eps\Ev)=\rho$ for some discrete approximation $\grad_h\cdot$
to the divergence operator. 

Some methods are devised so that the discrete method exactly conserves the charge.

Other approaches use an approximate method. 


Projection methods can take an approximation to $\Ev(\xv,t)\approx \Uv_\iv^*$ and
add a correction to ensure that $\grad_h(\eps\Uv_\iv^n)=\rho$ using
\begin{align}
  \Uv_\iv^n &= \Uv_\iv^* + \grad\phi  \\
  \Delta_h \phi &= \rho - \grad_h\cdot(\eps\Uv_\iv^*)
\end{align}
This requires the solution to the elliptic equation for $\phi$.


In another approach, we can add some extra terms to equation~\eqref{eq:Ett},
\begin{align}
 c^2\partial_t^2 \Ev &= \Delta \Ev  
                 -\grad(\frac{1}{\epsilon}\rho)- \mu \partial_t\Jv 
                 +\alpha \partial_t\grad( \grad\cdot(\eps\Ev)-\rho ) 
                 -\beta \partial_t\grad\phi \\
   \Delta \phi &= \grad\cdot(\eps\Ev)-\rho
\end{align}
which implies
\begin{align}
 \mu\partial_t^2 \grad\cdot\Ev &= \Delta(\grad\cdot\Ev-\frac{1}{\epsilon}\rho)
                 - \mu \partial_t\grad\cdot\Jv 
                 +\alpha \partial_t\Delta( \grad\cdot(\eps\Ev)-\rho ) 
                 -\beta \partial_t(\grad\cdot(\eps\Ev)-\rho)
\end{align}
If $\grad\cdot\Jv=-\rho_t$ and letting $\delta=\grad\cdot(\eps\Ev)-\rho$, then $\delta$
satisfies
\begin{align}
 \mu\partial_t^2 \delta &= \Delta(\delta)
                 +\alpha \partial_t\Delta \delta
                 -\beta \partial_t\delta
\end{align}
If $\delta$ is not zero then it will be diffused and damped by the two additional
terms.



For the first order system we could use
\begin{align}
   \eps\partial_t \Ev &=  \grad\times\Hv - \Jv +\alpha\grad( \grad\cdot(\eps\Ev)-\rho )
                    -\beta\grad\phi \\
    \Delta \phi &= \grad\cdot(\eps\Ev)-\rho
\end{align}
which leads to 
\begin{align}
 \partial_t \delta &= \grad\cdot(\grad\times\Hv) 
                 +\alpha \Delta \delta
                 -\beta \delta
\end{align}

% Alternatively we could solve
% \begin{align}
%    \eps\partial_t \Ev &=  \grad\times\Hv - \Jv  + \grad\phi \\
%      \phi_t & = {1\over\beta}(\grad\cdot(\eps\Ev)-\rho) 
% \end{align}

% -------------------------------------------------------------------------
\clearpage
\section{Moving Point Charge}

Consider a point charge with position $\rv(t)$ with charge $e$ and velocity $\vv$
with $\betav(t)=\vv(t)/c$.


The fields due to an accelerating point charge are~\cite{Jackson}
\begin{align}
 \Ev(\xv,t)&= e\Big[ {\nv-\betav \over \gamma^2(1-\betav\cdot\nv)^3 R^2 }\Big]_{\rm ret}
    +{e\over c}\Big[ {\nv\times\{ (\nv-\betav)\times\dot{\betav}
                  \over (1-\betav\cdot\nv)^3 R } \Big]_{\rm ret} \label{eq:movingPointChargeE}\\
 \Bv&=\Big[ \nv\times\Ev \Big]_{\rm ret}
\end{align}
where the expression is to be evaluated at the retarded time $\tau$ and where
\begin{align}
  \tau &= t - \vert \xv-\rv(\tau)\vert/c , \qquad\mbox{** check this **} \\
 \nv &= { \xv-\rv(\tau) \over \vert \xv-\rv(\tau) \vert} \\
 R & =\vert \xv-\rv(\tau)\vert, \\
%    & = x_0-r_0(\tau_0) ,\\
 \beta &=\vert \betav \vert ,\\
 \gamma &= {1\over \sqrt{ 1-\beta^2} }, \\
 d\tau &= dt\sqrt{1-\beta^2(t)} = {dt\over \gamma(t)}
\end{align}
$\tau$ is the {\em proper time of the particle}.

The corresponding potential is the Li\'enard-Wiechert potential
\[
   \phi(\xv,t) = e\Big[ {1\over (1-\betav\cdot\nv)R }\Big]_{\rm ret}
\]

The two terms in~\eqref{eq:movingPointChargeE} represent the {\em velocity fields}
and {\em acceleration fields}. The former are like electrostatic fields and
decay like $R^{-2}$ while the later are radiation fields that vary as $R^{-1}$.


\end{document}
