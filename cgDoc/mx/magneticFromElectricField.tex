\newcommand{\Ehv}{\widehat{\Ev}}
\newcommand{\Hhv}{\widehat{\Hv}}
\newcommand{\Ehrv}{\Ehv_r}
\newcommand{\Ehiv}{\Ehv_i}
\newcommand{\Hhrv}{\Hhv_r}
\newcommand{\Hhiv}{\Hhv_i}
\newcommand{\Etv}{\widetilde{\Ev}}
\newcommand{\Htv}{\widetilde{\Hv}}
\section{Computing the magnetic field and intensity from a time harmonic electric field}

For time harmonic solutions we can easily compute the magnetic field given
the electric field and its time derivative (or the electric field at two times).
Thus we can compute the intensity given the electric field without having
to explicitly solve Maxwell's equations for the magnetic field. 


Consider a (complex valued) time harmonic solution to Maxwell's equations,
\begin{align*}
   \Etv(\xv,t) &= e^{-i\omega t}\Ehv(\xv),\\
   \Htv(\xv,t) &= e^{-i\omega t}\Hhv(\xv),
\end{align*}
where the real and imaginary parts of $\Ehv$ and $\Hhv$ are denoted as
\begin{align*}
   \Ehv(\xv) &= \Ehrv(\xv) + i \Ehiv(\xv),\\
   \Hhv(\xv) &= \Hhrv(\xv) + i \Hhiv(\xv) .
\end{align*}
We actually compute the real part of $\Etv$ and $\Htv$,
\begin{align*}
   \Ev(\xv,t) &= \Re\{ (\cos(\omega t)-i\sin(\omega t))(\Ehrv + i \Ehiv)  \}  \\
               &= \cos(\omega t)\Ehrv + \sin(\omega t) \Ehiv\\
   \Hv(\xv,t) &=\cos(\omega t)\Hhrv + \sin(\omega t) \Hhiv
\end{align*}
From the time harmonic form of Maxwell's equations
\begin{align*}
  -i\omega \eps\Ehv &= \grad\times \Hhv, \\
  i\omega \mu \Hhv &=  \grad\times \Ehv, 
\end{align*}
we obtain the relation for $(\Hhrv,\Hhiv)$ in terms of $(\Ehrv,\Ehiv)$
\begin{align}
   \omega\mu \Hhrv &= \grad\times \Ehiv, \label{eq:HrfromE}\\
  -\omega\mu \Hhiv &= \grad\times \Ehrv.\label{eq:HifromE}
\end{align}
Thus if we know $(\Ehrv,\Ehiv)$, we can determine $\Hv$.
% 
Given the computed solution for the electric field at two times,
\begin{align*}
\Ev(\xv,t_1) &= \cos(\omega t_1)\Ehrv + \sin(\omega t_1) \Ehiv, \\
\Ev(\xv,t_2) &= \cos(\omega t_2)\Ehrv + \sin(\omega t_2) \Ehiv, 
\end{align*}
%\begin{align*}
%\Ev(\xv,t) &= \cos(\omega t)\Ehrv + \sin(\omega t) \Ehiv, \\
%\Ev(\xv,t+\dt) &= \cos(\omega t+\dt)\Ehrv + \sin(\omega t+\dt) \Ehiv 
%\end{align*}
we can solve for $(\Ehrv,\Ehiv)$,
\begin{align}
  \Ehrv(\xv) &= \Big(~~~\sin(\omega t_2)\Ev(\xv,t_1) -\sin(\omega t_1)\Ev(\xv,t_2) \Big)/\sin(\omega (t_2-t_1)),  \label{eq:Ehrv} \\
  \Ehiv(\xv) &= \Big(  -\cos(\omega t_2)\Ev(\xv,t_1) +\cos(\omega t_1)\Ev(\xv,t_2) \Big)/\sin(\omega (t_2-t_1)) . \label{eq:Ehiv}
\end{align}
Given $(\Ehrv,\Ehiv)$ we can compute $(\Hhrv,\Hhiv)$ from~\eqref{eq:HrfromE}-\eqref{eq:HifromE}. 
Thus we can compute $\Hv(\xv,t)$ given $\Ev$ at two times. 

The intensity is given by 
\begin{align*}
  \Ic &= {1\over P} \int_0^P \half c\eps \vert \Ev \vert^2 + \half c \mu \vert \Hv \vert^2  ~dt , 
\end{align*}
where $P$ is the period.
Now we could compute the intensity from the electric field and magnetic field values determined above,
\begin{align*}
   {1\over P} \int_0^P \vert \Ev \vert^2 ~dt &= 
              {1\over P} \int_0^P \vert \cos(\omega t)\Ehrv + \sin(\omega t) \Ehiv\vert^2 ~dt \\
            &= \half \Big( \vert\Ehrv\vert^2 +  \vert\Ehiv\vert^2 \Big)
\end{align*}
and thus the intensity can be computed using 
\begin{align}
  \Ic &= \frac{1}{4} c\eps \Big(  \vert\Ehrv\vert^2 +  \vert\Ehiv\vert^2 \Big) +
         \frac{1}{4} c\mu  \Big(  \vert\Hhrv\vert^2 +  \vert\Hhiv\vert^2 \Big) ,    \label{eq:intensityEHri}     \\
      &= \frac{1}{4} c\eps \Big(  \vert\Ehrv\vert^2 +  \vert\Ehiv\vert^2 \Big) +
         \frac{1}{4} \frac{c}{\mu\omega^2}  \Big(  \vert \grad\times\Ehiv\vert^2 +  \vert\grad\times\Ehrv\vert^2 \Big) , \label{eq:intensityri}
\end{align}
%
Alternatively, knowing $\Ev$ and $\grad\times\Ev$, we could compute the intensity by the time average
\begin{align*}
  \Ic &= {1\over P} \int_0^P \half c\eps \vert \Ev \vert^2 + \half c \mu \vert \Hv \vert^2  ~dt, \\
      &= {1\over P} \int_0^P \half c\eps \vert \Ev \vert^2 + \half \frac{c}{\mu\omega^2} \vert \grad\times\Ev \vert^2  ~dt , \\
\end{align*}
which follows from~\eqref{eq:intensityri}.


Here is the suggested approach for computing the intensity whenever it is needed for output.
Given $\Ev(\xv,t)$ at times $t_2=t$, and the previous time $t_1=t-\dt$,
\begin{enumerate}
  \item Compute $\Ehrv(\xv)$ and $\Ehiv(\xv)$  from~\eqref{eq:Ehrv}-\eqref{eq:Ehiv}.
  \item Compute $\grad\times \Ehrv(\xv)$ and $\grad\times\Ehiv(\xv)$ by difference approximation.
  \item Compute $\Ic$ from~\eqref{eq:intensityri}.
\end{enumerate}

Approach 2: If we know $\Hv$ as well as $\Ev$ then we can can compute $\Hhrv(\xv)$ and $\Hhiv(\xv)$ (following ~\eqref{eq:Ehrv}-\eqref{eq:Ehiv})
and then use~\eqref{eq:intensityEHri} without the need to compute $\grad\times\Ehrv(\xv)$ etc.

