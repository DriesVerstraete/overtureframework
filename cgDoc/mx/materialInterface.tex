\section{Boundary conditions at material interfaces in 2D}


In this section we derive conditions at a material interface that can be used to
implement higher-order accurate centered discretizations.


Consider the two-dimensional problem 
\begin{align}
  \epsilon(x) u_t & = w_y \label{eq:u} \\
  \epsilon(x) v_t &= -w_x \label{eq:v} \\
   w_t &= u_y - v_x  \label{eq:w} \\
  (\epsilon(x) u)_x + (\epsilon(x) v)_y &= 0 \label{eq:div}
\end{align}
The permitivity, $\epsilon(x)$ is assumed to be piecewise constant with a possible jump at $x=0$.
We assume that the solution to these equations remains bounded.


The above equations can be combined to show that the following second-order wave equations are satisfied,
\begin{align}
  \epsilon(x) u_{tt} & = \partial_x\big( {1\over\epsilon} (\epsilon u)_x\big) + u_{yy} \label{eq:uwave} \\
  \epsilon(x) v_{tt} &= v_{xx} + v_{yy} - {\epsilon_x\over\epsilon} u_y  \label{eq:vwave} \\
  \epsilon(x) w_{tt} &= w_{yy} + \epsilon \partial_x\big( {1\over\epsilon} w_x \big) \label{eq:wwave} 
\end{align}




Integrating equations~\eqref{eq:div}, \eqref{eq:w} and~\eqref{eq:v} 
from $x=0_-$ to $x=0_+$ implies the first three interface conditions,
\begin{align}
     [ \epsilon u ] &=0   \\
     [ v ] & = 0  \\
     [ w ] & = 0 
\end{align}
Here $[v] = v(0_+,y,t) - v(0_-,y,t)$ is the jump in $v$ across the interface.
Thus the tangential component of the electric field is continuous but the normal component may jump.


In the next step, conditions on the first (normal) derivative, $u_x$, $v_x$ are obtained.
We use the fact that if $[v]=0$ then the jump in any time derivatives is also zero, $[\partial_t^m v]=0$.
\begin{align}
     [ (\epsilon u)_x + (\epsilon(x) v_y) ] &=0 \qquad\mbox{(from equation~\eqref{eq:div})}  \\
     [ v_x - u_y ] &= 0  \qquad\mbox{(from equation~\eqref{eq:w} with $[w_t]=0$ )}  \\
     [ w_x/\epsilon ] & = 0 \qquad\mbox{(from equation~\eqref{eq:v} with $[v_t]=0$ )}
\end{align}
Note that we could also use $[ u_x + v_y ]=0$ since when $\epsilon$ is piecewise constant, $u_x+v_y=0$
for $x\ne 0$. Also note that we write the divergence equation, $(\epsilon u)_x + (\epsilon(x) v_y=0$,
as a jump condition even though the divergence is identically zero on either side. The reason for this
will be explained later.


The jump conditions for $u_{xx}$, $v_{xx}$ and $w_{xx}$ are
\begin{align}
     [ u_{xx} + u_{yy} ] &=0 \qquad\mbox{(from equation~\eqref{eq:uwave} with $[\epsilon u_{tt}]=0$ )}    \\
     [ (v_{xx} + v_{yy})/\epsilon ] &= 0  \qquad\mbox{(from equation~\eqref{eq:vwave} with $[v_{tt}]=0$ )}  \\
     [ (w_{xx} + w_{yy})/\epsilon ] &= 0  \qquad\mbox{(from equation~\eqref{eq:wwave} with $[w_{tt}]=0$ )}
\end{align}
Note that $[\partial_x\big( {1\over\epsilon} (\epsilon u)_x\big) + u_{yy}]=0$ implies $[ u_{xx} + u_{yy} ]=0$
if $\epsilon$ is piecewise constant.


The jump conditions for $u_{xxx}$, $v_{xxx}$ and $w_{xxx}$ are
\begin{align}
     [ \Delta u_x + \Delta v_y ] &=0 \qquad\mbox{(from equation~\eqref{eq:div})}  \\
     [ (v_{xxx}+v_{xyy} - u_{xxy} - u_{yyy})/\epsilon ] &= 0  \qquad\mbox{(from $[ v_{xtt} - u_{ytt} ]=0$)} \\
     [ (w_{xxx}+ w_{xyy})/\epsilon^2 ] & = 0 \qquad\mbox{(from $[ w_{xtt}/\epsilon ]=0$)} 
\end{align}

The fourth derivatives are obtained from
\begin{align}
     [ \Delta^2 u /\epsilon ] &=0 \qquad\mbox{(from equation~\eqref{eq:uwave} with $[\epsilon u_{tttt}]=0$ )}    \\
     [ \Delta^2 v/\epsilon^2] &= 0  \qquad\mbox{(from equation~\eqref{eq:vwave} with $[v_{tttt}]=0$ )}  \\
     [ \Delta^2 w/\epsilon^2] &= 0  \qquad\mbox{(from equation~\eqref{eq:wwave} with $[w_{tttt}]=0$ )}
\end{align}


In general, for $m=0,1,2,\ldots$ the jump conditions on the even derivatives $\partial_x^{2m}$ are
\begin{align}
     [ \Delta^m u /\epsilon^{m-1} ] &=0  \\
     [ \Delta^m v/\epsilon^m] &= 0  \\
     [ \Delta^m w/\epsilon^m] &= 0  
\end{align}
while the conditions on the odd derivatives are, for $m=0,1,2,\ldots$,
\begin{align}
     [ \Delta^m (u_x + v_y) ] &=0 \\
     [ (\Delta^m (v_x - u_y))/\epsilon^m ] &= 0  \\
     [ (\Delta^m w_{x})/\epsilon^{m+1} ] & = 0 
\end{align}


For a general interface oriented with normal $\nv$ and tangent $\tauv$, 
the jump conditions will be, for $m=0,1,2,\ldots$,
\begin{align}
     [ \nv\cdot\Delta^m \uv /\epsilon^{m-1} ] &=0  \\
     [ \tauv\cdot\Delta^m \uv/\epsilon^m] &= 0  \\
     [ \Delta^m w/\epsilon^m] &= 0  
\end{align}
and 
\begin{align}
     [ \Delta^m(u_x + v_y) ] &=0 \\
     [ (\Delta^m (v_x - u_y))/\epsilon^m ] &= 0  \\
     [ (\Delta^m w_{n})/\epsilon^{m+1} ] & = 0 
\end{align}

\newcommand{\curl}{\grad\times}
% \newcommand{\Div}{\grad\cdot}

% *****************************************************************************************
\clearpage
\section{Boundary conditions at material interfaces in 3D}
Consider the three-dimensional problem 
\begin{align}
  \epsilon(\xv) \Ev_t & = \curl \Hv \label{eq:E} \\
  \mu(\xv) \Hv_t &= -\curl \Ev \label{eq:H} \\
  \Div(\epsilon(\xv) \Ev) &=0 \label{eq:divE} \\
  \Div(\mu(\xv) \Hv) &=0 \label{eq:divH}
\end{align}
The dielectric permittivity, $\epsilon$, and magnetic permeability, $\mu$ are
assumed to be piecewise constant with a possible jump across a smooth interface ${\mathcal S}$,
with normal $\nv(\xv)$.
It is assumed that the solution to these equations remains bounded.
Note that our assumptions of piecewise constant coefficients
imply that on each side of the interface the electric and magnetic fields
satisfy
\begin{align}
  \Ev_{tt} & = {1\over\epsilon\mu}\Delta\Ev \label{eq:Ett} \\
  \Hv_{tt} & = {1\over\epsilon\mu}\Delta\Hv \label{eq:Htt}  \\
  \Div\Ev &=0 \label{eq:divE2}  \\
  \Div\Hv &=0 \label{eq:divH2}
\end{align}
% 
The basic jump conditions at a material interface are 
\begin{alignat}{3}
  [ \epsilon \nv\cdot\Ev] & =0 &\qquad& [\mu \nv\cdot\Hv] =0   \label{eq:jumpN0}\\
  [ \tau\cdot \Ev] &=0         &\qquad& [ \tau\cdot\Hv ] =0    \label{eq:jumpT0}
\end{alignat}
Here $\tau$ represents a tangent to the material interface. Since there are two linearly independent
tangents, $\tau_m$, $m=1,2$, there will be two linear independent conditions
$[\tau_m\cdot \Ev]=0$, $m=1,2$.
% 
Jump conditions on the first spatial 
derivatives are obtained from the divergence relations~(\ref{eq:divE}-\ref{eq:divH})
and equations~(\ref{eq:E}-\ref{eq:H}) combined with $[\tau\cdot \Ev_t] =0$, $[\tau\cdot\Hv_t]=0$,
\begin{alignat}{3}
  [ \Div(\Ev)] & =0               &\qquad& [\Div(\Hv)] =0  \\
  [ \mu^{-1}\tau\cdot\curl\Ev] &=0    &\qquad& [ \epsilon^{-1}\tau\cdot\curl\Hv ] =0
\end{alignat}
Note that it is also true that $[ \Div(\epsilon \Ev)] =0$ (or indeed for any piecewise constant 
$\nu$, it is true that $[ \Div(\nu \Ev)] =0$) .

Jump conditions on the second spatial derivatives follow from taking two time derivatives
of equations~(\ref{eq:jumpN0}-\ref{eq:jumpT0}), and using the vector wave equations (\ref{eq:Ett}-\ref{eq:Htt})
to replace the time derivatives by space derivatives,
\begin{alignat}{3}
  [ \epsilon\nv\cdot\Delta \Ev/(\epsilon\mu)] & =0 
       &\qquad& [\mu \nv\cdot\Delta\Hv/(\epsilon\mu)] =0   \label{eq:jumpN2}\\
  [ \tau\cdot\Delta \Ev/(\mu\epsilon)] &=0 &\qquad& [ \tau\cdot\Delta\Hv/(\mu\epsilon) ] =0    \label{eq:jumpT2}
\end{alignat}
% 
By continuing to take time derivatives it follows that for $m=0,1,2,3,\ldots$, 
\begin{alignat}{3}
  [ \epsilon\nv\cdot\Delta^m \Ev/(\epsilon\mu)^m] & =0  
          &\qquad& [\mu \nv\cdot\Delta^m\Hv/(\epsilon\mu)^m] =0 \label{eq:jumpNm}\\
  [ \tau\cdot\Delta^m \Ev/(\mu\epsilon)^m] &=0 
          &\qquad& [ \tau\cdot\Delta^m\Hv/(\mu\epsilon)^m ] =0    \label{eq:jumpTm} \\
  [ \Div(\Delta^m\Ev)] & =0               &\qquad& [\Div(\Delta^m\Hv)] =0  \\
  [ \mu^{-1}\tau\cdot\curl\Delta^m\Ev/(\mu\epsilon)^m] &=0    
          &\qquad& [ \epsilon^{-1}\tau\cdot\curl\Delta^m\Hv/(\mu\epsilon)^m ] =0  \label{eq:jumpTmII}
\end{alignat}
These interface jump conditions impose conditions for each spatial derivative of the solution.

Another way to write the jump conditions~\eqref{eq:jumpNm} through ~\eqref{eq:jumpTmII} that
doesn't involve the tangent vectors $\tauv$ is as 
\begin{align}
  \Big[ \big(\Ev  + ( (\epsilon-1) \nv\cdot\Ev )~\nv\big)/(\epsilon\mu)^m \Big] &=0, \\
  \Big[ \big(\mu^{-1}(\curl\Delta^m\Ev - (\nv\cdot \curl\Delta^m\Ev)~\nv)
             + \Div(\Delta^m\Ev)~\nv  \big)/(\epsilon\mu)^m \Big] &=0,
\end{align}
and 
\begin{align}
  \Big[ \big(\Hv  + ( (\mu-1) \nv\cdot\Hv )~\nv\big)/(\epsilon\mu)^m \Big] &=0 , \\
  \Big[ \big(\epsilon^{-1}(\curl\Delta^m\Hv - (\nv\cdot \curl\Delta^m\Hv)~\nv )
             + \Div(\Delta^m\Hv)~\nv  \big)/(\epsilon\mu)^m \Big] &=0 ~.
\end{align}
The former equations follow by taking the dot product of the above equations with $\nv$ or $\tauv$. 
This latter form may be convenient for discretizing the equations since there is no
need to define tangent vectors. 

\subsection{Discrete boundary conditions for material interfaces}


The material interface is assumed to lie along the boundary between two grids
as shown in figure ...


All discrete approximations impose the basic interface condition
\begin{align*}
    [\epsilon \nv\cdot\Ev]&=0 \\
    [\tauv\cdot\Ev]&=0 
\end{align*}




A second-order accurate approximation will use one ghost line on each side of the interface.
The values of $\Ev$ at these two ghost line values are determined by imposing the following
conditions:
\begin{align*}
    [ \grad_{2h}\cdot(\Ev)] & =0 \\
    [ \mu^{-1}\tau\cdot\grad_{2h}\times\Ev] &=0 \\
    [ \epsilon\nv\cdot\Delta_{2h} \Ev/(\epsilon\mu)] & =0 \\
    [ \tau\cdot\Delta_{2h} \Ev/(\mu\epsilon)] &=0 
\end{align*}
Here $\Div_{2h}$, $\curl_{2h}$, and $\Delta_{2h}$ are second-order accurate centered difference
approximations.


A fourth-order accurate approximation will use two ghost lines on each side of the interface.
The values at the ghost points can be determined using fourth-order accurate approximations
to the interface conditions for the first and second derivatives 
\begin{align*}
    [ \grad_{4h}\cdot(\Ev)] & =0 \\
    [ \mu^{-1}\tau\cdot\grad_{2h}\times\Ev] &=0 \\
    [ \epsilon\nv\cdot\Delta_{2h} \Ev/(\epsilon\mu)] & =0 \\
    [ \tau\cdot\Delta_{2h} \Ev/(\mu\epsilon)] &=0 
\end{align*}
together with second-order accurate approximations to the interface conditions for the
third and fourth derivatives,
\begin{align*}
    [ (\Div\Delta)_{2h}(\Ev)] & =0 \\
    [ \mu^{-1} \tau\cdot(\curl\Delta)_{2h}\Ev/(\epsilon\mu)] &=0 \\
    [ \epsilon \nv\cdot(\Delta^2)_{2h} \Ev/(\epsilon\mu)^2 ] & =0 \\
    [ \tau \cdot(\Delta^2)_{2h} \Ev/(\epsilon\mu)^2] &=0 
\end{align*}


% *********************************************************************************************
\clearpage
\subsection{Reflection and Transmission for a planar material interface}


Consider an interface at $x=0$ between two materials.
Assume that for $x<0$, $\eps=\eps_1$ and $c=c_1=\eps_1^{-1/2}$ while for $x>0$ 
$\eps=\eps_2$ and $c=c_2=\eps_2^{-1/2}$.
For an incident plane wave
of the form
\begin{align*}
     \uv^I &= \av e^{ i (\kv\cdot\xv - \omega t)} \\
      \av &= (-k_2,k_1)/\| \kv \| \qquad\mbox{(since $\grad\cdot\uv=0$)}
\end{align*}
the solution will consist of a reflected wave and refracted (transmitted) wave in addition to the incident wave.
The solution is given by
\begin{align*}
  \uv(\xv,t) &= \av e^{ i (k_1 x + k_2 y - \omega t) } + R \bv e^{ i (-k_1 x + k_2 y - \omega t)} 
                                                                     \qquad\mbox{for $x<0$} \\
             &= \tau \dv e^{ i (\kappa_1 x + \kappa_2 y - \omega t)} \qquad\mbox{for $x>0$}
\end{align*}
where $R$ is the reflection coefficient, $\tau$ is the transmission coefficient and 
where $\bv=(-k_2,-k_1)/\| \kv \| $ and $\dv=(-\kappa_2,\kappa_1)/\| \kappav \|$.

We have the conditions, $[\eps u]=0$, $[v]=0$ and $\omega=c_1 \|\kv\|=c_2\|\kappav\|$.
These imply that ** check these**
\begin{align*}
   \kappa_2 &= k_2 \\
   \kappa_1 &= \sqrt{ (c_1/c_2)^2 (k_1^2+k_2^2) - k_2^2 } \\
   \tau &= { 2 c_2 \cos(\theta_1) \over c_2 \cos(\theta_2) + c_1 \cos(\theta_1) } \\
   R &= { c_1 \cos(\theta_1) - c_2 \cos(\theta_2) \over c_2 \cos(\theta_2) + c_1 \cos(\theta_1) } \\
   \cos(\theta_1) & = k_1/\|\kv\| \qquad\mbox{($\theta_1$ is the angle of incidence)} \\
  \cos(\theta_2) & = \kappa_1/\|\kappav\| \qquad\mbox{($\theta_2$ is the angle of reflection)} \\
   \sin(\theta_1)/c_1 &= \sin(\theta_2)/c_2 \qquad\mbox{(Snell's law, from $\kappa_2 = k_2$)}
\end{align*}  


Note that when
\[
     {k_1^2\over k_2^2} < \left({c_2\over c_1}\right)^2 - 1 
\]
then $\kappa_1$ is imaginary. This corresponds to total internal reflection.
This solution is still valid (check this!) provided we take the branch 
of the square root with $Im(\kappa_1)\ge 0$. The solution has an evanescent wave that decays
exponentially into the second region. If we take  $\kappa_1=i\alpha$ with $\alpha\ge 0$
then the refracted wave is 
\begin{align*}
  \uv(\xv,t) &= \tau \dv e^{-\alpha x} e^{ i (\kappa_2 y - \omega t)} \qquad\mbox{for $x>0$}
\end{align*}



% --------------------------------------------------------------------------------------
\section{Characteristic conditions and solution to the interface Riemann problem}


In this section we derive the characteristic conditions and the solution
to a Riemann problem at a material interface. These conditions will determine how to
project the values on the interface to match the jump conditions byusing weighted
averages of the predicted values. 

Consider the two-dimensional problem 
\begin{align}
  \epsilon(x) u_t & = w_y \label{eq:u} \\
  \epsilon(x) v_t &= -w_x \label{eq:v} \\
  \mu w_t &= u_y - v_x  \label{eq:w} \\
  (\epsilon(x) u)_x + (\epsilon(x) v)_y &= 0 \label{eq:div}
\end{align}
The permitivity, $\epsilon(x)$ is assumed to be piecewise constant with a possible jump at $x=0$.
We assume that the solution to these equations remains bounded.

At an interface we have the jump conditions
\begin{align*}
    [\eps u]=0, \quad [v]=0, \quad [w]=0.
\end{align*}

Consider a Riemann problem with left states $u_L$, $v_L$, $w_L$, $\eps_L$ for $x<0$ and
right states $u_R$, $v_R$, $w_R$, $\eps_R$ for $x>0$.

The equations are
\begin{align*}
\begin{bmatrix} u \\ v \\ w \end{bmatrix}_t 
 + 
\begin{bmatrix} 0 & 0 & 0 \\
                0 & 0 & 1/\eps \\
                0 & 1/\mu & 0 
\end{bmatrix}
\begin{bmatrix} u \\ v \\ w \end{bmatrix}_x
+ 
\begin{bmatrix} 0 & 0 & -1/\eps \\
                0 & 0 &  0 \\
                -1/\mu & 0 & 0 
\end{bmatrix}
\begin{bmatrix} u \\ v \\ w \end{bmatrix}_y
= 0
\end{align*}
In the x-direction the equations become a wave equation for the transverse wave components, 
(the component $u$ is a characteristic variable with eigenvalue $\lambda=0$)
\begin{align*}
\begin{bmatrix} v \\ w \end{bmatrix}_t 
 + 
\begin{bmatrix} 
                 0 & 1/\eps \\
                 1/\mu & 0 
\end{bmatrix}
\begin{bmatrix} v \\ w \end{bmatrix}_x
= 0 
\end{align*}
where  the matrix $A$ is 
\begin{align*}
A = \begin{bmatrix} 
                 0 & 1/\eps \\
                 1/\mu & 0 
\end{bmatrix} .
\end{align*}
The eigenvalues and eigenvectors are 
\begin{align*}
&  
A - c I = \begin{bmatrix} 
                 -\frac{1}{\sqrt{\eps\mu}} & 1/\eps \\
                 1/\mu & -\frac{1}{\sqrt{\eps\mu}} \end{bmatrix}, \qquad 
\lambda_{\pm} = \pm c = \pm \frac{1}{\sqrt{\eps\mu}} , \\
&
   r_{\pm} = \begin{bmatrix} \pm\eta \\ 1\end{bmatrix}, ~\text{right eigenvectors}, \quad
   l_{\pm} = \begin{bmatrix} 1 \\ \pm\eta\end{bmatrix}, ~\text{left eigenvectors},
\end{align*}
where the electrical impedance $\eta$ is defined as 
\begin{align*}
  \eta = \sqrt{\frac{\mu}{\eps}} \quad \text{(electrical impedance)}. 
\end{align*}
The characteristic relations are thus (in one-dimension)
\begin{align*}
&  \chi_{\pm} = l_\pm^T \begin{bmatrix} v \\ w \end{bmatrix} = 
v \pm \sqrt{\frac{\mu}{\eps}} w =  v \pm \eta w \quad \text{ are constant along $\frac{dx}{dt}=\pm c$}. 
\end{align*}
(Note the analogy from solid mechanics where $\chi_{\pm} = \sigma \mp z v$, $\sigma=$stress, $v=$velocity).
Also note that by taking the time derivative of $\chi_\pm$ we get the characteristic relations in terms of
$v_t$ and $v_x$ (using $\mu w_t=-v_x$)
\begin{align*}
&  \frac{\partial\chi_{\pm}}{\partial t}  =  v_t   \mp \frac{1}{\mu}\sqrt{\frac{\mu}{\eps}} v_x
 =  v_t \mp c v_x 
\end{align*}
To solve for the interface values, $v^{I}$, $w^{I}$,
from the solution to the Riemann problem we impose
\begin{align*}
&  v^{I} +  \eta_L w^{I} = v_L  + \eta_L w_L, \\
&  v^{I} -  \eta_R w^{I} = v_R  - \eta_R w_R,
\end{align*}
to give 
\begin{align*}
 &  v^{I} = \frac{ \eta_L^{-1} v_L + \eta_R^{-1} v_R}{\eta_L^{-1} + \eta_R^{-1}} + \frac{ w_L - w_R}{\eta_L^{-1} + \eta_R^{-1}}, \\ 
 &  w^{I} = \frac{ \eta_L w_L + \eta_R w_R}{\eta_L + \eta_R} + \frac{ v_L - v_R}{\eta_L + \eta_R} .
\end{align*}
Thus the interface value $w^{I}$ is determined from an impedance weighted average of $w$,
while $v^{I}$ is determined from an inverse-impedance weighted average for $v$.

\medskip
Question: How do we set $u^I$? 
The characteristic and jump conditions for $u$ are
\begin{align*}
 &  \text{$u$ is constant along $\frac{dx}{dt}=0$ and $\eps_L u_L = \eps_R u_R$}. 
\end{align*}
Let us integrate the u equation 
\[
  \eps(x) u_t = w_y , \\
\]
and average over a cell of width $\dx$ that crosses the interface, 
\begin{align*}
   \left<\eps u_t\right>_I \equiv & \frac{1}{\dx} \int \eps(x) u_t \, dx \\
       & = \frac{1}{\dx}\int  w^I_y \, dx , \\
       & = \frac{ \eta_L \partial_y w_L + \eta_R \partial_y w_R}{\eta_L + \eta_R} + ... , \\
       & = \frac{ \eta_L \eps_L \partial_t u_L + \eta_R \eps_R\partial_t u_R}{\eta_L + \eta_R} + ... , \\
%  &  \eps^I u^I_t = w^I_y = \frac{ \eta_L \eps_L \partial_t u_L + \eta_R \eps_R\partial_t u_R}{\eta_L + \eta_R} + ... , \\
\end{align*}
where we have used the formula for $w^I$ along with the governing PDEs. This 
suggests, by integrating in time, that the average is given by 
\begin{align*}
 &  \left<\eps u\right>_I = \frac{ \eta_L\, (\eps_L u_L) + \eta_R\, (\eps_R u_R)}{\eta_L + \eta_R} ,
\end{align*}
Thus it appears that the zero-characteristic quantity $\eps u$ should be determined by an impedance weighted average,
similar to $w$. 
Thus we maybe we should take new left and right interface values from 
\begin{align*}
 &  \eps_L u_L^I = \eps_R u_R^I =  \frac{ \eta_L\, (\eps_L u_L) + \eta_R\, (\eps_R u_R)}{\eta_L + \eta_R} . 
\end{align*}
i.e.
\begin{align*}
 &   u_L^I = \frac{1}{\eps_L} ~  \frac{ \eta_L\, (\eps_L u_L) + \eta_R\, (\eps_R u_R)}{\eta_L + \eta_R} , \\
 &   u_R^I = \frac{1}{\eps_R} ~  \frac{ \eta_L\, (\eps_L u_L) + \eta_R\, (\eps_R u_R)}{\eta_L + \eta_R} . 
\end{align*}
