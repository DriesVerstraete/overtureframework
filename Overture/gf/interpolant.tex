\section{Interpolant: Interpolating Grid Functions} \label{Interpolant}\index{interpolant}


The Interpolant class is used for interpolating a CompositeGridFunction --
i.e. obtaining values at the interpolation points in terms of the
values at the other points. Once an Interpolant object has been made
it can be used in one of two ways. One can either use the {\ff interpolate}
function in the Interpolant class or one can use the {\ff interpolate}
function that appears in the grid function class.

When an Interpolant is associated with a Composite Grid, the CompositeGrid will
be changed and will hold a pointer to the Interpolant. Any GridFunction associated
with the CompositeGrid will be able to use the Interpolant found there. This
allows GridFunctions to magically know how to interpolant themselves.


\subsection{Member Functions}

For now it is only possible to interpolate a {\ff real\-Composite\-Grid\-Function}
or a {\ff real\-Multigrid\-Composite\-Grid\-Function}. 

\input InterpolateInclude.tex   

\subsection{Examples}

Here is an example of using the Interpolant class \index{interpolant!test routine}
(file {\ff \examples testInterpolant.C})
{\footnotesize
\listinginput[1]{1}{/home/henshaw/res/gf/testInterpolant.C}
}


\section{Other Interpolation Functions}


\subsection{interpolatePoints: Interpolate a CompositeGridFunction at some given points in space}
\index{interpolate!arbitrary points}

\input interpolatePointsInclude.tex



\noindent {\bf Example:}
In this example we show how to interpolate a grid function at arbitrary points in space.
\begin{verbatim}
  CompositeGrid cg(...);    // get a CompositeGrid from some-where
  Range all;
  realCompositeGridFunction u(cg,all,all,all,2);  // grid function with two components

  int numberOfPointsToInterpolate=1;
  realArray positionToInterpolate(numberOfPointsToInterpolate,3), 
            uInterpolated(numberOfPointsToInterpolate,2);
  for(;;)
  {
    cout << "Enter a point to interpolate (x,y) \n";
    cin >> positionToInterpolate(0,0) >> positionToInterpolate(0,1) ;

    int extrap = interpolatePoints(positionToInterpolate,u, uInterpolated);

    if( extrap < 0 )
      cout << " point was extrapolated" << endl;

    uInterpolated.display("Here is uInterpolated:");
  }
\end{verbatim}





\subsection{InterpolateAllPoints on one CompositeGridFunction from another CompositeGridFunction}

\input interpolateAllPointsInclude.tex

\noindent {\bf Example:} Here is an example
\begin{verbatim}
     CompositeGrid cgFrom(...); 
     realCompositeGridFunction uFrom(cgFrom);   
     uFrom=...;

     ...

     CompositeGrid cgTo(...);
     realCompositeGridFunction uTo(cgTo);   

     interpolateAllPoints(uFrom,uTo);        

\end{verbatim}



\subsection{InterpolateExposedPoints of a CompositeGridFunction for a Moving CompositeGrid}
\index{interpolate!exposed points on a moving grid}


\input interpolateExposedPointsInclude.tex


