\section{Cell-centred and Face-centred Grid Functions} \label{sec:cellFace}
\index{grid function!cell centred}
\index{grid function!face centred}

  In this section we discuss cell-centred and face-centred
grid functions -- how to create them and how to use them.

In a vertex-centred grid function the solution values are defined
at the vertices of the grid (i.e. at the positions defined by the {\tt vertex} array).
By default a grid function defined on a vertex-centred grid will be vertex centered.

In a cell-centred grid function the solution values are assumed to be defined at
the centers of the cells. 
By default a grid function defined on a cell-centred grid will be cell-centred.

In a face-centred grid function the solution values are defined on the face of the
cell. In 2D there are two possible types of face-centered grid functions and
in 3D there are three possibilities. A face centered grid function is vertex centred
in one direction and cell centred in the others.

In three space dimensions a grid function could also be defined on an edge. An edge-centred
grid function is vertex-centred in two directions and cell centred in one.

\subsection{Creating face/cell/vertex centred grid functions in standard form}

The easiest way to create any of the commonly used grid functions is to use
the constructor that takes a {\tt GridFunctionType} and optional Ranges to specify components:
{\footnotesize
\begin{verbatim}
MappedGridFunction(MappedGrid & mg,
                   const GridFunctionParameters::GridFunctionType & type, 
                   const Range & component0=nullRange,
                   const Range & component1=nullRange,
                   const Range & component2=nullRange,
                   const Range & component3=nullRange,
                   const Range & component4=nullRange )
\end{verbatim}
}
This constructor is used to create a grid function of some common types 
in {\sl standard form}. A grid function in standard form will have the first 3 index positions
occupied by the coordinate directions. Any component indices will follow the coordinate indicies.
In addition to the constructor there is a corresponding
{\tt updateToMatchGrid} function that can be used to change the type and/or component dimensions
of a grid function that has already been constructed.

The standard types of grid functions are defined in the enum GridFunctionType, (this enum is found in
the class GridFunctionParameters),
{\footnotesize
\begin{verbatim}
  enum GridFunctionType
  {
    general,
    vertexCentered,
    cellCentered,
    faceCenteredAll,
    faceCenteredAxis1,
    faceCenteredAxis2,
    faceCenteredAxis3
    };
\end{verbatim}
}

\noindent where

\begin{itemize}
  \item \Index{vertexCentered}  : grid function is vertex centred
  \item \Index{cellCentered}    : grid function is cell centred
  \item \Index{faceCenteredAll} : the grid function has components that are face centred in each direction
  \item \Index{faceCenteredAxis1} : grid function is face centred along axis1, (axis1==0)
  \item \Index{faceCenteredAxis2} : grid function is face centred along axis2, (axis2==1)
  \item \Index{faceCenteredAxis3} : grid function is face centred along axis3, (axis3==2)
%  \item general : means same as vertexCentered when used in this constructor
\end{itemize}  


% \newcommand{\half}{{1\over2}}
\newcommand{\BOX}{\makebox(0,0){\framebox(4,4){}}}

A grid function with type {\tt faceCenteredAxis1} will be vertex centered along {\tt axis1}
and cell-centered along the other axes. Thus in 2D the grid function will live on the
vertical faces of the grid cells as shown in figure~\ref{fig:faceCenteredAxis1}.
\begin{figure}[h]\begin{center}                                                    
 \begin{picture}(300,100)(0,-20)
 \put(100,  0){\framebox(60,60){}}
% \put(130, 30){\makebox(0,0){${\bf u}_{i,j}$}}
 \put(100, 30){\makebox(0,0)[r]{${\bf u}_{i-\half,j}$~~}}
 \put(100, 30){\makebox(0,0){$\BOX$}}
% \put(100, 30){\circle*{2}}
 \put(160, 30){\makebox(0,0)[l]{$~~{\bf u}_{i+\half,j}$}}
 \put(160, 30){\makebox(0,0){$\BOX$}}
% \put(130,-10){\makebox(0,0)[t]{${\bf u}_{i,j-\half}$}}
% \put(100,-10){\makebox(0,0)[r]{${\bf u}_{i-\half,j-\half}$}}
% \put(130,  0){\makebox(0,0){$\odot$}}
% \put(130, 70){\makebox(0,0)[b]{${\bf u}_{i,j+\half}$}}
% \put(130, 60){\makebox(0,0){$\odot$}}
% \put(100,  0){\makebox(0,0){$\Diamond$}}
 \end{picture}                                                                  
\caption{A grid function of type {\tt faceCenteredAxis1}} \label{fig:faceCenteredAxis1}
\end{center}\end{figure}   

\noindent
Similiarly a grid function with type {\tt faceCenteredAxis2} will be vertex centered along {\tt axis2}
and cell-centered along the other axes. Thus in 2D the grid function will live on the
horizontal faces of the grid cells as shown in figure~\ref{fig:faceCenteredAxis2}.
\begin{figure}[h]\begin{center}                                                    
 \begin{picture}(300,100)(0,-20)
 \put(100,  0){\framebox(60,60){}}
% \put(130, 30){\makebox(0,0){${\bf u}_{i,j}$}}
% \put(100, 30){\makebox(0,0)[r]{${\bf u}_{i-\half,j}$~~}}
% \put(100, 30){\makebox(0,0){$\BOX$}}
% \put(100, 30){\circle*{2}}
% \put(160, 30){\makebox(0,0)[l]{$~~{\bf u}_{i+\half,j}$}}
% \put(160, 30){\makebox(0,0){$\BOX$}}
 \put(130,-10){\makebox(0,0)[t]{${\bf u}_{i,j-\half}$}}
% \put(100,-10){\makebox(0,0)[r]{${\bf u}_{i-\half,j-\half}$}}
 \put(130,  0){\makebox(0,0){$\BOX$}}
 \put(130, 70){\makebox(0,0)[b]{${\bf u}_{i,j+\half}$}}
 \put(130, 60){\makebox(0,0){$\BOX$}}
% \put(100,  0){\makebox(0,0){$\Diamond$}}
 \end{picture}                                                                  
\caption{A grid function of type {\tt faceCenteredAxis2} } \label{fig:faceCenteredAxis2}
\end{center}\end{figure}   

\noindent
A grid function that is defined to be {\tt faceCenteredAll}
will have components that live on each of the faces. This is accomplished by adding an 
extra component index to the grid function; the dimension of this extra component being
equal to the number of space dimensions (i.e. the number of different faces).
% will have some components that are
% {\tt faceCenteredAxis1}, others that that are {\tt faceCenteredAxis2} and some that
% are {\tt faceCenteredAxis3} (if it is in 3D). 
Thus for example, one could 
make a grid function, {\tt faceNormals}, that holds vectors that are normals to the faces of every cell.
Some of the normals are {\tt faceCenteredAxis1}, some are {\tt faceCenteredAxis2} and
some are {\tt faceCenteredAxis3}.


The next code fragment
{\footnotesize  
\begin{verbatim} 
  MappedGrid mg(...);    // create a grid some-how
  realMappedGridFunction u(mg,GridFunctionParameters::cellCentered,2); 
\end{verbatim}  
}  
will create a cell-centered grid function with two components. The name {\tt cellCentered} is defined
within the Class {\tt GridFunctionParameters} (to avoid polluting the global name space) and so we must
indicate its scope when we use it. Below we show how to do this in an easier way. 
The above declaration would be equivalent to
{\footnotesize  
\begin{verbatim} 
  MappedGrid mg(...); // cell-centered MappedGrid
  Range all;
  realMappedGridFunction u(mg,all,all,all,2);
\end{verbatim}  
}  
assuming that the {\tt MappedGrid} were cell centered. 
Another way to do this would be to use the {\tt updateToMatchGrid} member function
{\footnotesize  
\begin{verbatim} 
  MappedGrid mg(...); 
  realMappedGridFunction u;
  u.updateToMatchGrid(mg,GridFunctionParameters::cellCentered,2);
\end{verbatim}  
}  

Before presenting any further examples 
let us first define some variables {\tt vertexCentered}, {\tt cellCentered}, etc. so that we do not
have to indicate the scope, {\tt GridFunctionParameters::vertexCentered}, when refering to
entries in the GridFunctionType enum,
{\footnotesize
\begin{verbatim}
   const GridFunctionParameters::GridFunctionType vertexCentered    =GridFunctionParameters::vertexCentered;
   const GridFunctionParameters::GridFunctionType cellCentered      =GridFunctionParameters::cellCentered;
   const GridFunctionParameters::GridFunctionType faceCenteredAll   =GridFunctionParameters::faceCenteredAll;
   const GridFunctionParameters::GridFunctionType faceCenteredAll   =GridFunctionParameters::faceCenteredAll;
   const GridFunctionParameters::GridFunctionType faceCenteredAxis1 =GridFunctionParameters::faceCenteredAxis1;
   const GridFunctionParameters::GridFunctionType faceCenteredAxis2 =GridFunctionParameters::faceCenteredAxis2;
   const GridFunctionParameters::GridFunctionType faceCenteredAxis3 =GridFunctionParameters::faceCenteredAxis3;
\end{verbatim}
}
Given these definitions here are some examples
{\footnotesize  
\begin{verbatim} 
  realMappedGridFunction u(mg,vertexCentered,2);   // u(mg,all,all,all,0:1);
  realMappedGridFunction u(mg,cellCentered,2,3);   // u(mg,all,all,all,0:1,0:2);
			      
  realMappedGridFunction u(mg,faceCenteredAxis1);  // u(mg,all,all,all);
  realMappedGridFunction u(mg,faceCenteredAxis2,Range(2,3));  // u(mg,all,all,all,2:3);
			      
  realMappedGridFunction u(mg,faceCenteredAll);    // u(mg,all,all,all,0:d-1);
  realMappedGridFunction u(mg,faceCenteredAll,2);  // u(mg,all,all,all,0:1,0:d-1);
  realMappedGridFunction u(mg,faceCenteredAll,3,2);// u(mg,all,all,all,0:2,0:1,0:d-1);
\end{verbatim}  
}  
The comment following each declaration defines the shape of the resulting grid function.
(Here {\tt d} is the number of space dimensions).
Note that the grid functions created always have the 3 coordinate indices first, followed by the
component indices ({\sl standard form}).
Also note that the grid functions with type {\tt faceCenteredAll} have an
extra index added to the end. The value of this index determines the face centered-ness.
Thus, in the grid function declared {\tt u(mg,faceCenteredAll)} 
the component,{\tt u(all,all,all,i)}, will be face centered
along axis=i, for $i=0,1,...,d-1$, as shown again below
{\footnotesize
\begin{verbatim}
    realMappedGridFunction u(mg,faceCenteredAll);

    u(all,all,all,0)  <---> faceCenteredAxis1
    u(all,all,all,1)  <---> faceCenteredAxis2   (if d>1)
    u(all,all,all,2)  <---> faceCenteredAxis3   (if d>2)
\end{verbatim}
}

Here are some more examples where we also indicate the value returned by some of the query member
functions
{\footnotesize  
\begin{verbatim} 
  realMappedGridFunction u(mg,faceCenteredAxis2,Range(0,1),Range(1,1)); // u(mg,all,all,all,0:1,1:1)
  u.getGridFunctionType()               == faceCenteredAxis2 
  u.getGridFunctionTypeWithComponents() == faceCenteredAxis2With2Components
  u.getNumberOfComponents()             == 2    // number of component indicies, 0=scalar, 1=vector, 2=matrix
  u.getFaceCentering()                  == direction1


  realMappedGridFunction u(mg,faceCenteredAxis1); // u(mg,all,all,all)
  u.getGridFunctionType()               == faceCenteredAxis1
  u.getGridFunctionTypeWithComponents() == faceCenteredAxis1With0Components
  u.getNumberOfComponents()             == 0
  u.getFaceCentering()                  == direction0


  realMappedGridFunction u(mg,faceCenteredAll,3); // u(mg,all,all,all,0:2,0:d-1)
  u.getGridFunctionType()               == faceCenteredAll
  u.getGridFunctionTypeWithComponents() == faceCenteredAllWith1Component
  u.getNumberOfComponents()             == 1
  u.getFaceCentering()                  == all
\end{verbatim}
}

If you need to determine the gridFunctionType for a subset of the components of a grid
function then you can pass optional arguments to the {\tt getGridFunctionType} member function
to indicate which components you want to determine the type of,
{\footnotesize  
\begin{verbatim} 
  realMappedGridFunction u(mg,faceCenteredAll,Range(0,1)); // u(mg,all,all,all,0:1,0:d-1)

  u.getGridFunctionType()             == faceCenteredAll
  u.getGridFunctionType(0)            == faceCenteredAll
  u.getGridFunctionType(Range(0,1))   == faceCenteredAll

  u.getGridFunctionType(0,0)          == faceCenteredAxis1
  u.getGridFunctionType(Range(0,1),0) == faceCenteredAxis1
  u.getGridFunctionType(0,1)          == faceCenteredAxis2    (if d>1)
  u.getGridFunctionType(0,2)          == faceCenteredAxis3    (if d>2)
\end{verbatim}
}
\noindent In the above example we see that a grid function of type {\tt faceCenteredAll} 
has some components that are {\tt faceCenteredAxis1}, some that are {\tt faceCenteredAxis2}
and some that are {\tt faceCenteredAxis3}.



You can change the {\tt gridFunctionType} of an existing grid function by using {\tt updateToMatchGrid}
as shown in the next example:
{\footnotesize  
\begin{verbatim} 
  realMappedGridFunction u(mg,faceCenteredAxis1,Range(0,1)); // u(mg,all,all,all,0:1)

  u.updateToMatchGrid(faceCenteredAxis2);                    // u(mg,all,all,all,0:1)
  u.updateToMatchGrid(mg,faceCenteredAxis3);                 // u(mg,all,all,all,0:1)

  u.updateToMatchGrid(cellCentered,Range(1,3),Range(-1,1))    // u(mg,all,all,all,1:3,-1:1)
\end{verbatim}
}
\noindent Note that the grid function retains its components if no new values are given.
  

% This type of grid function might be used if you had a vector defined on the
% grid (such as a face-normal) 

% The grid function {\tt u2} will be dimensioned {\tt u2(all,all,all)}
% and it will be face centered along {\tt axis=axis1=0}. Simililarly for {\tt u3} and {\tt u4}.






\subsection{Grid functions with arbitrary centredness}
\index{grid function!arbrtray centredness}

The normal user can probably ignore this section.

This section describes how to make grid function with arbitrary centeredness and also
how to make face-centered grid functions that are not in standard form.

In order to represent all these types of centering a grid function has an intArray
called {\tt isCellCentered} which indicates the centering of the grid function
in each coordinate direction. Actually each component of a grid function can have
its own centering but we will ignore this for now.
Thus, for example,

\vskip\baselineskip
\begin{centering}
\begin{tabular}{|l|c|c|c|}
\hline
     & \multicolumn{3}{c|}{\tt isCellCentered(axis)} \\ \hline
\emph{type}        &  axis=0 & axis=1 & axis=2 \\ \hline
vertex centred & FALSE & FALSE & FALSE \\
cell centred   & TRUE  & TRUE  & TRUE  \\
face centred along axis=0  & FALSE  & TRUE  & TRUE  \\
face centred along axis=1  &  TRUE  & FALSE & TRUE  \\
face centred along axis=2  & TRUE   & TRUE  & FALSE \\
edge centred               & TRUE & FALSE & FALSE \\
edge centred               & FALSE & TRUE  & FALSE \\
edge centred               & FALSE & FALSE & TRUE  \\
\hline
\end{tabular}
\end{centering}
\vskip\baselineskip


The member functions {\tt setIsCellCentered}, {\tt setIsFaceCentered} and {\tt setFaceCentering}
can be used to create grid-functions with various centerings. The most general function
is {\tt setIsCellCentered}. The most general function for creating face centered grid functions
is {\tt setIsFaceCentered} while the function {\tt setFaceCentering} can be used to create
face-centred grid functions of some standard forms.



\subsubsection{Semi-general face-centred grid functions}

Although the grid functions support a very general type of
face-centering, in practice one often uses more
specialized forms. The two common forms of face-centred
grid functions are
\begin{enumerate}
  \item A grid function for which all components are face-centred
        in the same direction (i.e. along the same axis). For example
  {\footnotesize\begin{verbatim}
    ...
    const int axis1=0, axis2=1, axis3=2;
    MappedGrid mg(...);
    Range all;                  // a null Range is used when constructing grid functions, it indicates
                                // the positions of the coordinate axes

    realMappedGridFunction u(mg,all,all,all,Range(0,1));
    u.setFaceCentering(axis1);               // u(I1,I2,I3,0:1) : all components face centred along direction 0

    u.updateToMatchGrid(Range(0,2),all,all,all);
    u.setFaceCentering(axis2);               // u(0:2,I1,I2,I3) : all components face centred along direction 1
  \end{verbatim}
  }
  \item A grid function where each component is face centred in all space directions, for example
  {\footnotesize\begin{verbatim}
    ...
    MappedGrid mg(...);
  
    realMappedGridFunction u(mg,all,all,all,Range(0,1),faceRange);  
    // u(I1,I2,I3,0:1,0)  : these components are face-centred along direction 0
    // u(I1,I2,I3,0:1,1)  : these components are face-centred along direction 1
    // u(I1,I2,I3,0:1,2)  : these components are face-centred along direction 2 (if the grid is 3D)

    u.updateToMatchGrid(mg,all,all,all,faceRange,Range(0,0),Range(0,2));
    // u(I1,I2,I3,0,0:0,0:2)  : these components are face-centred along direction 0
    // u(I1,I2,I3,1,0:0,0:2)  : these components are face-centred along direction 1
    // u(I1,I2,I3,2,0:0,0:2)  : these components are face-centred along direction 2 (if the grid is 3D)
  \end{verbatim}
  }
  Note that the special Range, {\ff faceRange} is used to indicate the position of the face centering 
  component. This component will be dimensioned with a Range = Range(0,numberOfDimensions-1).
  Note that only one occurence of {\ff faceRange} must appear in the argument list.
\end{enumerate}

To determine if a grid function is of a given type of face-centering use the function
{\ff getFaceCentering} which returns a value from the enumerator {\ff faceCenteringType}
(found in a MappedGridFunction):
{\footnotesize\begin{verbatim}
  enum faceCenteringType    // Here are some standard types of face centred grid functions
  { 
    none=-1,                // not face centred
    direction0=0,           // all components are face centred along direction (i.e. axis) = 0
    direction1=1,           // all components are face centred along direction (i.e. axis) = 1
    direction2=2,           // all components are face centred along direction (i.e. axis) = 2
    all=-2                  // components are face centred in all directions, positionOfFaceCentering determines
  };                        //   the Index position that is used for the "directions"
}
\end{verbatim}
}

To further illustrate these ideas, consider
the following function which determines whether a grid function is of one of the
standard types:
{\footnotesize\begin{verbatim}
void determineFaceCentering( realMappedGridFunction & u )
{
  switch (u.getFaceCentering())
  {
  case GridFunctionParameters::none :
    cout << " this is not a standard face centered variable \n";
    break;
  case GridFunctionParameters::all :
    cout << "this function has components that are face centred in all space dimensions\n";
    cout << "The value u.positionOfFaceCentering = " << u.positionOfFaceCentering << ", "
            "gives the position of the faceRange";
    break;
  case GridFunctionParameters::direction0 :
    cout << "all components are face centered along direction=0 \n";
    break;
  case GridFunctionParameters::direction1 :
    cout << "all components are face centered along direction=1 \n";
    break;
  case GridFunctionParameters::direction2 :
    cout << "all components are face centered along direction=2 \n";
    break;
  default:
    cout << "Unknown face-centering type! This case should not occur! \n";
  };
\end{verbatim}
}
For further examples see the program {\ff \GF /cellFace.C} which tests
various aspects of cell and face centred grid functions.