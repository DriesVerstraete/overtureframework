%-----------------------------------------------------------------------
%   Grids and Grid Functions
%
%-----------------------------------------------------------------------

\documentclass{article}
\usepackage[bookmarks=true]{hyperref}

% \input documentationPageSize.tex
\hbadness=10000 
\sloppy \hfuzz=30pt

\usepackage{calc}
% set the page width and height for the paper (The covers will have their own size)
\setlength{\textwidth}{7in}  
\setlength{\textheight}{9.5in} 
% here we automatically compute the offsets in order to centre the page
\setlength{\oddsidemargin}{(\paperwidth-\textwidth)/2 - 1in}
% \setlength{\topmargin}{(\paperheight-\textheight -\headheight-\headsep-\footskip)/2 - 1in + .8in }
\setlength{\topmargin}{(\paperheight-\textheight -\headheight-\headsep-\footskip)/2 - 1in -.2in }

\input homeHenshaw


%-----------------------------------------------------------------------
% Grid Function Classes
%-----------------------------------------------------------------------
%123456789 123456789 123456789 123456789 123456789 123456789 123456789 12
\usepackage{amsmath}
\usepackage{amssymb}

\usepackage{verbatim}
\usepackage{moreverb}
\usepackage{graphics}    
\usepackage{calc}
\usepackage{ifthen}
% \usepackage{fancybox}


\usepackage{makeidx} % index
\makeindex
\newcommand{\Index}[1]{#1\index{#1}}

% ---- we have lemmas and theorems in this paper ----
\newtheorem{assumption}{Assumption}
\newtheorem{definition}{Definition}

\newcommand{\OvertureHomePage}{http://www.llnl.gov/casc/Overture}

\newcommand{\primer}{\homeHenshaw/Overture/primer}
\newcommand{\GF}{Overture/tests}
\newcommand{\examples}{Overture/examples}
\newcommand{\gf}{\homeHenshaw/Overture/gf}
\newcommand{\mapping}{\homeHenshaw/Overture/mapping}
\newcommand{\ogshow}{\homeHenshaw/Overture/ogshow}
\newcommand{\oges}{\homeHenshaw/Overture/oges}
\newcommand{\figures}{\homeHenshaw/OvertureFigures}

\newcommand{\OVERTUREOVERTURE}{/n/\-c3servet/\-henshaw/\-Overture/\-Overture}
\newcommand{\OvertureOverture}{\homeHenshaw/Overture/Overture}

\newcommand{\RA}{realArray}
\newcommand{\MGF}{MappedGridFunction}
\newcommand{\RMGF}{realMappedGridFunction}
\newcommand{\RCGF}{realCompositeGridFunction}

\newcommand{\DABO}{Differential\-And\-Boundary\-Operators}

\newcommand{\MG}{Mapped\-Grid}
\newcommand{\GC}{Grid\-Collection}
\newcommand{\CG}{Composite\-Grid}
\newcommand{\MGCG}{Multigrid\-Composite\-Grid}

\newcommand{\MGO}{MappedGridOperators}
\newcommand{\GCO}{Grid\-Collection\-Operators}
\newcommand{\CGO}{Composite\-Grid\-Operators}
\newcommand{\MGCGO}{Multigrid\-Composite\-Grid\-Operators}

\begin{document}

% \input list.tex    %  defines Lentry enviroment for documenting functions

% -----definitions-----
\input wdhDefinitions
\def\comma  {~~~,~~}
\def\uvd    {{\bf U}}
\def\ud     {{    U}}
\def\pd     {{    P}}
\def\calo{{\cal O}}


\vspace{5\baselineskip}
\index{grid functions}
\begin{flushleft}
{\Large
Grid Functions for Overture\\
User Guide, Version 1.00000000000000000000000000000000003 \\
}
\vspace{2\baselineskip}
Bill Henshaw
\footnote{
        This work was partially
        supported by grant N00014-95-F-0067 from the Office of Naval
        Research
        } 
\vspace{\baselineskip}
Centre for Applied Scientific Computing \\
Lawrence Livermore National Laboratory    \\
Livermore, CA, 94551   \\
henshaw@llnl.gov \\
http://www.llnl.gov/casc/people/henshaw \\
http://www.llnl.gov/casc/Overture 
\vspace{\baselineskip}
\today
\vspace{\baselineskip}
UCRL-MA-132231
% LA-UR-96-3464

\vspace{4\baselineskip}

\noindent{\bf Abstract:}
We describe the grids and grid functions that can be used with
Overture. The grid functions are based on the 
fabulous A++ array class library.
\end{flushleft}

\tableofcontents
\listoffigures

\vfill\eject
%---------- End of title Page for a Research Report

\section{Introduction}



Figure \ref{fig:system} gives an overview of the classes that make
up Overture. In this document we will discuss grids and grid functions.

\input \primer/otherDocs.tex

\begin{figure} \label{fig:system}
  \begin{center}
  \includegraphics{\figures /system.idraw.ps}
  \caption{An overview of the Overture classes}
  \end{center}
\end{figure}


\section{Grids}


Grids and collections of grids are the fundamental objects that
PDE solvers have to deal with.
The grid classes that are described here are designed so that they
can be used by a a wide variety of PDE solvers including
\begin{itemize}
  \item solvers written for a single rectangular grid
  \item solvers written with AMR++ to perform adaptive computations
  \item solvers written with Overture for overlapping grids 
  \item solvers that combine AMR++ and Overture
\end{itemize}


\noindent
Here is a list of the various grid classes. The Generic grids are
not really used for anything except to derive from.
\begin{itemize}
  \item {\bf \Index{GenericGrid}} - Derive all grids from this class. This
   can be the base class for both structured and unstructured grids.
  \item {\bf \Index{MappedGrid} : GenericGrid} - Logically rectangular grid
   with a mapping. This class supports the creation of a variety of 
  geometry arrays such as the {\ff vertexCoord}, {\ff vertexDerivatives}
  and also holds boundary condition information etc.
  \item {\bf \Index{GenericGridGridCollection}} - A collection of GenericGrid's.
   Base class for all collections of grids.
  \item {\bf \Index{GridCollection} : GenericGridCollection} - A collection
  of MappedGrid's. Also contains connection (interpolation) information
  and {\ff mask} arrays and parent/child/sibling information for
  adaptive grids. May be a valid ``overlapping'' grid (grids
  cover the entire domain) or may just
  be a subset of the grids contained in a valid ``overlapping'' grid.
  \item {\bf \Index{CompositeGrid} : GridCollection} - This is a valid
   ``overlapping'' grid, grids cover the entire computational domain.
\end{itemize}


\begin{figure} \label{fig:grids}
  \begin{center}
  \includegraphics{\figures /grids.idraw.ps}
  \caption{Class diagram for grid classes}
  \end{center}
\end{figure}


% \subsection{Geoff's documentation on Grids}
% 
% This is file is in {\ff /home/geoff/src/c++/Overture/doc}.
% 
% {\footnotesize
% \listinginput[1]{1}{/home/geoff/src/c++/Overture/doc}
% }

\subsection{MappedGrid}\index{MappedGrid}

{\bf ***  This documentation is out of date *** }

  The mapped grid is a logically rectangular grid with a mapping
function. 

\begin{figure} \label{fig:mappedGrid}
  \begin{center}
  \includegraphics{\figures /mappedGrid.idraw.ps}
  \caption{Class diagram for a MappedGrid}
  \end{center}
\end{figure}

%====================================================================================
\input MappedGrid.tex
% {\footnotesize
% \listinginput[1]{1}{/home/geoff/src/Overture/Overture/MappedGrid.h}
% }
%====================================================================================

\subsection{GridCollection}\index{GridCollection}

See the description of a CompositeGrid.

% Here is the header file:
% {\footnotesize
% \listinginput[1]{1}{/home/geoff/src/Overture/Overture/GridCollection.h}
% }

\subsection{CompositeGrid}\index{CompositeGrid}

{\bf ***  This documentation is out of date *** }

A CompositeGrid is a collection of MappedGrid's along with the information needed
for interpolating between component grids.
 
%====================================================================================
\input CompositeGrid.tex
% Here is the header file:
% {\footnotesize
% \listinginput[1]{1}{/home/geoff/src/Overture/Overture/CompositeGrid.h}
% }
%====================================================================================


\vfill\eject
\section{GridFunctions}\index{grid function}

Almost all applications that use grids will also need to use grid functions.
For example, a PDE solver will need to store the values of the solution
(velocity, pressure, density, ...). These values are defined at each
point on the grid. A grid function will thus hold one or more values
for each point on the grid. The grid function will be associated with 
a grid and thus will know how to dimension itself. Grid functions
also come with a collection of operations. These operations include the standard
arithmetic operators as well as more sophisticated operations such as
differentiation and interpolation.


Here is a list of the available grid functions. Basically each type of
grid has a grid function associated with it.
The keyword ``{\ff type}''
can be any of {\ff double}, {\ff float} or {\ff int}.
\begin{itemize}
  \item {\bf typeMappedGridFunction : typeArray} - this grid function
    is associated with a {\ff MappedGrid} and is derived from an 
    A++ array.
  \item {\bf typeGridCollectionFunction} - this grid function is 
     a collection of {\ff typeMappedGridFunction}'s and is associated
     with a {\ff GridCollection}.
  \item {\bf typeCompositeFunction} - this grid function is 
     a collection of {\ff typeMappedGridFunction}'s and is associated
     with a {\ff CompositeGrid}.
\end{itemize}

\begin{figure} \label{fig:GridFunctions}
  \begin{center}
  \includegraphics{\figures /gf.idraw.ps}
  \caption{Class diagram for grid function classes}
  \end{center}
\end{figure}


%=============================================================================
\vfill\eject
\input MappedGridFunction.tex
%=============================================================================


%=============================================================================
\vfill\eject
\input GridCollectionFunction.tex
%=============================================================================


%==============================================================================
\vfill\eject
\input cellAndFace.tex
%==============================================================================


%==============================================================================
\vfill\eject
\input interpolant.tex
%==============================================================================


\bibliography{\homeHenshaw/papers/henshaw}
\bibliographystyle{siam}

\printindex

\end{document}


