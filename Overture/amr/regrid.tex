\section{Adaptive mesh regridding algorithms} \index{AMR!regridding}


The basic block-structured adaptive mesh refinement regridding algorithm
can be found in the thesis of Berger\cite{BergerThesis}.

The basic idea for building new refinement grids is illustrated in figure~(\ref{fig:amrRegrid}).
The goal is to cover a set of tagged cells by a set of non-overlapping boxes. For efficiency,
the boxes are not allowed to become too small, nor must they be too empty. Each box satisfies
an `efficiency' condition where the ratio of tagged cells to untagged cells must be larger
than some {\tt efficiencyFactor} $\approx .7$.
\begin{enumerate}
  \item Given an estimate of the error, tag cells where the error is too large. Fit a box to
        enclose the tagged cells.
  \item Recursively sub-divide the box. Split the box in the longest direction, at a position based on the 
     histogram formed from the sum of the number of tagged cells per row or column. The exact formula
     is given later.
  \item After splitting the box, fit new bounding boxes to each half and repeat the process. Continue
     until the efficiency of the box is larger than the {\tt efficiencyFactor}.
\end{enumerate}

\newcommand{\tekfig}[3]{
  \leavevmode
  \epsfxsize=#2
  \epsfclipon
  \fboxsep=0.5mm
  \fboxrule=0.2mm
% \centerline{\fbox{\epsfbox[141.7 6.4 531.7 396.4]{#1}}}
  \fbox{{\epsfbox[#3]{#1}}}
}

\begin{figure}[hbt]
  \begin{center}
   \tekfig{gridWithFlaggedCellsAndInitialBox.eps}{.31\linewidth}{50. 160. 450. 675.}
   \tekfig{gridWithFlaggedCellsAndTwoBoxes.eps}{.28\linewidth}{ 95. 165. 455. 700.} 
   \tekfig{gridWithFlaggedCellsAndTwoCompactBoxes.eps}{.29\linewidth}{75. 160. 450. 675.} 
  \caption{The 3 basic steps in regridding are (1) tag error cells and enclose in a box,
             (2) split the box into 2 based on a histogram of the column or
               row sums pf tagged cells, (3) fit new boxes to each split box and repeat if the ratio
               of tagged to untagged cells is too small.} 
          \label{fig:amrRegrid}
  \end{center}
\end{figure}



\subsection{Block based aligned grids}

This algorithm is based on the regridding procedure from LBL, found 
in their HAMR package. This algorithm is based on the algorithm
from Bell-Berger-Collela-Rigoustos-Saltzman-Welcome
\cite{Berger3,bergerColella:1989,bergerRigoutsos:1991,bellBergerSaltzmanWelcome:1993}.

The basic idea is to recursively bisect the region of tagged points
until the resulting patches satisfy an efficiency criteria.

Here is the algorithm
\begin{algorithm}
\begin{programbox}
\mbox{A block AMR grid generator:}
\bc{regrid}(l_b, l_f, G_l,e )
l_b \mbox{ : base level, this level and below do not change}
l_f \mbox{ : fine level, build this new level }
e  \mbox{ : errors defined on all grids for levels $l=l_b,\ldots,l_f-1$}
G_l \cc{Set of boxes for level } l
\overline{G_l} \cc{complement of $G_l$ with respect to some large bounding box.}
E_m \cc{Operator that expands each box by m cells};
R_r \cc{Operator that refines each box by ratio r}
n_b \cc{Number of buffer zones}
\{\qtab
   \mbox{Build proper nesting domains for the base level region $l_b$}
  
  P_{l_b} := \overline{ E_m \overline{ G_{l_b}}}  ~~\cc{ Boxes defining the properly nested region for $l_b$}

  \FOR l=l_b+1,\ldots,l_f-1
    P_l := \overline{R_r \overline{P_{l-1}}}  ~~~\cc{refined version of $P_{l_b}$}
  \END
   \   \\
  \mbox{Build levels from finest level down}
  
  \FOR l=l_f, l_f-1,\ldots,l_b+1 
    \mbox{Build a list of tagged cells for all grids at this level}
    tag \cc{ List that will hold indices of tagged cells}
    \FOR \mbox{each grid at this level}
      tag := tag \cup ( |error| > |tol|)  \cc{add points where error is large}
      tag := \mbox{add $n_b$ neighbours of tagged cells}
      \IF l<l_f \qquad\cc{Enforce proper nesting of level $l+1$ grids in level $l$}
         \mbox{add cells that lie beneath tagged cells (plus a buffer) on level l+1}
      \END
    \END
    b_0 := \bc{buildBox}(tag) \cc{build a box around all tagged cells}
    \mbox{Recursively sub-divide the box}
    B :=0 \cc{ The set that will contain the refinement boxes}
    \bc{split}( b_0,B ) 
    B := \bc{merge}( B ) 
  \END
\untab
\}
\end{programbox}
\end{algorithm}

\begin{programbox}
\mbox{Recursively sub-divide a box:}
\bc{split}(\bc{box}~b, \bc{boxSet}~B )
\{\qtab
  \IF b \mbox{ is efficient or too small}
    \keyword{if}~ b \subset P_l  
      ~~\mbox{b lies in the region of proper nesting, }  P_l
      ~~B := B \cup b;
    \keyword{else}
      ~~B_0 := b \intersect P_l \cc{set of boxes for intersection}
      ~~B := B \cup B_0  \cc{add boxes from } B_0
    \keyword{end}
   \untab\keyword{else}\tab
    \mbox{divide b into 2 boxes, split along the longest side}
    \{b_1,b_2\} := \bc{subDivide}(b)
    \bc{split}(b_1,B)
    \bc{split}(b_2,B)
  \END
\untab
\}
\end{programbox}

\begin{programbox}
\mbox{Sub-divide a box in a smart way:}
\bc{subDivide}(\bc{box}~b, \alpha )
\alpha \mbox{  : divide box along this axis}
\{\qtab

  h_i := \mbox{ number of tagged points with $i_\alpha=i$, } ~i=0,\ldots,N
  m:= N/2
  Z := \{ i \origbar h_i \equiv 0 \}
  \IF Z \ne \emptyset 
    \mbox{split at the middle most point where the $h_i=0$}
    j := m + \min_{i\in Z} \modbar{i-m} 
  \untab\keyword{else}\tab
     \mbox{split at the middle most point where the 2nd derivative changes sign,}
     \mbox{and the 3rd derivative is largest in magnitude}
     d_i := h_{i+1}-2h_i+h_{i-1}

     S := \{ i ~\origbar~  d_{i-1} d_i < 0 \mbox{ and } \modbar{d_i-d_{i-1} } \mbox{ is maximal} \}
     \keyword{if} ~S \ne \emptyset 
     ~~  j:= m + \min_{i\in S} \modbar{i-m}
     \keyword{else}
     ~~ j:=m        
     \keyword{end}
  \END
  \mbox{split box $b$ at index $j$ creating boxes $b_1$ and $b_2$}
  return \{b_1,b_2\}
\untab
\}
\end{programbox}



Figure~(\ref{fig:properNesting}) illustrates the proper nesting region ($P_l$ in the above algorithm) for a
set of boxes. The proper nesting region for a set of boxes is the region  interior to the set of
    boxes which lies a certain buffering distance from the coarse grid boundary. The proper
     nesting region defines the region into which new refinement patches are allowed to lie.

\begin{figure}[hbt]
  \begin{center}
   \tekfig{properNestingRegion.eps}{.4\linewidth}{100. 275. 500. 700.}
  \caption{The proper nesting region for a set of boxes is the region  interior to the set of
    boxes which lies a certain buffering distance from the coarse grid boundary. The proper
     nesting region defines the region into which new refinement patches are allowed to lie.} \label{fig:properNesting}
  \end{center}
\end{figure}


\begin{figure}[hbt]
  \begin{center}
  \epsfig{file=diag1.ps,width=.3\linewidth}
  \epsfig{file=diag2.ps,width=.3\linewidth} 
  \epsfig{file=diag3.ps,width=.3\linewidth} 
  \epsfig{file=diag4.ps,width=.3\linewidth} 
  \epsfig{file=diag5.ps,width=.3\linewidth} 
  \epsfig{file=diag.ps,width=.3\linewidth} 
  \caption{Steps in the AMR regridding algorithm. The grid is shown after 1,2,3,4,5 and all steps.
     The tagged error points are marked as black dots.}
  \end{center} \label{fig:diag}
\end{figure}



\begin{figure}[hbt]
  \begin{center}
  \epsfig{file=hollowCircle3levels.ps,width=.45\linewidth}
  \epsfig{file=cicCross3Levels.ps,width=.45\linewidth} 
  \caption{Some example amr grids made with the amrGrid test program.}
  \end{center} \label{fig:amrExamples}
\end{figure}
