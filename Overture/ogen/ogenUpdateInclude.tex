\subsubsection{Interactive updateOverlap}
 
\newlength{\ogenUpdateIncludeArgIndent}
\begin{flushleft} \textbf{%
int  \\ 
\settowidth{\ogenUpdateIncludeArgIndent}{updateOverlap(}%
updateOverlap( CompositeGrid \& cg, MappingInformation \& mapInfo )
}\end{flushleft}
  
\begin{description}
\item[{\bf Description:}] 
   Use this function to interactively create a composite grid.

\item[{\bf mapInfo (input) :}]  a MappingInformation object that contains a list of
   Mappings that can be used to make the composite grid. NOTE: If mapInfo.graphXInterface==NULL
   then it will be assumed that mapInfo is to be ignored and that 
   the input CompositeGrid cg will already have a set of grids in it to use.


 Here is a description of some of the commands that are available from the
 {\tt updateOverlap} function of {\tt Ogen}. This function is called when you 
 choose ``{\tt generate overlapping grid}'' from the {\tt ogen} program.
 
 \begin{description}
   \item[compute overlap] : this will compute the overlapping grid. As the grid is generated various
     information messages are printed out. Some of these messages may only make sense to the 
      joker who wrote this code.
   \item[change parameters] : make changes to parameters. See the next section for details.
   \item[display intermediate results] : this will toggle a debugging mode. When this mode
     is on, and you choose {\tt compute overlap} to generate the grid, then the overlapping grid 
     will be plotted at various stages in its algorithm. The algorithm is described in section
     (\ref{algorithm}). The program will pause at the end of each stage of the algorithm and
     allow you to either {\tt continue} or to {\tt change the plot} as described next.
      Experienced users will be able to see when something goes wrong and hopefully detect the cause.
   \item[change the plot] : this will cause the grid to be re-plotted. You will be in the grid plotter
     menu and you can make changes to the style of the plot (toggle grids on and off, plot interpolation
     points etc.). These changes will be retained when you exit back to the grid generator.
 
 \end{description}

\end{description}
\subsubsection{Non-interactive updateOverlap}
 
\begin{flushleft} \textbf{%
int  \\ 
\settowidth{\ogenUpdateIncludeArgIndent}{updateOverlap(}%
updateOverlap( CompositeGrid \& cg )
}\end{flushleft}
\begin{description}
\item[{\bf Description:}] 
 Build a composite grid non-interactively using the component grids found
 in cg. This function might be called if one or more grids have changed.
\item[{\bf Return value:}]  0=success, otherwise the number of errors encountered.
 
\end{description}
\subsubsection{Moving Grid updateOverlap}
 
\begin{flushleft} \textbf{%
int  \\ 
\settowidth{\ogenUpdateIncludeArgIndent}{updateOverlap(}%
updateOverlap(CompositeGrid \& cg, \\ 
\hspace{\ogenUpdateIncludeArgIndent}CompositeGrid \& cgOld, \\ 
\hspace{\ogenUpdateIncludeArgIndent}const LogicalArray \& hasMoved, \\ 
\hspace{\ogenUpdateIncludeArgIndent}const MovingGridOption \& option  =useOptimalAlgorithm)
}\end{flushleft}
\begin{description}
\item[{\bf Description:}] 
    Determine an overlapping grid when one or more grids has moved.
   {\bf NOTE:} If the number of grid points changes then you should use the 
   {\tt useFullAlgorithm} option.
 
\item[{\bf cg (input) :}]  grid to update
\item[{\bf cgOld (input) :}]  for grids that have not moved, share data with this CompositeGrid.
\item[{\bf hasMoved (input):}]  specify which grids have moved with hasMoved(grid)=TRUE
\item[{\bf option (input) :}]  An option from one of:
 {\footnotesize
 \begin{verbatim}
   enum MovingGridOption
   {
     useOptimalAlgorithm=0,
     minimizeOverlap=1,
     useFullAlgorithm
   };
 \end{verbatim}
 }
  The {\tt useOptimalAlgorithm} may result in the overlap increasing as the grid is moved.

\item[{\bf Return value:}]  0=success, otherwise the number of errors encountered.
 
\end{description}
