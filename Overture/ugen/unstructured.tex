%=======================================================================================================
% Unstructured Mesh Documentation
%=======================================================================================================
\documentclass{article}
\usepackage[bookmarks=true]{hyperref}

% \input documentationPageSize.tex
\hbadness=10000 
\sloppy \hfuzz=30pt

\usepackage{calc}
% set the page width and height for the paper (The covers will have their own size)
\setlength{\textwidth}{7in}  
\setlength{\textheight}{9.5in} 
% here we automatically compute the offsets in order to centre the page
\setlength{\oddsidemargin}{(\paperwidth-\textwidth)/2 - 1in}
% \setlength{\topmargin}{(\paperheight-\textheight -\headheight-\headsep-\footskip)/2 - 1in + .8in }
\setlength{\topmargin}{(\paperheight-\textheight -\headheight-\headsep-\footskip)/2 - 1in -.2in }

\input homeHenshaw

\usepackage{epsfig}
\usepackage{subfigure}

\begin{document}


\def\uvd    {{\bf U}}
\def\ud     {{    U}}
\def\pd     {{    P}}
\def\id     {i}
\def\jd     {j}
\def\kap {\sqrt{s+\omega^2}}

\newcommand{\primer}{\homeHenshaw/res/primer}
\newcommand{\gf}{\homeHenshaw/res/gf}
\newcommand{\mapping}{\homeHenshaw/res/mapping}
\newcommand{\ogshow}{\homeHenshaw/res/ogshow}
\newcommand{\oges}{\homeHenshaw/res/oges}
% \newcommand{\figures}{../docFigures}
\newcommand{\figures}{\homeHenshaw/OvertureFigures}

\begin{flushleft}
 ~~  \\
 ~~  \\
 ~~  \\
 ~~  \\
 ~~  \\
  {\Large 
   Unstructured Hybrid Mesh Support for Overture  \\ 
   A Description of the Ugen and AdvancingFront Classes  \\
   and Documentation for Additional Support Classes 
  }\\
\vspace{2\baselineskip}
Kyle K. Chand \\
% \footnote{
%         This work was partially
%         supported by 
%         }   \\
Centre for Applied Scientific Computing \\
Lawrence Livermore National Laboratory    \\
Livermore, CA, 94551   \\
chand@llnl.gov \\
http://www.llnl.gov/casc/people/chand \\
http://www.llnl.gov/casc/Overture \\
\vspace{1\baselineskip}
\today
\vspace{\baselineskip}

\end{flushleft}

\vspace{1\baselineskip}

\begin{abstract}
Overture's support for unstructured and hybrid mesh generation is implemented 
through the classes described in this document.  There are three tiers
of classes used for the generation of unstructured meshes.  The first tier consists
of container classes, essentially {\tt GeometricADT}, used by the mesh generator to 
perform geometric searches.  The {\tt AdvancingFront} class encapsulates the
logic of an advancing front mesh generator and uses {\tt GeometricADT}'s. Finally, 
{\tt Ugen} acts as the interface between the unstructured mesh generator and the 
rest of Overture.

The following classes are described in this document:
\begin{itemize}
 \item Ugen : Unstructured/Hybrid mesh generator interface
 \item AdvancingFront : unstructured mesh generation using the advancing front method
 \item GeometricADT : bounding box alternating digital tree for geometric searches
 \item GeometricADT::iterator : defines an iteration path through a GeometricADT
 \item GeometricADT::traversor : performs a search traversal of a GeometricADT
 \item NTreeNode$<$int degree, class Data$>$ : primitive container class used to build GeometricADT
 \item Exceptions : a list of the various exceptions thrown by the above classes
 \item CompositeGridHybridConnectivity : manages the mappings between unstructured and structrutred components of a CompositeGrid
\end{itemize}
\end{abstract}

\tableofcontents

\vspace{3\baselineskip}

\section{Introduction}
Hybrid meshes consist of regions of structured grids joined by unstructured 
meshes.  Figure~\ref{fig:overlapHybridComp} compares overlapping and hybrid meshes
for the same geometry.  Generation of unstructured hybrid meshes in Overture is orchestrated by the 
class {\tt Ugen}.  When provided with a {\tt CompositeGrid},
{\tt Ugen} removes the overlap, determines the hole boundaries and conducts the
generation of an unstructured mesh filling the spaces between component grids. 
Currently, the {\tt CompositeGrid} must contain an overlapping grid with holes 
cut using the {\tt ``compute for hybrid mesh''} option in {\tt Ogen}.  To generate
unstructured meshes, {\tt Ugen} utilizes an advancing front method implemented
in the class {\tt AdvancingFront}.  One the unstructured regions have been generated they
are placed into an {\tt UnstructuredMapping} and subsequently added to the 
original CompositeGrid.  Connectivity between the structured and unstructured regions 
can be accessed through the class {\tt CompositeGridHybridConnectivity}, an instance of
which is created in the original {\tt CompositeGrid}.

\begin{figure} [htb] 
\centering
 \mbox{
       \subfigure[]{\epsfig{file=\figures/overlappingExample.eps,width=.35\textwidth}}
       \hspace{.25in}
       \subfigure[]{\epsfig{file=\figures/hybridExample.eps,width=.35\textwidth}}}
 \caption{Comparison of Overlapping and Hybrid Meshes : (a) Overlapping Grid; (b) Hybrid Mesh} \label{fig:overlapHybridComp} 
\end{figure}

\section{Generating Hybrid Meshes}
The environment provided by {\tt Ugen} assumes that mappings have already been created and
that the user is prepared to assemble a hybrid mesh.  Selecting the ``generate hybrid mesh'' option
in the main {\tt Ogen} menu initially proceeds in a manner similar to the overlapping grid version.
A list of the mappings involved in the construction of the hybrid mesh is built.  As with the
overlapping grid case, a priority is assigned to each mapping with selections being made in 
ascending order of priority.  Higher priority grids will cut holes in lower priority ones.
Figure~\ref{ugenEnv.eps} displays the initial state of the hybrid mesh generation environment
just after the mappings for Figure~\ref{overlapHybridComp}a have been selected.  There are several
options now available via the pop-up menu :\\
\begin{itemize}
\item set plotting frequency (<1 for never) : This options selects how many unstructured elements will be generated before the plot is updated. Setting the value to -1 forces the mesh generator to continue until the mesh is completely generated or an error occurs.
\item continue generation : continue generating the unstructured mesh until the plotting frequency is reached; the mesh is complete; or an error occurs.
\item enlarge hole : enlarge the hole and reinitialize the mesh.  This option will destroy an already generated unstructured mesh
\item reset hole : destroy any unstuructured mesh elements and reinitialize the algorithm
\item plot component grids (toggle) : plot/do not plot component grids
\item plot control function (toggle) : plot/do not plot the control function
\item open graphics : open a graphics dump file
\item plot object : plot/refresh the image
\item change the plot : change plotting characteristics such as the component grids and control function
\item exit : exit the hybrid mesh generation interface, this will attempt to add the unstructured mesh to the {\tt CompositeGrid} and generate the connectivity between the unstructured and structured meshes.
\end{itemize}

\subsection{Preprocessing the Composite Grids}
\subsection{Class Ugen}
\input Ugen.tex

\section{Advancing Front Mesh Generation}
Advancing front mesh generation creates an unstructured mesh by ``growing'' 
elements off of an initial set of curves (2D) or surfaces (3D) which describe a connected domain.
Discretized representations of these curves or surfaces are collectively
known as the "front" which is successively "advanced" until the domain
is filled with an unstructured mesh.  Figure~\ref{fig:advfront} illustrates an advancing front. 
In this case, the initial curve is bold black,
the front is bold red with dots and the generated elements are black
The implmentation of the mesh generator follows that of Peraire, J, Peiro, J. and Morgan, K., Ref ??.
\begin{figure}
 \begin{center}
  \epsfig{file=\figures/advFront.eps}
  \caption{An advancing front (red) grows off of an initial curve (black)}\label{fig:advfront}
  \end{center}
\end{figure}

\subsubsection{Mesh Generation Algorithm}
Advancement of the ``advancing'' front refers
the the process by which new elements are created.  The process may be summarized as :
\begin{enumerate}
\item Choose and discretize an initial boundary
\item Select a face to advance
\item Find all nearby old vertices
\item Calculate possible new vertices
\item Prioritize and choose a vertex from the list of old and new vertices
\item Delete the starting face from the front and add any new faces
\item Return to 2 and repeat until the front is empty.
\end{enumerate}

Initially, a discretized
curve (2D) or surface (3D) must be created that encloses the region to be meshed.  
Assuming we start with 2D curve discretized into a set of line segments (called
``faces''), mesh generation consists of generating elements/zones from faces in the 
front.  As faces are used they are deleted from the front, while new faces are created
and added to it.  Eventually the front will collapse upon itself, becoming empty, 
resulting in a completed mesh.
Zooming into the dashed circle of Figure~\ref{fig:advfront} provides an
opportunity to examine a sample front advance.  Figure~\ref{fig:advfront-cand} shows
a face chosen for advancement and marks several important features of the algorithm.
The letter symbols indicate the points that could be used to complete a new element and
dashed lines represent the new elements they would create.
\begin{figure}
 \begin{center}
 \epsfig{file=\figures/advFront-cand.eps}
 \caption{A cluttered view of candidate vertices}\label{fig:advfront-cand}
 \end{center}
\end{figure} 
A new element may be created by using existing points belonging to other
faces in the front or a new point may be created.  

A user defined notion of
the ``ideal'' element in that region of the mesh leads to the calculation of
the ``ideal'' new point, $P_{\mbox{ideal}}$.  In a uniform 2D mesh the ideal element would be an
equliateral triangle, but typically mesh size and quality control
parameters can permit stretched elements.  Candidate points that already exist in
the front are sought within a circle with an origin at $P_{\mbox{ideal}}$ and passing throught the 
points in the face. Figure~\ref{fig:advfront-cand} shows this search circle 
in bold black and marks the candidates already in the front as $E_{1}$ and $E_{2}$.  
An element can also be created by generating a new point.  In addition to the ``ideal''
point, the locations of new points are the centers of the circles created using the two edge 
points and each existing candidate front point.  In this case, the two circles
are shown in red and blue for the circles using $E_{1}$ and $E_{2}$ respectively. $P_{1}$
and $P_{2}$ are the additional new points to be considered.  Now the algorithm must choose 
which point to use, or, equivalently, which element to create.

The candidates $E_{1}, E_{2}, P_{\mbox{ideal}}, P_{1}, P_{2}$, are prioritized and 
assembled into a queue.  Vertices that already exist, $E_{1}$ and $E_{2}$ are ordered 
according to how close the element they would create is to the ``ideal'' element.
Next comes the actual ``ideal'' vertex $P_{\mbox{ideal}}$.  Then the additional
candidate new ( $P_{1}, P_{2}$) vertices are ordered in a similar way to the 
existing vertices, the closer to ``ideal'' the element would be, the higher the priority.  
In Figure~\ref{fig:advfront-cand}, the ordering would be : 
$E_{1}, E_{2}, P_{\mbox{ideal}}, P_{1}, P_{2}$.  The first candidate in this queue that
generates a valid element (no intersections) is chosen to complete the new element.  Once the new
element has been created any new faces are added to the front.  The
original face, now buried beneath the front, is removed.  Figure~\ref{fig:advfront-newfront} 
illustrates the end result of our example.  This process is repeated until the front is empty.
\begin{figure}
 \begin{center}
 \epsfig{file=\figures/advFront-newfront.eps}
 \caption{The new front}\label{fig:advfront-newfront}
 \end{center}
\end{figure}	

\subsubsection{Mesh stretching control}
you'll have to wait for this subsection

\subsection{Class AdvancingFront}
\input AdvancingFront.tex

\vfill\eject

\section{Geometric Searching}
and this subsection 
\subsection{Class GeometricADT}
\input GeometricADT.tex

\subsection{Class GeometricADT::iterator}
Creation of a {\tt GeometricADT::iterator} requires the specification of a 
target representing a location in bounding box space.  The iterator will then
iterate from the root of a GeometricADT to the terminal leaf that would store
an object with the given target.  The default constructor is protected since 
the iterator requires the context of both a GeometricADT and a target bounding
box. 
\input GeometricADTIterator.tex
%\vfill\eject
\subsection{Class GeometricADT::traversor}
A {\tt GeometricADT::traversor} will traverse it's {\tt GeometricADT} yielding 
all the elements that overlap the target bounding box.  This class is the basic
tool used for performing geometric searches using the {\tt GeometricADT} class.
%\input GeometricADTTraversor.tex
%\vfill\eject
\section{Class NTreeNode$<$int degree, Class Data$>$}
\input NTreeNode.tex
%\vfill\eject
\section{Exception Classes}
%\input Exceptions.tex
%\vfill\eject
\section{Composite Grid Hybrid Connectivity}
\input CompositeGridHybridConnectivity.tex
\input CompositeGridHybridConnectivityImp.tex

\end{document}
