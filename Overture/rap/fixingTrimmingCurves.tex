\section{Fixing errors in trimming curves}\label{sec:fixingTrimCurves}


% -------------------------------------------------------------------
\subsection{Example: Self-intersecting trimming curve}

  
A trimming curve may be invalid if it intersects itself. One
common way this can happen is near a cusp in the geometry.
The CAD program may not care if the trim curve intersects itself
near the end of the cusp since it is not visible. We must fix this
problem, however, so that we can build a proper water-tight triangulation. 

\input selfIntersectionTrimCurveFig.tex

Figure~\ref{fig:selfIntersectingTrimCurveFig} shows an example. 
One good way to fix this problem is to 
\begin{enumerate}
  \item split one of the curves near the cusp but before
    the intersection point (usually one should split the sub-curve that
        has more curvature).
  \item hide the small portion of the split curve that contains
    the intersection.
  \item Join the split sub-curve to the end-point of the cusp with
        a straight-line
\end{enumerate}




% -------------------------------------------------------------------
\subsection{Example: A trimming curves with missing and self-intersecting segments}\label{sec:retroSurf11}


  In this example there are a number of mistakes in the trimming curves.
Figure~\ref{fig:retroSurf11Trim1} shows the original trim curve (plot the sub-curves in difference colours)
along with a gap between segments that is fixed by adding a line segment.

Figure~\ref{fig:retroSurf11Trim2} show two other problems where the trim curve intersects
itself. These are fixed using the {\em snap-to-intersection} option.


\input retroSurf11TrimCurveFig.tex

