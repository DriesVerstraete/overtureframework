%
% These sections also appear in GraphicsDoc.tex -- should share 
%
% 
%  Suggested title for this section is 
% ------------ \section{Mouse, Button and View Characteristics of the Graphics Window}
% ------------ \section{Features and Properties of the Overture Graphical User Interface}

This section describes some of the features and properties of the Overture graphical user interface
which is illustrated in figure~\ref{fig:clip-plane}.
Some of these features are

\begin{itemize}
%
  \item User defined dialog windows that can contain pulldown menus
  with push or toggle buttons, option menus, text labels (for
  inputting strings), push buttons, and toggle buttons.
%
  \item Multiple graphics windows and a single command window with a
  scrollable sub-window for outputting text, a command line, and a
  scrollable list of previous commands.
%
  \item Rotation buttons 
     \includegraphics{\figures/rot_x_p},
     \includegraphics{\figures/rot_y_p}, 
     \includegraphics{\figures/zrotp} ,... which
     rotate the object on the screen about fixed x, y, and z axes (the
     x axis is to the right, the y-axis is up and the z-axis is out of
     the screen).
%
  \item Translation buttons \includegraphics{\figures/right},
  \includegraphics{\figures/down}, 
  \includegraphics{\figures/zp},
  \includegraphics{\figures/zm},... which shift
  the object on the screen along a given axis. The two last buttons
  shift the object in and out of the screen, respectively. Since an
  othographic projection method is used in the graphics interface, these buttons only change
  the appearance when clipping planes are used.
%
  \item Push buttons for making the objects bigger:
  \includegraphics{\figures/zoom_in}, 
  or smaller: \includegraphics{\figures/zoom_out},
  and a reset button: \includegraphics{\figures/reset} to reset the view point, and a {\bf clear} button
  to erase all objects on the
  screen.\index{bigger,smaller,clear,reset}
%
  \item A push button \includegraphics{\figures/rotpnt} to set the rotation center (after selecting this
    button, pick a point on the displayed grid or surface to use as a new rotation point).
%
  \item A {\bf rubber band zoom} feature.
%
  \item Mouse driven {\bf translate, rotate and zoom}.
%
  \item A pop-up menu that is active on the command and graphics
  windows. This menu is defined by the application (= user program).
%
  \item Static pull-down menus ({\bf file}, {\bf view},
     and {\bf help}) on the graphics windows. Here, the screen can be saved in
     different formats, clipping planes and viewing characteristics
     can be set, annotations can be made (not fully implemented), and some help can be found. 
%
  \item Static pull-down menus ({\bf file} and {\bf help}) on the
     command window. Here command files can be read/saved, new
     graphics windows can be opened, the window focus can be set, and
     the application can be aborted.
%
  \item An optional pull-down menu ({\bf My menu} in this case) that is
     defined by the application.
%
  \item Pushbuttons ({\bf Pick 3D} and {\bf Plot} in this case) that are defined by the application.
%
  \item A file-selection dialog box (not shown in the figure).
%
  \item The option of typing any command on the command line or reading
  any command from a command file. All commands can be entered in this
  fashion, including any pop-up or pull-down menu item or any of the
  buttons, {\bf x+r:0}, {\bf y-r:0}, {\bf x+:0}, {\bf y+:0}, {\bf
  bigger:0}, etc. For the buttons, the :0 refers to the window number
  where the view should be modified, which in this case is window
  \#0. Furthermore, when typing a command, only the first
  distinguishing characters need to be entered.
%
  \item Recording or retrieving a command sequence in a command file.
\end{itemize}



\subsection{Using the mouse for rotating, scaling and picking}

Rotations are performed with respect to axes that are fixed relative
to the screen. The x-axis points to the right, the y-axis points
upward and the z-axis points out of the screen.  Rotations can either
be performed about the centre of the window, or about a user defined
point. This point can be set by first pressing the {\bf set
rotation point} icon on the graphics window and then clicking on the
screen with the left mouse button. The rotation point can also be set
by opening the ``Set View Characteristics'' dialog from the ``View''
pull-down menu (see section~\protect\ref{sec:view-characteristics} for
details). Note that the centre of the window is changed with the
translation commands {\bf x+}, {\bf x-}, {\bf y+}, etc.

Typing a rotation command with an argument (on the command line), 
such as {\bf x+r:1 45}, will cause the view in window number 1 to 
rotate by 45 degrees about the x-axis. If no window number is given
with the commands, such as, {\bf y-y 30}, the command will apply to
window 0.

Typing a translation command with an argument, such as {\bf x+:0 .25}
will cause the view in window number 0 to move to the right $.25$ units
(in normalized screen coordinates; the screen goes from -1 to 1).

{\bf Rubber band zoom:} \index{rubber band zoom} The middle mouse
button is used to ZOOM in. Press the middle button at one corner of a
square, drag the mouse to another corner and lift the button.  The
view will magnify to the square that was marked.  Use `reset' to reset
the view.

{\bf Mouse driven translate, rotate and zoom}: \index{mouse
button!translate, rotate and zoom} All these operations are performed
with the SHIFT key down. To translate, you hold the SHIFT key down,
press the left mouse button and drag the cursor; the plotted objects
will translate in the same direction as the mouse is moved. To rotate
the view, you hold the SHIFT key down and press the middle mouse
button and drag the cursor. Moving the cursor left or right will
rotate about the y-axis (the vertical screen direction) and moving up
or down will rotate about the x-axis (horizontal screen direction). To
zoom in or out, you hold the SHIFT key down, press the right mouse
button and drag the cursor vertically. To rotate about the z-axis, you
press the left mouse buttons and drag the cursor horizontally.

{\bf Picking (aka selecting):} While clicking the left mouse button,
you select an object and get the $(x,y,z)$ coordinate of the point on
the object where you clicked. The object that was selected is reported
in the selectionInfo data structure, which holds the global ID number,
the front and back z-buffer coordinates and the window and 3--D
coordinates. The global ID number can be used by the application to
identify the selected object.

It is also possible to select several objects on the screen by
specifying a rectangular region. To do this, you press the left mouse
button in one corner of the rectangle and drag it to the diagonally
opposite corner of the rectangle, where the mouse-button is
released. The program will draw a rectangular frame to indicate the
selected region. When you are happy with the selected region, you
release the mouse button. An imaginary viewing volume is defined by
translating the rectangular region into the screen and all objects
that intersect the viewing volume are selected. However, only the 3--D
coordinate of the closest object is computed. We remark that the
object with the lowest front z-buffer value is the closest to the
viewer. Also note:
\begin{enumerate}
\item Objects that are hidden by another object are also selected by
this method.
\item The selection takes clipping planes (see below) into
account, so it is not possible to select an object that has been
removed by a clipping plane.
\end{enumerate}
%
The mouse driven features are summarized in table~\protect\ref{tab:mouse}.
\begin{table}[ht]
\begin{center}
\begin{tabular}{r|r|l}  
Modifier      & Mouse button & Function                      \\ \hline\hline
              & left         & picking                       \\ \hline
              & middle       & rubber band zoom              \\ \hline
              & right        & pop-up menu                   \\ \hline
$<$SHIFT$>$   & left         & translate                     \\ \hline
$<$SHIFT$>$   & middle       & rotate around the x \& y axes \\ \hline
$<$SHIFT$>$   & right        & zoom (up \& down) and z-rotation (left \& right) \\ \hline
\end{tabular}
  \caption{Mouse driven features.}\label{tab:mouse}
\end{center}
\end{table}

\newcommand{\figWidth}{.95\linewidth}
% ----------------------------------------------------------------------------
\subsection{Using clipping planes}
The clipping plane dialog for each graphics window is opened from the
`View' pull-down menu on the menu bar in the graphics
window. Figure~\protect\ref{fig:clip-plane} shows an example from the test
program. The first clip plane is activated by clicking on the toggle
button in the top left corner. After it is activated, we can look
inside the cube and see the sphere. By dragging the slider bar for the
clipping plane, the clipping plane is moved closer or further away
from the eye. The direction of the normal of the clipping plane can
also be changed by editing the numbers in the `Normal' box.
\begin{figure}
  \begin{center}
     \includegraphics[width=\figWidth]{\figures/clipPlane}
  \caption{The Overture graphics interface. The clipping plane dialog window is also shown}\label{fig:clip-plane}
  \end{center}
\end{figure}

\subsection{Setting the view characteristics} \label{sec:view-characteristics}
The view characteristics dialog for each graphics window is opened
from the `View' pull-down menu on the menu bar in the graphics
window. Figure~\protect\ref{fig:view-char} shows an example from the test
program. NOTE: When entering numerical values into a text box, it is
necessary to hit $<$RETURN$>$ before the changes take effect.

Here follows a brief description of the functionality in this window:
\begin{description}
\item[Background colour:]Select a background colour from the menu by
clicking on the label with the current colour. In this example, the
background colour is white. Changing the background colour will take
effect immediately.
%
\item[Text colour:]Select a text (foreground) colour from the menu by
clicking on the label with the current colour. In this example, the
text colour is steel blue. Note that the text colour is used to colour
the axes, the labels, and sometimes also the grid lines. However, only
the axes will change colour immediately after a new colour is
chosen. To update the colour of the labels and the grid lines, it is
necessary to replot the object on the screen.
%
\item[Axes origin:] Click on the radio buttons to set the origin of
the coordinate axes either at the default location (lower, left,
back corner of the bounding box), or at the rotation point.
%
\item[Rotation point:] Enter the (X, Y, Z) coordinates of the rotation
point. The rotation point will remain fixed to the screen during both
interactive rotation with $<$SHIFT$>$+middle mouse button, and during
rotation with the buttons {\bf x+r}, {\bf y+r}, etc.
%
\item[Pick rotation point:]After clicking on this button, set the
rotation point by clicking with the left mouse button on a point
on an object in the graphics window. NOTE: 
\begin{enumerate}
\item Picking will only work in the window from which the view
characteristics dialog was opened. If you are unsure which window to
pick in, you can press the right mouse button and read the window
number from the title of the popup menu.
\end{enumerate}
%
\item[Lighting:]Activate/deactivate lighting in the graphics window.
%
\item[Light \#i:]Turn on/off light source number $i$,
$i=0,1,2$. Turning off all light sources is the same as deactivating
lighting with the above function.
%
\begin{description}
\item[Position (X, Y, Z):]The location (X, Y, Z) of light source
number i. 
\item[Ambient (R, G, B, A):]The ambient colour (R, G, B, A) of light source
number i.
\item[Diffusive (R, G, B, A):]The diffusive colour (R, G, B, A) of light source
number i.
\item[Specular (R, G, B, A):]The specular colour (R, G, B, A) of light source
number i.
\end{description}
%
\item[X-colour material properties:] The following properties
characterize the default material, which is used when a X-colour is
chosen for a lit object. The X-colours are distinguished from the
predefined ``special'' materials in that only their ambient and
diffuse reflections are defined, but not their specular and shininess
properties. (For those fluent in OpenGL, this functionality is
obtained by calling the function glColorMaterial with the argument
GL\_AMBIENT\_AND\_DIFFUSE.)  Note that the reflective properties only
influence objects that are lit. Also note that the objects need to be
re-plotted in order for the changes to take effect.

The predefined materials are listed in table~\protect\ref{tab:colours}. Hence,
any colour that is not in the table is considered to be an X-colour.
%
\begin{table}[h]
\begin{center}
\begin{tabular}{|l|l|l|l|} \hline
 emerald &  jade & obsidian &  pearl \\ \hline
 ruby &  turquoise &  brass &  bronze \\ \hline
 chrome &  copper &  gold & silver \\ \hline
 blackPlastic &  cyanPlastic &  greenPlastic &  redPlastic \\ \hline 
 whitePlastic &  yellowPlastic &  blackRubber &  cyanRubber \\ \hline
 greenRubber & redRubber &  whiteRubber &  yellowRubber \\ \hline
\end{tabular}
  \caption{Predefined materials.}\label{tab:colours}
\end{center}
\end{table}
%
\begin{description}
\item[Specular (R, G, B, A):]The specular reflective property of the
X-colour material. For example, by setting the first number to
zero, you will get a bluish reflection on the green sphere in the
example application. Note that the same effect could have been
obtained by changing the specular colour of the light sources.
\item[Shininess exponent:]A number between 0 and 128 that describes
how ``narrow'' the specular reflection of the X-colour material will
be. A lower number gives a wider reflection.
\end{description}
%
\end{description}
%
\begin{figure}
  \begin{center}
    \includegraphics[width=\figWidth]{\figures/viewChar}
  \caption{The view characteristics dialog window}\label{fig:view-char}
  \end{center}
\end{figure}


\subsection{Changing the default appearance of the windows}

Default settings for some parameters can be changed through a file
named .overturerc in your HOME directory. An example of an
.overturerc file is
\begin{verbatim}
commandwindow*width: 800
commandwindow*height: 150
graphicswindow*width:  650
graphicswindow*height: 500
backgroundcolour: mediumgoldenrod
foregroundcolour: steelblue
\end{verbatim}
The window sizes is specified in pixels. It is not necessary to
specify both the width and height, and the default size is obtained
either by omitting the command completely, or by setting the size to
-1. The default foreground colour is black and the default background
colour is white. If you change them, you must use one of the
following colours:
\begin{center}
\begin{tabular}{c|c|c|c|c}
black    &  white     &  red            & blue      &  green   \\ \hline
orange   &  yellow    &  darkgreen      & seagreen  &  skyblue \\ \hline
navyblue &  violet    &  pink           & turquoise &  gold    \\ \hline
coral    &  violetred &  darkturquoise  & steelblue &  orchid  \\ \hline
salmon   & aquamarine & mediumgoldenrod & wheat     &  khaki   \\ \hline
maroon   &  slateblue & darkorchid      & plum      &
\end{tabular}
\end{center}
