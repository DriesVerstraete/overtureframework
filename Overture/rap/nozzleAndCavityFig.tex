

%-rw-r--r--  1 henshaw  staff    186639 Feb 10 06:31 nacCavityBaseFrontVolume.ps
%-rw-r--r--  1 henshaw  staff    562121 Feb 10 06:31 nacCavityBaseFrontSurface.ps
%-rw-r--r--  1 henshaw  staff    528168 Feb 10 06:31 nacCavityBaseFrontChooseBoundaryCurves.ps
%-rw-r--r--  1 henshaw  staff    520956 Feb 10 06:31 nacCavityBaseFrontChooseInitialCurve.ps
%-rw-r--r--  1 henshaw  staff    241968 Feb 10 06:31 nacLeftCornerVolume.ps
%-rw-r--r--  1 henshaw  staff    593637 Feb 10 06:31 nacLeftCornerSurface.ps
%-rw-r--r--  1 henshaw  staff    542509 Feb 10 06:30 nacLeftCornerInitialCurve.ps
%-rw-r--r--  1 henshaw  staff    382952 Feb 10 06:30 nacLeftNozzleVolume.ps
%-rw-r--r--  1 henshaw  staff    417081 Feb 10 06:30 nacLeftNozzleSurface.ps
%-rw-r--r--  1 henshaw  staff    375145 Feb 10 06:30 nacLeftNozzleSurfaceChooseCurve.ps
%-rw-r--r--  1 henshaw  staff    304393 Feb 10 06:30 nozzleAndCavityTopo.ps
%-rw-r--r--  1 henshaw  staff    297843 Feb 10 06:30 nozzleAndCavityCAD.ps

{
\begin{figure}[hbt]
\newcommand{\figWidtha}{6cm}
\newcommand{\trimfig}[2]{\trimFigb{#1}{#2}{0}{.0}{.1}{.1}}
\begin{center}
 \begin{tikzpicture}[scale=1]
 \useasboundingbox (0,.75) rectangle (18,6);  % set the bounding box (so we have less surrounding white space)
%
 \draw (0,0.0) node[anchor=south west] {\trimfig{nacLeftCornerInitialCurve}{\figWidtha}};
 \draw (6,0.0) node[anchor=south west] {\trimfig{nacLeftCornerSurface}{\figWidtha}};
 \draw (12,0.0) node[anchor=south west] {\trimfig{nacLeftCornerVolume}{\figWidtha}};
%
% \draw[step=1cm,gray] (0,0) grid (18,6);
%  \draw (current bounding box.south west) rectangle (current bounding box.north east);
 \end{tikzpicture}
\end{center}
\caption{Nozzle and Cavity. Construct a grid to connect the nozzle and cavity (left-side). 
Left: choose a start curve by joining two edges. Middle: grow a surface grid in both directions
from the start curve. Right: grow a volume grid.}
\label{fig:nozzleAndCavityLeftCorner}
\end{figure}
}

%% ----------------------------------------------------------------------
%{
%\newcommand{\clipfig}[2]{\clipFigb{#1}{#2}{.0}{1.}{.0}{1.}}
%\newcommand{\figWidth}{5.5cm}
%\begin{figure}[hbt]
% \begin{center}
% \begin{pspicture}(0,0)(16,6)
%  \rput(3.0,3.0){\clipfig{nacLeftCornerInitialCurve.ps}{\figWidth}}
%  \rput(8.5,3.0){\clipfig{nacLeftCornerSurface.ps}{\figWidth}}
%  \rput(14.,3.0){\clipfig{nacLeftCornerVolume.ps}{\figWidth}}
% \psgrid[subgriddiv=2]
%\end{pspicture}
%\end{center}
%\caption{Nozzle and Cavity. Construct a grid to connect the nozzle and cavity (left-side). 
%Left: choose a start curve by joining two edges. Middle: grow a surface grid in both directions
%from the start curve. Right: grow a volume grid.}
%\label{fig:nozzleAndCavityLeftCorner}
%\end{figure}
%}

{
\begin{figure}[hbt]
\newcommand{\figWidtha}{6cm}
\newcommand{\trimfig}[2]{\trimFigb{#1}{#2}{0}{.0}{.1}{.1}}
\begin{center}
 \begin{tikzpicture}[scale=1]
 \useasboundingbox (0,.75) rectangle (18,6);  % set the bounding box (so we have less surrounding white space)
%
 \draw (0,0.0) node[anchor=south west] {\trimfig{nacCavityBaseFrontChooseInitialCurve}{\figWidtha}};
 \draw (6,0.0) node[anchor=south west] {\trimfig{nacCavityBaseFrontSurface}{\figWidtha}};
 \draw (12,0.0) node[anchor=south west] {\trimfig{nacCavityBaseFrontVolume}{\figWidtha}};
%
%\draw[step=1cm,gray] (0,0) grid (18,6);
%  \draw (current bounding box.south west) rectangle (current bounding box.north east);
 \end{tikzpicture}
\end{center}
\caption{Nozzle and Cavity. Construct a grid for the cavity. 
Left: choose a start curve from an edge (boundary curves are highlighted in blue). 
Middle: grow a surface grid along the base of the
cavity; the boundary conditions for this surface grid have been set to follow boundary curves that have been previously selected. 
Right: grow a volume grid.}
\label{fig:nozzleAndCavityBase}
\end{figure}
}

% ----------------------------------------------------------------------
%{
%\newcommand{\clipfig}[2]{\clipFigb{#1}{#2}{.0}{1.}{.0}{1.}}
%\newcommand{\figWidth}{5.5cm}
%\begin{figure}[hbt]
% \begin{center}
% \begin{pspicture}(0,0)(16,6)
%% nacCavityBaseFrontChooseBoundaryCurves.ps
%  \rput(3.0,3.0){\clipfig{nacCavityBaseFrontChooseInitialCurve.ps}{\figWidth}}
%  \rput(8.5,3.0){\clipfig{nacCavityBaseFrontSurface.ps}{\figWidth}}
%  \rput(14.,3.0){\clipfig{nacCavityBaseFrontVolume.ps}{\figWidth}}
%% 
% \psgrid[subgriddiv=2]
%\end{pspicture}
%\end{center}
%\caption{Nozzle and Cavity. Construct a grid for the cavity. 
%Left: choose a start curve from an edge (boundary curves are highlighted in blue). 
%Middle: grow a surface grid along the base of the
%cavity; the boundary conditions for this surface grid have been set to follow boundary curves that have been previously selected. 
%Right: grow a volume grid.}
%\label{fig:nozzleAndCavityBase}
%\end{figure}
%}

{
\begin{figure}[hbt]
\newcommand{\figWidtha}{12cm}
\newcommand{\trimfig}[2]{\trimFigb{#1}{#2}{0}{.0}{.1}{.1}}
\begin{center}
 \begin{tikzpicture}[scale=1]
 \useasboundingbox (0,.75) rectangle (12,10);  % set the bounding box (so we have less surrounding white space)
%
 \draw (0,0.0) node[anchor=south west] {\trimfig{nozzleAndCavityGrid}{\figWidtha}};
%
%\draw[step=1cm,gray] (0,0) grid (12,10);
%  \draw (current bounding box.south west) rectangle (current bounding box.north east);
 \end{tikzpicture}
\end{center}
\caption{Nozzle and Cavity. Overlapping grid.}
\label{fig:nozzleAndCavityGrid}
\end{figure}
}
%{
%\newcommand{\figWidtha}{12.cm}
%\newcommand{\figWidth}{14.cm}
%\newcommand{\clipfig}[2]{\clipFigb{#1}{#2}{.0}{1.}{.125}{.85}}
%\begin{figure}[hbt]
% \begin{center}
% \begin{pspicture}(0,0)(14,8)
%  \rput(7.,4){\clipfig{nozzleAndCavityGrid.ps}{\figWidtha}}
% \psgrid[subgriddiv=2]
%\end{pspicture}
%\end{center}
%\caption{Nozzle and Cavity. Overlapping grid.}
%\label{fig:nozzleAndCavityGrid}
%\end{figure}
%}
