{
\begin{figure}[hbt]
\newcommand{\figWidtha}{6.cm}
\newcommand{\trimfig}[2]{\trimFigb{#1}{#2}{0.05}{.05}{.05}{.05}}
\begin{center}
 \begin{tikzpicture}[scale=1]
 \useasboundingbox (0,.75) rectangle (13,13);  % set the bounding box (so we have less surrounding white space)
%
 \draw (0,6.75) node[anchor=south west] {\trimfig{retroSurf11Trim0}{\figWidtha}};
 \draw (1,12) node[anchor=west] {\smallss 1. original trim curve.};
%
 \draw (0,0.0) node[anchor=south west] {\trimfig{retroSurf11Trim0Zoom}{\figWidtha}};
 \draw (1,5.0) node[anchor=west] {\smallss 2. missing segment (upper right).};
%
 \draw (7,6.75) node[anchor=south west] {\trimfig{retroSurf11Trim0Join}{\figWidtha}};
 \draw (10,9) node[anchor=west] {\smallss 3. join with line segment.};
%
%\draw[step=1cm,gray] (0,0) grid (13,13);
%  \draw (current bounding box.south west) rectangle (current bounding box.north east);
 \end{tikzpicture}
\end{center}
\caption{A trimming curve that has a gap. The gap can be filled using the
{\em join-with-line-segment} option and picking one adjacent curve and then the other.
}
\label{fig:retroSurf11Trim1}
\end{figure}
}

%{
%\newcommand{\figWidth}{7.5cm}
%\newcommand{\clipfig}[2]{\clipFigb{#1}{#2}{.0}{1.}{.15}{.9}}
%\begin{figure}[hbt]
% \begin{center}
% \begin{pspicture}(0,0.5)(15.75,12.5)
%  \rput(3.50, 9.00){\clipfig{retroSurf11Trim0.ps}{\figWidth}}
%  \rput(3.50, 3.00){\clipfig{retroSurf11Trim0Zoom.ps}{\figWidth}}
%  \rput(11.5, 9.00){\clipfig{retroSurf11Trim0Join.ps}{\figWidth}}
%%  \rput(11.5, 2.00){\clipfig{cuspBrokenTrimJoin.ps}{\figWidth}}
%%
%\rput[l](1,11){\psframebox*{\smallss 1. original trim curve}}
%\rput[l](1,5){\psframebox*{\smallss 2. missing segment (upper right).}}
%\rput[l](9,11){\psframebox*{\smallss 3. join with line segment.}}
%%\rput[l](12,5){\psframebox*{\smallss 4. join with line.}}
%%\psgrid[subgriddiv=2]
%\end{pspicture}
%\end{center}
%\caption{A trimming curve that has a gap. The gap can be filled using the
%{\em join-with-line-segment} option and picking one adjacent curve and then the other.
%}
%\label{fig:retroSurf11Trim1}
%\end{figure}
%}
{
\begin{figure}[hbt]
\newcommand{\figWidtha}{6.cm}
\newcommand{\trimfig}[2]{\trimFigb{#1}{#2}{0.05}{.05}{.05}{.05}}
\begin{center}
 \begin{tikzpicture}[scale=1]
 \useasboundingbox (0,-6.5) rectangle (13,13);  % set the bounding box (so we have less surrounding white space)
%
 \draw (0,6.75) node[anchor=south west] {\trimfig{retroSurf11Trim0Zoom1}{\figWidtha}};
 \draw (2,8.5) node[anchor=west] {\smallss 1. zoom on lower right.};
%
 \draw (0,0.0) node[anchor=south west] {\trimfig{retroSurf11Trim0Snap1}{\figWidtha}};
 \draw (1.5,1.5) node[anchor=west] {\smallss 2. snap to intersection.};
%
 \draw (7,6.75) node[anchor=south west] {\trimfig{retroSurf11Trim0Zoom2}{\figWidtha}};
 \draw (9,12) node[anchor=west] {\smallss 3. zoom on lower left.};
%
 \draw (7,0.0) node[anchor=south west] {\trimfig{retroSurf11Trim0Snap2}{\figWidtha}};
 \draw (9,5.5) node[anchor=west] {\smallss 4. snap to intersection.};
%
 \draw (7,-7.0) node[anchor=south west] {\trimfig{retroSurf11Trim0Hide}{\figWidtha}};
 \draw (8.5,-2.5) node[anchor=west] {\smallss 4. hide extraneous segment.};
%
%\draw[step=1cm,gray] (0,-6) grid (13,13);
%  \draw (current bounding box.south west) rectangle (current bounding box.north east);
 \end{tikzpicture}
\end{center}
\caption{The trimming curve has two self intersecting segments. These can be eliminated using
the {\em snap-to-intersection} option and picking one adjacent curve and then the other. The
end points of the adjacent curves will be changed to the point of intersection. The small
left-over segments are hidden. Longer left-over segments are hidden explicitly.
}
\label{fig:retroSurf11Trim2}
\end{figure}
}

%{
%\newcommand{\figWidth}{7.5cm}
%\newcommand{\clipfig}[2]{\clipFigb{#1}{#2}{.0}{1.}{.2}{.9}}
%\begin{figure}[hbt]
% \begin{center}
% \begin{pspicture}(0,-5)(15.75,12.5)
%  \rput(3.50, 9.00){\clipfig{retroSurf11Trim0Zoom1.ps}{\figWidth}}
%  \rput(3.50, 3.00){\clipfig{retroSurf11Trim0Snap1.ps}{\figWidth}}
%  \rput(11.5, 9.00){\clipfig{retroSurf11Trim0Zoom2.ps}{\figWidth}}
%  \rput(11.5, 3.00){\clipfig{retroSurf11Trim0Snap2.ps}{\figWidth}}
%  \rput(11.5,-3.00){\clipfig{retroSurf11Trim0Hide.ps}{\figWidth}}
%%
%\rput[l](1,11){\psframebox*{\smallss 1. zoom on lower right.}}
%\rput[l](1,5){\psframebox*{\smallss 2. snap to intersection.}}
%\rput[l](9,11){\psframebox*{\smallss 3. zoom on lower left.}}
%\rput[l](9,5){\psframebox*{\smallss 4. snap to intersection.}}
%\rput[l](9,-1){\psframebox*{\smallss 5. hide extraneous segment.}}
%%\psgrid[subgriddiv=2]
%\end{pspicture}
%\end{center}
%\caption{The trimming curve has two self intersecting segments. These can be eliminated using
%the {\em snap-to-intersection} option and picking one adjacent curve and then the other. The
%end points of the adjacent curves will be changed to the point of intersection. The small
%left-over segments are hidden. Longer left-over segments are hidden explicitly.
%}
%\label{fig:retroSurf11Trim2}
%\end{figure}
%}
