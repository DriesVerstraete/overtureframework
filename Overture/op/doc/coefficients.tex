\section{Implicit operators and Coefficient Matrices}


The {\ff MappedGridOperator} functions such as {\ff laplacianCoefficient}, {\ff xCoefficient}
etc. generate a ``coefficient-matrix'' (sparse matrix representation) for the indicated operator.
In this section we describe how coefficient-matrices can be created to define a system 
of equations for a PDE boundary-value problem.

To create a coefficient matrix you should create a grid function in the following way
\begin{verbatim}
    MappedGrid mg; // from somewhere
    int stencilSize=9;       // number of points in the stencil, 9 points assuming 2D
    realMappedGridFunction coeff(mg,stencilSize,all,all,all);
    coeff.setIsACoefficientMatrix(TRUE,stencilSize);
\end{verbatim}

From this declaration we see the the elements of the stencil are stored as
\begin{verbatim}
    coeff(m,I1,I2,I3)  m=0,1,...,stencilSize-1
  where
    Index I1,I2,I3   : Index's for the grid function coordinate dimensions
\end{verbatim}
Thus all the coefficients of the stencil are stored in the first component.
For example, a nine point approximation to the Laplace operator might be stored as
\begin{verbatim}
    coeff(6,I1,I2,I3)=0    coeff(7,I1,I2,I3)=1    coeff(8,I1,I2,I3)=0
    coeff(3,I1,I2,I3)=1    coeff(4,I1,I2,I3)=-4   coeff(5,I1,I2,I3)=1
    coeff(0,I1,I2,I3)=0    coeff(1,I1,I2,I3)=1    coeff(2,I1,I2,I3)=0
\end{verbatim}
The typical user will not need to know exactly how the coefficients are
stored (indeed, there is more than one storage format). 
This {\it representation} of the sparse matrix should really 
be hidden. It is useful, however, to have an idea of the format of the
matrix coefficient array. The actual representation is stored in an object of
type {\ff SparseRep}. See section \ref{SparseRep} for more details.

Once a coefficient-matrix grid-function has been declared, the sparse matrix
representing a PDE boundary value problem can be formed as follows
{\footnotesize
\begin{verbatim}
  MappedGridOperators op(mg);                            // create some differential operators
  op.setStencilSize(stencilSize);
  coeff.setOperators(op);
  
  coeff=op.laplacianCoefficients();                      // get the coefficients for the Laplace operator
  // fill in the coefficients for the boundary conditions
  coeff.applyBoundaryConditionCoefficients(0,0,dirichlet,allBoundaries);     // equations on boundary
  coeff.applyBoundaryConditionCoefficients(0,0,extrapolate,allBoundaries);   // equations on the ghost line
  coeff.finishBoundaryConditions();
\end{verbatim}
}
In this example we form the Laplace operator with Dirichlet boundary conditions.
By default one ghost-line is used so we must supply equations there. (See the description
of the {\ff MappedGridFunction} member function {\ff setIsACoefficientMatrix} for details
on how to change the number of ghostlines that are used.)
The {\tt coeff} grid function can be give to a sparse matrix solver, such as
{\tt Oges}. See the examples for more details.


Let us consider, in a bit more detail, what happens in the above example. Let us suppose
that the we are dealing with a simple one-dimensional grid corresponding to a line on the
unit interval and that we have one ghost line value.
After the line {\tt coeff=op.laplacianCoefficients();} is executed the sparse matrix will be
filled in (at all interior points and boundary points) with a discrete approximation to the 
Laplacian, resulting in a (sparse) representation for the following matrix
\newcommand{\dxa}{{1\over h^2}}
\newcommand{\dxb}{-{2\over h^2}}
\newcommand{\dxc}{{1\over 2h}}
\begin{equation*}
   \left[\begin{array}{ccccccc}
         0    & 0    & 0    &\ldots&      &      &    \\
         \dxa & \dxb & \dxa & 0    &\ldots&      &   \\
         0    & \dxa & \dxb & \dxa &  0   &\ldots&   \\
         0    &  0   & \dxa & \dxb & \dxa &  0   &   \\
        \vdots&\vdots&      &\ddots&\ddots&\ddots&    \\
              &      &      &  0   & \dxa & \dxb & \dxa \\
              &      &      &\ldots&  0   &   0  &  0   
    \end{array} \right]
  \qquad
   \begin{array}{l}
    i=-1~~~~{\rm (ghost line)}\\i=0\\i=1\\i=2\\ \vdots \\ i=N\\ i=N+1~~~~{\rm (ghost line)}
   \end{array}
\end{equation*}
So far no equation is applied at the ghost lines (first and last rows).
Internally this matrix is stored in a sparse fashion with only 3 values stored per row (actually we need 4 values
per row in 1D since the extrapolation equations below use 4 points by default).
After the dirichlet boundary condition is applied with {\tt coeff.apply\-Boundary\-Condition\-Coefficients(0,0,dirichlet,allBoundaries);} the equation on the boundary will be replaced with the identity operator. The resulting
matrix is
\begin{equation*}
   \left[\begin{array}{ccccccc}
         0    & 0    & 0    &\ldots&      &      &    \\
           0  &  1   &  0   & 0    &\ldots&      &   \\
         0    & \dxa & \dxb & \dxa &  0   &\ldots&   \\
         0    &  0   & \dxa & \dxb & \dxa &  0   &   \\
        \vdots&\vdots&      &\ddots&\ddots&\ddots&    \\
              &      &      &  0   &  0   &  1   &  0   \\
              &      &      &\ldots&  0   &   0  &  0   
    \end{array} \right]
  \qquad
   \begin{array}{l}
    i=-1~~~~{\rm (ghost line)}\\i=0\\i=1\\i=2\\ \vdots \\ i=N\\ i=N+1~~~~{\rm (ghost line)}
   \end{array}
\end{equation*}

Finally the values at the ghost points are assigned using extrapolation,
\begin{equation*}
   \left[\begin{array}{ccccccc}
         1    & -3   &  3   & -1   &      &      &    \\
           0  &  1   &  0   & 0    &\ldots&      &   \\
         0    & \dxa & \dxb & \dxa &  0   &\ldots&   \\
         0    &  0   & \dxa & \dxb & \dxa &  0   &   \\
        \vdots&\vdots&      &\ddots&\ddots&\ddots&    \\
              &      &  0   &  0   &  0   &  1   &  0   \\
              &      &  0   & -1   &  3   & -3   &  1   
    \end{array} \right]
  \qquad
   \begin{array}{l}
    i=-1~~~~{\rm (ghost line)}\\i=0\\i=1\\i=2\\ \vdots \\ i=N\\ i=N+1~~~~{\rm (ghost line)}
   \end{array}
\end{equation*}


If we wanted to apply a Neumann boundary condition we could have said
{\footnotesize
\begin{verbatim}
  coeff=op.laplacianCoefficients();                      // get the coefficients for the Laplace operator
  coeff.applyBoundaryConditionCoefficients(0,0,neumann,allBoundaries);   // equations on the ghost line
  coeff.finishBoundaryConditions();
\end{verbatim}
}
which would result in the following matrix:
\begin{equation*}
   \left[\begin{array}{ccccccc}
         \dxc & 0    &-\dxc &  0   &\ldots&      &    \\
         \dxa & \dxb & \dxa & 0    &\ldots&      &   \\
         0    & \dxa & \dxb & \dxa &  0   &\ldots&   \\
         0    &  0   & \dxa & \dxb & \dxa &  0   &   \\
        \vdots&\vdots&      &\ddots&\ddots&\ddots&    \\
              &      &      &  0   & \dxa & \dxb & \dxa \\
              &      &\ldots&  0   &-\dxc &   0  &\dxc  
    \end{array} \right]
  \qquad
   \begin{array}{l}
    i=-1~~~~{\rm (ghost line)}\\i=0\\i=1\\i=2\\ \vdots \\ i=N\\ i=N+1~~~~{\rm (ghost line)}
   \end{array}
\end{equation*}
Note that the equation is applied on the boundary and the Neumann condition is the equation that sits
a the ghost line.


Given one of the above matrices it is now apparent how we must fill-in the right-hand-side function
when we are going to solve a problem. In the dirichlet boundary condition case we should give the RHS for
the Laplace operator, $u_{xx}=f(x)$ at all interior points and the 
dirichlet BC values, $u = g(x)$, on the boundary
(by default the Oges solver will fill in zero values at all extrapolation
equations, otherwise we would have to set the ghost line values to zero). 
\begin{equation*}
   \left[\begin{array}{ccccccc}
         1    & -3   &  3   & -1   &      &      &    \\
           0  &  1   &  0   & 0    &\ldots&      &   \\
         0    & \dxa & \dxb & \dxa &  0   &\ldots&   \\
         0    &  0   & \dxa & \dxb & \dxa &  0   &   \\
        \vdots&\vdots&      &\ddots&\ddots&\ddots&    \\
              &      &  0   &  0   &  0   &  1   &  0   \\
              &      &  0   & -1   &  3   & -3   &  1   
    \end{array} \right]
   \left[\begin{array}{l}
    u_{-1} \\ u_0 \\ u_1 \\ u_2 \\ \vdots \\ u_N \\ u_{N+1}
   \end{array}\right]
   = 
   \left[\begin{array}{l}
    0 \\ g(x_0) \\ f(x_1) \\ f(x_2) \\ \vdots \\ g(x_N) \\ 0
   \end{array}\right]
\end{equation*}
In the neumann case we should
give the RHS for the Laplace operator at the interior {\bf and} the boundary and we should give the RHS
for the neumann condition,  $\partial u / \partial n = k(x)$,
at the ghost line. In this case the RHS vector would look like
\begin{equation*}
   \left[\begin{array}{ccccccc}
         \dxc &  0   & -\dxc&  0   &      &      &    \\
           0  &  1   &  0   & 0    &\ldots&      &   \\
         0    & \dxa & \dxb & \dxa &  0   &\ldots&   \\
         0    &  0   & \dxa & \dxb & \dxa &  0   &   \\
        \vdots&\vdots&      &\ddots&\ddots&\ddots&    \\
              &      &  0   &  0   &  0   &  1   &  0   \\
              &      &  0   &  0   & -\dxc&  0   & \dxc   
    \end{array} \right]
   \left[\begin{array}{l}
    u_{-1} \\ u_0 \\ u_1 \\ u_2 \\ \vdots \\ u_N \\ u_{N+1}
   \end{array}\right]
   = 
   \left[\begin{array}{l}
    k(x_0) \\ f(x_0) \\ f(x_1) \\ f(x_2) \\ \vdots \\ f(x_N) \\ k(x_N)
   \end{array} \right]
\end{equation*}




For a system of equations the situation is a bit more complicated but as
for a single equation all coefficients of the stencil appear in the first
component.
{\footnotesize
\begin{verbatim}
  MappedGridOperators op(mg);                            // create some operators
  op.setStencilSize(stencilSize);
  op.setNumberOfComponentsForCoefficients(numberOfComponentsForCoefficients);
  coeff.setOperators(op);

  // Form a system of equations for (u,v)
  //     a1(  u_xx + u_yy ) + a2*v_x = f_0
  //     a3(  v_xx + v_yy ) + a4*u_y = f_1
  //  BC's:   u=given   on all boundaries
  //          v=given   on inflow
  //          v.n=given on walls
  const int a1=1., a2=2., a3=3., a4=4.;
  //  const int a1=1., a2=0., a3=1., a4=0.;

  coeff=a1*op.laplacianCoefficients(all,all,all,0,0)+a2*op.xCoefficients(all,all,all,0,1)
       +a3*op.laplacianCoefficients(all,all,all,1,1)+a4*op.yCoefficients(all,all,all,1,0);

  coeff.applyBoundaryConditionCoefficients(0,0,dirichlet,  allBoundaries);  
  coeff.applyBoundaryConditionCoefficients(0,0,extrapolate,allBoundaries);  
  // coeff.display("Here is coeff after dirichlet/extrapolate BC's for (0) ");

  coeff.applyBoundaryConditionCoefficients(1,1,dirichlet,  inflow);
  coeff.applyBoundaryConditionCoefficients(1,1,extrapolate,inflow);
  coeff.applyBoundaryConditionCoefficients(1,1,neumann,     wall);
  // coeff.display("Here is coeff with dirichlet (0) and neumann BC's on wall (1)");

  coeff.finishBoundaryConditions();
\end{verbatim}
}
See example 2 for more details.



\subsection{Poisson's equation on a MappedGrid}

In this example we solve Poisson's equation on a MappedGrid
(file {\ff \examples/tcm.C})
{\footnotesize
\listinginput[1]{1}{\op/tests/tcm.C}
}



\subsection{Systems of Equations on a MappedGrid}
  In the general case one can define a matrix for a boundary-value problem
for a system of equations....




In this example we generate the matrix corresponding to the following
system of equations
\begin{eqnarray*}
 a_1 \Delta u + a_2 v_x - u &=& g_0 \\
 a_3\Delta v + a_4 u_y &=& g_1 \\
   u=g_0 &,& v_n=g_1 ~~~\mbox{on the boundary}
\end{eqnarray*}
Note the use of the {\tt identityCoefficents} operator.


(file {\ff \examples/tcm2.C})
{\footnotesize
\listinginput[1]{1}{\op/tests/tcm2.C}
}


\subsection{Poisson's equation on a CompositeGrid}

In this example we solve Poisson's eqution on a CompositeGrid
(file {\ff \examples/tcm3.C})
{\footnotesize
\listinginput[1]{1}{\op/tests/tcm3.C}
}

\subsection{Systems of equations on a CompositeGrid}

(file {\ff \examples/tcm4.C})
{\footnotesize
\listinginput[1]{1}{\op/tests/tcm4.C}
}


\subsection{Solving Poisson's equation to fourth-order accuracy}

The file {\ff \examples/tcmOrder4.C} shows how to solve an elliptic problem to fourth-order
accuracy. In this case we use two ghost lines (really only needed for Neumann
boundary conditions). We need to tell the operators to use fourth order and we need to build
the coefficient matrix using 2 ghost lines. In order to extrapolate the second ghost line
we use a {\tt BoundaryConditionParameters} object. The order of extrapolation will be set
to the order of accuracy plus one, by default. This example will only work with version 15 or later.

\subsection{Multiplying a grid function or array times a coefficient matrix}

Suppose one wants to form the variable coefficient operator such as
\[
     L = x {\partial \over \partial x} + y {\partial \over \partial y}
\]
In order to multiply a grid function (or array of the correct shape) 
times a coefficient matrix one can
use the {\tt multiply} function as illustrated in the next example
\begin{verbatim}
  MappedGrid mg(...);
  Range all;
  MappedGridOperators op(mg);
  realMappedGridFunction coeff(mg,9,all,all,all);
  ...
  
  Index I1,I2,I3;
  getIndex(mg.dimension,I1,I2,I3);   // define Index's for the entire grid
  // form the operator  x d/dx + y d/dy

  RealArray x,y;
  x=mg.vertex(I1,I2,I3,0);  // make a copy since we cannot pass a view to multiply
  y=mg.vertex(I1,I2,I3,1);
  coeff= multiply(x,op.xCoefficients()) + multiply(y,op.yCoefficents());

\end{verbatim}
One cannot use the normal multiplication operator, $\star$, because the array operations would not be conformable.
Since the {\tt multiply} reshapes it's first argument in order to multiply it times the second argument
{\bf one cannot pass a view of an array as the first argument of the multiply function}. Passing a view will
result in an A++ errror.

A {\tt multiply} function is also defined for multiplying a scalar {\tt realCompositeGridFunction} times
a {\tt realCompositeGridFunction} coefficient matrix.



\subsection{SparseRep: Define a Storage Format for a Sparse Matrix} \label{SparseRep}

This section is primarily for the use of people who are writing new Operator classes.
Normal beings may not want to read this.


The class {\ff SparseRepForMGF} defines the sparse representation for coefficient-matrix
MappedGridFunction. 
A coefficient matrix contains a pointer to a {\ff SparseRepForMGF}. This 
object holds all the information that defines how the stencil is stored
in the first component. For example this object will know that the
value {\tt coeff(m,i1,i2,i3)} is the coefficient that multiples the
grid function value at the point {\tt (i1',i2',i3')}. 
To save this information in a compact form, each point on the grid is given
an equation number, (stored in the {\tt intMappedGridFunction equationNumber}),
 so that instead of saving the three numbers {\tt (i1',i2',i3')} 
only a single equationNumber need be saved. 

The {\ff SparseRepForMGF} object also contains an {\tt intMappedGridFunction classify}.
The {\tt classify} array holds a value for each point on the grid to indicate the
kind of equation ({\tt interior}, {\tt boundary}, {\tt extrapolation}, {\tt interpolation}...)
that is being applied at that point. This information can be used by a sparse solver
(such as {\tt Oges}) to automatically zero out the right-hand-side for certain equations,
such as extrapolation.

Here is how {\tt SparseRep} is used by the grid function classes.

If we have a {\tt realMappedGridFunction coeff} or a {\tt realCompositeGridFunction coeff}
then the statement
\begin{verbatim}
  coeff.setIsACoefficientMatrix(TRUE);
\end{verbatim}
will cause a {\tt SparseRep} object to be created (and coeff will keep a pointer to it).
The {\tt SparseRep} object will be initialized with a call to {\tt SparseRep::updateToMatchGrid}.
This will give initial values to the {\tt classify} and {\tt equationNumber} arrays assuming
a standard stencil.

When {\tt coeff} is filled in with values for the interior with a statement like
\begin{verbatim}
  coeff=op.laplacianCoefficients();
\end{verbatim}
then normally the default values for the {\tt classify} and {\tt equationNumber} arrays
will be correct.

However, when the boundary conditions are filled in with a statment like
\begin{verbatim}
  coeff.applyBoundaryConditionCoefficinets(0,neumann,...);
\end{verbatim}
then the default values for the {\tt classify} and {\tt equationNumber} arrays will
have to be changed. (On a vertex grid the neumann boundary condition is the equation for the ghost-line
but it is centred on the boundary and thus the default equation numbers are wrong.)
For some examples look at the implementation of the
{\tt applyBoundaryConditionCoefficients} in the file {\tt BoundaryOperators.C}.


If {\tt coeff} is a {\tt realMappedGridFunction} then the statement
\begin{verbatim}
  coeff.finishBoundaryConditions();
\end{verbatim}
will add extrapolation equations at corners and insert equations for periodic
boundary conditions.

If {\tt coeff} is a {\tt realCompositeGridFunction} then the statement
\begin{verbatim}
  coeff.finishBoundaryConditions();
\end{verbatim}
will call {\tt coeff[grid].finishBoundaryConditions()} for each component grid function
and in addition it will add in the interpolation equations to {\tt coeff}.





\input SparseRepInclude.tex

