\subsection{Constructor}
 
\newlength{\LineMappingIncludeArgIndent}
\begin{flushleft} \textbf{%
\settowidth{\LineMappingIncludeArgIndent}{LineMapping(}% 
LineMapping(const real xa\_, \\ 
\hspace{\LineMappingIncludeArgIndent}const real xb\_, \\ 
\hspace{\LineMappingIncludeArgIndent}const int numberOfGridPoints )
}\end{flushleft}
\begin{description}
\item[{\bf Description:}]  Build a mapping for a line in 1D.
\item[{\bf xa\_, xb\_ (input) :}]  End points of the interval.
\end{description}
\subsection{Constructor}
 
\begin{flushleft} \textbf{%
\settowidth{\LineMappingIncludeArgIndent}{LineMapping(}% 
LineMapping(const real xa\_,const real ya\_, \\ 
\hspace{\LineMappingIncludeArgIndent}const real xb\_,const real yb\_,\\ 
\hspace{\LineMappingIncludeArgIndent}const int numberOfGridPoints)
}\end{flushleft}
\begin{description}
\item[{\bf Description:}]  Build a mapping for a line in 2D.
\item[{\bf xa\_, ya\_, xb\_, yb\_ (input) :}]  End points of the line.
\end{description}
\subsection{Constructor}
 
\begin{flushleft} \textbf{%
\settowidth{\LineMappingIncludeArgIndent}{LineMapping(}% 
LineMapping(const real xa\_,const real ya\_,const real za\_, \\ 
\hspace{\LineMappingIncludeArgIndent}const real xb\_,const real yb\_,const real zb\_,\\ 
\hspace{\LineMappingIncludeArgIndent}const int numberOfGridPoints)
}\end{flushleft}
\begin{description}
\item[{\bf Description:}]  Build a mapping for a line in 3D.
\item[{\bf xa\_, ya\_,za\_,  xb\_, yb\_,zb\_ (input) :}]  End points of the line.
\end{description}
\subsection{getPoints}
 
\begin{flushleft} \textbf{%
int  \\ 
\settowidth{\LineMappingIncludeArgIndent}{getPoints(}%
getPoints( real \& xa\_, real \& xb\_ ) const
}\end{flushleft}
\begin{description}
\item[{\bf Description:}]  Get the end points of the line.
\item[{\bf xa\_, xb\_ (output) :}]  End points of the line.
\end{description}
\subsection{getPoints}
 
\begin{flushleft} \textbf{%
int  \\ 
\settowidth{\LineMappingIncludeArgIndent}{getPoints(}%
getPoints( real \& xa\_, real \& ya\_,\\ 
\hspace{\LineMappingIncludeArgIndent}real \& xb\_, real \& yb\_ ) const
}\end{flushleft}
\begin{description}
\item[{\bf Description:}]  Get the end points of the line.
\item[{\bf xa\_, ya\_, xb\_, yb\_ (output) :}]  End points of the line.
\end{description}
\subsection{getPoints}
 
\begin{flushleft} \textbf{%
int  \\ 
\settowidth{\LineMappingIncludeArgIndent}{getPoints(}%
getPoints( real \& xa\_, real \& ya\_, real \& za\_, \\ 
\hspace{\LineMappingIncludeArgIndent}real \& xb\_, real \& yb\_, real \& zb\_ ) const
}\end{flushleft}
\begin{description}
\item[{\bf Description:}]  Get the end points of the line.
\item[{\bf xa\_, ya\_,za\_,  xb\_, yb\_,zb\_ (output) :}]  End points of the line.
\end{description}
\subsection{setPoints}
 
\begin{flushleft} \textbf{%
int  \\ 
\settowidth{\LineMappingIncludeArgIndent}{setPoints(}%
setPoints( const real \& xa\_, const real \& xb\_ )
}\end{flushleft}
\begin{description}
\item[{\bf Description:}]  Specify the end points for a line in 1D.
\item[{\bf xa\_, xb\_ (input) :}]  End points of the interval.
\end{description}
\subsection{setPoints}
 
\begin{flushleft} \textbf{%
int   \\ 
\settowidth{\LineMappingIncludeArgIndent}{setPoints(}%
setPoints( const real \& xa\_, const real \& ya\_,\\ 
\hspace{\LineMappingIncludeArgIndent}const real \& xb\_, const real \& yb\_ )
}\end{flushleft}
\begin{description}
\item[{\bf Description:}]  Specify the end points for a line in 2D.
\item[{\bf xa\_, ya\_, xb\_, yb\_ (input) :}]  End points of the line.
\end{description}
\subsection{setPoints}
 
\begin{flushleft} \textbf{%
int   \\ 
\settowidth{\LineMappingIncludeArgIndent}{setPoints(}%
setPoints( const real \& xa\_, const real \& ya\_, const real \& za\_, \\ 
\hspace{\LineMappingIncludeArgIndent}const real \& xb\_, const real \& yb\_, const real \& zb\_ )
}\end{flushleft}
\begin{description}
\item[{\bf Description:}]  Specify the end points for a line in 3D.
\item[{\bf xa\_, ya\_,za\_,  xb\_, yb\_,zb\_ (input) :}]  End points of the line.
\end{description}
