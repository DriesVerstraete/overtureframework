%--------------------------------------------------------------
\section{SmoothedPolygon}
\index{smoothed-polygon mapping}\index{Mapping!SmoothedPolygonMapping}
%-------------------------------------------------------------

This mapping can be used to create a grid whose one edge is
a polgyon with smoothed corners. The grid is created by extending
normals from the smoothed polygon.

% \begin{figure} 
%   \begin{center}
%   \includegraphics[0in,1in][8in,5in]{poly1.ps}
%   \caption{SmoothPolygon Grid} \label{fig:SP1}
%   \end{center}
% \end{figure}
% 
% \begin{figure} 
%   \begin{center}
%   \includegraphics[0in,1in][8in,5in]{poly2.ps}
%   \caption{SmoothPolygon Grid (periodic)} \label{fig:SP2}
%   \end{center}
% \end{figure}
% 
% \begin{figure} 
%   \begin{center}
%   \includegraphics[0.in,1.in][8in,4in]{poly3.ps}
%   \caption{SmoothPolygon Grid} \label{fig:SP3}
%   \end{center}
% \end{figure}

The smoothed polygon is defined by a sequence of vertices 
$$
     \xv_v(i) = (x_v(i),y_v(i))), ~~~i=0,1,\ldots,n_v-1
$$      
The curve is parameterized by a pseudo-arclength $s$, $0\le s \le 1$
with the value of $s$ at vertex $i$ defined by the relative distance
along the (un-smoothed) polygon:
$$
     s(i)= { \sum_{j=0}^{i-2} \| \xv_v(j+1)-\xv_v(j) \| \over
             \sum_{j=0}^{n_v-2} \| \xv_v(j+1)-\xv_v(j) \| }
$$
The smoothed polygon is defined using the {\it interval} functions
$V_j(s)$ of the StretchMapping class. The interval functions can be used to 
smoothly transition from one slope to a second slope.
For example, the x component of the smoothed polygon is defined as
$$
    x(s) = \left[ s+\sum_{j=0}^{n_v}( V_j(s)-V_j(0) )\right]c_1 + c_0
$$
Recall that the interval function $V_j$ is dependent on the three
parameters $d_j$, $e_j$ and $f_j$.
The parameter $d_j$ for $V_j$ is given by
$$
       d_j = { x_v(j+1)-x_v(j) \over s(j+1)-s(j)}  ~~~j=0,1,\ldots,n_v-2
$$
while
$$
       f_j=s(j)
$$
The value of $e_j$ is specified by the user (default = 40) and determines
the sharpness of the curve at the vertex. 

A grid is defined from this smooth polygon by extending normals. The length
of the normal can be constant or can be made to vary.
If $r_1$ parameterizes the curve in the tangential direction and $r_2$ in the
normal direction then the parameterization of the grid is given by
$$
\xv(r_1,r_2) = \xv(r_1) + r_2 N(r_1) \nv(r_1)
$$
The function $N(r_1)$ is itself defined in terms of stretching functions.

The user has the option to stretch the grid lines in the tangential
direction in order to concentrate grid lines near the vertices. 
The user may also stretch the grid lines in the normal direction.
Of course the grid lines may also be stretched by composing this
mapping with a StretchMapping. 

{\bf Note:}  Unfortunately the smoothed polygon only
matches the corners exponentially close with respect to the {\tt sharpness} parameter.
Moreover the higher numbered vertices will  have larger
errors (cf. the formula above). 
If you choose small values for the sharpness then the SmoothedPolygon
will not match the vertices very well, nor will it be symmetric.

% --------------------------------------------------------------------------
%% \include SmoothedPolygonInclude.tex
% --------------------------------------------------------------------------

\subsection{update(MappingInformation \&)}

The SmoothPolygon Mapping is defined interactively through a graphics
interface:
{\footnotesize
\begin{verbatim}

  GL_GraphicsInterface graphicsInterface;          // create a GL_GraphicsInterface object
  MappingInformation mappingInfo;
  mappingInfo.graphXInterface=&graphicsInterface;

  ...
  SmoothPolygon poly;
  poly.interactiveConstructor( mappingInfo );      // interactively create the smoothed polygon

\end{verbatim}
}
The user must specify the vertices of the polygon. The user may then
optionally change various parameters from their default values.

\begin{itemize}
  \item {\bf sharpness} : Specify how sharp the corners are (exponent). 
        Choose the value for $e_j$ in $V_j$.
        Note that if you choose small values for the sharpness then the SmoothedPolygon
         will not match the vertices very well, nor will it necessarily be symmetric.
  \item {\bf t-stretch} : Specify stretching in tangent-direction. Specify
        the values for $a_j$ and $b_j$ for the exponential layer stretching
        at corner $j$
  \item {\bf n-stretch} : Specify stretching in normal-direction. Specify
        the values for $a_i$, $b_i$ and $c_i$ for the exponential layer function
        for stretching in the normal direction and specify the number of
        layer functions. By default there is one layer function and the grid lines
        are concentrated near the smoothed polygon with values 
        $a_0=1.$, $b_0=4.$ and $c_0=0.$
  \item {\bf corners  } : Fix the grid corners to specific positions. Use this
        option to fix the positions of the four corners of the grid. The corners
        of the grid that lie at a normal distance from the smoothed polygon
        may not be exactly where you want them because the normal may be
        slightly different from the line which is perpendicular to the 
        straight line which joins the vertices. This option applies a bi-linear
        transformation to the entire grid in order to deform the corners to the
        specified positions.
  \item {\bf n-dist   } : Specify normal distance at vertex(+-epsilon)
        Choose the normal distance for the grid to extend from the
        polygon. Optionally the normal distance can be made to vary;
        a separate normal distance can be given at the position just
        before vertex $i$ and just after vertex $i$. 
  \item {\bf curve or area (toggle)} : Change the mapping from defining an area to
        define a curve (or vice versa). In other words toggle the domain dimension between 1 and 2.
  \item {\bf isPeriodic}: Specify periodicity array.
         Indicate whether the grid periodic in the tangential direction.
         Set this value to $2$ if the grid is closed and periodic
         or to $1$ if the grid is not closed but the derivative of the
         curve is periodic.
  \item {\bf help     } : Print this list
  \item {\bf exit     } : Finished with parameters, construct grid
\end{itemize}

\subsection{Examples}

Here are some sample command files that create some SmoothedPolygon mappings.
These command files can be read, for example, by the overlapping grid
generator {\tt ogen} from within the {\tt create mappings} menu.

\noindent
\begin{minipage}{.4\linewidth}
{\footnotesize
\listinginput[1]{1}{\mapping/poly1.cmd}
}
\end{minipage}\hfill
\begin{minipage}{.6\linewidth}
  \begin{center}
   \includegraphics[width=9cm]{\figures/poly1} \\
   % \epsfig{file=\figures/poly1.ps,height=4.in}  \\
  {SmoothedPolygon example 1}  \label{fig:poly1}
  \end{center}
\end{minipage}

\noindent
\begin{minipage}{.4\linewidth}
{\footnotesize
\listinginput[1]{1}{\mapping/poly2.cmd}
}
\end{minipage}\hfill
\begin{minipage}{.6\linewidth}
  \begin{center}
   \includegraphics[width=9cm]{\figures/poly2} \\
   % \epsfig{file=\figures/poly2.ps,height=4.in}  \\
  {SmoothedPolygon example 2}  \label{fig:poly2}
  \end{center}
\end{minipage}

\noindent
\begin{minipage}{.4\linewidth}
{\footnotesize
\listinginput[1]{1}{\mapping/poly3.cmd}
}
\end{minipage}\hfill
\begin{minipage}{.6\linewidth}
  \begin{center}
   \includegraphics[width=9cm]{\figures/poly3} \\
   % \epsfig{file=\figures/poly3.ps,height=4.in}  \\
  {SmoothedPolygon example 3}  \label{fig:poly3}
  \end{center}
\end{minipage}




