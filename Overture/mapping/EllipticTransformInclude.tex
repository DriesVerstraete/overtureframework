\subsubsection{Constructor}
 
\newlength{\EllipticTransformIncludeArgIndent}
\begin{flushleft} \textbf{%
\settowidth{\EllipticTransformIncludeArgIndent}{EllipticTransform(}% 
EllipticTransform() 
}\end{flushleft}
\begin{description}
\item[{\bf Purpose:}]  
    Create a mapping that can be used to generate an elliptic grid
   from an existing grid. This can be useful to smooth out an existing Mapping.
 
\end{description}
\subsubsection{get}
 
\begin{flushleft} \textbf{%
int  \\ 
\settowidth{\EllipticTransformIncludeArgIndent}{get(}%
get( const GenericDataBase \& dir, const aString \& name)
}\end{flushleft}
\begin{description}
\item[{\bf Description:}] 
    Get a mapping from the database.
\item[{\bf dir (input):}]  get the Mapping from a sub-directory of this directory.
\item[{\bf name (input) :}]  name of the sub-directory to look for the Mapping in.
\end{description}
\subsubsection{put}
 
\begin{flushleft} \textbf{%
int  \\ 
\settowidth{\EllipticTransformIncludeArgIndent}{put(}%
put( GenericDataBase \& dir, const aString \& name) const
}\end{flushleft}
\begin{description}
\item[{\bf Description:}] 
    Save a mapping into a database.
\item[{\bf dir (input):}]  put the Mapping into a sub-directory of this directory.
\item[{\bf name (input) :}]  name of the sub-directory to save the Mapping in.
\end{description}
\subsubsection{generateGrid}
 
\begin{flushleft} \textbf{%
void  \\ 
\settowidth{\EllipticTransformIncludeArgIndent}{generateGrid(}%
generateGrid(GenericGraphicsInterface *gi  = NULL, \\ 
GraphicsParameters \& parameters  =Overture::nullMappingParameters())
}\end{flushleft}
\begin{description}
\item[{\bf Description:}] 
    This function performs the iterations to solve the elliptic grid equations.
\item[{\bf gi (input) :}]  supply a graphics interface if you want to see the grid as it
    is being computed.
\item[{\bf parameters (input) :}]  optional parameters used by the graphics interface.
\end{description}
