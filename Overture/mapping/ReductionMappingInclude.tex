\subsection{Constructor}
 
\newlength{\ReductionMappingIncludeArgIndent}
\begin{flushleft} \textbf{%
\settowidth{\ReductionMappingIncludeArgIndent}{ReductionMapping(}% 
ReductionMapping() 
}\end{flushleft}
\begin{description}
\item[{\bf Purpose:}]  Default Constructor
\end{description}
\subsection{Constructor}
 
\begin{flushleft} \textbf{%
\settowidth{\ReductionMappingIncludeArgIndent}{ReductionMapping(}% 
ReductionMapping(Mapping \& mapToReduce, \\ 
\hspace{\ReductionMappingIncludeArgIndent}const real \& inactiveAxis1Value  =0.,\\ 
\hspace{\ReductionMappingIncludeArgIndent}const real \& inactiveAxis2Value  =-1.,\\ 
\hspace{\ReductionMappingIncludeArgIndent}const real \& inactiveAxis3Value  =-1.)
}\end{flushleft}
\begin{description}
\item[{\bf Purpose:}]  Create a reduction mapping.

\item[{\bf mapToReduce (input):}]  reduce the domain dimension of this mapping.
\item[{\bf inactiveAxis1Value (input):}]  if this value is between [0,1] then the r value
      for axis1 will be fixed to this value and axis1 will become an in-active axis;
       otherwise axis1 will remain active.
\item[{\bf inactiveAxis2Value (input):}]  fix an r value for axis2. See comments for inactiveAxis1Value.
\item[{\bf inactiveAxis3Value (input):}]  fix an r value for axis3. See comments for inactiveAxis1Value.

\end{description}
\subsection{Constructor}
 
\begin{flushleft} \textbf{%
\settowidth{\ReductionMappingIncludeArgIndent}{ReductionMapping(}% 
ReductionMapping(Mapping \& mapToReduce, \\ 
\hspace{\ReductionMappingIncludeArgIndent}const int \& inactiveAxis,\\ 
\hspace{\ReductionMappingIncludeArgIndent}const real \& inactiveAxisValue )
}\end{flushleft}
\begin{description}
\item[{\bf Purpose:}]  Create a reduction mapping.

\item[{\bf mapToReduce (input):}]  reduce the domain dimension of this mapping.
\item[{\bf inactiveAxis (input):}]  This is the inactive axis.
\item[{\bf inactiveAxisValue (input):}]  This is the value of the inactive axis in [0,1].

\end{description}
\subsection{set}
 
\begin{flushleft} \textbf{%
int   \\ 
\settowidth{\ReductionMappingIncludeArgIndent}{set(}%
set(Mapping \& mapToReduce, \\ 
\hspace{\ReductionMappingIncludeArgIndent}const real \& inactiveAxis1Value  =0. ,\\ 
\hspace{\ReductionMappingIncludeArgIndent}const real \& inactiveAxis2Value  =-1.,\\ 
\hspace{\ReductionMappingIncludeArgIndent}const real \& inactiveAxis3Value  =-1.)
}\end{flushleft}
\begin{description}
\item[{\bf Purpose:}]  Set parameters for a reduction mapping.

\item[{\bf mapToReduce (input):}]  reduce the domain dimension of this mapping.
\item[{\bf inactiveAxis1Value (input):}]  if this value is between [0,1] then the r value
      for axis1 will be fixed to this value and axis1 will become an in-active axis;
       otherwise axis1 will remain active.
\item[{\bf inactiveAxis2Value (input):}]  fix an r value for axis2. See comments for inactiveAxis1Value.
\item[{\bf inactiveAxis3Value (input):}]  fix an r value for axis3. See comments for inactiveAxis1Value.

\end{description}
\subsection{set}
 
\begin{flushleft} \textbf{%
int   \\ 
\settowidth{\ReductionMappingIncludeArgIndent}{set(}%
set(Mapping \& mapToReduce,\\ 
\hspace{\ReductionMappingIncludeArgIndent}const int \& inactiveAxis,\\ 
\hspace{\ReductionMappingIncludeArgIndent}const real \& inactiveAxisValue ) 
}\end{flushleft}
\begin{description}
\item[{\bf Purpose:}]  Set parameters for a reduction mapping.

\item[{\bf mapToReduce (input):}]  reduce the domain dimension of this mapping.
\item[{\bf inactiveAxis (input):}]  This is the inactive axis.
\item[{\bf inactiveAxisValue (input):}]  This is the value of the inactive axis in [0,1].

\end{description}
\subsection{setInActiveAxes}
 
\begin{flushleft} \textbf{%
int  \\ 
\settowidth{\ReductionMappingIncludeArgIndent}{setInActiveAxes(}%
setInActiveAxes( const real \& inactiveAxis1Value  =0.,\\ 
\hspace{\ReductionMappingIncludeArgIndent}const real \& inactiveAxis2Value  =-1.,\\ 
\hspace{\ReductionMappingIncludeArgIndent}const real \& inactiveAxis3Value  =-1.)
}\end{flushleft}
\begin{description}
\item[{\bf Purpose:}]  Specify the in-active axes.

\item[{\bf inactiveAxis1Value (input):}]  if this value is between [0,1] then the r value
      for axis1 will be fixed to this value and axis1 will become an in-active axis;
       otherwise axis1 will remain active.
\item[{\bf inactiveAxis2Value (input):}]  fix an r value for axis2. See comments for inactiveAxis1Value.
\item[{\bf inactiveAxis3Value (input):}]  fix an r value for axis3. See comments for inactiveAxis1Value.
\end{description}
\subsection{setInActiveAxes}
 
\begin{flushleft} \textbf{%
int  \\ 
\settowidth{\ReductionMappingIncludeArgIndent}{setInActiveAxes(}%
setInActiveAxes(const int \& inactiveAxis,\\ 
\hspace{\ReductionMappingIncludeArgIndent}const real \& inactiveAxisValue ) 
}\end{flushleft}
\begin{description}
\item[{\bf Purpose:}]  Set parameters for a reduction mapping.

\item[{\bf inactiveAxis (input):}]  This is the inactive axis.
\item[{\bf inactiveAxisValue (input):}]  This is the value of the inactive axis in [0,1].

\end{description}
