%=======================================================================================================
% Mapping Documentation
%=======================================================================================================
\documentclass{article}

\voffset=-1.25truein
\hoffset=-1.truein
\setlength{\textwidth}{7in}      % page width
\setlength{\textheight}{9.5in}    % page height for xdvi

 \usepackage{epsfig}
 \usepackage{graphics}    
 \usepackage{moreverb}
 \usepackage{amsmath}

% \usepackage{fancybox}
 \usepackage{subfigure}
 \usepackage{multicol}

\usepackage{makeidx} % index
\makeindex
\newcommand{\Index}[1]{#1\index{#1}}

\begin{document}


\input wdhDefinitions

\def\uvd    {{\bf U}}
\def\ud     {{    U}}
\def\pd     {{    P}}
\def\id     {i}
\def\jd     {j}
\def\kap {\sqrt{s+\omega^2}}

\newcommand{\mapping}{/home/henshaw/Overture/mapping}
\newcommand{\figures}{../docFigures}

\vspace{3\baselineskip}
\begin{flushleft}
  {\Large 
   Mappings for Overture  \\ 
   A Description of the Mapping Class  \\
   and Documentation for Many Useful Mappings \\
  }
\vspace{2\baselineskip}
William D. Henshaw               
\footnote{
        This work was partially
        supported by grant N00014-95-F-0067 from the Office of Naval
        Research
        }   \\
Centre for Applied Scientific Computing \\
Lawrence Livermore National Laboratory    \\
Livermore, CA, 94551   \\
henshaw@llnl.gov \\
http://www.llnl.gov/casc/people/henshaw \\
http://www.llnl.gov/casc/Overture
\vspace{1\baselineskip}
\today
\vspace{\baselineskip}
UCRL-MA-132239
% LA-UR-96-3469

\end{flushleft}

\vspace{1\baselineskip}

\begin{abstract}
This document describes the class {\ff Mapping}. The Mapping class is
used to define transformations. These transformations are used within
Overture to define grids and stretching functions and rotations etc.
The base class is called {\ff Mapping}. Particular mappings such as
a sphere or an annulus are defined by deriving a class from the
base class and defining the particular transformation. 
A number of derived Mappings have been written including
\begin{itemize}
 \item  Various Analytical mappings: LineMapping, SquareMapping, CircleMapping, 
        AnnulusMapping, BoxMapping, CylinderMapping, PlaneMapping, QuadraticMapping, SphereMapping
 \item  AirfoilMapping : for creating airfoil related grids and curves (including some NACA airfoils).
 \item  ComposeMapping : for composing two mappings  
 \item  CompositeSurface : a mapping that represents a collection of sub-surfaces.
 \item  CrossSectionMapping : define a surface by cross-sections
 \item  DataPointMapping : mappings defined by data points  
 \item  DepthMapping : create a 3D grid from a 2D grid by adding a variable depth.
 \item  EllipticTransform: smooth a mapping with an elliptic transform (thanks to Eugene Sy)
 \item  FilletMapping: create a fillet or collar grid to join two intersecting surfaces.
 \item  HyperbolicMapping: create volume grids using hyperbolic grid generation (described else-where).
 \item  IntersectionMapping: a mapping that is the intersection between two other mappings, such as
        the curve of intersection between two surfaces.        
 \item  JoinMapping: create a mapping that can join two intersecting mappings.
 \item  MatrixMapping : define a matrix transformation by rotations, scaling, shifts etc.  
 \item  MatrixTransform : apply a matrix transformation to another mapping
 \item  NormalMapping : define a new mapping by extending normals
 \item  NurbsMapping : define a mapping by a NURBS, non-uniform rational b-spline.
 \item  OrthographicTransform : define an orthographic transform
 \item  ReductionMapping : make a new Mapping from the face or edge of another mapping.
 \item  ReparameterizationTransform : reparameterize a mapping (e.g. remove singularities)
 \item  RestrictionMapping : define a restriction to a sub-rectangle.
 \item  RevolutionMapping : create a surface or volume of revolution
 \item  RocketMapping : create curves related to rocket geometries.
 \item  SmoothedPolygon : for polygons with smoothed corners  
 \item  StretchMapping : one-dimensional stretching transformations  
 \item  StretchedSquare : stretch grid lines on the unit interval.
 \item  StretchTransform : stretch grid lines along the coordinate directions 
 \item  SweepMapping : Sweep a 2D Mapping along a curve in 3D.
 \item  SplineMapping: define a cubic spline curve.
 \item  TFIMapping : define a grid from given boundary curves by transfinite-interpolation (Coon's patch).
 \item  TrimmedMapping : define a trimmed surface in 3D, the surface has portions removed
        from it (``trimmed'').
 \item  UnstructuredMapping : create an unstructured representation for an existing mapping or 
	read in an manipulate and unstructured mesh.
\end{itemize}
All these classes are described in this document.
\end{abstract}



\tableofcontents

\vspace{3\baselineskip}

\section{Introduction}

The C++ class ``Mapping'' can be used to define the ``mappings'' (transformations)
and their properties. For example,
each component grid in an overlapping grid will contain a member
function that defines the mapping from the unit square (or unit cube)
onto the domain covered by the grid. This mapping may in turn be
defined in terms of the curves (or surfaces) that form its boundaries.
Stretching functions as well as rotations and scalings are all
defined by mappings.


\end{document}