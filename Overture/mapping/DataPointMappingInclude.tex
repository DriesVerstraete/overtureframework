\subsection{Constructor}
 
\newlength{\DataPointMappingIncludeArgIndent}
\begin{flushleft} \textbf{%
\settowidth{\DataPointMappingIncludeArgIndent}{DataPointMapping(}% 
DataPointMapping() 
}\end{flushleft}
\begin{description}
\item[{\bf Purpose:}]  Default Constructor. 
\end{description}
\subsection{getDataPoints}
 
\begin{flushleft} \textbf{%
\settowidth{\DataPointMappingIncludeArgIndent}{getDataPoints(}% 
getDataPoints()
}\end{flushleft}
\begin{description}
\item[{\bf Description:}] 
  Return the array of data points. It will not be the same array as was given to
    setDataPoints since ghostlines will have been added. Use getGridIndexRange
    to determine the index positions for the grid boundaries.
\item[{\bf Return value:}]  array of data points, xy(I1,I2,I3,0:r-1), r=rangeDimension
\end{description}
\subsection{getGridIndexRange}
 
\begin{flushleft} \textbf{%
const IntegerArray \&  \\ 
\settowidth{\DataPointMappingIncludeArgIndent}{getGridIndexRange(}%
getGridIndexRange()
}\end{flushleft}
\begin{description}
\item[{\bf Description:}] 
  Return the gridIndexRange array for the data points. These values indicate
  the index positions for the grid boundaries.
\item[{\bf Return value:}]  The gridIndexRange(0:1,0:2).
\end{description}
\subsection{getDimension}
 
\begin{flushleft} \textbf{%
const IntegerArray \&  \\ 
\settowidth{\DataPointMappingIncludeArgIndent}{getDimension(}%
getDimension()
}\end{flushleft}
\begin{description}
\item[{\bf Description:}] 
  Return the dimension array for the data points. These values indicate
  the index positions for the array dimensions.
\item[{\bf Return value:}]  The dimension(0:1,0:2).
\end{description}
\subsection{setDataPoints}
 
\begin{flushleft} \textbf{%
int  \\ 
\settowidth{\DataPointMappingIncludeArgIndent}{setDataPoints(}%
setDataPoints(const realArray \& xd,\\ 
\hspace{\DataPointMappingIncludeArgIndent}const int positionOfCoordinates  =3, \\ 
\hspace{\DataPointMappingIncludeArgIndent}const int domainDimension\_  =-1,\\ 
\hspace{\DataPointMappingIncludeArgIndent}const int numberOfGhostLinesInData  = 0,\\ 
const IntegerArray \& xGridIndexRange  = Overture::nullIntArray()) 
}\end{flushleft}
\begin{description}
\item[{\bf Purpose:}]  
   Supply data points as
    \begin{enumerate}
      \item  xd(0:r-1,I,J,K) if positionOfCoordinates==0 $\rightarrow$ domainDimension=domainDimension\_
      \item  xd(I,0:r-1)     if positionOfCoordinates==1 $\rightarrow$ domainDimension=1
      \item  xd(I,J,0:r-1)   if positionOfCoordinates==2 $\rightarrow$ domainDimension=2
      \item  xd(I,J,K,0:r-1) if positionOfCoordinates==3 $\rightarrow$ domainDimension=domainDimension\_
    \end{enumerate}
   where r=number of dimensions (range dimension)
\item[{\bf xd (input):}]  An array of values defining the coordinates of a grid of points. This routine make a COPY
   of this array.
\item[{\bf positionOfCoordinates (input):}]  indicates the "shape" of the input array xd.
\item[{\bf domainDimension\_ (input):}]  As indicated above this parameter defines the domainDimension when
    positionOfCoordinates is 0 or 3.
\item[{\bf numberOfGhostLinesInData (input) :}]  The data includes the coordinates of this many ghost lines (for all sides).
    These values are over-ridden by the index array argument.
\item[{\bf xGridIndexRange (input):}]  If this array is not null and size (2,0:r-1) then these values indicate the 
    points in the array xd that represent the boundary points on the grid. Use this option to specify
    arbitrary number of ghost points on any side.
\item[{\bf Remarks:}] 
   Note that by default the DataPointMapping will have the properties
   \begin{itemize}
     \item domainSpace = parameterSpace
     \item rangeSpace = cartesianSpace
     \item not periodic
     \item boundary conditions all 1
   \end{itemize}
   You will have to change the above properties as appropriate.  
   NOTE: you should set the periodicity of this mapping before supplying data points.      
\end{description}
\subsection{setDataPoints}
 
\begin{flushleft} \textbf{%
int  \\ 
\settowidth{\DataPointMappingIncludeArgIndent}{setDataPoints(}%
setDataPoints(const realArray \& xd, \\ 
\hspace{\DataPointMappingIncludeArgIndent}const int positionOfCoordinates, \\ 
\hspace{\DataPointMappingIncludeArgIndent}const int domainDimension\_,\\ 
\hspace{\DataPointMappingIncludeArgIndent}const int numberOfGhostLinesInData[2][3],\\ 
const IntegerArray \& xGridIndexRange  = Overture::nullIntArray())
}\end{flushleft}
\begin{description}
\item[{\bf Description:}] 
    Supply data points: Same as above routine except that the numberOfGhostLinesInData can
  be defined as separate values for each face.
\item[{\bf numberOfGhostLinesInData[side][axis] :}]  specify the number of ghostlines in the input data
    for each face.
\end{description}
\subsection{computeGhostPoints}
 
\begin{flushleft} \textbf{%
int  \\ 
\settowidth{\DataPointMappingIncludeArgIndent}{computeGhostPoints(}%
computeGhostPoints( IndexRangeType \& numberOfGhostLinesOld, \\ 
\hspace{\DataPointMappingIncludeArgIndent}IndexRangeType \& numberOfGhostLinesNew )
}\end{flushleft}
\begin{description}
\item[{\bf Access Level:}]  protected
\item[{\bf Description:}] 
   Determine values at ghost points that have not been user set:  extrapolate or use periodicity
   Ghost lines on sides with boundaryCondition>0 are extrapolated with a stretchingFactor (see below)
   so that the grid lines get
   further apart. This is useful for highly stretched grids so that the ghost points move away from
   the boundary.
\end{description}
\subsection{setNumberOfGhostLines}
 
\begin{flushleft} \textbf{%
int  \\ 
\settowidth{\DataPointMappingIncludeArgIndent}{setNumberOfGhostLines(}%
setNumberOfGhostLines( IndexRangeType \& numberOfGhostLinesNew )
}\end{flushleft}
\begin{description}
\item[{\bf Description:}] 
    Specify the number of ghost lines.
\item[{\bf numberOfGhostLinesNew(side,axis) :}]  specify the number of ghostlines.
\end{description}
\subsection{projectGhostPoints}
 
\begin{flushleft} \textbf{%
int  \\ 
\settowidth{\DataPointMappingIncludeArgIndent}{projectGhostPoints(}%
projectGhostPoints(MappingInformation \& mapInfo )
}\end{flushleft}
\begin{description}
\item[{\bf Description:}] 
   Project the ghost points on physical boundaries onto the closest mapping
  found in a list of Mapping's

\item[{\bf mapInfo (input):}]   Project onto the closest mapping found in mapInfo.mappingList.

\end{description}
\subsection{setDataPoints( fileName )}
 
\begin{flushleft} \textbf{%
int  \\ 
\settowidth{\DataPointMappingIncludeArgIndent}{setDataPoints(}%
setDataPoints( const aString \& fileName )
}\end{flushleft}
\begin{description}
\item[{\bf Description:}] 
   Assign the  data points from a file of data. By default this routine will
    attempt to automaticall determine the format of the file.
\item[{\bf fileName (input) :}]  name of an existing file of data (such as a plot3d file)
 
\end{description}
\subsection{setMapping}
 
\begin{flushleft} \textbf{%
int  \\ 
\settowidth{\DataPointMappingIncludeArgIndent}{setMapping(}%
setMapping( Mapping \& map )             
}\end{flushleft}
\begin{description}
\item[{\bf Description:}] 
    Build a data point mapping from grids points obtained by evaluating a 
  mapping. 
\item[{\bf map (input) :}]  Mapping to get data points from.
 
\end{description}
\subsection{setMapping}
 
\begin{flushleft} \textbf{%
int  \\ 
\settowidth{\DataPointMappingIncludeArgIndent}{setMapping(}%
setMapping( Mapping \& map )             
}\end{flushleft}
\begin{description}
\item[{\bf Description:}] 
    Build a data point mapping from grids points obtained by evaluating a 
  mapping. 
\item[{\bf map (input) :}]  Mapping to get data points from.
 
\end{description}
\subsection{setOrderOfInterpolation}
 
\begin{flushleft} \textbf{%
void  \\ 
\settowidth{\DataPointMappingIncludeArgIndent}{setOrderOfInterpolation(}%
setOrderOfInterpolation( const int order )
}\end{flushleft}
\begin{description}
\item[{\bf Purpose:}]  
   Set the order of interpolation, 2 or 4.
\item[{\bf order (input) :}]  A value of 2 or 4.
\end{description}
\subsection{setOrderOfInterpolation}
 
\begin{flushleft} \textbf{%
int  \\ 
\settowidth{\DataPointMappingIncludeArgIndent}{getOrderOfInterpolation(}%
getOrderOfInterpolation()
}\end{flushleft}
\begin{description}
\item[{\bf Purpose:}]  
   Get the order of interpolation.
\item[{\bf Return value:}]  The order of interpolation.
\end{description}
\subsection{useScalarArrayIndexing}
 
\begin{flushleft} \textbf{%
void   \\ 
\settowidth{\DataPointMappingIncludeArgIndent}{useScalarArrayIndexing(}%
useScalarArrayIndexing(const bool \& trueOrFalse  =FALSE) 
}\end{flushleft}
\begin{description}
\item[{\bf Purpose:}]  
    Turn on or off the use of scalar indexing. Scalar indexing for array
 operations can be faster when the length of arrays are smaller.
\item[{\bf trueOrFalse (input) :}]  TRUE means turn on scalra indexing.
\end{description}
\subsubsection{sizeOf}
 
\begin{flushleft} \textbf{%
real  \\ 
\settowidth{\DataPointMappingIncludeArgIndent}{sizeOf(}%
sizeOf(FILE *file  = NULL) const
}\end{flushleft}
\begin{description}
\item[{\bf Description:}] 
   Return size of this object  
\end{description}
\subsection{update}
 
\begin{flushleft} \textbf{%
int  \\ 
\settowidth{\DataPointMappingIncludeArgIndent}{update(}%
update( MappingInformation \& mapInfo ) 
}\end{flushleft}
\begin{description}
\item[{\bf Purpose:}]  
   Interactively change parameters describing the Mapping.
   The user may choose to read in data points from a file. The current supported
  file formats are
   \begin{itemize}
     \item plot3d 
   \end{itemize}
         
\end{description}
