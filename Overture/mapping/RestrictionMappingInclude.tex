\subsection{Default Constructor}
 
\newlength{\RestrictionMappingIncludeArgIndent}
\begin{flushleft} \textbf{%
\settowidth{\RestrictionMappingIncludeArgIndent}{RestrictionMapping(}% 
RestrictionMapping(const real ra\_   =0.,\\ 
\hspace{\RestrictionMappingIncludeArgIndent}const real rb\_   =1.,\\ 
\hspace{\RestrictionMappingIncludeArgIndent}const real sa\_   =0.,\\ 
\hspace{\RestrictionMappingIncludeArgIndent}const real sb\_   =1.,\\ 
\hspace{\RestrictionMappingIncludeArgIndent}const real ta\_   =0.,\\ 
\hspace{\RestrictionMappingIncludeArgIndent}const real tb\_   =1.,\\ 
\hspace{\RestrictionMappingIncludeArgIndent}const int dimension  =2 ,\\ 
\hspace{\RestrictionMappingIncludeArgIndent}Mapping *restrictedMapping  =NULL)
}\end{flushleft}
\begin{description}
\item[{\bf Purpose:}]  Default Constructor
 The restriction is a Mapping from {\tt parameter} space to {\tt parameter} space
  defined by 
 \begin{align*}
     x(I,axis1) &= (rb-ra) r(I,axis1)+ra \\
     x(I,axis2) &= (sb-sa) r(I,axis2)+sa \\
     x(I,axis3) &= (tb-ta) r(I,axis3)+ta  
 \end{align*}
\item[{\bf ra\_,rb\_,sa\_,sb\_,ta\_,tb\_ (input):}]  Parameters in the definition of the 
   {\tt RestrictionMapping}.
\item[{\bf dimension (input):}]  define the domain and range dimension (which are equal).
\item[{\bf restrictedMapping (input) :}]  optionally pass the Mapping being restricted. 
  This Mapping is used to set spaceIsPeriodic.
\end{description}
\subsection{scaleBounds}
 
\begin{flushleft} \textbf{%
int  \\ 
\settowidth{\RestrictionMappingIncludeArgIndent}{scaleBounds(}%
scaleBounds(const real ra\_ =0.,\\ 
\hspace{\RestrictionMappingIncludeArgIndent}const real rb\_ =1., \\ 
\hspace{\RestrictionMappingIncludeArgIndent}const real sa\_ =0.,\\ 
\hspace{\RestrictionMappingIncludeArgIndent}const real sb\_ =1.,\\ 
\hspace{\RestrictionMappingIncludeArgIndent}const real ta\_ =0.,\\ 
\hspace{\RestrictionMappingIncludeArgIndent}const real tb\_   =1.)
}\end{flushleft}
\begin{description}
\item[{\bf Purpose:}]  
    Scale the current bounds. Define a sub-rectangle of the current restriction.
  These parameters apply to the current restriction as if it were the entire unit square
  or unit cube. For example for the "r" variable the transformation from old values of
   (ra,rb) to new values of (ra,rb) is defined by:
  \begin{align*}
     rba &= rb-ra \\
     rb &= ra+rb\_ ~rba \\
     ra &= ra+ra\_ ~rba
  \end{align*}
   
\item[{\bf ra\_,rb\_,sa\_,sb\_,ta\_,tb\_ (input):}]  These parameters define a 
 sub-rectangle of the current restriction.
\end{description}
\subsection{getBounds}
 
\begin{flushleft} \textbf{%
int  \\ 
\settowidth{\RestrictionMappingIncludeArgIndent}{getBounds(}%
getBounds(real \& ra\_, real \& rb\_, real \& sa\_, real \& sb\_, real \& ta\_, real \& tb\_ ) const
}\end{flushleft}
\begin{description}
\item[{\bf Description:}] 
  Get the bounds for a restriction mapping.
   {\tt RestrictionMapping} for further details.
\item[{\bf ra\_,rb\_,sa\_,sb\_,ta\_,tb\_ (output):}]  
\end{description}
\subsection{setBounds}
 
\begin{flushleft} \textbf{%
int  \\ 
\settowidth{\RestrictionMappingIncludeArgIndent}{setBounds(}%
setBounds(const real ra\_ =0., \\ 
\hspace{\RestrictionMappingIncludeArgIndent}const real rb\_ =1., \\ 
\hspace{\RestrictionMappingIncludeArgIndent}const real sa\_ =0.,\\ 
\hspace{\RestrictionMappingIncludeArgIndent}const real sb\_ =1.,\\ 
\hspace{\RestrictionMappingIncludeArgIndent}const real ta\_ =0.,\\ 
\hspace{\RestrictionMappingIncludeArgIndent}const real tb\_   =1.)
}\end{flushleft}
\begin{description}
\item[{\bf Purpose:}]  
  Set absolute bounds for the restriction.
\item[{\bf ra\_,rb\_,sa\_,sb\_,ta\_,tb\_ (input):}]  Parameters in the definition of the 
   {\tt RestrictionMapping}.
\end{description}
\subsection{setSpaceIsPeriodic}
 
\begin{flushleft} \textbf{%
int  \\ 
\settowidth{\RestrictionMappingIncludeArgIndent}{setSpaceIsPeriodic(}%
setSpaceIsPeriodic( int axis, bool trueOrFalse  = true)
}\end{flushleft}
 Description:
    Indicate whether the space being restricted is periodic. For example if you
 restrict an AnnulusMapping then you should set periodic1=true since the Annulus
 is periodic along axis1
