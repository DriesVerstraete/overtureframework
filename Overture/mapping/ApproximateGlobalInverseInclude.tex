\subsubsection{constructor}
 
\newlength{\ApproximateGlobalInverseIncludeArgIndent}
\begin{flushleft} \textbf{%
\settowidth{\ApproximateGlobalInverseIncludeArgIndent}{ApproximateGlobalInverse(}% 
ApproximateGlobalInverse( Mapping \& map0 )
}\end{flushleft}
\begin{description}
\item[{\bf Description:}] 
   Build an approximate inverse to go with a given mapping.
\end{description}
\subsubsection{getGrid}
 
\begin{flushleft} \textbf{%
const RealArray \&  \\ 
\settowidth{\ApproximateGlobalInverseIncludeArgIndent}{getGrid(}%
getGrid() const
}\end{flushleft}
\begin{description}
\item[{\bf Description:}] 
 return the grid used for the inverse
\end{description}
\subsubsection{getBoundingBox}
 
\begin{flushleft} \textbf{%
const RealArray \&  \\ 
\settowidth{\ApproximateGlobalInverseIncludeArgIndent}{getBoundingBox(}%
getBoundingBox() const
}\end{flushleft}
\begin{description}
\item[{\bf Description:}] 
   Return the bounding box for the entire mapping.
\end{description}
\subsubsection{getBoundingBoxTree}
 
\begin{flushleft} \textbf{%
const BoundingBox \&  \\ 
\settowidth{\ApproximateGlobalInverseIncludeArgIndent}{getBoundingBoxTree(}%
getBoundingBoxTree(int side, int axis) const
}\end{flushleft}
\begin{description}
\item[{\bf Description:}] 
   Return the bounding box tree for a given boundary of the mapping.
\end{description}
\subsubsection{getParameter}
 
\begin{flushleft} \textbf{%
real  \\ 
\settowidth{\ApproximateGlobalInverseIncludeArgIndent}{getParameter(}%
getParameter( const  realParameter \& param ) const
}\end{flushleft}
\begin{description}
\item[{\bf Description:}] 
   Return the value of a parameter.
\item[{\bf param (input) :}]  One of {\tt MappingParameters::THEboundingBoxExtensionFactor}
     or {\tt MappingParameters::THEstencilWalkBoundingBoxExtensionFactor}.
 
\end{description}
\subsubsection{getParameter}
 
\begin{flushleft} \textbf{%
int  \\ 
\settowidth{\ApproximateGlobalInverseIncludeArgIndent}{getParameter(}%
getParameter( const  intParameter \& param ) const
}\end{flushleft}
\begin{description}
\item[{\bf Description:}] 
   Return the value of a parameter.
\item[{\bf param (input) :}]  One of {\tt MappingParameters::THEfindBestGuess}
 
\end{description}
\subsubsection{setParameter}
 
\begin{flushleft} \textbf{%
void  \\ 
\settowidth{\ApproximateGlobalInverseIncludeArgIndent}{setParameter(}%
setParameter( const  realParameter \& param, const real \& value ) 
}\end{flushleft}
\begin{description}
\item[{\bf Description:}] 
   Set the value of a parameter.
\item[{\bf param (input) :}]  One of {\tt MappingParameters::THEboundingBoxExtensionFactor}
     or {\tt MappingParameters::THEstencilWalkBoundingBoxExtensionFactor}.
\item[{\bf value (input) :}]  value for the parameter.
 
\end{description}
\subsubsection{setParameter(int)}
 
\begin{flushleft} \textbf{%
void  \\ 
\settowidth{\ApproximateGlobalInverseIncludeArgIndent}{setParameter(}%
setParameter( const  intParameter \& param, const int \& value ) 
}\end{flushleft}
\begin{description}
\item[{\bf Description:}] 
   Set the value of a parameter.
\item[{\bf param (input) :}]  One of {\tt MappingParameters::THEboundingBoxExtensionFactor}
     or {\tt MappingParameters::THEstencilWalkBoundingBoxExtensionFactor}.
\item[{\bf value (input) :}]  value for the parameter.
 
\end{description}
\subsubsection{useRobustInverse}
 
\begin{flushleft} \textbf{%
void  \\ 
\settowidth{\ApproximateGlobalInverseIncludeArgIndent}{useRobustInverse(}%
useRobustInverse(const bool trueOrFalse  =true)
}\end{flushleft}
\begin{description}
\item[{\bf Description:}] 
    If true use the more robust approximate inverse that will work with highly
 stretched grids where the closest grid point x, to a given point may be many cells
 away from the cell containing the point x.
\end{description}
\subsubsection{usingRobustInverse}
 
\begin{flushleft} \textbf{%
bool  \\ 
\settowidth{\ApproximateGlobalInverseIncludeArgIndent}{usingRobustInverse(}%
usingRobustInverse() const 
}\end{flushleft}
\begin{description}
\item[{\bf Description:}] 
    Return true if using the more robust approximate inverse that will work with highly
 stretched grids where the closest grid point x, to a given point may be many cells
 away from the cell containing the point x.
\end{description}
\subsubsection{sizeOf}
 
\begin{flushleft} \textbf{%
real  \\ 
\settowidth{\ApproximateGlobalInverseIncludeArgIndent}{sizeOf(}%
sizeOf(FILE *file  = NULL) const
}\end{flushleft}
\begin{description}
\item[{\bf Description:}] 
   Return size of this object  
\end{description}
\subsubsection{get}
 
\begin{flushleft} \textbf{%
int  \\ 
\settowidth{\ApproximateGlobalInverseIncludeArgIndent}{get(}%
get( const GenericDataBase \& dir, const aString \& name)
}\end{flushleft}
\begin{description}
\item[{\bf Description:}] 
    Get this object from a sub-directory called "name"
\end{description}
\subsubsection{put}
 
\begin{flushleft} \textbf{%
int   \\ 
\settowidth{\ApproximateGlobalInverseIncludeArgIndent}{put(}%
put( GenericDataBase \& dir, const aString \& name) const
}\end{flushleft}
\begin{description}
\item[{\bf Description:}] 
 save this object to a sub-directory called "name"
\end{description}
\subsubsection{inverse}
 
\begin{flushleft} \textbf{%
void  \\ 
\settowidth{\ApproximateGlobalInverseIncludeArgIndent}{inverse(}%
inverse(const RealArray \& x, \\ 
\hspace{\ApproximateGlobalInverseIncludeArgIndent}RealArray \& r, \\ 
\hspace{\ApproximateGlobalInverseIncludeArgIndent}RealArray \& rx,\\ 
\hspace{\ApproximateGlobalInverseIncludeArgIndent}MappingWorkSpace \& workSpace, \\ 
\hspace{\ApproximateGlobalInverseIncludeArgIndent}MappingParameters \& params )
}\end{flushleft}
\begin{description}
\item[{\bf Purpose:}] 
   Find an approximate inverse of the mapping; this approximate inverse
   should be good enough so that Newton will converge

\item[{\bf Method:}]  
 \begin{enumerate}
   \item If space is periodic (e.g. if the grids all live on a background square which has
      one or more periodic edges) then we need to worry about values of x that are outside 
      the basic periodic region. These points may have periodic images that lie inside the
      periodic region. We thus add new points to the list that are the periodic images that
      lie inside the basic square. ***NOTE*** space periodic rarley occurs and probably hasn't
      been tested enough.
 \begin{verbatim}
       --------------------
       |                  |
       | x                |   X
       | periodic         |   initial point to invert
       | image            |
       |                  |
       |                  |
       |                  |
       --------------------
 \end{verbatim}
  \item For all points to invert, find the closest point on the reference grid that goes with
    the mapping. This grid is usually just the grid that is used when plotting the mapping.
     This step is performed by the function {\tt findNearestGridPoint}
 \end{enumerate}

\item[{\bf Notes:}] 
   The results produced by this routine are saved in the object workSpace.
\item[{\bf workSpace.x0 (output) :}]  list of points to invert with possible extra points if space is periodic.
\item[{\bf workSpace.r0 (output) :}]  unit square coordinates of the closest point.
\item[{\bf workSpace.I0 (output) :}]  Index object that demarks the active points in x0 and r0.
\item[{\bf workSpace.index0 (output) :}]  indirect addressing array that points back to the original r array; used
     when there are extra points added for periodicity in space.
\item[{\bf workSpace.index0IsSequential (output) :}]  if true then space is periodic and the index0 indirect addressing
     array should be used when storing results back in the user arrays r and rx.

\end{description}
