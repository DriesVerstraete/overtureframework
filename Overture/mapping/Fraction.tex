\section{Class Fraction}

This class is used to define ``fractions'', the ratio of two integers.
Fractions can represent infinity and -infinity with a
zero numerator and nonzero denominator. Thus $1/0$ is infinity
and $-1/0$ is -infinity. We define $2/0$ be be greater than $1/0$.

% \subsection{Constructors}
% 
% 
% \begin{tabbing}
% {\ff Fraction( int n, int d )12345 } \=  \kill
% {\ff Fraction( int n, int d=1 ) } 
% \> define a fraction, {\ff n} = numerator, {\ff d} = denominator \\
% \end{tabbing}
% 
\noindent
Note that we do not know how to construct a fraction from a real number.
% 
% \subsection{Member Functions}

The relational operators $\le$, $<$, $\ge$, $>$ and $==$ are defined for
the comparison of two fractions or a fraction and a real number.
In addition, the arithmetic operators $+$, $-$, $*$ and $/$ are defined for
two objects of type Fraction (or a Fraction and a real or int). 

\noindent
NOTE: By definition
the result of the operators $+$, $-$, $*$, or $/$ between a Fraction and
a real results in a real. 

% \noindent 
% Here are the member functions that can be used to access the numerator
% and denominator
% \begin{tabbing}
% {\ff xxxxxxxxxxxxxxxxxxx12345} \= \kill
% {\ff int setNumerator() } \> set the numerator \\
% {\ff int setDenominator() } \> set the denominator \\
% {\ff int getNumerator() } \> get the numerator \\
% {\ff int getDenominator() } \> get the denominator \\
% \end{tabbing}



\section{Class Bound}

A bound is defined as a real number, a fraction or null. The bound
class implements the bound and supplies functions for comparing
bounds. Bounds allow rational numbers to be specified precisely.
Bounds can represent infinity and -infinity by fractions with a
zero numerator and nonzero denominator. Thus $1/0$ is infinity
and $-1/0$ is -infinity. We define $2/0$ be be greater than $1/0$.


% \subsection{enum types}
% 
% {\footnotesize
% \begin{verbatim}
% 
% enum boundType{ realNumber, fraction, null };
% 
% \end{verbatim}
% }
% 
% \subsection{Constructors}
% \begin{tabbing}
% {\ff Bound( Fraction f0 ) } \= \kill
% {\ff Bound() } \> default constructor, boundType=null \\
% {\ff Bound( real x0 ) } \> define a bound from a real number \\
% {\ff Bound( int i ) } \> define a bound from a int \\
% {\ff Bound( Fraction f0 ) } \> define a bound from a fraction \\
% \end{tabbing}
% 
% 
% \subsection{Member Functions}
% 
The relational operators $\le$, $<$, $\ge$, $>$ and $==$ are defined for
the comparison of two bounds or a bound and a real number,
or a bound and a fraction.
In addition the arithmetic operators $+$, $-$, $*$ and $/$ are defined for
two objects of type Bound.
% There are also member functions to assign and retrieve values
% \begin{tabbing}
% {\ff void get( boundType bt, real x, Fraction f ) }xxxxx \= \kill
% {\ff void set( real value ) }  \> assign a real value to the bound  \\
% {\ff void set( int value ) }  \> assign an integer value to the bound  \\
% {\ff void set( int n, int d ) }  \> assign a numerator and denominator  \\
% {\ff void get( boundType bt, real x, Fraction f ) }  \> get boundType and value  \\
% \end{tabbing}
% 