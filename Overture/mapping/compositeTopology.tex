%=======================================================================================================
%  Documentation for compositeTopology
%=======================================================================================================
\documentclass[11pt]{article}

\voffset=-1.25truein
\hoffset=-1.truein
\setlength{\textwidth}{7in}      % page width
\setlength{\textheight}{9.5in}    % page height for xdvi

\usepackage{program}
\newtheorem{algorithm}{Algorithm}[section] 

\usepackage{epsfig}
\usepackage{graphics}    
\usepackage{moreverb}
\usepackage{amsmath}
% \usepackage{fancybox}
% \usepackage{subfigure}

\usepackage{float}

% bc = boldCommand
\newcommand{\bc}[1]{\mbox{\bf#1}}   % bold name
\newcommand{\cc}[1]{\mbox{  : #1}}  % comment

\usepackage{makeidx} % index
\makeindex
\newcommand{\Index}[1]{#1\index{#1}}


\begin{document}


\input  wdhDefinitions

% \newcommand{\primer}{/home/henshaw/res/primer}
% \newcommand{\gf}{/home/henshaw/res/gf}
\newcommand{\mapping}{/home/henshaw/Overture/mapping}
% \newcommand{\ogshow}{/home/henshaw/res/ogshow}
% \newcommand{\oges}{/home/henshaw/res/oges}
\newcommand{\figures}{./}

\vspace{3\baselineskip}
\begin{center}
  {\Large 
%   Determining the Connectivity of Patched CAD Models and  \\
%   Fast Projection Algorithms \\
   An Algorithm for Projecting Points onto a Patched CAD Model \\
  }
\vspace{ 2\baselineskip}
William D. Henshaw  \\
Centre for Applied Scientific Computing \\
Lawrence Livermore National Laboratory    \\
Livermore, CA, 94551   \\
henshaw@llnl.gov \\
\vspace{1\baselineskip}
\today \\
\vspace{\baselineskip}
% UCRL-MA-134240

\end{center}

\vspace{1\baselineskip}

\begin{abstract}
  We are interested in building structured overlapping grids for
geometries defined by computer-aided-design (CAD) packages.  Geometric
information defining the boundary surfaces of a computation domain is
often provided in the form of a collection of possibly hundreds of
trimmed patches. The first step in building an overlapping volume grid
on such a geometry is to build overlapping surface grids. A surface
grid is typically built using hyperbolic grid generation; starting
from a curve on the surface, a grid is grown by marching over the
surface. A given hyperbolic grid will typically cover many of the
underlying CAD surface patches. The fundamental operation needed for
building surface grids is that of projecting a point in space
onto the closest point on the CAD surface. We describe an fast
algorithm for performing this projection, it will make use of a fairly
coarse global triangulation of the CAD geometry. We describe how to
build this global triangulation by first determining the connectivity
of the CAD surface patches.  This step is necessary since it often the
case that the CAD description will contain no information specifying
how a given patch connects to other neighbouring patches. Determining
the connectivity is difficult since the surface patches may contain
mistakes such as gaps or overlaps between neighbouring patches.

% In order to build In this paper we describe an algorithm for determining the
% connectivity of a patched surface and an algorithm for building a
% global triangulation of the surface.  The global triangulation can
% serve as a basis for a fast projection algorithm for projecting points
% onto the patched surface. This projection algorithm can be used, for
% example, by a hyperbolic surface grid generator to grow new structured grids on the
% surface for use with the overlaping grid approach. 
\end{abstract}


% \vfill\eject
% \tableofcontents

\clearpage
\section{Introduction}

   We are motivated by the problem of of grid generation on
geometrical configurations defined by computer aided design (CAD)
packages. The description of the geometry is often in the form
of a collection of trimmed patches, see figure~(\ref{patchedSurface}).
The output from a CAD program will often be saved in a standard file
format such as IGES or the newer STEP specification. The description
defines a boundary-representation (B-REP) of the geometry, as opposed
to say a solid-model representation. Unfortunately the still widely
used IGES format does not include any connectivity information. There
is no information specifying how a given patch connects to other
neighbouring patches. As a first step in the grid generation process
this connectivity information must be determined. To further
complicate matters the trimmed patches will often be inaccurate, or
contain mistakes, making it difficult to determine where two
neighbouring patches should be joined.

\begin{figure}[hbt]
  \begin{center}
   \epsfig{file=\figures/topology.cat.ps,width=.65\linewidth}
  \end{center}
\caption{A CAD geometry represented as a collection of trimmed surface patches.} \label{patchedSurface}
\end{figure}

In this paper we describe an algorithm for determining the
connectivity of a patched surface and an algorithm for building a
global triangulation of the surface. 
The approach we take to determine the connectivity of patched CAD model
is based on the ``Edge-Curve'' approach described by Steinbrenner, Wyman and Chawner~\cite{Steinbrenner00}.
In this technique we first build
curves (edge-curves) on the boundaries of all trimmed-patches and then
attempt to identify where an edge-curve from one patch matches to the
edge curve of a neighbouring patch. It is usually necessary to split
the edge curves at appropriate locations in order to perform the
matching. When two edge-curves are identified to be the same we say
the edges have been merged and choose one edge-curve to define the
boundary segment for both patches. The details of algorithm described
here differ in a variety of ways from that of 
Steinbrenner et.al.~\cite{Steinbrenner00}. Since the details are important
we attempt to carefully describe our approach.


Once the edge curves have been matched we then can form a global
triangulation for the patchd surface. The first step in forming this
global triangulation is to build triangulations on each trimmed patch.
The triangulation of each patch is performed in the two-dimensional
parameter space, permitting the use of fast triangulation algorithms.
The triangulation on each patch will have boundary nodes that are
defined by the merged edge-curves.  This means that the separate
surface triangulations can be connected together since they will share
boundary nodes with a neighbouring patch.


The global triangulation can serve as a basis for a fast projection
algorithm for projecting points onto the patched surface. This
projection algorithm can be used, for example, by a hyperbolic surface
grid generator to grow new structured grids on the surface for use
with the overlaping grid approach. To project a point onto the patched
surface we first project the point onto the global
triangulation. Finding the closest triangle is performed by a
walking-algorithm if an initial guess is known or by a global search
using an ADT tree. Since each triangle belongs to just one sub-patch
we can then project the point onto the sub-patch using Newton's
method.

 The algorithm for determining the connectivity by merging edge curves
was motivated by the approach described by~\cite{gridGen}. We also
tried another technique where we first triangulated each trimmed-patch
and then attempted to stitch together neighbouring patches by adding
boundary nodes of one patch-triangulation to a nearby
patch-triangulation. This latter method worked reasonably well in many
cases but ran into difficulties when the trimmed patches did not match
very well (overlapping patches were especially troublesome) and for
patches containing very thin regions. In constrast, the edge-curve
merging approach works well since it assumes the boundaries of the
trimmed-patches consist of piecewise smooth segments that should
either match to a smooth segment of a neighbouring patch or be on the
boundary of the surface. This assumption is correct for the CAD
surfaces that we deal with.

The algorithms we describe here have been implemented within the
Overture object oriented framework\cite{??} and will be made available
with the Overture software which can be obtained from {\tt
http://\-www.llnl.gov/\-casc/\-Overture}.



\begin{figure}[hbt]
  \begin{center}
   \epsfig{file=\figures/topology.trim3d.2.ps,width=.31\linewidth}  
   \epsfig{file=\figures/topology.untrimmed.2.ps,width=.31\linewidth}
   \epsfig{file=\figures/topology.trim2d.2.ps,width=.31\linewidth}
  \end{center}
\caption{The trimmed patch (left) is formed from an untrimmed surface (middle) and a set of one or
more trimming curves (right). The untrimmed surface is a mapping from two-dimensional parameter space
into three-dimensional cartesian space. The trimming curves are defined in parameter space.} \label{trimmedPatched}
\end{figure}


\section{Determning the Connectivity of a Patched Surface}


A patched-surface consists of a set of sub-surfaces. There are often
hundreds of sub-surfaces.  A sub-surface may defined in a variety of
ways such as with a spline, B-spline or non-uniform-rational-bspline
(NURBS).  In general the sub-surface will be trimmed, in which case
only a portion of the surface will be used, the valid region is
defined by trimming curves, see figure~(\ref{trimmingCurves}).

It is often the case that the CAD file contains no topology
information, that is there is no information to say which sub-surface
connects to which other sub-surfaces. The purpose of the connectivity
algorithm is to determine how the sub-surfaces are connected. Once the
connection information is computed a triangulation for the whole
surface can be found. 

A useful feature of the connectivity algorithm is that it will aid in
the discovery of errors in the trimmed surfaces.  Gross errors in the
trimming curves are detected when the geometry is first read from the
database file. Errors detected at this time include trim curves that
lie outside the unit square in parameter space, trim curves that don't
close on themselves (i.e. they should be periodic), and trim curves
that self-intersect. These gross errors should be fixed before
proceeding to the connectivity stage. In Overture we have the ability
to edit the trim curves to fix these types of errors.  Errors detected
at the connectivity stage would include large gaps between patches or
multiple definition of patches (sometimes the exact same trimmed patch
may appear more than once in the CAD file!). These errors are usually
easily found by visually inspecting the set of merged and unmerged
curves.  There should only be unmerged curves on the boundary of the
surface.

There are two main steps in determining how sub-surfaces are connected.
\begin{description}
   \item[build edge curves] : build curve-segments that lie on the boundary of each sub-surface. 
      A sub-surface defined by a NURBS, for example, will have 4 boundary curve segments.  A sub-surface
      defined by a trimmed-mapping will have boundary segements corresponding to each trimming curve.
      A single trimming curve may be split into multiple boundary-segments (if the trimming curve
      was originally represented this way in the CAD file).
   \item[merge edge curves] : We examine the curve-segments to look for matching segments. If two segments
     agree (to some tolerance) we declare that the segments {\it are the same} (i.e. that they both
      represent the {\it true} boundary curve). Where two segments are the same, we also declare that
      their respective sub-surfaces are joined. It may be necessary to {\bf split} a curve-segment
      into two or more pieces so that the pieces can be joined to other segments. After merging all possible
      curve-segments we should have matched all sub-surfaces where they join other sub-surfaces, thus
      determing the topology of the surface.
\end{description}


\begin{algorithm}
\begin{programbox}
\bc{determineConnectivity}(\Cv )
\bc{Purpose} \cc{Determine the connectivity of a patched surface and build a global triangulation for the surface}
\Cv \cc{CompositeSurface}
\{\qtab

   \bc{buildEdgeCurves}

   \bc{mergeEdgeCurves}

   \bc{buildGlobalTriangulation}

\untab
\}
\end{programbox}
\end{algorithm}


\begin{figure}[hbt]
  \begin{center}
   \epsfig{file=\figures/topology.threePlanes.edges.ps,width=.31\linewidth}
   \epsfig{file=\figures/topology.threePlanes.merged.ps,width=.31\linewidth}
   \epsfig{file=\figures/topology.threePlanes.triangulation.ps,width=.31\linewidth}
   \end{center}
\caption{Figure showing the three stages of determining the connectivity. Edge curves are built
   on each side of each patch (left). The edge curves are merged and then split and merged (middle). Green curves
  have been merged, blue and red curves have not been merged. A red curve is an original curve that has been 
split.
Triangulations are built separately for each patch and then stitched together at the common boundary points (right).}
\label{connectivity}
\end{figure}


Trimming curves are usually defined in the two-dimensional parameter
space of the patch.  In order to compare edge-curves from different
patches we must build three-dimensional representations for the edge
curves. In some cases we can build an exact representation of the edge
curve. For example, if the edge curve is a parameter line on a NURBS
then the edge curve is itself a NURBS. In other cases it would be too
difficult to build an exact representation of the curve so instead we
sample the curve at some appropriate number of points and then fit a
curve to these points. The two-dimensional arclength and curvature of
the curve are used to determine how many points to use. Usually we
parameterize the edge-curve using the parameterization of the trimming
curve unless the parameterization is poor and then we parameterize by
the three-dimensional arclength. Usually a trimming curve will be
represented in the CAD file as a collection of sub-curves with each
sub-curve being smooth. These sub-curves will usually correspond to
the curve of intersection between two surface patches and thus be
exactly the edge-curves that we wish to merge. When a trimming curve
is created from a CAD file the sub-curves are merged into a single
composite curve; however, we also keep the the sub-curves.  In some
cases a trimming curve will not be smooth; a piece-wise linear NURBS
can have sharp corners, and even higher-order NURBS can represent
corners using multiple knots. Such trimming curves are split into
smooth sub-curves by looking for multiple knots and detecting corners
where the tangent changes rapidly.


Here is the algorithm for building the edge curves.
\begin{algorithm}
\begin{programbox}
\bc{buildEdgeCurves}(\Cv, \Ev )
\bc{Purpose} \cc{Build edge curves for all sub-surfaces of $\Cv$}
\Cv \cc{CompositeSurface}
\Ev \cc{List of edge curves}
\{\qtab

  e:=0 \cc{counts edge curves}
  \FOR s=0,1,\ldots,n_s

    \mbox{Build edge curves for sub-surface $s$}

    \cv := \Cv[s]

    \IF \mbox{$\cv$ is not trimmed}
      \FOR b=1,2,3,4
         \mbox{Build a curve for boundary edge $b$}

         \IF \mbox{$\cv$ is a NURBS we can build an exact representation of the edge}
            \Ev[e] = \cv.\bc{buildCurveOnSurface}
         \ELSE
 
          \mbox{Build a curve that interpolates the physical space trim curve}
         
           \Ev[e] = \bc{buildInterpolantEdgeCurve}(\xv_i)

           \IF \mbox{arclength of $\Ev[e] \ge$ mergeTolerance }         
             e=e+1
           \ELSE
              \mbox{throw this short segment away}
           \END
         \END
      \END
    \ELSIF
      \FOR c=0,1,\ldots,\mbox{number of trimming curves}

        \tv := \mbox{trimming curve $c$}
        \FOR \mbox{each sub-curve of the trimming curve}

          \mbox{Build a curve that interpolates the physical space trim curve}

          \Ev[e] = \bc{buildInterpolantEdgeCurve}(\xv_i)
          \IF \mbox{arclength of $\Ev[e] \ge$ mergeTolerance}       
            e=e+1
          \ELSE
             \mbox{throw this short segement away}
          \END

        \END

      \END

    \END  

  \END

\untab
\}
\end{programbox}
\end{algorithm}


After the edge curves have been built we then attempt to merge the
edge curves.  The merging step consists of two phases. In the first
phase we examine all the original edge curves and look for matching
curves. We use an ADT to search for possible matching edge-curves. If
two edges agree then we define the edges to be merged.  In the second
phase we consider all curves that were not merged in the first
phase. We attempt to split these curves into sub-curves which may then
be merged. An edge curve can be split where it touches the end-point
of another edge curve.  For each un-merged edge we look for the
endpoints of nearby edge-curves that will cause a split.  A split is
not allowed if it lies too close to the start or end of the un-merged
edge.


\begin{algorithm}
\begin{programbox}
\bc{mergeEdgeCurves}(\Cv, \Ev )
\bc{Purpose} \cc{Merge edge curves}
\Cv \cc{CompositeSurface}
\Ev \cc{List of edge curves}
\{\qtab

  \mbox{Merge edge curves}
  \FOR e=1,\ldots,numberOfEdgeCurves
    \IF edgeCurveStatus[e]==edgeCurveIsNotMerged 
      \mbox{Attempt to merge edge curve e with other edge curves}
      \bc{merge}(e)
    \END
  \END

  \mbox{Split and merge edge curves}
   \FOR e=1,\ldots,numberOfEdgeCurves
     \IF edgeCurveStatus[e]==edgeCurveIsNotMerged
       \mbox{attempt to split this edge curve}

       e2 : = \mbox{ edge curve to split with, choosen with an ADT tree}
       \IF \mbox{e2 is from a different sub-surface and edgeCurveStatus[e2]!=edgeCurveIsRemoved}
          \mbox{Check the endpoints of e2}
          x = \mbox{closest point on e to the endpoint of e2}
          dist = \mbox{distance from endpoint of e2 to edge curve e}
          dista = \mbox{shortest distance of x to the end point of e}

          splitTolerance = splitToleranceFactor * mergeTolerance
          \IF \mbox{$dist<mergeTolerance$ and $dista>splitTolerance$}

            \mbox{split this edge curve into two pieces}

            \mbox{attempt to merge the two new pieces with other edge curves}

          \END
        \END
     \END
   \END

\untab
\}
\end{programbox}
\end{algorithm}

Here is the algorithm that decides if two edge-curves can be merged.
In practice we have only found it necessary to compare the edge-curves at the end points and 
the midpoint.  

\begin{algorithm}
\begin{programbox}
\bc{merge}(e, \Ev )
\bc{Purpose} \cc{Attempt to merge edge curve e with other edge curves}
\Ev \cc{List of edge curves}
\{\qtab


  \FOR \mbox{check edge curve e2 chosen from an ADT search tree}

    \IF \mbox{end points of e and e2 match, $\| \xv_1 - \xv_2\|_1 < mergeTolerance$}
     
      \mbox{check distance at midpoint of e}
      \xv_m = \Ev[e].map(.5)  \cc{check the midpoint of e}
      \rv = \Ev[e2].inverseMap(\xv_m)
      \xv = \Ev[e2].map(\rv)
      \IF \| \xv - \xv_m \|_1 < mergeTolerance 
        \mbox{merge these curves}

	edgeCurveStatus[e]=edgeCurveIsMerged;
	edgeCurveStatus[e2]=edgeCurveIsRemoved;
        edgeCurveInfo(e,edgeCurveMatchingEdge)=e2;
	edgeCurveInfo(e2,edgeCurveMatchingEdge)=e;

      \END
    \END
  \END


\untab
\}
\end{programbox}
\end{algorithm}

\begin{alignat*}{3}
   \text{edgeCurveInfo}(.,&0) &&\mbox{: surface number} \\
                   &1  &&\mbox{: trim curve number} \\
                   &2  &&\mbox{: index of next segment in the trim curve} \\
                   &3  &&\mbox{: matching edge} \\ 
                   &4  &&\mbox{: orientation relative to the matching edge.}
\end{alignat*}


\begin{description}
 \item[edgeCurvePriority(e,0)==e2]  : the first point on edge curve e should really be defined as
                                the start or end (side2) of edge curve e2
 \item[edgeCurvePriority(e,2)==side2]
 \item[edgeCurvePriority(e,1)==e3]   : the last point on edge curve e should really be defined as
                               the start or end (side3) of edge curve ee
 \item[edgeCurvePriority(e,3)==side3]
\end{description}  



\section{Building a Global Triangulation}


A global triangulation can be built once the connectivity information has been determined by
the merging of edge curves. The global triangulation is formed by first triangulating each
surface patch. The nodes on the boundary of the patch are defined by nodes on the merged edge curves. This ensures
that the boundary nodes will match to neighbouring patches. The surface patches are triangulated
in the parameter space of the patch. This allows us to use fast two-dimensional triangulation
algorithms. We use the ``triangle'' program from Jonathon Shewchuck\cite{Shewchuck} to compute a
constrained Delaunay triangulation. We add additional nodes to the interior of the triangulation
but we prevent new nodes from being added to the boundary. Since the merged edge-curve is defined
in three-dimensional space we must determine the corresponding parameter space coordinates. In some cases
the parameter space coordinates are known from the time when the edge-curve was generated. In other
cases we must project the 3D points onto the surface patch. In addition to being more expensive this
projection step can also be error prone if the surface-patch is defined by a poor parameterization.
For example, it it not uncommon that the surface has a coordinate singularity where one face is
collapsed to a point. We double check the result of the projection step by comparing the projected
3D point to the original point being projected. If these points are not close we instead project the
point onto the boundary edge of the surface.





For each boundary node on a triangulation we identify the corresponding point on the fundamental edge curve.
\begin{align*}
   \text{boundaryNodeInfo}(i,0) &\mbox{: index into fundamental edge curve} \\
   \text{boundaryNodeInfo}(i,1) &\mbox{: edge curve number} \\
\end{align*}


\begin{algorithm}
\begin{programbox}
\bc{buildGlobalTriangulation}(\Ev )
\bc{Purpose} \cc{Build a global triangulation for the composite surface}
\Ev \cc{List of edge curves}
\{\qtab

  \FOR s=0,1,\ldots,\mbox{numberOfSubSurfaces}
    \mbox{Determine which edges should be used for this sub-surface}

     trimCurve(j,i) = \mbox{edge curve to use for trim curve i, sub-curve j of i.}

     \mbox{define a priority for the end points of edge curves}
     edgeCurvePriority(e,0..3) 
                                  

     \mbox{Build a triangulation in the parameter space of this sub-surface}
     \FOR i=1,2,\ldots,\mbox{numberOfTrimCurves}
        \mbox{Make a list of parameter space coordinates on all the trim sub-curves}
     \END

     \mbox{Build a triangulation in parameter space}

     \mbox{Map parameter space coordinates to physical space}

  \END

  \mbox{Build the global triangulation}
  \FOR s=0,1,\ldots,\mbox{numberOfSubSurfaces}

    \mbox{First add boundary nodes from sub-surface s}

    \FOR i=0,\ldots,\mbox{numberOfBoundaryNodes}
      j= boundaryNodeInfo(i,0);
      e = boundaryNodeInfo(i,1);

      \IF edgeNodeInfop[e](j)==-1 
        \mbox{first time we see an edge node we give it a global number}
	edgeNodeInfop[e](j)=globalNumber
        nodes(globalNumber,R3)=sNodes(i,R3);
	globalNumber++;
      \ELSE
	nodeIsNew(i)=false; // this node is already there.
      nodeTranslation(i)=edgeNodeInfop[e](j);
    \END

    \mbox{Now add interior nodes}
    \FOR i=0,\ldots,\mbox{numberOfInteriorNodes}
      nodeTranslation(i)=globalNumber
      nodes(globalNumber,R3)=sNodes(i,R3)
      globalNumber++
    \END

    \bc{connectivity}
    \mbox{Assign elements that come from sub-surface s}

    \mbox{Assign new faces}


  \END

\untab
\}
\end{programbox}
\end{algorithm}


Figure~(\ref{catTriangles}) shows a global triangulation computed for a CAD description of a
diesel engine. This example shows that a relatively coarse triangulation can be computed, if requested.
For this example it took $7.0$ seconds to build $1130$ edge curves, $4.0$ seconds to merge the curves and
$13.7$ seconds to build the global triangulation (Sun Ultra 430?? Mhz).

\begin{figure}[hbt]
  \begin{center}
   \epsfig{file=\figures/cat.triangulation.coarse.ps,width=.65\linewidth}
  \end{center}
\caption{Global triangulation for the diesel engine geometry.} \label{catTriangulation}
\end{figure}

\begin{figure}[hbt]
  \begin{center}
   \epsfig{file=\figures/kcs.stern.cad.ps,width=.475\linewidth}
   \epsfig{file=\figures/kcs.stern.triangulation.ps,width=.475\linewidth}
  \end{center}
\caption{Global triangulation for a tanker.} \label{catTriangulation}
\end{figure}

\begin{figure}[hbt]
  \begin{center}
   \epsfig{file=\figures/asmo.cad.ps,width=.475\linewidth}
   \epsfig{file=\figures/asmo.triangulation.ps,width=.475\linewidth}
  \end{center}
\caption{Global triangulation for the ``asmo'' model car. This model looks deceptively simple
      but contains 249 surfaces, many with very bad or singular parameterizations.} \label{amsoTriangulation}
\end{figure}


% \begin{figure}[hbt]
%   \begin{center}
%    \epsfig{file=\figures/asmo.triangulation.ps,width=.65\linewidth}
%   \end{center}
% \end{figure}


\section{Projecting Points onto the Patched Surface}


  The global triangulation for a patched surface can be used to define a
fast projection algorithm.  Given a target point, $\pv_t$, in space near the surface we
wish project the point onto the surface, i.e. we want to find the
closest point on the surface to a given point, defined in some norm. This projection algorithm
is used by the hyperbolic surface grid generator in Overture in order to
build structured grids on the surface. The projection algorithm consist of two steps.
First find the closest point on the triangulation. Secondly, uses the closest triangle to
determine the closest surface patch and project onto the surface patch.

Given a good initial guess as to the closest triangle to $\pv_t$, we 
find the closest point on the global triangulation we use a walking method\cite{}.
The walking method starts at a given triangle and marches to a neighbouring triangle that is
closer to the target point. 
This marching continues until it reaches the boundary of the 
triangulation or else reaches an extremal triangle. An extremal triangle will be one
where the line passing through the target point in the direction normal to the triangle
face intersects the triangle.
An extremal triangle could be a local
maximum, a local minimum or a saddle point in the distance from the target point to the surface.
We rely on the initial guess being good enough and the triangulation to be sufficiently fine
for this walking method to give a reasonable answer.


\newcommand{\com}[1]{\{\mbox{\it #1}\}}

\begin{algorithm}
\begin{programbox}
\bc{getTriangleCoordinates( $e, \xv, \xv_e, r,s, inside$ )}
\bc{Purpose} \cc{Compute the barycentric? triangle coordinates}
e \cc{element number for the triangle}
\xv \cc{find the coordinates of this point}
\xv_e \cc{output: closest point on the triangle to $\xv$}
r,s \cc{output: barycentric coordinates}
\{\qtab
  \com{Element $e$ has vertices $\xv_0$, $\xv_1$, and $\xv_2$}

  \av = \xv_1 - \xv_0
  \bv = \xv_2 - \xv_0

  \com{Compute $r,s$ such that}
  \xv-\xv_0 = r \av + s \bv
   
   inside = 
\untab
\}
\end{programbox}
\end{algorithm}



\begin{algorithm}
\begin{programbox}
\bc{projectOntoTriangulation($\xv,e_0$)}
\bc{Purpose} \cc{Project a point $\xv$ onto a triangulated surface}
\xv \cc{point to project}
e_0 \cc{initial guess for the closest element, $e_0=-1$ if no guess is known}
\{\qtab
   \IF e_0 \ge 0 
     \com{Perform an incremental search}
     e := e_0
     done := false
     s=0
     \WHILE \mbox{ $!$done and $s<\rm{maxNumberOfSteps}$}
       s=s+1
       getTriangleCoordinates( e, \xv, \xv_e, r,s, inside )
       \IF inside 
         done:=true
       \ELSE
         e_1 := \mbox{element adjacent to e in the direction of $\xv$}

         \IF \mbox{$e_1$ is not on the boundary}
           e := e_1
         \ELSE
           \IF \mbox{near a corner on the boundary}
             e:=-1 \com{force a global search}
             done:=true
         \END

         \IF \mbox{$e = e_{\rm old}$ or $e = e_{\rm older}$}
           \com{check for cycles}
           done = true;
         \ELSIF e_0 \ge 0 
           \com{Check for corners on the surface}
           \theta = \mbox{angle between the normals to $e$ and $e_{\rm old}$}
           \IF \theta > \theta_0 
              \com{Adjust the position of $\xv$ to account for the corner}
           \END
         \END
       \END
     \END
   \END

   \IF e<0 
     \bc{globalSearch($\xv$)}
   \END 
   
\untab
\}
\end{programbox}
\end{algorithm}


If we do not have an initial guess we use a global search to find the closest point on the trianglulation.
The global search uses an alternating-digital-tree (ADT) tree 
in which we save the bounding boxes for all triangles
on the global mesh. An ADT tree is a special type of binary search tree. 
We look for the intersection of a box around the target point
with the triangle bounding boxes; this will determine potential triangles to check. The ADT tree 
is a fast way to answer his query. Given a list of potetntial triangles we check each one
to determine the closest point. The only parameter in this search is the size of the bounding
box around the target point. It should not be too large nor too small. We usually start with a
safe value and then increase or decrease the box size for subsequent searches depeneding on the
number of intersections found.


Each triangle on the global triangulation lies on exactly one surface-patch. 
Once the closest triangle has been found we then suppose that the closest
point on the surface will lie on the patch pointed to by the triangle. This assumes
that the surface triangulation resolves the surface to a reasonable degree. 
If the target point lies very near the boundary between two patches it could be that
the true projected point is on the neighbouring patch. For now we have ignored this possibility,
although it would be possible to deal with this case. For our purposes so far it doesn't
seem to be an issue.

\begin{figure}[hbt]
  \begin{center}
   \epsfig{file=\figures/electrode.inner.grid.ps,width=.675\linewidth}
  \end{center}
\caption{The hyprbolic surface grid generator uses the fast projection algorithm to grow surface grids.}
\end{figure}



\vfill\eject
\bibliography{/home/henshaw/papers/henshaw}
\bibliographystyle{siam}

% \printindex

\end{document}