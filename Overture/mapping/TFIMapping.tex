\section{TFIMapping: Transfinite-Interpolation}
\index{tfi mapping}\index{Mapping!TFIMapping}\index{interpolation!transfinite}\index{transfinite interpolation}
\index{Coon's patch}

{\bf Thanks} to Brian Miller for help with this Mapping.

Use this Mapping to create a transfinite-interpolation mapping (also known as a Coon's patch).
Transfinite-interpolation creates a patch (grid) in 2D (3D) using curves (surfaces) that define the boundaries.
The quality of this algebaric grid is strongly dependent on the parameterizations
of the boundary curves.
//
In its simplest form this mapping blends two given curves to create a grid
in the region between the curves. 
Given the two curves $\cv_i(r_1)$ the linear interpolation formula is
\[
  \xv(r_1,r_2) = \cv_0(r_1) (1-r_2) + \cv_1(r_1) r_2
\]
   
With an hermite interpolation it is possible to also specify the $r_2$ derivative of the
patch at the boundaries. The formula for hermite interpolation between two curves
is
\[
  \xv(r_1,r_2) = \cv_0(r_1)\Phi_0(r_2) + \cv(r_1) \Phi_1(r_2)
            ~+~  \dot{\cv}_0(r_1)\Psi_0(r_2) + \dot{\cv}(r_1) \Psi_1(r_2)
\]
where the blending functions are given by
\begin{eqnarray*} 
  \Phi_0(r)=(1+2r)(1-r)^2 && \Phi_1(r)=(3-2r) r^2 \\
  \Psi_0(r)= r    (1-r)^2 && \Psi_1(r)=(r-1)  r^2
\end{eqnarray*} 
The patch will satisfy the conditions
\[
 \xv(r_1,i) =  \cv_i(r_1) ~~\mbox{and}~~ {\partial \over\partial r_2} \xv(r_1,0) =  \dot{\cv}_i(r_1)~~~i=0,1 ~~
\]
The grid lines will be normal to the boundary provided that we choose the derivatives $\dot{\cv}_i$
to be proportional to the normal vectors. The scaling of this vector will determine the grid
spacing near the boundary. We choose 
\[
    \dot{\cv}_i(r) = \| \cv_1(r)-\cv_0(r) \| ~\nv_i(r)
\]
 where $\nv_i(r)$ is the unit normal vector and $\| (a,b) \| =\sqrt{a^2+b^2}$. 

 The derivative of the hermite-interpolant involves the second derivative of the
 boundary curves and is given by
\[
  { \partial\over \partial r_1 } \xv(r_1,r_2) = { \partial\over \partial r_1 }\cv_0(r_1)\Phi_0(r_2) 
        +{ \partial\over \partial r_1 } \cv(r_1) \Phi_1(r_2)
    ~+~ { \partial\over \partial r_1 } \dot{\cv}_0(r_1)\Psi_0(r_2) 
      + { \partial\over \partial r_1 } \dot{\cv}_1(r_1) \Psi_1(r_2)
\]
where
\[
  { \partial\over \partial r_1 }  \dot{\cv_i} =   
      { (\cv_1-\cv_0)\cdot(\cv_{1,r_1}-\cv_{0,r_1}) \dot{\cv_i} \over \| \cv_1-\cv_0 \|^2 }
    - { (\dot{\cv_i}\cdot\cv_{i,r_1 r_1}) \cv_{i,r_1} \over \| \cv_{i,r_1} \|^2 }    
\]

\newcommand{\cvLeft}{\cv_{\rm left}}
\newcommand{\cvRight}{\cv_{\rm right}}
\newcommand{\cvBottom}{\cv_{\rm bottom}}
\newcommand{\cvTop}{\cv_{\rm top}}
\newcommand{\cvBack}{\cv_{\rm back}}
\newcommand{\cvFront}{\cv_{\rm Front}}
More generally a transfinite-interpolation mapping can interpolate 4 curves in 2D/3D or up to 6 curves in 3D.
Define
\begin{align*}
  \cvLeft    & :  \mbox{side corresponding to $r_1=0$.}  \\
  \cvRight   & :  \mbox{side corresponding to $r_1=1$.}  \\
  \cvBottom  & :  \mbox{side corresponding to $r_2=0$.}  \\
  \cvTop     & :  \mbox{side corresponding to $r_2=1$.}  \\
  \cvBack    & :  \mbox{side corresponding to $r_3=0$.}  \\
  \cvFront   & :  \mbox{side corresponding to $r_3=1$.}
\end{align*}
The notation `left', `right', etc. comes from the fact that when a cube is plotted in the graphics window
the left side is the face $r_1=0$, the right face is $r_1=1$ etc.

In two dimensions, when 4 sides are specified, the formula for linear interpolation is
\begin{align*}
    \cv(r_1,r_2) & = \cvLeft(r_2) (1-r_1) + \cvRight(r_2) r_1 \\
                 & + \cvBottom(r_1)(1-r_2) + \cvTop(r_1) r_2  \\  
                 & -\Big\{ (1-r_2)( (1-r_1)\cvLeft(0) + r_1 \cvRight(1) ) 
                          +   r_2( (1-r_1)\cvBottom(0) + r_1 \cvTop(1) ) \Big\}.
\end{align*}
The last line in this formula represents a correction, a bilinear function that passes through the
four corners, that ensures the mapping, $\cv$, matches the four boundary curves.
In three-dimensions with 6 sides specified
\begin{align*}
    \cv(r_1,r_2) & = \cvLeft(r_2,r_3) (1-r_1) + \cvRight(r_2,r_3) r_1 \\
                 & + \cvBottom(r_1,r_3)(1-r_2)  +   \cvTop(r_1,r_3) r_2  \\  
                 & + \cvBack(r_1,r_2)(1-r_3)  +   \cvFront(r_1,r_2) r_3  \\  
                 & -\Big\{ (1-r_2)( (1-r_1)\cvLeft(0,r_3) + r_1 \cvRight(0,r_3) )  
                           +    r_2( (1-r_1)\cvBottom(0,r_3) + r_1 \cvTop(1,r_3) )   \\  
                 &         + (1-r_2)( (1-r_3)\cvBottom(r_1,0) + r_3 \cvTop(r_1,1) )      
                           +    r_2( (1-r_3)\cvBack(r_1,1) + r_3 \cvFront(r_1,1) )   \\  
                 &         + (1-r_3)( (1-r_1)\cvBack(0,r_2) + r_1 \cvFront(1,r_2) )      
                           +    r_3( (1-r_1)\cvLeft(r_2,1) + r_1 \cvRight(r_2,1) )  \Big\}  \\  
                 & +\Big[ (1-r_3)[ (1-r_2)( (1-r_1)\cv(0,0,0) + r_1 \cv(1,0,0) )      
                                     + r_2 ( (1-r_1)\cv(0,1,0) + r_1 \cv(1,1,0)) )   \\  
                 &           +r_3 [ (1-r_2)( (1-r_1)\cv(0,0,1) + r_1 \cv(1,0,1) )      
                                     + r_2 ( (1-r_1)\cv(0,1,1) + r_1 \cv(1,1,1)) )   \Big].
\end{align*}
In the 3d case we must first subtract off corrections as in 2D (3 such corrections) and then
add back a trilinear function that passes through the 8 vertices to ensure that all 6 sides are
matched.

\subsection{Compatibility conditions}
  The above TFI formulae will only give a reasonable grid if
\begin{enumerate}
  \item The curves that define the faces match at vertices (and edges in 3d).
  \item The curves are parameterized in the `same direction', otherwise the
        grids lines could cross.
  \item Curves on opposite sides are parameterized in a similar way.
\end{enumerate}
Even if all these conditions are met the grid lines may still cross if the boundary curves
are strange enough.


\subsection{Examples}
\subsubsection{2D linear TFI mapping with 2 sides specified}
\noindent
\begin{minipage}{.4\linewidth}
{\footnotesize
\listinginput[1]{1}{\mapping /tfi.cmd}
}
\end{minipage}\hfill
\begin{minipage}{.6\linewidth}
  \begin{center}
   \includegraphics[width=9cm]{\figures/tfi1} \\
   % \epsfig{file=\figures/tfi1.ps,height=4.in}  \\
  {A grid created with linear trans-finite interpolation}  \label{fig:tfi1}
  \end{center}
\end{minipage}

\subsubsection{2D hermite TFI mapping with 2 sides specified}
\noindent
\begin{minipage}{.4\linewidth}
{\footnotesize
\listinginput[1]{1}{\mapping /tfi2.cmd}
}
\end{minipage}\hfill
\begin{minipage}{.6\linewidth}
  \begin{center}
   \includegraphics[width=9cm]{\figures/tfi2} \\
   % \epsfig{file=\figures/tfi2.ps,height=4.in}  \\
  {A grid created with hermite trans-finite interpolation; grid lines are normal to the 
       bottom and top boundaries}  \label{fig:tfi2}
  \end{center}
\end{minipage}

\subsubsection{2D linear TFI mapping with 4 sides specified}
\noindent
\begin{minipage}{.4\linewidth}
{\footnotesize
\listinginput[1]{1}{\mapping /tfi2d4.cmd}
}
\end{minipage}\hfill
\begin{minipage}{.6\linewidth}
  \begin{center}
   \includegraphics[width=9cm]{\figures/tfi2d4c} \\
   \includegraphics[width=9cm]{\figures/tfi2d4} \\
   % \epsfig{file=\figures/tfi2d4c.ps,height=4.in}  \\
   % \epsfig{file=\figures/tfi2d4.ps,height=4.in}  \\
  {A grid created with linear trans-finite interpolation, all four sides are specified. The top
     figure shows the 4 boundary curves before interpolation.} \label{fig:tfi2d4}
  \end{center}
\end{minipage}

\subsubsection{3D linear TFI mapping with 2 sides specified}
\noindent
\begin{minipage}{.4\linewidth}
{\footnotesize
\listinginput[1]{1}{\mapping /tfi4.cmd}
}
\end{minipage}\hfill
\begin{minipage}{.6\linewidth}
  \begin{center}
   \includegraphics[width=9cm]{\figures/tfi4} \\
   % \epsfig{file=\figures/tfi4.ps,height=4.in}  \\
  {A grid created with linear trans-finite interpolation between two Annulus mappings.}
  \end{center}
\end{minipage}



%% \input TFIMappingInclude.tex
