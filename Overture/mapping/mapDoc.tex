%=======================================================================================================
% Mapping Documentation
%    Short version for looking at the write up for one or two mappings.
%=======================================================================================================
\documentclass{article}

\voffset=-1.25truein
\hoffset=-1.truein
\setlength{\textwidth}{7in}      % page width
\setlength{\textheight}{9.5in}    % page height for xdvi

\usepackage{epsfig}
\usepackage{graphics}    
\usepackage{moreverb}
\usepackage{amsmath}
\usepackage{fancybox}
\usepackage{subfigure}
\usepackage{multicol}

\begin{document}


\input wdhDefinitions

\def\uvd    {{\bf U}}
\def\ud     {{    U}}
\def\pd     {{    P}}
\def\id     {i}
\def\jd     {j}
\def\kap {\sqrt{s+\omega^2}}

\newcommand{\primer}{/home/henshaw/res/primer}
\newcommand{\gf}{/home/henshaw/res/gf}
\newcommand{\mapping}{/home/henshaw/res/mapping}
\newcommand{\ogshow}{/home/henshaw/res/ogshow}
\newcommand{\oges}{/home/henshaw/res/oges}
\newcommand{\figures}{../docFigures}


% \input DataPointMapping.tex

% \input inverse
\input NurbsMapping.tex
% \input HyperbolicMapping.tex

% \vfill\eject
% \input SweepMapping.tex

% \input TrimmedMapping.tex
% \input RevolutionMapping.tex
% \input EllipticGridGenerator.tex
% \vfill\eject
% \input CompositeSurface.tex
% \vfill\eject

% \input FilletMapping.tex
% \input JoinMapping.tex
% \input CrossSectionMapping.tex


\end{document}
