\subsubsection{Constructor}
 
\newlength{\SplineMappingIncludeArgIndent}
\begin{flushleft} \textbf{%
\settowidth{\SplineMappingIncludeArgIndent}{SplineMapping(}% 
SplineMapping(const int \& rangeDimension\_  =2) 
}\end{flushleft}
\begin{description}
\item[{\bf Purpose:}] 
    Default Constructor: create a spline curve with the given range dimension.
 Use this Mapping to create a cubic spline curve in two dimensions.
 This spline is defined by a set of points (knots), $x(i),y(i)$.
 The spline is normally parameterized by arclength. The pline can also be parameterized
 by a weighting of arclength and curvature so that more points are placed in regions
 with high curvature.
 For a spline which is periodic in space, the Mapping will automatically
 add an extra point if the first point is not equal to the last point.

\item[{\bf rangeDimension\_ :}]  1,2, 3

 The SplineMapping uses `{\bf TSPACK}: Tension Spline Curve Fitting Package'
 by Robert J. Renka; available from Netlib. See the TSPACK documentation
 and the reference 
 \begin{description}
   \item[RENKA, R.J.] Interpolatory tension splines with automatic selection
   of tension factors. SIAM J. Sci. Stat. Comput. {\bf 8}, (1987), pp. 393-415.
 \end{description}

\end{description}
\subsection{shift}
 
\begin{flushleft} \textbf{%
int  \\ 
\settowidth{\SplineMappingIncludeArgIndent}{shift(}%
shift(const real \& shiftx  =0., \\ 
\hspace{\SplineMappingIncludeArgIndent}const real \& shifty  =0., \\ 
\hspace{\SplineMappingIncludeArgIndent}const real \& shiftz /* =0.*/ )
}\end{flushleft}
\begin{description}
\item[{\bf Purpose:}]  Shift the SPLINE in space.
\end{description}
\subsection{scale}
 
\begin{flushleft} \textbf{%
int  \\ 
\settowidth{\SplineMappingIncludeArgIndent}{scale(}%
scale(const real \& scalex  =0., \\ 
\hspace{\SplineMappingIncludeArgIndent}const real \& scaley  =0., \\ 
\hspace{\SplineMappingIncludeArgIndent}const real \& scalez /* =0.*/ )
}\end{flushleft}
\begin{description}
\item[{\bf Purpose:}]  Scale the SPLINE in space.
\end{description}
\subsection{rotate}
 
\begin{flushleft} \textbf{%
int  \\ 
\settowidth{\SplineMappingIncludeArgIndent}{rotate(}%
rotate( const int \& axis, const real \& theta )
}\end{flushleft}
\begin{description}
\item[{\bf Purpose:}]  Perform a rotation about a given axis. This rotation is applied
   after any existing transformations. Use the reset function first if you
   want to remove any existing transformations.
\item[{\bf axis (input) :}]  axis to rotate about (0,1,2)
\item[{\bf theta (input) :}]  angle in radians to rotate by.
\end{description}
\subsubsection{setParameterizationType}
 
\begin{flushleft} \textbf{%
int  \\ 
\settowidth{\SplineMappingIncludeArgIndent}{setParameterizationType(}%
setParameterizationType(const ParameterizationType \& type)
}\end{flushleft}
\begin{description}
\item[{\bf Description:}] 
   Specify the parameterization for the Spline. With {\tt index} parameterization
 the knots on the spline are parameterized as being equally spaced. With {\tt arclength}
  parameterization the knots are parameterized by arclength or a weighted combination
 of arclength and curvature. With {\tt userDefined} parameterization the user must supply
 the parameterization through the {\tt setParameterization} function.
\item[{\bf type (input) :}]  One of {\tt index} or {\tt arcLength} or {\tt userDefined}.
\end{description}
\subsubsection{getParameterization}
 
\begin{flushleft} \textbf{%
const RealArray \&  \\ 
\settowidth{\SplineMappingIncludeArgIndent}{getParameterizationS(}%
getParameterizationS() const
}\end{flushleft}
\begin{description}
\item[{\bf Description:}] 
   Return the current parameterization.
\end{description}
\subsubsection{getNumberOfKnots}
 
\begin{flushleft} \textbf{%
int  \\ 
\settowidth{\SplineMappingIncludeArgIndent}{getNumberOfKnots(}%
getNumberOfKnots() const
}\end{flushleft}
\begin{description}
\item[{\bf Purpose:}]  
    Return the number of knots on the spline. 
\end{description}
\subsubsection{setParameterization}
 
\begin{flushleft} \textbf{%
int   \\ 
\settowidth{\SplineMappingIncludeArgIndent}{setParameterization(}%
setParameterization(const RealArray \& s\_ )
}\end{flushleft}
\begin{description}
\item[{\bf Description:}] 
   Supply a user defined parameterization. This routine will set the parameterization type
 to be {\tt userDefined}.
\item[{\bf s\_ (input) :}]  An increasing sequence of values that are to be
     used to parameterize the spline points. These values must cover the interval [0,1] which
   will be the interval defining the mapping. You could add values outside [0,1] to define the
 behaviour of the spline at "ghost points".  The number of points in the array must
     be equal to the number of points supplied when the {\tt setPoints} function is called.
\end{description}
\subsubsection{parameterize}
 
\begin{flushleft} \textbf{%
int  \\ 
\settowidth{\SplineMappingIncludeArgIndent}{parameterize(}%
parameterize(const real \& arcLengthWeight\_ /* =1.*/, \\ 
\hspace{\SplineMappingIncludeArgIndent}const real \& curvatureWeight\_ /* =0.*/ )
}\end{flushleft}
\begin{description}
\item[{\bf Description:}] 
   Set the `arclength' parameterization parameters. The parameterization is chosen to
 redistribute the points to resolve the arclength and/or the curvature of the curve.
 By default the spline is parameterized by arclength only. To resolve regions of high
 curvature choose the recommended values of {\tt arcLengthWeight\_=1.} and
  {\tt curvatureWeight\_=.5}.

  To determine the parameterization we equidistribute the weight function 
  \[
     w(r) = 1. + {\rm arcLengthWeight} {s(r)\over |s|_\infty}  
               + {\rm curvatureWeight} {c(r)\over |c|_\infty}
  \]
  where $s(r)$ is the local arclength and $c(r)$ is the curvature. Note that we normalize
 $s$ and $c$ by their maximum values.
  
\item[{\bf arcLengthWeight\_ (input):}]  A weight for arclength. A negative value may give undefined results.
\item[{\bf curvatureWeight\_ (input):}]  A weight for curvature. A negative value may give undefined results.
\end{description}
\subsubsection{setEndConditions}
 
\begin{flushleft} \textbf{%
int   \\ 
\settowidth{\SplineMappingIncludeArgIndent}{setEndConditions(}%
setEndConditions(const EndCondition \& condition, \\ 
const RealArray \& endValues  =Overture::nullRealDistributedArray())
}\end{flushleft}
\begin{description}
\item[{\bf Description:}] 
   Specify end conditions for the spline
\item[{\bf condition (input) :}]  Specify an end condition.
    \begin{description}
      \item[monontone parabolic fit] : default BC for the shape preserving spline.
      \item[first derivative] : user specified first derivatives.
      \item[second derivative] : user specified second derivatives.
    \end{description}
\item[{\bf endValues (input) :}]  if {\tt condition==firstDerivative} (or {\tt condition==secondDerivative})
  then endValues(0:1,0:r-1) should
  hold the values for the first (or second) derivatives of the spline at the start and end. Here
   r=rangeDimension.
     
\end{description}
\subsubsection{setPoints}
 
\begin{flushleft} \textbf{%
int  \\ 
\settowidth{\SplineMappingIncludeArgIndent}{setPoints(}%
setPoints( const RealArray \& x )
}\end{flushleft}
\begin{description}
\item[{\bf Purpose:}]  Supply spline points for a 1D curve.
\item[{\bf x (input) :}]  array of spline knots.
  The spline is parameterized by a NORMALIZED index, i/(number of points -1), i=0,1,...
\end{description}
\subsubsection{setPoints}
 
\begin{flushleft} \textbf{%
int  \\ 
\settowidth{\SplineMappingIncludeArgIndent}{setPoints(}%
setPoints( const RealArray \& x, const RealArray \& y )
}\end{flushleft}
\begin{description}
\item[{\bf Purpose:}]  Supply spline points for a 2D curve. Use the points (x(i),y(i)) i=x.getBase(0),..,x.getBound(0)
\item[{\bf x,y (input) :}]  array of spline knots.
\end{description}
\subsubsection{setPoints}
 
\begin{flushleft} \textbf{%
int  \\ 
\settowidth{\SplineMappingIncludeArgIndent}{setPoints(}%
setPoints( const RealArray \& x, const RealArray \& y, const RealArray \& z )
}\end{flushleft}
\begin{description}
\item[{\bf Purpose:}]  Supply spline points for a 3D curve. Use the points (x(i),y(i),z(i)) i=x.getBase(0),..,x.getBound(0)
\item[{\bf x,y,z (input) :}]  array of spline knots.
\end{description}
\subsubsection{setShapePreserving}
 
\begin{flushleft} \textbf{%
int  \\ 
\settowidth{\SplineMappingIncludeArgIndent}{setShapePreserving(}%
setShapePreserving( const bool trueOrFalse  = true)
}\end{flushleft}
\begin{description}
\item[{\bf Description:}] 
   Create a shape preserving (monotone) spline or not
\item[{\bf trueOrFalse (input) :}]  if true, create a spline that preserves the shape. For a one dimensional
   curve the shape preserving spline will attempt to remain montone where the knots ar montone.
   See the comments with TSPACK for further details.

\end{description}
\subsubsection{setTension}
 
\begin{flushleft} \textbf{%
int   \\ 
\settowidth{\SplineMappingIncludeArgIndent}{setTension(}%
setTension( const real \& tensionFactor )
}\end{flushleft}
\begin{description}
\item[{\bf Description:}] 
   Specify a constant tension factor. Specifying this value will turn off the shape preseeving feature.
\item[{\bf tensionFactor (input):}]  A value from 0. to 85. A value of 0. corresponds to no tension.

\end{description}
\subsubsection{setDomainInterval}
 
\begin{flushleft} \textbf{%
int   \\ 
\settowidth{\SplineMappingIncludeArgIndent}{setDomainInterval(}%
setDomainInterval(const real \& rStart\_  =0., \\ 
\hspace{\SplineMappingIncludeArgIndent}const real \& rEnd\_  =1.)
}\end{flushleft}
\begin{description}
\item[{\bf Description:}] 
 Restrict the domain of the spline.
 By default the spline is parameterized on the interval [0,1].
 You may choose a sub-section of the spline by choosing a new interval [rStart,rEnd].
 For periodic splines the interval may lie in [-1,2] so the sub-section can cross the branch cut.
 You may even choose rEnd<rStart to reverse the order of the parameterization.
\item[{\bf rStart\_,rEnd\_ (input) :}]  define the new interval.
\end{description}
\subsubsection{getDomainInterval}
 
\begin{flushleft} \textbf{%
int   \\ 
\settowidth{\SplineMappingIncludeArgIndent}{getDomainInterval(}%
getDomainInterval(real \& rStart\_, real \& rEnd\_) const
}\end{flushleft}
\begin{description}
\item[{\bf Description:}] 
  Get the current domain interval.
\item[{\bf rStart\_,rEnd\_ (output) :}]  the current domain interval.
\end{description}
\subsubsection{setIsPeriodic}
 
\begin{flushleft} \textbf{%
void  \\ 
\settowidth{\SplineMappingIncludeArgIndent}{setIsPeriodic(}%
setIsPeriodic( const int axis, const periodicType isPeriodic0 )
}\end{flushleft}
\begin{description}
\item[{\bf Description:}] 
\item[{\bf axis (input):}]  axis = (0,1,2) (or axis = (axis1,axis2,axis3)) with $axis<domainDimension$.
\item[{\bf Notes:}] 
    This routine has some side effects. It will change the boundaryConditions to be consistent
  with the periodicity (if necessary).
\end{description}
\subsubsection{useOldSpline}
 
\begin{flushleft} \textbf{%
int   \\ 
\settowidth{\SplineMappingIncludeArgIndent}{useOldSpline(}%
useOldSpline( const bool \& trueOrFalse  =true)
}\end{flushleft}
\begin{description}
\item[{\bf Description:}] 
  Use the old spline routines from FMM, Forsythe Malcolm and Moler. This is for backward
 compatability.
\item[{\bf trueOrFalse (input) :}]  If true Use the old spline from FMM, otherwise use the tension splines.

\end{description}
\subsubsection{map}
 
\begin{flushleft} \textbf{%
void  \\ 
\settowidth{\SplineMappingIncludeArgIndent}{mapS(}%
mapS( const RealArray \& r, RealArray \& x, RealArray \& xr, MappingParameters \& params )
}\end{flushleft}
\begin{description}
\item[{\bf Purpose:}]  Evaluate the spline and/or derivatives. 
\end{description}
\subsubsection{update}
 
\begin{flushleft} \textbf{%
int  \\ 
\settowidth{\SplineMappingIncludeArgIndent}{update(}%
update( MappingInformation \& mapInfo ) 
}\end{flushleft}
\begin{description}
\item[{\bf Purpose:}]  Interactively create and/or change the spline mapping.
\item[{\bf mapInfo (input):}]  Holds a graphics interface to use.
\end{description}
