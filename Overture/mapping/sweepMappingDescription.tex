\documentclass[12pt]{article}
\usepackage{graphics}
\usepackage{listings}
\input epsf
\begin{document}
\begin{center}
\Large \bf Sweep Mapping
\end{center}
\underline{\underline{Purpose}}:\\
Given a planar surface (or curve) $\bf S(r_1,r_2)$ (or $ \bf S(r_1)$), and a 
$3D$ curve $\bf C(r_3)$, we would like to 
generate a $3D$ volume or surface by sweepping $\bf S$ perpendicularly
to $\bf C$ in such a way that the center of each $\bf S_k$ ring lie on
the curve $\bf C$. At $r_3 = 0$, it is assumed that $\bf S = S_0$ is
orthogonal to $\bf C$ and the tangent to $\bf C$ coincide with the
normal $\bf n $ to $\bf S$. To make sure that the center of $\bf S =
S_0$ lies at $\bf C(0)$, We first find the center $(x_0,y_0,z_0)$ as the
average of all the points that make up the sweep surface $\bf S$, namely
$x_0={\sum_{i=0}^n {x_i}\over n+1}$, $y_0={\sum_{i=0}^n {y_i}\over n+1}$,
and $z_0={\sum_{i=0}^n {z_i}\over n+1}$.
Then a translation that maps $\bf C(0)$ to $(x_0,y_0,z_0)$ is applied 
to $\bf C$.
\\
\\
\underline{\underline{Strategy}}:\\
With a sufficient number of grid points in each direction, 
we incrementally compute the matrix transformation to be used the
following way. At $k=0$ corresponding to $r_3=0$, the identity matrix is
used since $\bf S$ and $\bf C$ satisfy the required conditions and $\bf
S_0 = S$. For $k>0$, the ring $S_k$ is gotten from the ring $S_{k-1}$
the following way:\\
A translation that maps the center of $S_{k-1}$ (which is the same
point as $C_{k-1}$) to the point $C_k$ is applied to $S_{k-1}$. A
rotation is then applied to the resulting points is such a way that the
unit normal to the surface $\bf S_{k-1}$ coincides with the tangent to the
curve $\bf C$ at the point $C_k$. To implement this,
the unit vector $\bf n_0$  of the surface $\bf S_{k-1}$ is chosen to 
be the first vector in a new orthonormal basis. The second basis vector 
$\bf n_1$ is given by $\bf n_1 = {n_0\times t \over \| n_0\times t\|}$
where $\bf t = {\partial C (r_3+\Delta r_3) \over \partial r_3}$. The third 
basis vector $\bf n_2$ is given by $\bf n_0\times n_1\over \|n_0\times n_1\|$. In the
new coordinate system, the rotation is about $\bf n_1$ with center at
$C_k$. Since $n_0$ is rotated to coincide with $t$, the rotation angle
is given by $\cos \theta = n_0 \cdot t$ and $\sin \theta = t \cdot n_2$.
The overall matrix transformation is therefore a product of three matrices; 
first the matrix transformation from the canonic basis of the
3D vector space to the basis $(n_0, n_1, n_2)$, the rotation of angle
$\theta$ with center
$(0,0,0)$ around $\bf n_1$ and finaly the matrix transformation from the
basis $(n_0, n_1, n_2)$ to the canonic basis.\\
For the simplification of the mapping calculations, the discrete values 
of the global transformation $M(r_{1k},r_{2k},r_{3k})$ are considered 
as the points for three splines. With these splines we can calculate the 
image of any triplet 
$(r_1,r_2,r_3)$. If $\alpha (r_3)$ is the value of the scalar we will multiply
(also stored in a spline), the image $X(r_1,r_2,r_3)$ is given by\\
\begin{displaymath}
X(r_1,r_2,r_3)=\left\{M(r_1,r_2,r_3)*\left[{\bf S}(r_1,r_2) - {\bf C}(0)\right]\right\}\alpha(r_3)+{\bf C}(r_3)
\end{displaymath}
\\
\\
\underline{\underline{Remark}}\\
At the limit ($\Delta r_3 \rightarrow 0$) corresponding to  the continuous case, the basis $(n_0, n_1, n_2)$ becomes proportional to ${\partial C(r_3)
\over \partial r_3},\, {\partial^2 C(r_3) \over \partial r_3^2}, \,
{\partial C(r_3) \over \partial r_3} \times {\partial^2 C(r_3) \over
\partial r_3^2}$. In fact when $\Delta r_3$ is very small then
\begin{eqnarray*}
n_1 & \approx & {\partial C(r_3) \over \partial r_3} \times {\partial C(r_3 + \Delta r_3) \over \partial r_3}\\
n_1   & \approx & {\partial C(r_3) \over \partial r_3} \times \left(
{\partial C(r_3) \over \partial r_3} + \Delta r_3 {\partial^2 C(r_3)
\over \partial r_3^2} + \cdots \right)\\
    & \approx &  \Delta r_3 {\partial C(r_3) \over \partial r_3}\times
    {\partial^2 C(r_3) \over \partial r_3^2}
\end{eqnarray*}
Here are the description of some functions of the class
\subsection{Constructor}
 
\newlength{\SweepMappingIncludeArgIndent}
\begin{flushleft} \textbf{%
\settowidth{\SweepMappingIncludeArgIndent}{SweepMapping(}% 
SweepMapping(Mapping *sweepmap  = NULL,\\ 
\hspace{\SweepMappingIncludeArgIndent}Mapping *dirsweepmap  = NULL,\\ 
\hspace{\SweepMappingIncludeArgIndent}Mapping *scale  = NULL,\\ 
\hspace{\SweepMappingIncludeArgIndent}const int domainDimension0  =3)
}\end{flushleft}
\begin{description}
\item[{\bf Description:}]  Define a sweep mapping or an extruded mapping.

 Build a mapping defined by a sweep surface or curve
 (a mapping with domainDimension=2 rangeDimension=3 or 
 domainDimension=1, rangeDimension=3) and a sweep
 curve or line (domainDimension=1, rangeDimension=3).

\item[{\bf sweepmap (input) :}]  is the mapping for the sweep surface or curve; default: an
            annulus with inner radius=0 and outer radius=1
\item[{\bf dirsweepmap (input) :}]  The mapping for the sweep curve; default: a half circle
              of radius=4.
\item[{\bf scale (input) :}]  to scale up $(>1)$ or down $(0<s<1)$; default $1$.
 
\item[{\bf Author:}]  Thomas Rutaganira. 
\item[{\bf Changes:}]  WDH + AP
\end{description}
\subsection{SetSweepSurface}
 
\begin{flushleft} \textbf{%
void  \\ 
\settowidth{\SweepMappingIncludeArgIndent}{setSweepSurface(}%
setSweepSurface(Mapping *sweepmap)
}\end{flushleft}
\begin{description}
\item[{\bf Description:}]  Specify the mapping to use as the sweepMap,
               a 3D surface or a 3D curve. If it is a 3D
               surface, the resulting SweepMapping will be a
               3D volume and if it is a 3D curve, the SweepMapping
               will be a 3D surface.
\end{description}
\subsection{setCentering}
 
\begin{flushleft} \textbf{%
int  \\ 
\settowidth{\SweepMappingIncludeArgIndent}{setCentering(}%
setCentering( CenteringOptionsEnum centering )
}\end{flushleft}
\begin{description}
\item[{\bf Description:}]  Specify the centering.
\item[{\bf centering (input) :}]  Specify the manner in which the reference surface should be centered.
   One of {\bf useCenterOfSweepSurface}, {\bf useCenterOfSweepCurve} or {\bf specifiedCenter}.
   See the documentation for further details.
\end{description}
\subsection{setOrientation}
 
\begin{flushleft} \textbf{%
int  \\ 
\settowidth{\SweepMappingIncludeArgIndent}{setOrientation(}%
setOrientation( real orientation\_  =1.)
}\end{flushleft}
\begin{description}
\item[{\bf Description:}]  Specify the orientation of the sweepmapping, +1 or -1.
   When the sweep surface is rotated to align with the sweep curve it may
 face in a forward or reverse direction depending on the orientation. Thus if a
 swept surface appears `inside-out' one should change the orientation.
\end{description}
\subsection{setExtrudeBounds}
 
\begin{flushleft} \textbf{%
int  \\ 
\settowidth{\SweepMappingIncludeArgIndent}{setExtrudeBounds(}%
setExtrudeBounds(real za\_  =0., \\ 
\hspace{\SweepMappingIncludeArgIndent}real zb\_  =1.)
}\end{flushleft}
\begin{description}
\item[{\bf Description:}]  Specify the bounds on an extruded mapping.
\item[{\bf za\_,zb\_ (input) :}]  
\end{description}
\subsection{setStraightLine}
 
\begin{flushleft} \textbf{%
int  \\ 
\settowidth{\SweepMappingIncludeArgIndent}{setStraightLine(}%
setStraightLine(real lx  =0. */, real ly /* =0. */, real lz /* =1.)
}\end{flushleft}
\begin{description}
\item[{\bf Description:}]  Specify the straight line of a tabulated cylinder mapping
\item[{\bf lx,ly,lz (input) :}]  
\end{description}
\subsection{SetSweepCurve}
 
\begin{flushleft} \textbf{%
void  \\ 
\settowidth{\SweepMappingIncludeArgIndent}{setSweepCurve(}%
setSweepCurve(Mapping *dirsweepmap)
}\end{flushleft}
\begin{description}
\item[{\bf Description:}]  Specify the mapping to use as the curve to
               sweep along (a  3D curve).
\end{description}
\subsection{SetScaleSpline}
 
\begin{flushleft} \textbf{%
void  \\ 
\settowidth{\SweepMappingIncludeArgIndent}{setScale(}%
setScale(Mapping *scale)
}\end{flushleft}
\begin{description}
\item[{\bf Description:}]  Specify the mapping to use as the curve to
               sweep along (a  3D curve).
\end{description}
\subsection{setMappingProperties}
 
\begin{flushleft} \textbf{%
int  \\ 
\settowidth{\SweepMappingIncludeArgIndent}{setMappingProperties(}%
setMappingProperties()
}\end{flushleft}
 Access: protected.
\begin{description}
\item[{\bf Description:}]  Initialize the parameters of the
  sweep mapping. 

\end{description}
\subsection{FindRowSplines}
 
\begin{flushleft} \textbf{%
void SweepMapping  \\ 
\settowidth{\SweepMappingIncludeArgIndent}{findRowSplines(}%
findRowSplines(void)
}\end{flushleft}
\begin{description}
\item[{\bf Description:}] 
 This function initializes the splines rowSpline0, 1, 2 that will
 gives the matrix transformation as well as its derivatives for
 the mapping calculations. A point of the spline gives
 a row for the matrix transformation. 
\end{description}
\subsection{map}
 
\begin{flushleft} \textbf{%
void  \\ 
\settowidth{\SweepMappingIncludeArgIndent}{mapS(}%
mapS(const RealArray \& r, RealArray \& x, RealArray \& xr, MappingParameters \& params)
}\end{flushleft}
\begin{description}
\item[{\bf Description:}]  Use the transformations defined by rowSpline0, 
 rowSpline1, and rowSpline2 and the additional scaling mapping 
 to compute the image(s) and/or the derivatives for the parameter
 point(s) defined by $r$.
\end{description}

The following command file generates the geometry for the aortic arch.
{\ttfamily \scriptsize
\labelstyle{\ttfamily}
\keywordstyle{\ttfamily}
\commentstyle{\ttfamily}
\stringstyle{\ttfamily}
\postlisting{\bigbreak}
\inputlisting{fourpipes3.cmd}}
This lead to the following plot.
\begin{figure}
\centerline{\epsfxsize=\textwidth \epsffile{fourpipes.eps}}
\caption{Aortic arch}
\end{figure}
\end{document}
