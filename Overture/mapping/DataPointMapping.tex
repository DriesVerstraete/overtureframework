%--------------------------------------------------------------
\section{DataPointMapping: create a mapping from an array of grid points}
\index{data-point mapping}\index{Mapping!DataPointMapping}\index{Mapping!discrete mapping}\index{Mapping!external}
\index{Mapping!plot3d}
%-------------------------------------------------------------

\subsection{Description}

The {\ff DataPointMapping} can be used to create a mapping from
a set of grid points.  The grid points may be in a file (such as Plot3D
format) or can be in an A++ array.


\newcommand{\rh}{\hat{r}}
\newcommand{\rvh}{\hat{\rv}}

The {\ff DataPointMapping} defines a Mapping transformation by interpolating
the grid points. 

For {\tt orderOfInterpolation=2} the transformation is
defined as a linear interpolation (i.e. 2 points in each direction). In 2D this would
be
\[
   \xv(\rv) = (1-\rh_2) [ (1-\rh_1)\xv_{00} + \rh_1\xv_{10} ]
            +    \rh_2  [ (1-\rh_1)\xv_{01} + \rh_1\xv_{11} ]
\]
where $\xv_{mn}$ are the grid points that define the cell and $\rh_m$
are the relative distances of the point $\rv$ from the r-coordinates
of the corner point $\xv_{00}$
\[
        \rh_1 = { (r_1-r_{00}) \over \Delta r_1 } ~~~~,~~~~
        \rh_2 = { (r_2-r_{00}) \over \Delta r_2 }
\]
The derivatives returned by the Mapping are just the derivatives of the
above expression. 


For {\tt orderOfInterpolation=4} the transformation is
defined as a cubic interpolation (i.e. 4 points in each direction).
In 2D this would be defined as
\begin{align}
   \xv(\rv) & =  \sum_{m=0}^3 q_m(\rh_2) \sum_{n=0}^3 q_n(\rh_1) \xv_{nm}  \\
   q_m(s) & = \prod_{n\ne m} (s-n)/(m-n) \qquad \mbox{Lagrange polynomials,} \quad q_m(n)=\delta_{mn}
\end{align}
The derivatives returned by the Mapping are just the derivatives of the
above expression. 


Figure \ref{fig:usa} shows a grid for part of the coast of the USA, created by
Lotta Olsson. The grid was created with the help of the HYPGEN hyperbolic grid
generator and saved in Plot3D format.
\begin{figure}[h]
  \begin{center}
  \includegraphics[width=10cm]{\figures/usa}
  % \epsfig{file=\figures/usa.ps}
  \caption{A DataPointMapping created from a Plot3D file.}
  \end{center}
\label{fig:usa}
\end{figure}

\subsection{Fast Approximate Inverse}

 Since the DataPointMapping is extensively used, a specialized fast approximate
inverse has been defined for both the second order (bi-linear/tri-linear) and
fourth order representations..

The inverse consists of the steps:
\begin{enumerate}
  \item Find the closest vertex on the grid to the point, $\xv$, to be inverted.
  \item Find the hexahedral that $\xv$ is in.
  \item Invert the local interpolant with Newton's method.
\end{enumerate}


The linear interpolant within a given cell is
\[
   \xv(\rv) = (1-\rh_0) [ (1-\rh_1)\xv_{00} + \rh_1\xv_{01} ]
            +    \rh_0  [ (1-\rh_1)\xv_{10} + \rh_1\xv_{11} ]
\]
in 2D or
\begin{align*}
   \xv(\rv) = (1-\rh_0) &[ (1-\rh_1)((1-\rh_2)\xv_{000}+\rh_2\xv_{001}) + \rh_1((1-\rh_2)\xv_{010}+\rh_2\xv_{011}) ] \\
             +    \rh_0  &[ (1-\rh_1)((1-\rh_2)\xv_{100}+\rh_2\xv_{101}) + \rh_1((1-\rh_2)\xv_{110}+\rh_2\xv_{111}) ]
\end{align*}
in 3D where $\xv_{lmn}$ are the grid points that define the cell and $\rh_m$ are the scaled unit square coordinates,
$\rh_m \in [0,1]$ for points within the cell.


A Newton iteration to invert the mapping would look like
\[
  \xv(\rv_0+\delta\rv) = \xv(\rv_0) + {\partial \xv \over \partial \rv }(\rv_0) \delta \rv.
\]
where
\[
  \left[ \partial \xv \over \partial \rv \right] =
   \begin{bmatrix}
      \av_0 & \av_1 & \av_2
   \end{bmatrix}
\]
where
\begin{align*}
  \av_0 & = (1-\rh_1)((1-\rh_2)(\xv_{100}-\xv_{000})+\rh_2(\xv_{101}-\xv_{001})) 
        + \rh_1((1-\rh_2)(\xv_{110}-\xv_{010})+\rh_2(\xv_{111}-\xv_{011}))      \\
  \av_1 &=(1-\rh_0)((1-\rh_2)(\xv_{010}-\xv_{000})+\rh_2(\xv_{011}-\xv_{001})) 
       + \rh_1((1-\rh_2)(\xv_{110}-\xv_{100})+\rh_2(\xv_{111}-\xv_{101}))       \\
  \av_2 &=  (1-\rh_0)((1-\rh_1)(\xv_{001}-\xv_{000})+\rh_2(\xv_{101}-\xv_{100})) 
       + \rh_1((1-\rh_2)(\xv_{011}-\xv_{010})+\rh_2(\xv_{111}-\xv_{110})) 
\end{align*}

In the special case when the cell is a regular 'diamond' shape the linear interpolant simplifies to
\[
   \xv(\rv)-\xv_{000} = \rh_0 ( \xv_{100}-\xv_{000} ) + \rh_1 ( \xv_{010}-\xv_{000} )+ \rh_2 ( \xv_{001}-\xv_{000} )
\]
or
\[
   \xv(\rv) = \rh_0 \xv_{100} + \rh_1 \xv_{010} + \rh_2 \xv_{001}
\]
where the 'inverse' is computed by solving a $d\times d$ matrix
\[
   \begin{bmatrix} \xv_{100} & \xv_{010} & \xv_{001} \end{bmatrix}
   \begin{bmatrix} \rh_0 \\ \rh_1 \\ \rh_2  \end{bmatrix} =
   \begin{bmatrix} \xv   \end{bmatrix}
\]
** The newton step will do this case exactly**


% If the cell is not regular we can compute an approximate inverse by first computing coordinates
% $\rvh^{lmn}$ for each vertex, assuming a regular diamond shape at the vertex $\xv_{lmn}$ formed
% from the points that connect to the vertex.


\subsection{Examples}


Grid points saved in plot3d format can be read in to form a DataPointMapping.
To illustrate this we first build a plot3d file by creating a mapping and
saving it in plot3d format. This is accomplished by the following
command file ({\tt Overture/sampleMappings/createCylinderPlot3d.cmd})
{\footnotesize
\listinginput[1]{1}{\mapping/createCylinderPlot3d.cmd}
}
There are a variety of options for defining the contents of the plot3d file. For example
it can be a binary file or ascii file, it can contain the {\em iblank} array (i.e. mask array) or not.
These options are available from the {\tt save plot3d file} menu.


A plot3d file can be either read from the DataPointMapping menu or the
{\tt create mappings} menu. Here is a command file to read the plot3d file 
created in the previous step, ({\tt Overture/sampleMappings/readCylinderPlot3d.cmd})
{\footnotesize
\listinginput[1]{1}{\mapping/readCylinderPlot3d.cmd}
}

%% \input DataPointMappingInclude.tex



