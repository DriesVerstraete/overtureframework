\subsection{Constructor}
 
\newlength{\SweepMappingIncludeArgIndent}
\begin{flushleft} \textbf{%
\settowidth{\SweepMappingIncludeArgIndent}{SweepMapping(}% 
SweepMapping(Mapping *sweepmap  = NULL,\\ 
\hspace{\SweepMappingIncludeArgIndent}Mapping *dirsweepmap  = NULL,\\ 
\hspace{\SweepMappingIncludeArgIndent}Mapping *scale  = NULL,\\ 
\hspace{\SweepMappingIncludeArgIndent}const int domainDimension0  =3)
}\end{flushleft}
\begin{description}
\item[{\bf Description:}]  Define a sweep mapping or an extruded mapping.

 Build a mapping defined by a sweep surface or curve
 (a mapping with domainDimension=2 rangeDimension=3 or 
 domainDimension=1, rangeDimension=3) and a sweep
 curve or line (domainDimension=1, rangeDimension=3).

\item[{\bf sweepmap (input) :}]  is the mapping for the sweep surface or curve; default: an
            annulus with inner radius=0 and outer radius=1
\item[{\bf dirsweepmap (input) :}]  The mapping for the sweep curve; default: a half circle
              of radius=4.
\item[{\bf scale (input) :}]  to scale up $(>1)$ or down $(0<s<1)$; default $1$.
 
\item[{\bf Author:}]  Thomas Rutaganira. 
\item[{\bf Changes:}]  WDH + AP
\end{description}
\subsection{SetSweepSurface}
 
\begin{flushleft} \textbf{%
void  \\ 
\settowidth{\SweepMappingIncludeArgIndent}{setSweepSurface(}%
setSweepSurface(Mapping *sweepmap)
}\end{flushleft}
\begin{description}
\item[{\bf Description:}]  Specify the mapping to use as the sweepMap,
               a 3D surface or a 3D curve. If it is a 3D
               surface, the resulting SweepMapping will be a
               3D volume and if it is a 3D curve, the SweepMapping
               will be a 3D surface.
\end{description}
\subsection{setCentering}
 
\begin{flushleft} \textbf{%
int  \\ 
\settowidth{\SweepMappingIncludeArgIndent}{setCentering(}%
setCentering( CenteringOptionsEnum centering )
}\end{flushleft}
\begin{description}
\item[{\bf Description:}]  Specify the centering.
\item[{\bf centering (input) :}]  Specify the manner in which the reference surface should be centered.
   One of {\bf useCenterOfSweepSurface}, {\bf useCenterOfSweepCurve} or {\bf specifiedCenter}.
   See the documentation for further details.
\end{description}
\subsection{setOrientation}
 
\begin{flushleft} \textbf{%
int  \\ 
\settowidth{\SweepMappingIncludeArgIndent}{setOrientation(}%
setOrientation( real orientation\_  =1.)
}\end{flushleft}
\begin{description}
\item[{\bf Description:}]  Specify the orientation of the sweepmapping, +1 or -1.
   When the sweep surface is rotated to align with the sweep curve it may
 face in a forward or reverse direction depending on the orientation. Thus if a
 swept surface appears `inside-out' one should change the orientation.
\end{description}
\subsection{setExtrudeBounds}
 
\begin{flushleft} \textbf{%
int  \\ 
\settowidth{\SweepMappingIncludeArgIndent}{setExtrudeBounds(}%
setExtrudeBounds(real za\_  =0., \\ 
\hspace{\SweepMappingIncludeArgIndent}real zb\_  =1.)
}\end{flushleft}
\begin{description}
\item[{\bf Description:}]  Specify the bounds on an extruded mapping.
\item[{\bf za\_,zb\_ (input) :}]  
\end{description}
\subsection{setStraightLine}
 
\begin{flushleft} \textbf{%
int  \\ 
\settowidth{\SweepMappingIncludeArgIndent}{setStraightLine(}%
setStraightLine(real lx  =0. */, real ly /* =0. */, real lz /* =1.)
}\end{flushleft}
\begin{description}
\item[{\bf Description:}]  Specify the straight line of a tabulated cylinder mapping
\item[{\bf lx,ly,lz (input) :}]  
\end{description}
\subsection{SetSweepCurve}
 
\begin{flushleft} \textbf{%
void  \\ 
\settowidth{\SweepMappingIncludeArgIndent}{setSweepCurve(}%
setSweepCurve(Mapping *dirsweepmap)
}\end{flushleft}
\begin{description}
\item[{\bf Description:}]  Specify the mapping to use as the curve to
               sweep along (a  3D curve).
\end{description}
\subsection{SetScaleSpline}
 
\begin{flushleft} \textbf{%
void  \\ 
\settowidth{\SweepMappingIncludeArgIndent}{setScale(}%
setScale(Mapping *scale)
}\end{flushleft}
\begin{description}
\item[{\bf Description:}]  Specify the mapping to use as the curve to
               sweep along (a  3D curve).
\end{description}
\subsection{setMappingProperties}
 
\begin{flushleft} \textbf{%
int  \\ 
\settowidth{\SweepMappingIncludeArgIndent}{setMappingProperties(}%
setMappingProperties()
}\end{flushleft}
 Access: protected.
\begin{description}
\item[{\bf Description:}]  Initialize the parameters of the
  sweep mapping. 

\end{description}
\subsection{FindRowSplines}
 
\begin{flushleft} \textbf{%
void SweepMapping  \\ 
\settowidth{\SweepMappingIncludeArgIndent}{findRowSplines(}%
findRowSplines(void)
}\end{flushleft}
\begin{description}
\item[{\bf Description:}] 
 This function initializes the splines rowSpline0, 1, 2 that will
 gives the matrix transformation as well as its derivatives for
 the mapping calculations. A point of the spline gives
 a row for the matrix transformation. 
\end{description}
\subsection{map}
 
\begin{flushleft} \textbf{%
void  \\ 
\settowidth{\SweepMappingIncludeArgIndent}{mapS(}%
mapS(const RealArray \& r, RealArray \& x, RealArray \& xr, MappingParameters \& params)
}\end{flushleft}
\begin{description}
\item[{\bf Description:}]  Use the transformations defined by rowSpline0, 
 rowSpline1, and rowSpline2 and the additional scaling mapping 
 to compute the image(s) and/or the derivatives for the parameter
 point(s) defined by $r$.
\end{description}
