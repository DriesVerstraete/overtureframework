\newcommand{\egr}{EllipticTransform.C}
\def\Grad{\mathop{\nabla}}
\def\ParS#1#2{\mathord{\frac{\partial^2{#1}}{\partial{#2}^2}}}
\def\vel{\mathord{\mathbf{V}}}
\def\Par#1#2{\mathord{\frac{\partial{#1}}{\partial{#2}}}}
\def\ParC#1#2#3{\mathord{\frac{\partial^2{#1}}{\partial{#2}\partial{#3}}}}
\def\velt{\mathord{\mathbf{V}_{\! \mathnormal{t}}}}
\def\bx{\mathord{\mathbf{x}}}
\def\Lap{\mathop{{\nabla}^2}}
\def\TrInt{\int\!\int\!\int}
\def\DbInt{\int\!\int}
\def\DA{{d}{\bf A}}
\def\velnp{\mathord{\mathbf{V}}^{n+1}}
\def\E{\mathord{\mathbf{E}}}
\def\Ei{\mathord{\mathbf{E}^{-1}}}
\def\M{\mathord{\mathbf{M}}}
\def\Mi{\mathord{\mathbf{M}^{-1}}}
\def\D{\mathord{\mathbf{D}}}
\def\xxi{\bx_\xi}
\def\xeta{\bx_\eta}
\def\xxixi{\bx_{\xi\xi}}
\def\xetaeta{\bx_{\eta\eta}}

\section{EllipticTransform}\index{elliptic mapping}\index{Mapping!EllipticTransform}
  This Mapping was originally created by Eugene Sy. Changes made by Bill Henshaw.

 **** This documentation is out of date as I have made lots of changes ****


\subsection{Introduction}

The program {\egr} performs smoothing on a desired mapping by solving a Poisson equation
on the domain.  A new elliptic mapping is created from the original mapping which is
supplied as an initial condition.

The original mapping $\M$ maps the unit square $U$ into physical space $\Omega_p$.
With the elliptic transform, a one to one function $E'$ taking $U$ into 
$\Omega_p$ is first created.  Then,
$\Mi$ is applied to the solution in $\Omega_p$ to map the points back into the unit
square.  In this way, a mapping $\D=\Mi(E')$ is obtained, where $\D:U\mapsto U$, and 
$\D$ is a data point mapping.  The elliptic mapping $\E$ is then given by a 
composition, where

\begin{eqnarray*}
\E(U) = \M(\D(U))
\end{eqnarray*}

\subsection{The Governing Equations}

The focus from here on will be on the creation of the mapping $\E$.  Assume
that $U$ has coordinate directions $\xi$ and $\eta$.  The equations to 
be solved are:

\begin{eqnarray*}
\Lap\xi=P(\xi,\eta) \\
\Lap\eta=Q(\xi,\eta)
\end{eqnarray*}
These are transformed to computational space and take the form:

\begin{eqnarray*}
\alpha(\ParS{x}{\xi}+P\Par{x}{\xi})+\beta(\ParS{x}{\eta}+Q\Par{x}{\eta})-
2\gamma\ParC{x}{\xi}{\eta} &=& 0	\\
\alpha(\ParS{y}{\xi}+P\Par{y}{\xi})+\beta(\ParS{y}{\eta}+Q\Par{y}{\eta})-
2\gamma\ParC{y}{\xi}{\eta} &=& 0        \\
\end{eqnarray*}
\begin{eqnarray*}
\alpha &=& (\Par{x}{\eta})^2 + (\Par{y}{\eta})^2	\\
\beta &=& (\Par{x}{\xi})^2 + (\Par{y}{\xi})^2	\\
\gamma &=& \Par{x}{\xi}\Par{x}{\eta} + \Par{y}{\xi}\Par{y}{\eta}    \\
\end{eqnarray*}
These equations are solved on the 
unit square $U$ using finite difference methods.

\subsection{Control of the Boundary}

The existing code has the ability to evaluate Dirichlet, orthogonal, and periodic
boundary conditions for solving the above equations.

\subsubsection{Dirichlet Conditions}
If the boundary conditions are Dirichlet
the locations of the boundary points must be 
correctly specified in the initial condition.  These points are considered
by the elliptic grid generator  
to be fixed, and computations are done only on the interior 
of the grid.

\subsubsection{Orthogonal Boundary Conditions}
If the user desires that the gridlines meet the boundary orthogonally, then
the points on the boundary are given a degree of freedom, and are allowed to
move along the boundary.  For example, consider the boundary $\xi=0$.
Assume that the $(n+1)$st iteration is being computed, and let $j$ represent
the $\eta$ coordinate.  
The following two equations are solved to give the location of the boundary points.

\begin{eqnarray}
{\bx}_{\eta}\cdot{\bx}_{\xi} &=& 0      \\
\bx^{n+1} &=& \bx^n + (\Delta\bx\cdot{\hat{\bx}}_{\eta}){\hat{\bx}}_{\eta}
\end{eqnarray}

Equation 1 is the orthogonality condition, and equation 2 prevents the boundary 
points from moving off the boundary.  $\Delta\bx$ represents the change in 
position of the point $\bx$ dictated from equation 1.  In other words, if $\bx^n$
is the location of the point at the $n$th iteration step, equation 1 will steer 
the point towards a position $\bx^*$ for the $(n+1)$st iterate.  It follows that
\begin{eqnarray*}
\Delta\bx&=&\bx^*-\bx^n\\
\Delta\bx\cdot{\hat{\bx}}_{\eta}&=&{\bx}^{*}\cdot{\hat{\bx}}_{\eta}-
{\bx}^{n}\cdot{\hat{\bx}}_{\eta}
\end{eqnarray*}
%\def\mimo{\mathord{\frac{\mbox{mass out}}{\mbox{mass in}}}}

To solve the equations, ${\bx}_{\xi}$ in equation 1 is first forward differenced 
to obtain the following result.
\begin{eqnarray}
{x^*}_{0,j}x_{\eta} &=& y_{\xi}y_{\eta}+x_{1,j}x_{\eta}\\
{y^*}_{0,j}y_{\eta} &=& x_{\xi}x_{\eta}+y_{1,j}y_{\eta}
\end{eqnarray}
This gives an expression for ${\bx}^{*}\cdot{\bx}_{\eta}$, where
$\bx^*$ represents the position the boundary point wants to move to
in order to satisfy the orthogonality condition.  $\bx_{\eta}$ and
all the terms on the right hand side
are calculated from the solution at the $n$th iterate.
Normalizing by $\|{\bx}_{\eta}\|$
yields the following:
\begin{eqnarray}
{x^*}_{0,j}{\hat{x}}_{\eta} &=& y_{\xi}{\hat{y}}_{\eta}+x_{1,j}{\hat{x}}_{\eta}\\
{y^*}_{0,j}{\hat{y}}_{\eta} &=& x_{\xi}{\hat{x}}_{\eta}+y_{1,j}{\hat{y}}_{\eta}
\end{eqnarray}
This is ${\bx}^{*}\cdot{\hat{\bx}}_{\eta}$.  Then, since 
${\bx}^{n}\cdot{\hat{\bx}}_{\eta}$ is also readily calculated from the $n$th iterate,
the entire right hand side of equation 2 is known.  This allows calculation of 
${\bx}^{n+1}$, and the continuation of the iteration.

Orthogonality at the boundary can also be obtained by a different means.  As explained in
Thompson, et al. [3], the boundary points can be fixed, and boundary orthogonality 
enforced
by utilization of the proper forcing functions $P$ and $Q$.  In addition, the 
thickness of the boundary layer can be specified by the user.  

Again, let $\xi=0$, as above, and let $j$ represent the $\eta$ coordinate.
Assume that the boundary layer thickness $\|\bx_{\xi}\|$ is chosen,
and calculate ${\bx_{\eta}}$ and ${\bx_{\eta\eta}}$ from
the fixed locations of the boundary points.  
All this, together with the orthogonality condition
\begin{eqnarray*}
{\bx}_{\eta}\cdot{\bx}_{\xi} = 0    
\end{eqnarray*}
allows for the determination of $\bx_{\xi}$.  As shown in Knupp and Steinberg[2].  
$\bx_{\xi}$ is given by:
\begin{eqnarray}
\bx_{\xi}=\frac{\|\bx_{\xi}\|}{\|\bx_{\eta}\|}{\bx_{\eta}}^{\perp}
\end{eqnarray}
Here, ${\bx_{\eta}}^{\perp}$ is the vector perpendicular to the vector ${\bx_{\eta}}$.  
In addition to this, the method requires $\bx_{\xi\xi}$, and this is calculated by means of the Pade
approximation 
\begin{eqnarray}
\bx_{\xi\xi}|_0=\frac{-7\bx_{1,j}+8\bx_{1,j}-\bx_{3,j}}{2\Delta\eta^2}-3\frac{\bx_{\xi}|_0}{\Delta\eta}
\end{eqnarray}
Using this information, the proper forcing functions $P$ and $Q$ can be determined.
\begin{eqnarray*}
P(\xi,\eta)&=&-\frac{\xxi\cdot\xxixi}{\|\xxi\|^2}-\frac{\xxi\cdot\xetaeta}{\|\xeta\|^2} \\
Q(\xi,\eta)&=&-\frac{\xeta\cdot\xetaeta}{\|\xeta\|^2}-\frac{\xeta\cdot\xxixi}{\|\xxi\|^2}
\end{eqnarray*}
Once the appropriate $P$ and $Q$ are determined for the boundary, the values are interpolated onto
the interior points using a linear scaling, and the iteration is continued.

Note that because second order differences are being lagged, heavy underrelaxation is 
required for this scheme to converge (Knupp and Steinberg [2]).  

\subsubsection{Periodic Boundaries}

Two types of periodic boundaries exist.  The first is derivative periodic and the second
is function periodic.  Derivative periodicity involves identical derivatives on the 
boundary, but not necessarily identical positions.  Function periodicity involves
matching both the derivatives and the positions at the boundary points (as in the case of an
annulus).

In either case, the values beyond the boundary are evaluated by means of ghost points.   
A ghost array allows for calculation of values on the boundary in much the same way they
are found on the interior.

\subsection{Sources}

Should clustering of points or lines be necessary in the interior, certain points or lines
may be designated as being lines of attraction.  This involves manipulation of the source 
terms $P$ and $Q$ before iteration begins.  If a coordinate line is to be made a
line of attraction,
the attraction power $\pi$ must be specified along with the diffusivity
$\delta$.  With these two parameters, the following expressions
for $P$ and $Q$ are evaluated at all points in the field.
\begin{eqnarray*}
P_{line}(\xi,\eta) &=& -\sum_{i=1}^{N}\pi_{i}Sign(\xi-\xi_i)e^{-\delta_i|\xi-\xi_i|} \\
Q_{line}(\xi,\eta) &=& -\sum_{j=1}^{M}\pi_{j}Sign(\eta-\eta_j)e^{-\delta_j|\eta-\eta_j|}
\end{eqnarray*}
Here, $N$ and $M$ represent the number of $\xi$ and $\eta$ lines of attraction respectively.
Should points of attraction be desired, power and diffusivity are specified
for each point, and two more sums are evaluated.
\begin{eqnarray*}
P_{point}(\xi,\eta) &=& -\sum_{i=1}^{L}\pi_{i}Sign(\xi-\xi_i)e^{-\delta_i|\xi-\xi_i|} \\
Q_{point}(\xi,\eta) &=& -\sum_{j=1}^{L}\pi_{j}Sign(\eta-\eta_j)e^{-\delta_j|\eta-\eta_j|}
\end{eqnarray*}
Here, L is the number of point sources.  Note that a source can be made into a sink by
merely changing the sign in front of the power $\pi$.

\subsection{Using the Elliptic Grid Generator With Ogen}

The user is assumed to be familiar with generation of mappings in Ogen.  A mapping
must first be made as an initial condition for the elliptic smoother.

\subsubsection{Grid Dimensions}
The first thing to specify after choosing a mapping is the amount of grid refinement desired.
The number of grid lines in i and j (or $\xi$ and $\eta$ respectively) should be 
entered before all else, as these parameters are needed for correct implementation of
the boundary conditions.

\subsubsection{Boundary Conditions}
The appropriate GRID boundary conditions should be entered next.
These are not to be confused with the boundary conditions for the physical problem
to be solved later, and the switches are completely independent.  The following
choices are available:

\begin{itemize}
 \item -1:  Refers to a {\bf periodic} boundary condition.  This
  is selected by default if a periodic boundary is declared when the original mapping
  is made.  If a 
  periodic boundary is not specified when the original mapping was created, this boundary
  condition should not be used.
 \item  1:  Refers to a {\bf Dirichlet} boundary condition.  The positions of the boundary 
  points in the original mapping are used as the boundary condition for the elliptic map.
 \item  2:  Refers to a {\bf orthogonal} boundary condition.  This forces the gridlines to  
  meet the boundary in an orthogonal fashion.  The boundary points are free to move, but
  only along the boundary.
 \item  3:  Refers to the {\bf combined} boundary condition.  The boundary points are fixed
  as in the Dirichlet case, but the sources and sinks are modified so as to guarantee 
  orthogonality at the boundary.  This boundary condition requires that the user 
  specify the thickness of the boundary layer.
\end{itemize}

The boundary conditions are stored in a 2-dimensional array {\bf gridBc(i,j)}, where
i and j range from 0 to 1.  On the unit square, the following are the locations of the 
boundaries:

\begin{itemize}
 \item gridBc(0,0):  Refers to the boundary condition on $0\leq x \leq1, y=0$.
 \item gridBc(0,1):  Refers to the boundary condition on $x=0, 0\leq y \leq1$.
 \item gridBc(1,0):  Refers to the boundary condition on $0\leq x \leq1, y=1$.
 \item gridBc(1,1):  Refers to the boundary condition on $x=1, 0\leq y \leq1$.
\end{itemize}
 
\subsubsection{Sources and Sinks}

If the user decides that lines or points of attraction (repulsion) are desired for 
the elliptic grid, these can be specified.  The selection \emph{Poisson i-line sources} 
creates lines of constant $\xi$ into lines of attraction.  Similarly, \emph{Poisson
j-line sources} turns lines of constant $\eta$ into lines of attraction, and 
\emph{Poisson point sources} creates points of attraction in $\xi-\eta$ space.

The power and diffusivity of each source must be selected carefully.  Too much power or 
too little diffusivity can mar convergence of the grid.

\subsubsection{Other Functions}

There are several other switches that can be used.  These are:

\begin{itemize}
 \item \emph{change SOR parameter}:  This changes the value of $\omega$ used for either 
  overrelaxation or underrelaxation.  Setting $\omega=1$ indicates a point Gauss-
  Seidel method, and is a good conservative first choice.  Note that if {\bf combined}
  boundary conditions are used, then $\omega$ needs to be quite small, usually on
  the order of 0.05 or 0.1.  This is because of the lagged second order quantities
  that the scheme requires.
 \item \emph{set maximum number of iterations}:  This controls the maximum amount of iterations
  for the grid to converge.
 \item \emph{set epsilon for convergence}:  This value is the indicator for the convergence
  of the elliptic grid.  If the differences between successive iterations drops below
  epsilon, then the grid is said to have converged.  This 
  is preset to $10^{-5}$.
 \item \emph{set number of periods}:  If the grid is periodic (either derivative or function),
  then any sources or sinks in the field need to be made periodic as well.  The number 
  entered here indicates the number of periods desired for these sources and sinks.
  1 is the default value.  
  Increasing this number is crucial if strong source terms exist, and 5 or 7 may 
  be needed to properly resolve the periodicity.
 \end{itemize}

\subsection{In Conclusion}

The elliptic transform is a fine way to smooth out a grid, and 
frees the user from the restriction to simple geometries.
Unfortunately, convergence of the routine is
not guaranteed for all geometries, and for all powers of sources and sinks.  

The routine, however, does provide an ability to deal with boundaries well, and 
can greatly simplify many complex computations.



%% \subsection{Member functions}
%% \input EllipticTransformInclude.tex


\subsection{Examples}

\subsubsection{Smoothed out diamond airfoil}

In the left column is the command file that was used to generate the grid on the bottom right.


\noindent
\begin{minipage}{.4\linewidth}
{\footnotesize
\listinginput[1]{1}{\mapping/ellaf.cmd}
}
\end{minipage}\hfill
\begin{minipage}{.6\linewidth}
  \begin{center}
   \includegraphics[width=8cm]{\figures/ellaf1}\\
  % \epsfig{file=\figures/ellaf1.ps,height=3.in}  \\
  {Diamond airfoil before elliptic transform.}    \\
   \includegraphics[width=8cm]{\figures/ellaf2}\\
   % \epsfig{file=\figures/ellaf2.ps,height=3.in}  \\
  {Diamond airfoil after elliptic transform.}
  \end{center}
\end{minipage}




 
% \end{document}



