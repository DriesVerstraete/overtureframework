%--------------------------------------------------------------
\section{ComposeMapping: compose two mappings}\index{compose mapping}\index{Mapping!ComposeMapping}
%-------------------------------------------------------------

This mapping can be used to create a new mapping by composing
two existing mappings. The Mapping's {\tt MatrixTransform}, {\tt StretchTransform},
{\tt ReparameterizationTransform}, abnd {\tt EllipticTransform} are all derived
from the ComposeMapping.

% \subsection{Constructors}
% 
% \begin{tabbing}
% {\ff Mapping()xxxxxxxxxxxxxxxxxxxxxxxxxxxxxxxxxx123456}\= \kill
% {\ff Mapping()}    \> Default constructor\\
% {\ff Mapping( Mapping \& mapa, Mapping \& mapb)}  \> 
%          create a mapping, {\ff mapb} $\circ$ {\ff mapa } \\
% \end{tabbing}
% 
% 
% \subsection{Member Functions}
% 
% \begin{tabbing}
% 0123456789012345678901234567689012345678901234567890123 \= \kill
% {\ff void map( realArray \& r, realArray \& x, realArray \& xr )} 
%                \> evaluate the mapping and derivative  \\
% {\ff void inverseMap( realArray \& x, realArray \& r, realArray \& rx )} 
%                \> evaluate the inverse mapping and derivative  \\
% {\ff void get( const Dir \& dir, const String \& name)} \> get from a database file \\
% {\ff void put( const Dir \& dir, const String \& name)} \> put to a database file \\
% \end{tabbing}
% 
% \noindent
Here is an example of the use of the {\ff ComposeMapping} class.
The composed mapping consists of a mapping for a cube followed by
a rotation mapping.
{\footnotesize
\begin{verbatim}
#include "maputil.h"

void main()
{
  BoxMapping box(0.,.5,0.,.5,0.,.5)  ;           // Define grid to be a cube

  MatrixMapping rotation  ;                      // Define a matrix mapping 
  rotation.rotate( zAxis, Pi/2. );               // rotate about z axis
  
  ComposeMapping rotatedBox( box,rotation );     // define a mapping by composition

  r(axis1)=.5; r(axis2)=.5; r(axis3)=.5;
  rotatedBox.map( r,x,xr );                      // evaluate the mapping
}
\end{verbatim}
}
\noindent

