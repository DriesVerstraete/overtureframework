\subsection{Default Constructor}
 
\newlength{\ReparameterizationTransformIncludeArgIndent}
\begin{flushleft} \textbf{%
\settowidth{\ReparameterizationTransformIncludeArgIndent}{ReparameterizationTransform(}% 
ReparameterizationTransform() 
}\end{flushleft}
\begin{description}
\item[{\bf Purpose:}]  Default Constructor
    The {\tt ReparameterizationTransform} can reparameterize a given Mapping
 in one of the following ways:
   \begin{description}
      \item[orthographic:] Remove a polar singularity by using a orthographic projection
         to define a new patch over the singularity.
      \item[restriction:] restrict the parameter space to a sub-rectangle of the
          original parameter space. Use this, for example, to define a refined patch in an
          adaptive grid.
      \item[equidistribution:] reparameterize a curve in 2D or 3D so as to equi-distribute
         a weighted sum of arclength and curvature.   
   \end{description}   
\end{description}
\subsection{Constructor(Mapping,ReparameterizationTypes)}
 
\begin{flushleft} \textbf{%
\settowidth{\ReparameterizationTransformIncludeArgIndent}{ReparameterizationTransform(}% 
ReparameterizationTransform(Mapping \& map, \\ 
\hspace{\ReparameterizationTransformIncludeArgIndent}const ReparameterizationTypes type  = defaultReparameterization) 
}\end{flushleft}
\begin{description}
\item[{\bf Description:}]  Constructor for a Reparameterization. 
\item[{\bf map (input) :}]  mapping to reparameterize.
\item[{\bf type (input) :}]  
\end{description}
\subsection{Constructor(MappingRC,ReparameterizationTypes)}
 
\begin{flushleft} \textbf{%
\settowidth{\ReparameterizationTransformIncludeArgIndent}{ReparameterizationTransform(}% 
ReparameterizationTransform(MappingRC \& mapRC, \\ 
\hspace{\ReparameterizationTransformIncludeArgIndent}const ReparameterizationTypes type  = defaultReparameterization) 
}\end{flushleft}
\begin{description}
\item[{\bf Description:}]  Constructor for a Reparameterization. 
    See the comments in the constructor member function
\end{description}
\subsection{constructor(MappingRC,ReparameterizationTypes)}
 
\begin{flushleft} \textbf{%
void  \\ 
\settowidth{\ReparameterizationTransformIncludeArgIndent}{constructor(}%
constructor(Mapping \& map, const ReparameterizationTypes type)
}\end{flushleft}
\begin{description}
\item[{\bf Description:}]  This is a protected routine, used internally.
    Constructor for a Reparameterization. This constructor will
   check to see if you are trying to reparameterize a Mapping that is already
   the same type of reparameterization of another mapping. For example you may
   be making a sub-mapping (restriction) of a sub-mapping. In this case this
   constructor will eliminate the multiple restriction operations and replace
   it by a single restriction. You should then use the scaleBounds member function
   to define a new restriction. This function will scale the bounds found in map.
\end{description}
\subsection{constructorForMultipleReparams}
 
\begin{flushleft} \textbf{%
void  \\ 
\settowidth{\ReparameterizationTransformIncludeArgIndent}{constructorForMultipleReparams(}%
constructorForMultipleReparams(ReparameterizationTransform \& rtMap )
}\end{flushleft}
\begin{description}
\item[{\bf Description:}]  **This is a protected routine**
   If you want to reparameterize a mapping that is already Reparameterized then use this
   constructor.  It will replace multiple reparams of the same type with just one reparam
\item[{\bf Notes:}] 
   
\end{description}
\subsection{scaleBound}
 
\begin{flushleft} \textbf{%
int  \\ 
\settowidth{\ReparameterizationTransformIncludeArgIndent}{scaleBounds(}%
scaleBounds(const real ra =0.,\\ 
\hspace{\ReparameterizationTransformIncludeArgIndent}const real rb =1., \\ 
\hspace{\ReparameterizationTransformIncludeArgIndent}const real sa =0.,\\ 
\hspace{\ReparameterizationTransformIncludeArgIndent}const real sb =1.,\\ 
\hspace{\ReparameterizationTransformIncludeArgIndent}const real ta =0.,\\ 
\hspace{\ReparameterizationTransformIncludeArgIndent}const real tb   =1.)
}\end{flushleft}
\begin{description}
\item[{\bf Description:}]  
    Scale the current bounds for a restriction Mapping. See the documentation for the
   {\tt RestrictionMapping} for further details.
\item[{\bf ra,rb,sa,sb,ta,tb (input):}]  
\end{description}
\subsection{getBounds}
 
\begin{flushleft} \textbf{%
int  \\ 
\settowidth{\ReparameterizationTransformIncludeArgIndent}{getBounds(}%
getBounds(real \& ra, real \& rb, real \& sa, real \& sb, real \& ta, real \& tb ) const
}\end{flushleft}
\begin{description}
\item[{\bf Description:}] 
  Get the bounds for a restriction mapping.
   {\tt RestrictionMapping} for further details.
\item[{\bf ra,rb,sa,sb,ta,tb (output):}]  
\end{description}
\subsection{setBounds}
 
\begin{flushleft} \textbf{%
int  \\ 
\settowidth{\ReparameterizationTransformIncludeArgIndent}{setBounds(}%
setBounds(const real ra =0., \\ 
\hspace{\ReparameterizationTransformIncludeArgIndent}const real rb =1., \\ 
\hspace{\ReparameterizationTransformIncludeArgIndent}const real sa =0.,\\ 
\hspace{\ReparameterizationTransformIncludeArgIndent}const real sb =1.,\\ 
\hspace{\ReparameterizationTransformIncludeArgIndent}const real ta =0.,\\ 
\hspace{\ReparameterizationTransformIncludeArgIndent}const real tb   =1.)
}\end{flushleft}
\begin{description}
\item[{\bf Description:}] 
  Set absolute bounds. See the documentation for the
   {\tt RestrictionMapping} for further details.
\item[{\bf ra,rb,sa,sb,ta,tb (input):}]  
\end{description}
\subsubsection{getBoundsForMulitpleReparameterizations}
 
\begin{flushleft} \textbf{%
int  \\ 
\settowidth{\ReparameterizationTransformIncludeArgIndent}{getBoundsForMultipleReparameterizations(}%
getBoundsForMultipleReparameterizations( real mrBounds[6] ) const
}\end{flushleft}
\begin{description}
\item[{\bf Description:}] 
  Get the bounds for multiple reparameterizations. This routine will usually only be
 called by the Grid class.
\item[{\bf mrBounds (output):}]  
\end{description}
\subsubsection{setBoundsForMulitpleReparameterizations}
 
\begin{flushleft} \textbf{%
int  \\ 
\settowidth{\ReparameterizationTransformIncludeArgIndent}{setBoundsForMultipleReparameterizations(}%
setBoundsForMultipleReparameterizations( real mrBounds[6] )
}\end{flushleft}
\begin{description}
\item[{\bf Description:}] 
  Set the bounds for multiple reparameterizations. This routine will usually only be
 called by the Grid class.
\item[{\bf mrBounds (input):}]  
\end{description}
\subsubsection{setEquidistributionParameters}
 
\begin{flushleft} \textbf{%
int  \\ 
\settowidth{\ReparameterizationTransformIncludeArgIndent}{setEquidistributionParameters(}%
setEquidistributionParameters(const real \& arcLengthWeight\_ /* =1.*/, \\ 
\hspace{\ReparameterizationTransformIncludeArgIndent}const real \& curvatureWeight\_ /* =0.*/,\\ 
\hspace{\ReparameterizationTransformIncludeArgIndent}const int \& numberOfSmooths  = 3)
}\end{flushleft}
\begin{description}
\item[{\bf Description:}] 
   Set the `arclength' parameterization parameters. The parameterization is chosen to
 redistribute the points to resolve the arclength and/or the curvature of the curve.
 By default the curve is parameterized by arclength only. To resolve regions of high
 curvature choose the recommended values of {\tt arcLengthWeight\_=1.} and
  {\tt curvatureWeight\_=1.}.

  To determine the parameterization we equidistribute the weight function 
  \[
     w(r) =      {\rm arcLengthWeight} {s(r)\over |s|_\infty}  
               + {\rm curvatureWeight} {c(r)\over |c|_\infty}
  \]
  where $s(r)$ is the local arclength and $c(r)$ is the curvature. Note that we normalize
 $s$ and $c$ by their maximum values.
  
 \[
      c = |x_{ss}| = { | x_{rr} | \over  |x_r|^2 }
 \]
 
\item[{\bf arcLengthWeight\_ (input):}]  A weight for arclength. A negative value may give undefined results.
\item[{\bf curvatureWeight\_ (input):}]  A weight for curvature. A negative value may give undefined results.
\item[{\bf numberOfSmooths (input):}]  Number of times to smooth the equidistribution weight function.

\end{description}
