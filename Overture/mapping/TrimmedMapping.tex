%--------------------------------------------------------------
\section{TrimmedMapping: define a trimmed surface in 3D }
\index{trimmed mapping}\index{Mapping!TrimmedMapping}\index{Nurbs!trimmed}
%-------------------------------------------------------------

\subsection{Description}

  A trimmed surface consists of a standard (i.e. logically rectangular) mapping ("surface") 
which has regions removed from it. The portions removed are defined by curves in the
parameter space of the mapping. The IN-ACTIVE part of the trimmed surface is any point
that is outside the outer boundary or inside any of the inner curves. Thus one inner
curve cannot be inside another inner curve. None of the trimming curves are allowed
to intersect each other.
  
Here is how you should evaluate a trimmed mapping (accessing the mask array to indicate
whether the point is inside or outside):
{\footnotesize
\begin{verbatim}
 TrimmedMapping trim;
 ... assign the TrimmedMapping somehow ...
 RealArray r(10,2), x(10,3), xr(10,3,2);
 ... assign r ....
 MappingParameters params;   // we need to pass this option argument to "map"         
 trim.map(r,x,xr,params);
 IntegerArray & mask = params.mask;  //  mask(i) = 0 if point is outside, =1 if inside
 for( int i=0; i<9; i++; )
 {
   if( mask(i)==0 )
     // point is outside, x(i,0:2) are the coordinates of the untrimmed surface at r(i,0:1)
   else
     // point is inside, x(i,0:2) are the coordinates of the trimmed surface at r(i,0:1)
 }
\end{verbatim}
}


\begin{figure}[ht]
  \begin{center}
    \includegraphics[width=9cm]{\figures/trim1}
    % \epsfig{file=\figures/trim1.ps,width=4in}
  \caption{A trimmed mapping with an outer trimming curve and 2 inner trimming curves.
     To plot the mapping we project points which are just outside the trimmed region onto the boundary}
  \end{center}
\label{fig:TrimmedMapping1}
\end{figure}

\begin{figure}[ht]
  \begin{center}
    \includegraphics[width=9cm]{\figures/trim2}
    % \epsfig{file=\figures/trim2.ps,width=4in}
  \caption{A trimmed mapping with an outer trimming curve and 1 inner trimming curve}
  \end{center}
\label{fig:TrimmedMapping2}
\end{figure}

%% \input TrimmedMappingInclude.tex
