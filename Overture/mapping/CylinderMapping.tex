\section{CylinderMapping}\index{cylinder mapping}\index{Mapping!CylinderMapping}

This mapping defines a cylindrical volume or surface in three-dimensions.
\begin{align}
 \theta &=  2\pi( \theta_0 + r_0( \theta_1-\theta_0) ) ,\\
 R(r_1) &= (R_0 + r_2 (R_1-R_0))  , \\
 \xv(r_0,r_1,r_2) &= ( R \cos(\theta ) + x_0 , R\sin(\theta) + y_0 , s_0 + r_1(s_1-s_0) + z_0 ) 
\end{align}
The above cylinder has the z-axis as the axial direction. It is also possible to to have the
axial direction to point in any of the coordinate direction using the 
({\tt cylAxis1}, {\tt cylAxis2}, {\tt cylAxis3}) variables (which should be a permutation of (0,1,2)):
Changing these variables will permute the definition of $(x_0,x_1,x_2)$: 
\begin{align}
    (x_{\mathtt cylAxis1},x_{\mathtt cylAxis2},x_{\mathtt cylAxis3}) = ( R \cos(\theta ) 
             + x_0 , R\sin(\theta) + y_0 , s_0 + r_2(s_1-s_0) + z_0 )
\end{align}
NOTE that the parameter space coordinates are always $(\theta,\rm{axial},\rm{radial})$. 

\begin{figure}[hbt]
\newcommand{\figWidth}{10cm}
\newcommand{\trimfig}[2]{\trimFig{#1}{#2}{0.05}{.025}{.05}{.05}}
\begin{center}\small
% ------------------------------------------------------------------------------------------------
\begin{tikzpicture}
  \useasboundingbox (0,0.25) rectangle (10,10);  % set the bounding box (so we have less surrounding white space)
% 
  \draw (0, 0) node[anchor=south west,xshift=-4pt,yshift=-4pt] {\trimfig{\figures/cylinder}{\figWidth}};
% grid:
%  \draw[step=1cm,gray] (0,0) grid (10,10);
\end{tikzpicture}
% ----------------------------------------------------------------------------------------
 \caption{CylinderMapping. This is the volume representation. The cylinder may also represent a surface. }
\label{fig:CylinderMapping}
\end{center}
\end{figure}


%- \begin{figure}[ht]
%-   \begin{center}
%-   \includegraphics[width=14cm]{\figures/CylinderMapping_idraw}
%-   % \epsfig{file=\figures/CylinderMapping.idraw.ps,width=14cm}
%-   \caption{The CylinderMapping defines a cylinder in three-dimensions.}
%-   \end{center}
%- \label{fig:CylinderMapping}
%- \end{figure}
%- 
%- 
%- %% \input CylinderMappingInclude.tex
%- 
%- 
%- \begin{figure}[htb]
%- \centering
%-   \includegraphics[width=10cm]{\figures/cylinder}
%-   % \epsfig{file=\figures/cylinder.ps,width=.5\textwidth}
%-  \caption{CylinderMapping. This is the volume representation. The cylinder may
%-    also represent a surface. }
%- \end{figure}
