%--------------------------------------------------------------
\section{NurbsMapping: define a new mapping as a NURBS.}\index{nurbs mapping}\index{Mapping!NurbsMapping}
\index{spline!surface}\index{spline!curve}
%-------------------------------------------------------------

The {\tt NurbsMapping} class defines mapings in terms of a non-uniform
rational b-spline, NURBS.
The implementation here is based on the reference, {\sl The NURBS Book}
Les Piegl and Wayne Tiller, Springer, 1997. 

The n-th degree Berstein polynomial is
\[
     B_{i,n}(u) = \binom{n}{i} u^i (1-u)^{n-i}
\]
and the n-th degree Bezier curve
\[
    \Cv(u) = \sum_{i=0}^n B_{i,n}(u) \Pv_i , \qquad 0\le u \le 1
\]
with control points $\Pv_i$.

The n-th degree rational Bezier curve is
\begin{align*}
    \Cv(u) &= { \sum_{i=0}^n B_{i,n}(u) w_i \Pv_i  \over \sum_{i=0}^n B_{i,n}(u) w_i } , \qquad 0\le u \le 1 \\
           &= \sum_{i=0}^n R_{i,n}(u) \Pv_i  
\end{align*}
with weights $w_i$.

Written using homogeneous coordinates
\[
    \Cv^w(u) = \sum_{i=0}^n B_{i,n}(u) \Pv^w_i  
\]
where $\Pv^w_i=(w_i\Pv_i,w_i)$.

B-spline basis functions are defined as
\begin{align*}
   N_{i,0}(u) &= \begin{cases} 1 & u_i \le u < u_{i+1} \\
                           0 & \text{otherwise} 
              \end{cases} \\
   N_{i,p}(u) &= {u-u_i \over u_{i+p} - u_i } N_{i,p-1}(u) + {u_{i+p+1}-u  \over u_{i+p+1} - u_{i+1} } N_{i+1,p-1}(u)
\end{align*}
where $\Uv=\{ u_0,\ldots,u_m\}$ are the knots, $u_i\le u_{i+1}$.

We only use nonperiodic (clamped or open) knot vectors,
\[
   \Uv =\{ a,\ldots,a, u_{p+1},\ldots,u_{m-p-1},b,\ldots,b \}
\]
with the end knots repeated $p+1$ times.

NonUniform Rational B-Spline (NURBS). p-th degree NURBS curve
\begin{align*}
    \Cv(u) &= { \sum_{i=0}^n N_{i,p}(u) w_i \Pv_i  \over \sum_{i=0}^n N_{i,p}(u) w_i} , \qquad a\le u \le b\\
           &= \sum_{i=0}^n R_{i,p} (u) \Pv_i  
\end{align*}
Written using homogeneous coordinates
\[
    \Cv^w(u) = \sum_{i=0}^p N_{i,p}(u) \Pv^w_i  
\]


%% \input NurbsMappingInclude.tex

\subsection{Examples}
% 
% \noindent
% \begin{minipage}{.4\linewidth}
% {\footnotesize
% \listinginput[1]{1}{\mapping/nurbs1.cmd}
% }
% \end{minipage}\hfill
% \begin{minipage}{.6\linewidth}
  \begin{center}
  \includegraphics[width=9cm]{\figures/nurbs} \\
  % \epsfig{file=\figures/nurbs.ps,height=.6\linewidth}  \\
  {A 2D NURBS curve defined by specifying control points.} \\
  \includegraphics[width=9cm]{\figures/nurbs1} \\
  % \epsfig{file=\figures/nurbs1.ps,height=.6\linewidth}  \\
  {A 3D NURBS surface defined by specifying control points.}
  \end{center}
% \end{minipage}
