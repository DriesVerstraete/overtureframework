\subsubsection{constructor}
 
\newlength{\EllipticGridGeneratorIncludeArgIndent}
\begin{flushleft} \textbf{%
\settowidth{\EllipticGridGeneratorIncludeArgIndent}{EllipticGridGenerator(}% 
EllipticGridGenerator()
}\end{flushleft}
\begin{description}
\item[{\bf Description:}] 
   Default constructor.

\end{description}
\subsubsection{solution}
 
\begin{flushleft} \textbf{%
const RealMappedGridFunction \&  \\ 
\settowidth{\EllipticGridGeneratorIncludeArgIndent}{solution(}%
solution() const
}\end{flushleft}
\begin{description}
\item[{\bf Description:}] 
     Return a reference to the current solution.
\end{description}
\subsubsection{setup}
 
\begin{flushleft} \textbf{%
void  \\ 
\settowidth{\EllipticGridGeneratorIncludeArgIndent}{setup(}%
setup(Mapping \& mapToUse, \\ 
\hspace{\EllipticGridGeneratorIncludeArgIndent}Mapping *projectionMappingToUse  =NULL)
}\end{flushleft}
\begin{description}
\item[{\bf Description:}] 
     Setup the EllipticGridGenerator
\item[{\bf mapToUse:}]  This mapping defines the ...
\end{description}
\subsubsection{updateForNewBoundaryConditions}
 
\begin{flushleft} \textbf{%
int      \\ 
\settowidth{\EllipticGridGeneratorIncludeArgIndent}{updateForNewBoundaryConditions(}%
updateForNewBoundaryConditions()
}\end{flushleft}
\begin{description}
\item[{\bf Access:}]  {\bf Protected}.
\item[{\bf Description:}] 
   Update some variables that depend on the boundary consitions.
 gridIndex : interior points plus boundaries where interior equations are applied.

\end{description}
\subsubsection{plot}
 
\begin{flushleft} \textbf{%
int  \\ 
\settowidth{\EllipticGridGeneratorIncludeArgIndent}{plot(}%
plot( const RealMappedGridFunction \& v, const aString \& label )
}\end{flushleft}
\begin{description}
\item[{\bf Access:}]  {\bf Protected}.
\item[{\bf Description:}] 
   Plot a grid function.
\end{description}
\subsubsection{projectBoundaryPointst}
 
\begin{flushleft} \textbf{%
int  \\ 
\settowidth{\EllipticGridGeneratorIncludeArgIndent}{projectBoundaryPoints(}%
projectBoundaryPoints(const int \& level,\\ 
\hspace{\EllipticGridGeneratorIncludeArgIndent}RealMappedGridFunction \& uu, \\ 
\hspace{\EllipticGridGeneratorIncludeArgIndent}const int \& side,  \\ 
\hspace{\EllipticGridGeneratorIncludeArgIndent}const int \& axis,\\ 
\hspace{\EllipticGridGeneratorIncludeArgIndent}const Index \& I1,\\ 
\hspace{\EllipticGridGeneratorIncludeArgIndent}const Index \& I2,\\ 
\hspace{\EllipticGridGeneratorIncludeArgIndent}const Index \& I3 )
}\end{flushleft}
\begin{description}
\item[{\bf Access:}]  {\bf Protected}.
\item[{\bf Description:}] 
   Project boundary points onto the Mappings`s that define the actual boundary.

 boundaryProjectionMap[2][3] : pointer to the mapping to use to project a point onto a boundary.
\end{description}
\subsubsection{applyBoundaryConditions}
 
\begin{flushleft} \textbf{%
int  \\ 
\settowidth{\EllipticGridGeneratorIncludeArgIndent}{applyBoundaryConditions(}%
applyBoundaryConditions(const int \& level,\\ 
\hspace{\EllipticGridGeneratorIncludeArgIndent}RealMappedGridFunction \& uu )
}\end{flushleft}
\begin{description}
\item[{\bf Access:}]  {\bf Protected}.
\item[{\bf Description:}] 
    Apply boundary conditions. 

  {\bf slip orthogonal boundary:}

  Adjust the points on the boundary to make the grid orthogonal at the boundary.
  To do this we compute the amount to shift the point in the unit square coordinates.
  We then recompute the $\xv$ coordinates by evaluating the Mapping on the boundary.

  Suppose that  $r,s$ are the coordinates tangential to the boundary and $t$ is the coordinate
  normal to the boundary. Let $\xv_0$ be the grid point on the boundary that we
 want to adjust and let $\xv_1$ be the grid point one line away from the boundary.
  Then we want to choose a new boundary 
 point $\xv(r,s)$ so that
 \begin{align*}
    (\xv-\xv_1) \cdot \xv_r &= 0 \\
    (\xv-\xv_1) \cdot \xv_s &= 0
 \end{align*}
 We use the approximation
 \[
    \xv(r,s) \approx \xv_0 + \Delta r \xv_r^0 + \Delta s \xv_s^0 
 \]
 and thus the equation for $(\Delta r,\Delta s)$ is
 \begin{align*}
   \begin{bmatrix}  
       \xv_r^0\cdot\xv_r^0 & \xv_r^0\cdot\xv_s^0 \\
       \xv_r^0\cdot\xv_s^0 & \xv_s^0\cdot\xv_s^0 
   \end{bmatrix}  
   \begin{bmatrix} \Delta r \\ \Delta s \end{bmatrix} =
   \begin{bmatrix} (\xv_1-\xv_0)\cdot \xv_r^0 \\ (\xv_1-\xv_0)\cdot \xv_s^0 \end{bmatrix}
 \end{align*}
 with solution
 \begin{align*}
   \begin{bmatrix} \Delta r \\ \Delta s \end{bmatrix} =
   {1\over \xv_r^0\cdot\xv_r^0 \xv_s^0\cdot\xv_s^0 - (\xv_r^0\cdot\xv_s^0)^2 }
    \begin{bmatrix}  
       \xv_s^0\cdot\xv_s^0 &-\xv_r^0\cdot\xv_s^0 \\
      -\xv_r^0\cdot\xv_s^0 & \xv_r^0\cdot\xv_r^0 
   \end{bmatrix}  
   \begin{bmatrix} (\xv_1-\xv_0)\cdot \xv_r^0 \\ (\xv_1-\xv_0)\cdot \xv_s^0 \end{bmatrix}
 \end{align*}
 In 2D this reduces to
 \[
  \Delta r = {1\over \xv_r^0\cdot\xv_r^0} (\xv_1-\xv_0)\cdot \xv_r^0
 \]

\end{description}
\subsubsection{stretchTheGrid}
 
\begin{flushleft} \textbf{%
int  \\ 
\settowidth{\EllipticGridGeneratorIncludeArgIndent}{stretchTheGrid(}%
stretchTheGrid(Mapping \& mapToStretch)
}\end{flushleft}
\begin{description}
\item[{\bf Access:}]  {\bf Protected}.
\item[{\bf Description:}] 

   Determine a starting guess for grids with stretched boundary layer spacing.

  If the user has requested a very small spacing near the boundary we can
 can explicitly stretch the initial grid to approximately statisfy the grid spacing.
 To do this we measure the actual grid spacing near each boundary that needs to be stretched.
 We then determine use stretching functions to determine a new grid by composing a 
 stretched
 
 
 
 
\end{description}
\subsubsection{determineBoundarySpacing}
 
\begin{flushleft} \textbf{%
int  \\ 
\settowidth{\EllipticGridGeneratorIncludeArgIndent}{determineBoundarySpacing(}%
determineBoundarySpacing(const int \& side, \\ 
\hspace{\EllipticGridGeneratorIncludeArgIndent}const int \& axis,\\ 
\hspace{\EllipticGridGeneratorIncludeArgIndent}real \& averageSpacing,\\ 
\hspace{\EllipticGridGeneratorIncludeArgIndent}real \& minimumSpacing,\\ 
\hspace{\EllipticGridGeneratorIncludeArgIndent}real \& maximumSpacing )
}\end{flushleft}
\begin{description}
\item[{\bf Access:}]  {\bf Protected}.
\item[{\bf Description:}] 
   Compute the current spacing of the first grid line from the boundary
    
\item[{\bf side,axis (input) :}]  determine spacing for this side.
\item[{\bf averageSpacing,minimumSpacing,maximumSpacing (output) :}] 

\end{description}
\subsubsection{redBlack}
 
\begin{flushleft} \textbf{%
int  \\ 
\settowidth{\EllipticGridGeneratorIncludeArgIndent}{redBlack(}%
redBlack(const int \& level, \\ 
\hspace{\EllipticGridGeneratorIncludeArgIndent}RealMappedGridFunction \& uu )
}\end{flushleft}
\begin{description}
\item[{\bf Access:}]  {\bf Protected}.
\item[{\bf Description:}] 
     Red black smooth.
\item[{\bf uu (input/output) :}]  On input and output : current solution valid at all points, including periodic points.

\end{description}
\subsubsection{jacobi}
 
\begin{flushleft} \textbf{%
int  \\ 
\settowidth{\EllipticGridGeneratorIncludeArgIndent}{jacobi(}%
jacobi(const int \& level, \\ 
\hspace{\EllipticGridGeneratorIncludeArgIndent}RealMappedGridFunction \& uu )
}\end{flushleft}
\begin{description}
\item[{\bf Access:}]  {\bf Protected}.
\item[{\bf Description:}] 
     Jacobi smooth.
\item[{\bf uu (input/output) :}]  On input and output : current solution valid at all points, including periodic points.

\end{description}
\subsubsection{lineSmoother}
 
\begin{flushleft} \textbf{%
int  \\ 
\settowidth{\EllipticGridGeneratorIncludeArgIndent}{lineSmoother(}%
lineSmoother(const int \& direction,\\ 
\hspace{\EllipticGridGeneratorIncludeArgIndent}const int \& level,\\ 
\hspace{\EllipticGridGeneratorIncludeArgIndent}RealMappedGridFunction \& uu )
}\end{flushleft}
\begin{description}
\item[{\bf Access:}]  {\bf Protected}.
\item[{\bf Description:}] 
   Perform a line smooth.
\item[{\bf direction (input) :}]  perform a line smooth along this axis, 0,1, or 2.
\item[{\bf uu (input/output) :}]  On input and output : current solution valid at all points, including periodic points.

\end{description}
\subsubsection{smooth}
 
\begin{flushleft} \textbf{%
int  \\ 
\settowidth{\EllipticGridGeneratorIncludeArgIndent}{smooth(}%
smooth(const int \& level, \\ 
\hspace{\EllipticGridGeneratorIncludeArgIndent}const SmoothingTypes \& smoothingType,\\ 
\hspace{\EllipticGridGeneratorIncludeArgIndent}const int \& numberOfSubIterations  =1)
}\end{flushleft}
\begin{description}
\item[{\bf Access:}]  {\bf Protected}.
\item[{\bf Description:}] 
   Handles the different smoothing methods.
\end{description}
\subsubsection{fineToCoarse}
 
\begin{flushleft} \textbf{%
int  \\ 
\settowidth{\EllipticGridGeneratorIncludeArgIndent}{fineToCoarse(}%
fineToCoarse(const int \& level, \\ 
\hspace{\EllipticGridGeneratorIncludeArgIndent}const RealMappedGridFunction \& uFine, \\ 
\hspace{\EllipticGridGeneratorIncludeArgIndent}RealMappedGridFunction \& uCoarse,\\ 
\hspace{\EllipticGridGeneratorIncludeArgIndent}const bool \& isAGridFunction  = FALSE)
}\end{flushleft}
\begin{description}
\item[{\bf Access:}]  {\bf Protected}.
\item[{\bf Description:}] 
   Compute the restriction of the defect
\item[{\bf isAGridFunction (input) :}]  If true then this variable defines some (x,y,z) coordinates on a grid.
    This is used to get the correct periodicity.
\end{description}
\subsubsection{coarseToFine}
 
\begin{flushleft} \textbf{%
int  \\ 
\settowidth{\EllipticGridGeneratorIncludeArgIndent}{coarseToFine(}%
coarseToFine(const int \& level,  \\ 
\hspace{\EllipticGridGeneratorIncludeArgIndent}const RealMappedGridFunction \& uCoarse, \\ 
\hspace{\EllipticGridGeneratorIncludeArgIndent}RealMappedGridFunction \& uFineGF,\\ 
\hspace{\EllipticGridGeneratorIncludeArgIndent}const bool \& isAGridFunction  = FALSE)
}\end{flushleft}
\begin{description}
\item[{\bf Access:}]  {\bf Protected}.
\item[{\bf Description:}] 
             Correct a Component Grid
      u(i,j) = u(i,j) + P[ u2(i,j) ]   ( P : Prolongation )
  cp21,cp22,cp23 : coeffcients for prolongation, 2nd order
  cp41,cp41,cp43 : coeffcients for prolongation, 4th order

\end{description}
\subsubsection{multigrid}
 
\begin{flushleft} \textbf{%
int  \\ 
\settowidth{\EllipticGridGeneratorIncludeArgIndent}{multigridVcycle(}%
multigridVcycle(const int \& level )
}\end{flushleft}
\begin{description}
\item[{\bf Access:}]  {\bf Protected}.
\item[{\bf Description:}] 
   Multigrid V cycle.
\end{description}
\subsubsection{generateGrid}
 
\begin{flushleft} \textbf{%
int  \\ 
\settowidth{\EllipticGridGeneratorIncludeArgIndent}{generateGrid(}%
generateGrid()
}\end{flushleft}
\begin{description}
\item[{\bf Access:}]  {\bf Protected}.
\item[{\bf Description:}] 
     Perform some multigrid iterations.

\item[{\bf u0 (input) :}]  initial guess.     // ****** fix this -- we should keep ghost values ------------------

\end{description}
\subsubsection{update}
 
\begin{flushleft} \textbf{%
int  \\ 
\settowidth{\EllipticGridGeneratorIncludeArgIndent}{update(}%
update(DataPointMapping \& dpm,\\ 
\hspace{\EllipticGridGeneratorIncludeArgIndent}GenericGraphicsInterface *gi\_  = NULL, \\ 
GraphicsParameters \& parameters  =Overture::nullMappingParameters())
}\end{flushleft}
\begin{description}
\item[{\bf Description:}] 
    Prompt for changes to parameters and compute the grid.
\item[{\bf dpm (input) :}]  build this mapping with the grid.
\item[{\bf gi (input) :}]  supply a graphics interface if you want to see the grid as it
    is being computed.
\item[{\bf parameters (input) :}]  optional parameters used by the graphics interface.

\end{description}
