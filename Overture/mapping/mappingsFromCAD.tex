%=======================================================================================================
% MappingsFromCAD documentation
%=======================================================================================================
\documentclass{article}

\voffset=-1.25truein
\hoffset=-1.truein
\setlength{\textwidth}{7in}      % page width
\setlength{\textheight}{9.5in}    % page height for xdvi

\usepackage{epsfig}
\usepackage{graphics}    
\usepackage{moreverb}
\usepackage{amsmath}
\usepackage{fancybox}
\usepackage{subfigure}
\usepackage{multicol}


\usepackage{makeidx} % index
\makeindex
\newcommand{\Index}[1]{#1\index{#1}}

\begin{document}


\input wdhDefinitions

\def\uvd    {{\bf U}}
\def\ud     {{    U}}
\def\pd     {{    P}}
\def\id     {i}
\def\jd     {j}
\def\kap {\sqrt{s+\omega^2}}

\newcommand{\mapping}{/home/henshaw/Overture/mapping}
\newcommand{\figures}{../docFigures}

\vspace{3\baselineskip}
\begin{flushleft}
  {\Large 
   Mappings from CAD, \\
   A Description of the IgesReader and MappingsFromCAD Classes\\
  }
\vspace{2\baselineskip}
William D. Henshaw, \\
Anders Petersson, \\             
Centre for Applied Scientific Computing \\
Lawrence Livermore National Laboratory    \\
Livermore, CA, 94551   \\
henshaw@llnl.gov \\
http://www.llnl.gov/casc/people/henshaw \\
http://www.llnl.gov/casc/Overture
\vspace{1\baselineskip}
\today
\vspace{\baselineskip}
% UCRL-MA-132239

\end{flushleft}

\vspace{1\baselineskip}

\begin{abstract}
\end{abstract}



% \tableofcontents

\section{IgesReader}\index{IgesReader}\index{iges}\index{CAD}

The {\tt IgesReader} is a class that can be used to read an IGES file that contains
objects created from a computer-aided-design package (such as proENGINEER).
For full details on the IGES file format,
see the reference guide, {\sl IGES, Initial Graphics Exchange Specification}\cite{iges}.
The IGES file is composed of six sections,

\begin{description}
  \item[1. flag] section (binary or compressed files only).
  \item[2. start] section.
  \item[3. global] section contains various global parameters about the file such as units, date created, etc.
  \item[4. directory entry] section (DE) contains a short entry for each entity in the file describing what
     the entity is and containing pointers to the parameter data.
  \item[5. parameter data] section (PD) contains the parameter data for each entity in the file.
  \item[6. terminate] section.
\end{description}

The IgesReader will read an IGES file and build a list of all the items of interest (i.e. items
that we currently know what to do with). This list is the {\tt entityInfo} list.

Here are some definitions
\begin{description}
  \item[entity type] is an enumerated list of entities found in the {\tt IgesReader} class
   that includes enums such as {\tt parametric\-Spline\-Surface=114}. In this case IGES entity type
    114 is a parametric spline surface; look for the description of entity 114 in the IGES manual.
  \item[sequence number] : each directory entry (DE) line in the file ends with a ``D'' followed by
     a sequence number -- this is the line number relative to the start of the DE section.
     The sequence number for an ``item'', saved in {\tt entityInfo(1,item)}
      will be the line number of the first line of the DE entry
     for that item.
\end{description}         


\input IgesReaderInclude.tex

\section{MappingsFromCAD}\index{CAD}\index{NURBS}

 The {\tt MappingsFromCAD} class builds selected Mappings from a CAD description. For example,
Mappings can be created from an IGES file.

\input MappingsFromCADInclude.tex

\vfill\eject
\bibliography{/home/henshaw/papers/henshaw}
\bibliographystyle{siam}

\printindex

\end{document}








