%--------------------------------------------------------------
\section{CompositeSurface: define a surface formed from many sub-surfaces}
\index{composite surface mapping}\index{Mapping!CompositeSurface}\index{patched surface}
%-------------------------------------------------------------

The {\tt CompositeSurface} Mapping is used to represent a surface that
is formed from a collection of sub-surfaces. This Mapping is not a
normal Mapping since it does not represent a transformation from the
unit square. 

As an example, a CompositeSurface is used to represent the collection of
NURBS and trimmed-NURBS surfaces that can be created by CAD packages. A single
CompositeSurface can hold any number of these surfaces. Each sub-surface
in a CompositeSurface is just any Mapping. Usually every sub-surface
will actually be a surface in 3D but this is not necessary.


The most common use for a CompositeSurface is in combination with
the Hyperbolic surface grid generator. This surface grid generator
can grow a grid over a portion of a CompositeSurface, starting from
some intial curve on the surface. 



\subsection{Projection onto the composite surface}

  The CompositeSurface has a function {\tt project} that can be used 
to take one or more points in space, $\xv_i$, and project these points
onto the CompositeSurface, giving new points $\xv_i^p$. 

The hyperbolic surface grid generator, for example, will march a line
of points over the CompositeSurface. At each step in it's marching algorithm,
new positions will be predicted for the next position for the line of points.
These predicted values are then projected exactly onto the CompositeSurface.

The projection algorithm make use of the following variables:
\begin{description}
  \item[$\xv$] : point near the surface that needs to be projected.
  \item[$s_0$] : initial guess for the sub-surface patch on which to look (may be omitted).
  \item[$\xv_0$] : a previous point on the CompositeSurface that is near to $\xv$. This 
    may be the previous location of $\xv$ from a surface grid generator (may be omitted).
  \item[$\nv_0$] : normal to the CompositeSurface at the point $\xv_0$  (may be omitted).
  \item[$\xv_p$] : projected point on the surface.
  \item[$s_p$] : subsurface index where the point was projected.
  \item[$\nv_p$] : normal to the CompositeSurface at the point $\xv_p$.
\end{description}
Here is the basic projection algorithm
\begin{enumerate}
  \item Project $\xv$ on the sub-surface patch $s_0$, giving the point $\yv=P_{s_0}(\xv)$. If $\yv$
    is in the interior of the sub-surface then we are done.
  \item If $\yv$ is on the boundary of the sub-surface $s_0$ then compute the distance
     $\dv_0=\| \xv-\yv \|$. We will try to find a sub-surface that is closer than this
     distance.
  \item Choose a new sub-surface to check, $s_1$, and project $\xv$ onto this surface,
           $\yv_1=P_{s_0}(\xv)$.  
\end{enumerate}

\subsubsection{Moving around sharp corners}

If the point $\xv$ to be projected is near a sharp corner in the surface
then there is some ambiguity as to the desired projection point $\xv_p$. 


If we are
marching over the surface then we usually want the projected point $\xv_p$
to be some specified distance from the old point $\xv_0$. In this case 
we may have to adjust the projected point and move it away from the corner.


%% *wdh* 2012/10/12 \input cs1.latex



%% \input CompositeSurfaceInclude.tex

\subsection{Examples}
% 
% \noindent
% \begin{minipage}{.4\linewidth}
% {\footnotesize
% \listinginput[1]{1}{\mapping/nurbs1.cmd}
% }
% \end{minipage}\hfill
% \begin{minipage}{.6\linewidth}
  \begin{center}
   \includegraphics[width=10cm]{\figures/compositeCyl4} \\
   % \epsfig{file=\figures/compositeCyl4.ps,height=.6\linewidth}  \\
  {A CompositeSurface for a cylindrical surface read from an IGES file created by pro/ENGINEER}
  \end{center}
% \end{minipage}
