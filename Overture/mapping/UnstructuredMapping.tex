\section{UnstructuredMapping}

% \newcommand{\figures}{.}

The {\tt UnstructuredMapping} class encapsulates the connectivity for an 
unstructured mesh.  Currently the class supports the ``Finite Element Zoo'' 
collection of element types.  This zoo consists of quadrilaterals and triangles
in surface meshes and hexahedra, triangle prisms, pyramids and tetrahedra in 
volume meshes.  

\subsection{Connectivity Iterator Interface}
\input newConnectivity.tex


\subsection{Implementation Details}
Since the supported element types are limited to the zoo, a canonical
ordering of the vertices, faces, edges, sides, etc., in each element
type can be constructed.  Currently implemented three dimensional
orderings are illustrated by Figure~\ref{fig:elementOrderings}. In two
dimensions the vertices and faces are simply ordered counter-clockwise
starting from 0.

\begin{figure} [htb] 
\centering
    \includegraphics[width=6cm]{\figures/hex}
    \includegraphics[width=6cm]{\figures/triprism}
    \includegraphics[width=6cm]{\figures/pyramid}
    \includegraphics[width=6cm]{\figures/tet}
%  \mbox{
%        \subfigure[]{\epsfig{file=\figures/hex.eps,width=.35\textwidth}}
%        \hspace{.25in}
%        \subfigure[]{\epsfig{file=\figures/triprism.eps,width=.35\textwidth}}}
%  \mbox{
%        \subfigure[]{\epsfig{file=\figures/pyramid.eps,width=.35\textwidth}}
%        \hspace{.25in}
%        \subfigure[]{\epsfig{file=\figures/tet.eps,width=.35\textwidth}}}
 \caption{Canonical orderings for the 3D finite element zoo: (a) hexahedra; (b) triangle prisms; (c) pyramids; (d) tetrahedra. Black indicates vertex numbers, green indicates face indices. Smple corners are drawn in red and sides are in blue.} \label{fig:elementOrderings} 
\end{figure}

\subsection{Iterations using the old connectivity interface}
Iterating through the connectivity using the old interface consists of using a few inlinable functions
which abstract away the underlying representation, including the canonical 
orderings of the elements.  Normally, a user will never even know the
orderings exist (at least they should beware of depending upon them!). Iterations
are also independent of the dimension of the mesh since the connectivity for
all dimensions share the same nomenclature ( eg. an edge in 2D is the same as 
an edge in 3D ).  The following subsections provide examples of how to 
navigate and use the limited set of iterations available in an {\tt UnstructuredMapping}.
\subsubsection{Element iteration}
{\footnotesize
\begin{verbatim}
  // assuming an UnstructuredMapping named um exists...
  const IntegerArray &elements = um.getElements();
  for ( int e=0; e<um.getNumberOfElements(); e++ ) {
      // ... do stuff with the element index
      elementScalar(e) = whatever;
  }
\end{verbatim}
}
\subsubsection{Vertex iteration}
{\footnotesize
\begin{verbatim}
  // assuming an UnstructuredMapping named um exists...
  const IntegerArray &vertices = um.getVertices();
  for ( int v=0; v<um.getNumberOfVertices(); v++ ) {
      // ... do stuff with the vertex index
      vertexScalar(v) = whatever;
  }
\end{verbatim}
}
\subsubsection{Iteration through the vertices in an element}
{\footnotesize
\begin{verbatim}
  for ( int e=0; e<um.getNumberOfElements(); e++ ) {
     for ( int v=0; v<um.getNumberOfVerticesThisElement(e); v++ ) {
        vGlobalIndex = um.elementGlobalVertex(e,v);
        // ... do stuff with global vertex index
        vertexScalar(vGlobalIndex) = whatever;
     }
  }
\end{verbatim}
}
\subsubsection{Iteration through the faces}
{\footnotesize
\begin{verbatim}
  const IntegerArray &elements = um.getElements();
  const IntegerArray &faceElements = um.getFaceElements();
  for ( int f=0; f<um.getNumberOfFaces(); f++ ) {
     // get the elements on either side of the face
     int element0 = faceElements(f, 0);
     int element1 = faceElemetns(f, 1);
     // ... do stuff with the face and element indices
     faceScalar(f) = whatever;
     elementScalar(element0) += -face;
     elementScalar(element1) +=  face;
  }
\end{verbatim}
}
\subsubsection{Iteration through the vertices in a face}
{\footnotesize
\begin{verbatim}
  const IntegerArray &elements = um.getElements();
  for ( int f=0; f<um.getNumberOfFaces(); f++ ) {
     const IntegerArray &faceVertices = um.getFaceVertices(f);
     for ( int v=0; v<um.getNumberOfVerticesThisFace(f); v++ )
        vGlobalIndex = um.faceGlobalVertex(f,v);
        // ... do stuff with global vertex index
        vertexScalar(vGlobalIndex) = whatever;
     }
\end{verbatim}
}

\subsection{Enum Types}
{\tt ElementType} enumerates the supported unstructured entity types.
{\footnotesize
\begin{verbatim}
enum ElementType 
  {
    triangle,
    quadrilateral,
    tetrahedron,
    pyramid,
    triPrism,
    septahedron,  // pray we never need...
    hexahedron,
    other,
    boundary
  };
\end{verbatim}
}
{\tt EntityTypeEnum} is used to determine the topological dimension of a given
entity.  
{\footnotesize
\begin{verbatim}
enum EntityTypeEnum
  {
    Invalid=-1,
    Vertex=0,
    Edge,
    Face,
    Region,
    Mesh, // kkc put this here to enable "Mesh" tagging...
    NumberOfEntityTypes
  }
\end{verbatim}
}

\noindent
In 2D, only triangles and quadrilaterals are supported.  More types are supported
in 3D, but in general these consist of the ``finite element zoo''.  These elements are
hexahedra or degenerate hexahedra.  Currently the septahedron is not supported as this 
shape is rather unusual and rarely (?) encountered (7 nodes, 6 faces). {\tt other} 
implies any shape not described by the previous enums and at this point could
include arbitrary polyhedra (although the connectivity to support arbitrary 
polyhedra is not implemented).  {\tt boundary} elements are typically placeholders.
{\tt boundaries} will not have a specific geometry associated with them and
may only consist of a limited set of connectivity.

\subsection{File Formats}
{\tt UnstructuredMapping}'s can be written to two different kinds of files using the
member functions {\tt get} and {\tt put}. Using {\tt get} or {\tt put} with an Overture
{\tt GenericDatabase} class as the first argument performs io directly to an 
Overture database file.  During a {\tt put}, the instance's arrays {\tt node} and {\tt element}
are written to the database file.  A {\tt get} retrieves these arrays and reconstructs the 
connectivity.  {\tt UnstructuredMapping}'s can also be read/written to ASCII files using a simplified 
version of a format commonly called ``ingrid'', or ``DYNA''.  This io method can be invoked 
by calling {\tt get} and {\tt put} with a string, the filename, as the first argument.  
The resulting file looks like :
{\footnotesize
\begin{verbatim}
A text header line (optional)
number of meshes, number of nodes, number of elements, domain dimension(optional), range dimension(optional)
node0 ID, x0, y0, z0
node1 ID, x1, y1, z1
.
.
.
nodeN ID, xN, yN,zN
element0 ID, tag0, n1,n2,n3,n4,n5,n6,n7,n8
element1 ID, tagN, node ID list
.
.
.
elementN ID, tagN, node ID list
\end{verbatim}
}
\noindent
Typically, Overture writes ``OVERTUREUMapping'' in the comment line and uses the optional
spaces for the domain dimension and range dimension.  These details, however, are not 
required and meshes from a variety of mesh generation tools have been read in.  The node ID
lists in the element lines are lists of global node ID's (listed in the node section),
that are in each element.  The ordering of the nodes in the list follows the canonical ordering
described in Figure~\ref{fig:elementOrderings}.  Currently the code requires that the nodes
be listed in ascending order of thier node ID's and that the node IDs be contiguous.  By the way,
the number of meshes in the file is ignored, only one mesh per file is supported at the moment.

\subsection{Relationship to Normal Overture Mappings}
While {\tt UnstructuredMapping} inherits from class {\tt Mapping}, 
there are a few caveats.  By its very nature, the inverse does not exist for 
an {\tt UnstructuredMapping}.  Any use of an {\tt UnstructuredMapping} in the context of 
mapping inverses should be prevented; all member functions dealing with inverses now throw exceptions.
However, a domainDimension and rangeDimension are both still used in the mapping to
help donote the difference between 2D meshes, 3D surface meshes and 3D volume meshes.
With these exceptions, {\tt UnstructuredMapping}s should play nicely with conventional
sturctured ones.  In particular, a structured mapping can be converted into an unstructured
one by using the member function {\tt buildFromAMapping}. 

%% \subsection{New Connectivity Interface}
% \input newConnectivity.tex
%% \subsection{Member Function Descriptions}
%% \input UnstructuredMappingImp.tex
%% \input UnstructuredMappingInclude.tex
