\subsection{Constructor}
 
\newlength{\JoinMappingIncludeArgIndent}
\begin{flushleft} \textbf{%
\settowidth{\JoinMappingIncludeArgIndent}{JoinMapping(}% 
JoinMapping() 
}\end{flushleft}
\begin{description}
\item[{\bf Description:}]  
  Define a mapping that tranforms a "source-mapping" so that
 it intersects another "clip-surface" "exactly". For example,
 a Mapping for a wing (source-mapping)  can be joined to a fuselage (clip-surface).

\end{description}
\subsection{Constructor}
 
\begin{flushleft} \textbf{%
\settowidth{\JoinMappingIncludeArgIndent}{JoinMapping(}% 
JoinMapping(Mapping \& sourceMapping, \\ 
\hspace{\JoinMappingIncludeArgIndent}Mapping \& clipSurface)
}\end{flushleft}
\begin{description}
\item[{\bf Description:}]  
  Define a mapping that tranforms a "source-mapping" so that
 it intersects another "clip-surface" "exactly". For example,
 a Mapping for a wing (source-mapping)  can be joined to a fuselage (clip-surface).
 
\item[{\bf sourceMapping (input):}]  defines the source-mapping. This is the Mapping
  that will be changed. 
\item[{\bf clipMapping :}]  defines the clip-surface. This Mapping will clip away a
    portion of the sourceMapping. Use  the setEndOfJoin function to specify
    which portion of the sourceMapping to retain. 
\end{description}
\subsection{setCurves}
 
\begin{flushleft} \textbf{%
int  \\ 
\settowidth{\JoinMappingIncludeArgIndent}{setCurves(}%
setCurves(Mapping \& sourceMapping, \\ 
\hspace{\JoinMappingIncludeArgIndent}Mapping \& clipSurface )
}\end{flushleft}
\begin{description}
\item[{\bf Description:}]  
   Supply the source-mapping and clip-surface from which the join will be defined.
 
\item[{\bf sourceMapping (input):}]  defines the source-mapping. This is the Mapping
  that will be changed. 
\item[{\bf clipMapping :}]  defines the clip-surface. This Mapping will clip away a
    portion of the sourceMapping. Use  the setEndOfJoin function to specify
    which portion of the sourceMapping to retain. 
\end{description}
\subsection{setEndOfJoin}
 
\begin{flushleft} \textbf{%
int  \\ 
\settowidth{\JoinMappingIncludeArgIndent}{setEndOfJoin(}%
setEndOfJoin( const real \& endOfJoin\_ )
}\end{flushleft}
\begin{description}
\item[{\bf Description:}]  
 Specify the r value for the end of the join opposite the curve
 of intersection. Use this to specify which portion of the source-mapping to retain.
 For example, choosing a value of $0$ or $1$ will select the portion of the
 source-mapping that lies on one side of the clip-surface or the other side. Choosing
 a value of $.8$, for example, will shorten the join-mapping. 
 
\item[{\bf endOfJoin\_ (input) :}]  a value in [0,1].
\end{description}
\subsection{map}
 
\begin{flushleft} \textbf{%
void  \\ 
\settowidth{\JoinMappingIncludeArgIndent}{map(}%
map( const realArray \& r, realArray \& x, realArray \& xr, MappingParameters \& params )
}\end{flushleft}
\begin{description}
\item[{\bf Purpose:}]  Evaluate the TFI and/or derivatives. 
\end{description}
\subsection{update}
 
\begin{flushleft} \textbf{%
int  \\ 
\settowidth{\JoinMappingIncludeArgIndent}{update(}%
update( MappingInformation \& mapInfo ) 
}\end{flushleft}
\begin{description}
\item[{\bf Purpose:}]  Interactively create and/or change the Join mapping.
\item[{\bf mapInfo (input):}]  Holds a graphics interface to use.
\end{description}
