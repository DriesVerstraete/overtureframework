\subsection{Constructor}
 
\newlength{\MatrixTransformIncludeArgIndent}
\begin{flushleft} \textbf{%
\settowidth{\MatrixTransformIncludeArgIndent}{MatrixTransform(}% 
MatrixTransform() 
}\end{flushleft}
\begin{description}
\item[{\bf Purpose:}]  Build a mapping for matrix transform.
\end{description}
\subsection{Constructor(Mapping\&)}
 
\begin{flushleft} \textbf{%
\settowidth{\MatrixTransformIncludeArgIndent}{MatrixTransform(}% 
MatrixTransform(Mapping \& map) 
}\end{flushleft}
\begin{description}
\item[{\bf Purpose:}]  Build a Mapping for matrix transformation of another Mapping.
\end{description}
\subsection{reset}
 
\begin{flushleft} \textbf{%
void  \\ 
\settowidth{\MatrixTransformIncludeArgIndent}{reset(}%
reset()
}\end{flushleft}
\begin{description}
\item[{\bf Purpose:}]  Reset the transformation to the identity.
\end{description}
\subsection{rotate}
 
\begin{flushleft} \textbf{%
void  \\ 
\settowidth{\MatrixTransformIncludeArgIndent}{rotate(}%
rotate( const int axis, const real theta )
}\end{flushleft}
\begin{description}
\item[{\bf Purpose:}]  Perform a rotation about a given axis.
\item[{\bf axis (input) :}]  axis to rotate about (0,1,2)
\item[{\bf theta (input) :}]  angle in radians to rotate by.
\end{description}
\subsection{rotate}
 
\begin{flushleft} \textbf{%
void  \\ 
\settowidth{\MatrixTransformIncludeArgIndent}{rotate(}%
rotate( const RealArray \& rotate, bool incremental  =false)
}\end{flushleft}
\begin{description}
\item[{\bf Purpose:}]  Perform a "rotation" using a $3\times3$ matrix. This does not really have to
  be a rotation. 
\item[{\bf rotate (input):}]  If incremental=false then the upper $3\times3$ portion of the $4\times4$ transformation
    matrix will be replaced by the matrix {\tt rotate(0:2,0:2)}. Otherwise this rotation matrix
    will mutliply the existing transformation.
\item[{\bf incremental (input) :}]  if true apply this rotation to the existing transformation,
    otherwise replace the existing rotation.
\end{description}
\subsection{scale}
 
\begin{flushleft} \textbf{%
void  \\ 
\settowidth{\MatrixTransformIncludeArgIndent}{scale(}%
scale( const real scalex  =1.,\\ 
\hspace{\MatrixTransformIncludeArgIndent}const real scaley  =1., \\ 
\hspace{\MatrixTransformIncludeArgIndent}const real scalez  =1., \\ 
\hspace{\MatrixTransformIncludeArgIndent}bool incremental   =true)
}\end{flushleft}
\begin{description}
\item[{\bf Purpose:}]  Perform a scaling
\item[{\bf scalex, scaley, scalez (input):}]  Scale factors along each axis.
\item[{\bf incremental (input) :}]  if true then incrementally transform the 
       existing mapping, other transform the original mapping.
\end{description}
\subsection{shift}
 
\begin{flushleft} \textbf{%
void  \\ 
\settowidth{\MatrixTransformIncludeArgIndent}{shift(}%
shift( const real shiftx  =0., \\ 
\hspace{\MatrixTransformIncludeArgIndent}const real shifty  =0.,\\ 
\hspace{\MatrixTransformIncludeArgIndent}const real shiftz  =0., \\ 
\hspace{\MatrixTransformIncludeArgIndent}bool incremental   =true)
}\end{flushleft}
\begin{description}
\item[{\bf Purpose:}]  Perform a shift.
\item[{\bf shitx, shity, shitz (input):}]  shifts along each axis.
\item[{\bf incremental (input) :}]  if true then incrementally transform the 
       existing mapping, other transform the original mapping.
\end{description}
