%--------------------------------------------------------------
\section{RevolutionMapping: create a surface or volume of revolution}
\index{revolution mapping}\index{Mapping!RevolutionMapping}\index{body of revolution}\index{axisymmetric}
%-------------------------------------------------------------

\subsection{Description}

The {\tt RevolutionMapping} revolves a two-dimensional mapping (ie. a 2D curve or 2D region)
in the plane around a given line to create
a three-dimension mapping in three-space. The {\tt RevolutionMapping} can also be used to revolve a 
curve in 3D about a line to create a surface in 3D.


\begin{figure}[ht]
  \begin{center}
   \includegraphics[width=9cm]{\figures/revolution_idraw}
  % \epsfig{file=\figures/revolution.idraw.ps,width=12cm}
  \caption{The RevolutionMapping can be used to revolve a 2D mapping about a line 
           through a given angle. It can also be used to revolve a 3D curve.}
  \end{center}
\label{fig:RevolutionMapping}
\end{figure}

The revolution mapping is defined in the following manner. (This description applies to the case
when the mapping to be revolved is a 2D region. A similar definition applies in the other cases).
Let
\begin{align*}
    \xv_0 &= {\tt lineOrigin(0:2)} = \mbox{a point on the line of revolution} \\
    \vv   &= {\tt lineTangent(0:2)} = \mbox{unit tangent to the line of revolution} \\
    \Pv   &:~  \mbox{The two dimensional mapping in the plane that we will rotate}
\end{align*}
We first evaluate the two-dimensional mapping and save in a three dimension vector, $\yv$,
\[
                   \rv(0:1) \rightarrow (\Pv(\rv),0) = (y(0:1),0)  \equiv \yv
\]
Now decompose $\yv-\xv_0$ into a component parallel             and a component
orthogonal to the line of revolution:
\begin{align*}
        \yv-\xv_0 &= \av + \bv \\      
    \av &=  (\av\cdot\vv) \vv \qquad \mbox{component parallel to $\vv$}
\end{align*}
Then rotate the part orthogonal to the line:
\[
      \xv-\xv_0 = \av + R \bv  \quad \mbox{where  $R$ is the rotation matrix}
\]
To compute $R$ we determine a vector $\cv$ orthogonal to $\bv$ and $\vv$,
\[
             \cv = \vv \times \bv
\]
Then
\[
         R \bv = \cos(\theta) \bv + \sin(\theta) \cv
\]
where
\[
          \theta = r(2,I) \delta+{\rm startAngle} ~2\pi, \quad \delta={\rm (endAngle-startAngle)} 2 \pi
\]
In summary the revolution mapping is defined by
\begin{align*}
        \xv &= \av + \cos(\theta)\bv + \sin(\theta)\cv + \xv_0 \\
        \av &= ((\yv-\xv_0)\cdot\vv)\vv  \\
        \bv &= \yv-\xv_0-\av \\
        \cv &= \vv \times \bv
\end{align*}
The derivatives of the mapping are defined as
\begin{align*}
   {\partial \xv \over \partial r_i} &=  {\partial \av \over \partial r_i} + 
                \cos(\theta) {\partial \bv \over \partial r_i} 
              + \sin(\theta) {\partial \cv \over \partial r_i} \qquad \mbox{for $i=0,1$} \\
         {\partial \av \over \partial r_i} &= ({\partial \yv \over \partial r_i}\cdot\vv)\vv \\
         {\partial \bv \over \partial r_i} &= {\partial \yv \over \partial r_i} - {\partial \av \over \partial r_i} \\
         {\partial \cv \over \partial r_i} &= \vv \times {\partial \bv \over \partial r_i} \\
    {\partial \xv \over \partial r_2}  &= \delta*( -\sin(\theta)\bv +\cos(\theta)\cv )
\end{align*}


\subsection{Inverse of the mapping}

  When the mapping to be revolved is in the x-y plane, 
the {\tt RevolutionMapping} can be inverted analytically in terms of the inverse
of the mapping that is being revolved. (When revolving a 3D curve we do not define
a special inverse). Here is how we do the inversion.

Given a value $\xv$ we need to determine the corresponding value of $\rv$.
To do this we can first decide how to rotate the point $\xv$ (about the
line through $\xv_0$ with tangent $\vv$) into the x-y plane. This will determine 
$\theta$ and $\yv$. Given $\yv$ we can invert the two-dimensional mapping to
determine ($r_0$,$r_1$).

To perform the rotation back to the $x-y$ plane we decompose $\xv-\xv_0$ into
\[
    \xv-\xv_0 = \av + \hat{\bv}
\]
where $\av$ is the component parallel to $\vv$, (the same $\av$ as above),

\[
   \av = ((\xv-\xv_0)\cdot\vv) \vv
\]
and
\[
    \hat{\bv} = \xv-\xv_0-\av.
\]
Note that $\hat{\bv}$ is not the same as $\bv$. Letting 
\[
   \hat{\cv} = \pm \vv\times {\hat\bv}
\]
where the correct sign must be chosen, then
we can rotate back to the $x-y$ plane with the transformation
\begin{equation}
  \yv = \av + \cos(-\theta) \hat{\bv} + \sin(-\theta) \hat{\cv}   \label{eq:yeqn}
\end{equation}
Since $y_3=0$ ($\yv=(y_1,y_2,y_3)$) it follows from the third
component of this last equation that 
\[
    0 = a_3 + \cos(\theta) \hat{b}_3 - \sin(\theta) \hat{c}_3 
\]
Now assuming that $a_3=0$ (which assumes that the line of rotation is in
the $x-y$ plane) then
\[
     \tan(\theta) = {\hat{b}_3 \over \hat{c}_3 }
\]
Given $\theta$ we then know the first two components of $\yv=(y_1,y_2,0)$ from (\ref{eq:yeqn}).
We now determine the inverse of $(y_1,y_2)$ using the {\tt inverseMap} of the two-dimensional
mapping,
\[
    \rv(0:1) = \Pv^{-1}(\yv(0:1))
\]

\subsection{Reparameterized to spherical-like coordinates}

If the body of revolution created by this mapping has a spherical
polar singularity at one or both ends we may wish to create a new
mapping near the pole that does not have a singularity by using an
orthographic mapping (created from the {\tt reparameterize} menu in
{\tt ogen}). The orthographic transform expects the mapping to be
parameterized like a sphere with parameters $(\phi,\theta,r)$.
Thus we will want to change the order of the parameters in the above
definition of the body of revolution:
\[
   \tilde{\xv}( r_1,r_2,r_3) = \xv(r_1,r_3,r_2) \mbox{~~~or~~~} 
   \tilde{\xv}( r_1,r_2,r_3) = \xv(r_3,r_1,r_2) 
\]
so that the new variable $\tilde{\xv}$ will be parameterized like a sphere.

This re-ordering is done automatically if the body of revolution is detected
to have a spherical polar type singularity.


\subsection{Examples}

\noindent
\begin{minipage}{.4\linewidth}
{\footnotesize
\listinginput[1]{1}{\mapping/revSP.cmd}
}
\end{minipage}\hfill
\begin{minipage}{.6\linewidth}
  \begin{center}
   \includegraphics[width=9cm]{\figures/revSP1}\\
   % \epsfig{file=\figures/revSP1.ps,width=\linewidth}  \\
   {A two-dimensional smoothed polygon.} \\
   \includegraphics[width=9cm]{\figures/revSP2}\\
   %\epsfig{file=\figures/revSP2.ps,width=\linewidth}  \\
   {A body of revolution created by revolving a two-dimensional smoothed polygon.}
  \end{center}
\end{minipage} 

\begin{minipage}{.4\linewidth}
{\footnotesize
\listinginput[1]{1}{\mapping/revCircle.cmd}
}
\end{minipage}\hfill
\begin{minipage}{.6\linewidth}
  \begin{center}
   \includegraphics[width=9cm]{\figures/revCircle}\\
  % \epsfig{file=\figures/revCircle.ps,width=.8\linewidth}  \\
   {A body of revolution for a torus is created by revolving a circle.}
  \end{center}
\end{minipage}  \\




%% \input RevolutionMappingInclude.tex


