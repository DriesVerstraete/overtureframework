\subsection{setExitCommand}
 
\newlength{\DialogDataIncludeArgIndent}
\begin{flushleft} \textbf{%
int  \\ 
\settowidth{\DialogDataIncludeArgIndent}{setExitCommand(}%
setExitCommand(const aString \&exitC, const aString \&exitL)
}\end{flushleft}
\begin{description}
\item[{\bf Description:}]  Set the exit command on the dialog window in the GUIState. Note that
  the dialog window will apear after pushGUI has been called.

\item[{\bf exitC(input):}]  The command hat will be issued when the exit button is pressed.
\item[{\bf exitL(input):}]  The text label that will appear on the exit button.

\item[{\bf Return values:}]  The function returns 1 on a successful completion and 0 if an error occured.
\item[{\bf Author:}]  AP
\end{description}
\subsection{setToggleButtons}
 
\begin{flushleft} \textbf{%
int  \\ 
\settowidth{\DialogDataIncludeArgIndent}{setToggleButtons(}%
setToggleButtons(const aString tbCommands[], const aString tbLabels[], const int initState[], \\ 
\hspace{\DialogDataIncludeArgIndent}int numberOfColumns  = 2)
}\end{flushleft}
\begin{description}
\item[{\bf Description:}]  Set the toggle buttons of the dialog window in the GUIState. The buttons
 will appear on the dialog window after pushGUI has been called.

\item[{\bf tbCommands(input):}]  Array of strings containing the commands for the toggle buttons. 
   The array must be terminated by an empty string ("").
\item[{\bf tbLabels(input):}]  Array of strings containing the text labels that will be put on
   the toggle buttons. The array must be terminated by an empty string ("").
\item[{\bf initState(input):}]  Array that describes the initial state of each toggle button.
\item[{\bf numberOfColumns(input):}]  Optional argument that specifies the number of columns in which the 
   toggle buttons shall be organized in the dialog window.
\item[{\bf Return values:}]  The function returns 1 on a successful completion and 0 if an error occured.
\item[{\bf Author:}]  AP
\end{description}
\subsection{deleteToggleButtons}
 
\begin{flushleft} \textbf{%
int  \\ 
\settowidth{\DialogDataIncludeArgIndent}{deleteToggleButtons(}%
deleteToggleButtons()
}\end{flushleft}
\begin{description}
\item[{\bf Description:}]  Delete the toggle buttons.
\item[{\bf Author:}]  WDH 
\end{description}
\subsection{setPushButtons}
 
\begin{flushleft} \textbf{%
int  \\ 
\settowidth{\DialogDataIncludeArgIndent}{setPushButtons(}%
setPushButtons(const aString pbCommands[], const aString pbLabels[], int numberOfRows  = 2)
}\end{flushleft}
\begin{description}
\item[{\bf Description:}]  Set the push buttons of the dialog window in the GUIState. The buttons
 will appear on the dialog window after pushGUI has been called.

\item[{\bf pbCommands(input):}]  Array of strings containing the commands for the push buttons. 
   The array  must be terminated by an empty string ("").
\item[{\bf pbLabels(input):}]  Array of strings containing the text labels that will be put on
   the push buttons. The array must be terminated by an empty string ("").
\item[{\bf numberOfRows(input):}]  Optional argument that specifies the number of rows in which the 
   push buttons shall be organized in the dialog window.

\item[{\bf Return values:}]  The function returns 1 on a successful completion and 0 if an error occured.
\item[{\bf Author:}]  AP
\end{description}
\subsection{setTextBoxes}
 
\begin{flushleft} \textbf{%
int  \\ 
\settowidth{\DialogDataIncludeArgIndent}{setTextBoxes(}%
setTextBoxes(const aString textCommands[], const aString textLabels[], const aString initString[])
}\end{flushleft}
\begin{description}
\item[{\bf Description:}]  Set the text boxes of the dialog window in the GUIState. The boxes
 will appear on the dialog window after pushGUI has been called.

\item[{\bf textCommands(input):}]  Array of strings containing the commands for the text boxes. 
   The array  must be terminated by an empty string ("").
\item[{\bf textLabelsLabels(input):}]  Array of strings containing the text labels that will be put 
   in front of the text boxes. The array must be terminated by an empty string ("").
\item[{\bf initString(input):}]   Array of strings containing the initial text that will be put
   in each text box. The array must be terminated by an empty string ("").

\item[{\bf Return values:}]  The function returns 1 on a successful completion and 0 if an error occured.
\item[{\bf Author:}]  AP
\end{description}
\subsection{addInfoLabel}
 
\begin{flushleft} \textbf{%
int  \\ 
\settowidth{\DialogDataIncludeArgIndent}{addInfoLabel(}%
addInfoLabel(const aString \& textLabel)
}\end{flushleft}
\begin{description}
\item[{\bf Description:}]  Add a new info label to the dialog window.

\item[{\bf textLabel(input):}]  The new text string.

\item[{\bf Return code:}]  The number of the new info label in the GUI, or -1 if there was no
 space left. (There is only space for MAX\_INFO\_LABELS (=10 by default) in each dialog window.)
\item[{\bf Author:}]  AP
\end{description}
\subsection{addInfoLabel}
 
\begin{flushleft} \textbf{%
int  \\ 
\settowidth{\DialogDataIncludeArgIndent}{deleteInfoLabels(}%
deleteInfoLabels()
}\end{flushleft}
\begin{description}
\item[{\bf Description:}]  Delete the existing info labels.

\item[{\bf Return code:}]  0 
\item[{\bf Author:}]  WDH
\end{description}
\subsection{setTextLabel}
 
\begin{flushleft} \textbf{%
int  \\ 
\settowidth{\DialogDataIncludeArgIndent}{setTextLabel(}%
setTextLabel(const aString \& textLabel, const aString \&buff)
}\end{flushleft}
\begin{description}
\item[{\bf Description:}]  Set the text string textlabel with label "textLabel" in the currently
 active GUIState.

\item[{\bf textLabel(input):}]  The label of the text label in the array given to setTextBoxes during setup.
\item[{\bf buff(input):}]  The new text string.

\item[{\bf Author:}]  WDH
\end{description}
\subsection{addOptionMenu}
 
\begin{flushleft} \textbf{%
int  \\ 
\settowidth{\DialogDataIncludeArgIndent}{addOptionMenu(}%
addOptionMenu(const aString \&opMainLabel, const aString opCommands[], const aString opLabels[], int initCommand)
}\end{flushleft}
\begin{description}
\item[{\bf Description:}]  Add an option menu to the dialog window. The option menu will appear when the
 dialog window is displayed, i.e., after pushGUI has been called.

\item[{\bf opMainLabel(input):}]  The descriptive label that will appear to the left of the option menu
  on the dialog window.
\item[{\bf opCommands(input):}]  An array of strings with the command that will be issued when each menu
  item is selected. The array must be terminated by an empty string ("").
\item[{\bf opLabels(input):}]  An array of strings with the label that will be put on each menu
  item. The array must be terminated by an empty string ("").
\item[{\bf initCommand(input):}]  The index of the initial selection in the opLabels array. This label will
  appear on top of the option menu to indicate the initial setting.
\item[{\bf Return values:}]  The function returns 1 on a successful completion and 0 if an error occured.
\item[{\bf Author:}]  AP
\end{description}
\subsection{deleteOptionMenus}
 
\begin{flushleft} \textbf{%
int  \\ 
\settowidth{\DialogDataIncludeArgIndent}{deleteOptionMenus(}%
deleteOptionMenus()
}\end{flushleft}
\begin{description}
\item[{\bf Description:}]  Delete all option menus.
\item[{\bf Author:}]  WDH 
\end{description}
\subsection{addRadioBox}
 
\begin{flushleft} \textbf{%
bool  \\ 
\settowidth{\DialogDataIncludeArgIndent}{addRadioBox(}%
addRadioBox(const aString \&rbMainLabel, const aString rbCommands[], const aString rbLabels[], int initCommand,\\ 
\hspace{\DialogDataIncludeArgIndent}int columns  = 1)
}\end{flushleft}
\begin{description}
\item[{\bf Description:}]  Add a radio box to the dialog window. The radio buttons will appear when the
 dialog window is displayed, i.e., after pushGUI has been called.

\item[{\bf rbCommands(input):}]  An array of strings with the command that will be issued when the radio 
  button is pressed. The array must be terminated by an empty string ("").
\item[{\bf rbLabels(input):}]  An array of strings with the label that will be put on each radio button. 
 The array must be terminated by an empty string ("").
\item[{\bf initCommand(input):}]  The index of the initial selection in the rbLabels array. This radio 
  button will be marked initially.
\item[{\bf Return values:}]  The function returns true on a successful completion and false if an error occured.
\item[{\bf Author:}]  AP
\end{description}
\subsection{addPulldownMenu}
 
\begin{flushleft} \textbf{%
int  \\ 
\settowidth{\DialogDataIncludeArgIndent}{addPulldownMenu(}%
addPulldownMenu(const aString \&pdMainLabel, const aString commands[], const aString labels[], button\_type bt, \\ 
\hspace{\DialogDataIncludeArgIndent}int *initState  = NULL)
}\end{flushleft}
\begin{description}
\item[{\bf Description:}]  Add a pulldown menu to the dialog window. The pulldown menu will appear when the
 dialog window is displayed, i.e., after pushGUI has been called. Successive pulldown menus 
 will be stacked from left to right on the menu bar.

\item[{\bf pdMainLabel(input):}]  The label that will appear on the menu bar.

\item[{\bf commands(input):}]  An array of strings with the command that will be issued when each menu
  item is selected. The array must be terminated by an empty string ("").

\item[{\bf labels(input):}]  An array of strings with the label that will be put on each menu
  item. The array must be terminated by an empty string ("").

\item[{\bf bt(input):}]  The type of buttons in the menu. Can be either GI\_PUSHBUTTON or GI\_TOGGLEBUTTON.

\item[{\bf initState(input):}]  Optional argument that only is used when bt == GI\_TOGGLEBUTTON. This argument
  is an array that specifies the initial state of each toggle buttons. If this argument is absent
  when bt == GI\_TOGGLEBUTTON, no menu items are marked as beeing selected.

\item[{\bf Return values:}]  The function returns 1 on a successful completion and 0 if an error occured.

\item[{\bf Author:}]  AP
\end{description}
\subsection{changeOptionMenu}
 
\begin{flushleft} \textbf{%
bool  \\ 
\settowidth{\DialogDataIncludeArgIndent}{changeOptionMenu(}%
changeOptionMenu(const aString \& opMainLabel, const aString opCommands[], const aString opLabels[], int initCommand)
}\end{flushleft}
\begin{description}
\item[{\bf Description:}]  
    Change an option menu with a given name

\item[{\bf opMainLabel( input):}]  name of the option menu
\item[{\bf // /opCommands, opLabels (input);:}] // /opCommands, opLabels (input);
\item[{\bf initCommand (input) :}]  

\item[{\bf Return values:}]  None.
\item[{\bf Author:}]  wdh
\end{description}
\subsection{setWindowTitle}
 
\begin{flushleft} \textbf{%
void  \\ 
\settowidth{\DialogDataIncludeArgIndent}{setWindowTitle(}%
setWindowTitle(const aString \&title)
}\end{flushleft}
\begin{description}
\item[{\bf Description:}]  Set the title of the dialog window in the GUIState. The title
 will appear on the dialog window after pushGUI has been called.

\item[{\bf title(input):}]  The new title.

\item[{\bf Return values:}]  None.
\item[{\bf Author:}]  AP
\end{description}
\subsection{setOptionMenuColumns}
 
\begin{flushleft} \textbf{%
void  \\ 
\settowidth{\DialogDataIncludeArgIndent}{setOptionMenuColumns(}%
setOptionMenuColumns(int columns)
}\end{flushleft}
\begin{description}
\item[{\bf Description:}]  Set the number of columns in which the option menus should 
 be organized on the dialog window.
\item[{\bf columns(input):}]  The number of columns.
\item[{\bf Return values:}]  None.
\item[{\bf Author:}]  AP
\end{description}
\subsection{setLastPullDownIsHelp}
 
\begin{flushleft} \textbf{%
void  \\ 
\settowidth{\DialogDataIncludeArgIndent}{setLastPullDownIsHelp(}%
setLastPullDownIsHelp(int trueFalse)
}\end{flushleft}
\begin{description}
\item[{\bf Description:}]  Specify whether the last pulldown menu should appear in the
 right end of the menu bar on the dialog window, where the help menu often is
 located.
\item[{\bf trueFalse(input):}]  1 if the last pulldown menu should be placed in the right end. 
  Otherwise, the last pulldown menu is placed just to the right of the second last 
  pulldown menu 
\item[{\bf Return value:}]  None.
\item[{\bf Author:}]  AP
\end{description}
\subsection{getPulldownMenu}
 
\begin{flushleft} \textbf{%
PullDownMenu\&  \\ 
\settowidth{\DialogDataIncludeArgIndent}{getPulldownMenu(}%
getPulldownMenu(int n)
}\end{flushleft}
\begin{description}
\item[{\bf Description:}]  
    return the n'th pull-down menu, $0 \leq n < n_{pullDownMenu}$.
\item[{\bf Author:}]  WDH
\end{description}
\subsection{getPulldownMenu}
 
\begin{flushleft} \textbf{%
PullDownMenu\&  \\ 
\settowidth{\DialogDataIncludeArgIndent}{getPulldownMenu(}%
getPulldownMenu(const aString \& label)
}\end{flushleft}
\begin{description}
\item[{\bf Description:}]  
    Find the pulldown menu with the given main label.
\item[{\bf label (input) :}]  the label given to the pulldown
\item[{\bf Return values:}]  the pulldown menu with the given main label, return PulldownMenu 0
 if the label was not found.
\item[{\bf Author:}]  WDH
\end{description}
\subsection{getOptionMenu}
 
\begin{flushleft} \textbf{%
OptionMenu\&  \\ 
\settowidth{\DialogDataIncludeArgIndent}{getOptionMenu(}%
getOptionMenu(int n)
}\end{flushleft}
\begin{description}
\item[{\bf Description:}]  
    return the n'th option menu, $0 \leq n < n_{optionMenu}$.
\item[{\bf Author:}]  WDH
\end{description}
\subsection{getOptionMenu}
 
\begin{flushleft} \textbf{%
OptionMenu\&  \\ 
\settowidth{\DialogDataIncludeArgIndent}{getOptionMenu(}%
getOptionMenu(const aString \& opMainLabel)
}\end{flushleft}
\begin{description}
\item[{\bf Description:}]  
    Find the option menu with the given main label.
\item[{\bf opMainLabel (input) :}]  the label given to an option menu.
\item[{\bf Return values:}]  the OptionMenu with the given main label, return OptionMenu 0
 if the label was not found.
\item[{\bf Author:}]  WDH
\end{description}
\subsection{getRadioBox}
 
\begin{flushleft} \textbf{%
RadioBox\&  \\ 
\settowidth{\DialogDataIncludeArgIndent}{getRadioBox(}%
getRadioBox(int n)
}\end{flushleft}
\begin{description}
\item[{\bf Description:}]  
    return the n'th radio box, $0 \leq n < n_{radioBoxes}$.
\item[{\bf Author:}]  AP
\end{description}
\subsection{getRadioBox}
 
\begin{flushleft} \textbf{%
RadioBox\&  \\ 
\settowidth{\DialogDataIncludeArgIndent}{getRadioBox(}%
getRadioBox(const aString \& radioLabel)
}\end{flushleft}
\begin{description}
\item[{\bf Description:}]  
    Find the radio box menu with the given main label.
\item[{\bf radioLabell (input) :}]  the label given to an radio box
\item[{\bf Return values:}]  the RadioBox with the given label, return RadioBox 0
 if the label was not found.
\item[{\bf Author:}]  WDH
\end{description}
\subsection{getToggleValue}
 
\begin{flushleft} \textbf{%
bool  \\ 
\settowidth{\DialogDataIncludeArgIndent}{getToggleValue(}%
getToggleValue( const aString \& answer, const aString \& label, bool \& target )
}\end{flushleft}
\begin{description}
\item[{\bf Description:}]  
    If `answer' requests a change in a toggle state then set `target' and adjust the
    toggle state.
\item[{\bf Return values:}]  true if answer requested a change in a toggle state, return false oterwise.
\item[{\bf Author:}]  WDH
\end{description}
\subsection{getTextValue(real)}
 
\begin{flushleft} \textbf{%
bool  \\ 
\settowidth{\DialogDataIncludeArgIndent}{getTextValue(}%
getTextValue( const aString \& answer, const aString \& label, const aString \& format, real \& target )
}\end{flushleft}
\begin{description}
\item[{\bf Description:}]  
    If `answer' requests a change in a real text value with label=`label' 
      then set `target' and adjust the
    text label.
\item[{\bf answer (input) :}]  check this answer
\item[{\bf label (input):}]  check if answer is of the form "label ..."
\item[{\bf target (output) :}]  fill in this value if answer begins with "label"
\item[{\bf format (input) :}]  use this format to reset the text label field with a new value.
\item[{\bf return value:}]  true if found, false otherwise
\item[{\bf Author:}]  WDH
\end{description}
\subsection{getTextValue(int)}
 
\begin{flushleft} \textbf{%
bool  \\ 
\settowidth{\DialogDataIncludeArgIndent}{getTextValue(}%
getTextValue( const aString \& answer, const aString \& label, const aString \& format, int \& target )
}\end{flushleft}
\begin{description}
\item[{\bf Description:}]  
    If `answer' requests a change in a int text value with label=`label' 
      then set `target' and adjust the
    text label.
\item[{\bf answer (input) :}]  check this answer
\item[{\bf label (input):}]  check if answer is of the form "label ..."
\item[{\bf target (output) :}]  fill in this value if answer begins with "label"
\item[{\bf format (input) :}]  use this format to reset the text label field with a new value.
\item[{\bf return value:}]  true if found, false otherwise
\item[{\bf Author:}]  WDH
\end{description}
\subsection{getTextValue(string)}
 
\begin{flushleft} \textbf{%
bool  \\ 
\settowidth{\DialogDataIncludeArgIndent}{getTextValue(}%
getTextValue( const aString \& answer, const aString \& label, const aString \& format, aString \& target )
}\end{flushleft}
\begin{description}
\item[{\bf Description:}]  
    If `answer' requests a change in a string text value with label=`label' 
      then set `target' and adjust the
    text label.
\item[{\bf answer (input) :}]  check this answer
\item[{\bf label (input):}]  check if answer is of the form "label ..."
\item[{\bf target (output) :}]  fill in this value if answer begins with "label"
\item[{\bf format (input) :}]  use this format to reset the text label field with a new value.
\item[{\bf return value:}]  true if found, false otherwise
\item[{\bf Author:}]  WDH
\end{description}
\subsection{constructor}
 
\begin{flushleft} \textbf{%
\settowidth{\DialogDataIncludeArgIndent}{PullDownMenu(}% 
PullDownMenu()
}\end{flushleft}
\begin{description}
\item[{\bf Description:}]  default constructor
\end{description}
\subsection{setPullDownMenu}
 
\begin{flushleft} \textbf{%
bool  \\ 
\settowidth{\DialogDataIncludeArgIndent}{setPullDownMenu(}%
setPullDownMenu(const aString \&pdMainLabel, const aString commands[], const aString labels[], button\_type bt, \\ 
\hspace{\DialogDataIncludeArgIndent}int *initState  = NULL)
}\end{flushleft}
\begin{description}
\item[{\bf Description:}]  Fill in all fields of a pulldown menu object except menupane which will be set to NULL 
   and sensitive which will be set to true.
   This function can for example be used to setup the optionMenu argument to makeGraphicsWindow.

\item[{\bf pdMainLabel(input):}]  The label that will appear on the menu bar.

\item[{\bf commands(input):}]  An array of strings with the command that will be issued when each menu
  item is selected. The array must be terminated by an empty string ("").

\item[{\bf labels(input):}]  An array of strings with the label that will be put on each menu
  item. The array must be terminated by an empty string ("").

\item[{\bf bt(input):}]  The type of buttons in the menu. Can be either GI\_PUSHBUTTON or GI\_TOGGLEBUTTON.

\item[{\bf initState(input):}]  Optional argument that only is used when bt == GI\_TOGGLEBUTTON. This argument
  is an array that specifies the initial state of each toggle buttons. If this argument is absent
  when bt == GI\_TOGGLEBUTTON, no menu items are marked as beeing selected.

\item[{\bf Return values:}]  The function returns true on a successful completion and false if an error occured.

\item[{\bf Author:}]  AP
\end{description}
\subsection{setSensitive}
 
\begin{flushleft} \textbf{%
void  \\ 
\settowidth{\DialogDataIncludeArgIndent}{setSensitive(}%
setSensitive(int trueFalse)
}\end{flushleft}
\begin{description}
\item[{\bf Description:}]  Set the sensitivity of a DialogData object.
\item[{\bf trueOrFalse:}]  The new state of the DialogData widget
\item[{\bf Return valuse:}]  None
\item[{\bf Author:}]  AP \& WDH
\end{description}
\subsection{setSensitive}
 
\begin{flushleft} \textbf{%
void  \\ 
\settowidth{\DialogDataIncludeArgIndent}{setSensitive(}%
setSensitive(bool trueOrFalse, WidgetTypeEnum widgetType, int number )
}\end{flushleft}
\begin{description}
\item[{\bf Description:}] 
    Set the sensitivity of a widget in the DialogData
\item[{\bf trueOrFalse (input):}]  set senstive or not
\item[{\bf widgetType (input):}]  choose a widget type to assign. One of 
 \begin{verbatim}
  enum WidgetTypeEnum  
  {
    optionMenuWidget,
    pushButtonWidget,
    pullDownWidget,
    toggleButtonWidget,
    textBoxWidget,
    radioBoxWidget
 };
 \end{verbatim}
\item[{\bf number (input) :}]  set sensitivity for this widget. 
\end{description}
\subsection{setSensitive}
 
\begin{flushleft} \textbf{%
void  \\ 
\settowidth{\DialogDataIncludeArgIndent}{setSensitive(}%
setSensitive(bool trueOrFalse, WidgetTypeEnum widgetType, const aString \& label)
}\end{flushleft}
\begin{description}
\item[{\bf Description:}] 
    Set the sensitivity of a widget in the DialogData
\item[{\bf trueOrFalse (input):}]  set senstive or not
\item[{\bf widgetType (input):}]  choose a widget type to assign. One of 
 \begin{verbatim}
  enum WidgetTypeEnum  
  {
    optionMenuWidget,
    pushButtonWidget,
    pullDownWidget,
    toggleButtonWidget,
    textBoxWidget,
    radioBoxWidget
 };
 \end{verbatim}
\item[{\bf label (input) :}]  set sensitivity for the widget with this label
\end{description}
\subsection{changeOptionMenu}
 
\begin{flushleft} \textbf{%
bool  \\ 
\settowidth{\DialogDataIncludeArgIndent}{changeOptionMenu(}%
changeOptionMenu(int nOption, const aString opCommands[], const aString opLabels[], int initCommand)
}\end{flushleft}
\begin{description}
\item[{\bf Description:}]  Change the menu items in an option menu after it has been created (by pushGUI)

\item[{\bf nOption(input):}]  Change option menu \# nOption.
\item[{\bf opCommands(input):}]  An array of strings with the command that will be issued when each menu
  item is selected. The array must be terminated by an empty string ("").
\item[{\bf opLabels(input):}]  An array of strings with the label that will be put on each menu
  item. The array must be terminated by an empty string ("").
\item[{\bf initCommand(input):}]  The index of the initial selection in the opLabels array. This label will
  appear on top of the option menu to indicate the initial setting.
\item[{\bf Return values:}]  The function returns true on a successful completion and false if an error occured.
\item[{\bf Author:}]  AP
\end{description}
\subsection{showSibling}
 
\begin{flushleft} \textbf{%
int  \\ 
\settowidth{\DialogDataIncludeArgIndent}{showSibling(}%
showSibling()
}\end{flushleft}
\begin{description}
\item[{\bf Description:}]  Show a sibling (dialog) window that previously was allocated with 
 getDialogSibling() and created with pushGUI().

\item[{\bf Returnvalues:}]  The function returns 1 if the sibling could be shown, otherwise 0 
 (in which case it doesn't exist or already is shown).
\item[{\bf Author:}]  AP
\end{description}
\subsection{hideSibling}
 
\begin{flushleft} \textbf{%
int  \\ 
\settowidth{\DialogDataIncludeArgIndent}{hideSibling(}%
hideSibling() 
}\end{flushleft}
\begin{description}
\item[{\bf Description:}]  Hide a sibling (dialog) window that previously was allocated with 
 getDialogSibling(), created with pushGUI() and shown with showSibling().

\item[{\bf Returnvalues:}]  The function returns 1 if the sibling could be hidden, otherwise 0 
 (in which case it doesn't exist or already is hidden).
\item[{\bf Author:}]  AP
\end{description}
\subsection{setTextLabel}
 
\begin{flushleft} \textbf{%
int  \\ 
\settowidth{\DialogDataIncludeArgIndent}{setTextLabel(}%
setTextLabel(int n, const aString \&buff)
}\end{flushleft}
\begin{description}
\item[{\bf Description:}]  Set the text string in textlabel \# n in the currently
 active GUIState.

\item[{\bf n(input):}]  The index of the text label in the array given to setTextBoxes during setup.
\item[{\bf buff(input):}]  The new text string.

\item[{\bf Author:}]  AP
\end{description}
\subsection{setInfoLabel}
 
\begin{flushleft} \textbf{%
bool  \\ 
\settowidth{\DialogDataIncludeArgIndent}{setInfoLabel(}%
setInfoLabel(int n, const aString \&buff)
}\end{flushleft}
\begin{description}
\item[{\bf Description:}]  Set the text string in info label \# n in the currently
 active GUIState.

\item[{\bf n(input):}]  The index of the text label returned by addInfoLabel during the setup.
\item[{\bf buff(input):}]  The new text string.

\item[{\bf Return code:}]  true if the label could be changed successfully, otherwise false

\item[{\bf Author:}]  AP
\end{description}
\subsection{setToggleState}
 
\begin{flushleft} \textbf{%
int  \\ 
\settowidth{\DialogDataIncludeArgIndent}{setToggleState(}%
setToggleState(int n, int trueFalse)
}\end{flushleft}
\begin{description}
\item[{\bf Description:}]  Set the state of toggle button \# n in the currently
 active GUIState.

\item[{\bf n(input):}]  The index of the toggle button in the array given to setToggleButtons during setup.
\item[{\bf trueFalse(input):}]  trueFalse==1 turns the toggle button on, all other values turn it off.

\item[{\bf Author:}]  AP
\end{description}
\subsection{setToggleState}
 
\begin{flushleft} \textbf{%
int  \\ 
\settowidth{\DialogDataIncludeArgIndent}{setToggleState(}%
setToggleState( const aString \& toggleButtonLabel,  int trueOrFalse)
}\end{flushleft}
\begin{description}
\item[{\bf Description:}]  Set the toggle state for the toggle button with the given label.

\item[{\bf toggleButtonLabel(input):}]  The label of the toggle button to set.
\item[{\bf trueOrFalse(input):}]  The new state.

\item[{\bf Author:}]  wdh
\end{description}
