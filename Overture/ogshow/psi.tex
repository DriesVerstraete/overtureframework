\subsection{Constructors}
 
\begin{flushleft} \textbf{%
\newlength{\argIndent}\settowidth{\argIndent}{PlotStuff(}% 
PlotStuff()
}\end{flushleft}
\begin{Lentry}
\item[Description] 
   Default constructor; the default constructor will open a window
\item[Author]  WDH

\end{Lentry}

 
\begin{flushleft} \textbf{%
\settowidth{\argIndent}{PlotStuff(}% 
PlotStuff(int \& argc, char *argv[]) 
}\end{flushleft}
\begin{Lentry}
\item[Description] 
   This Constructor takes the argc and argv from the main program -- The GLUT 
   window manager will strip off any parameters that it recognizes such as the
   size of the window. See the GLUT manual for further details.

\item[argc (input/output)]  The argument count to main.
\item[argv (input/output)]  The arguments to main.

\end{Lentry}

 
\begin{flushleft} \textbf{%
\settowidth{\argIndent}{PlotStuff(}% 
PlotStuff( const bool initialize )
}\end{flushleft}
\begin{Lentry}
\item[Description] 
    This Constructor will only create a window if the the argument initialize=TRUE
    To create a window later call createWindow()

\item[initialize (input)]  If TRUE then a window will be created. If FALSE no window will
         be created and you will have to call {\ff createWindow} to make the window.

\item[Author]  WDH.  
\end{Lentry}
\subsection{Plot a MappedGrid}
 
\begin{flushleft} \textbf{%
\settowidth{\argIndent}{void plot(}%
void plot(const MappedGrid \& mg, \\ 
\hspace{\argIndent}      PlotStuffParameters \& parameters  = defaultPlotStuffParameters)
}\end{flushleft}
\begin{Lentry}
\item[Description] 
   Plot a MappedGrid and optionally supply parameters that define the plot characteristics.
   In two-dimensions grid-lines are plotted. 
   In three dimensions, by default only the block boundaries of the grid are plotted.
   You may also plot grid lines on boundaries and/or plot shaded boundary surfaces.

   Grids and boundary conditions are plotted with different colours. Grids are numbered
   and boundary conditions are numbered. For each number there corresponds a colour.
   The colour associated with each number is plotted in the lower left corner.
 

\item[mg (input)]  MappedGrid to plot.
\item[parameters (input/output)]  Supply optional parameters to alter plot characteristics.

\item[Author]  WDH

\end{Lentry}
\subsection{Contour a realMappedGridFunction}
 
\begin{flushleft} \textbf{%
\settowidth{\argIndent}{void contour(}%
void contour(const realMappedGridFunction \& u, \\ 
\hspace{\argIndent}         PlotStuffParameters \& parameters  = defaultPlotStuffParameters)
}\end{flushleft}
\begin{Lentry}

\item[Description] 
   Plot contours of a realMappedGridFunction in 2D or 3D.
  Optionally supply parameters that define the plot characteristics.
  In two-dimensions plotting options include
  \begin{itemize}
    \item plot shaded surface 
    \item plot (colour) contour lines
    \item plot wire mesh surface (hidden lines are not supported here due to
          limitations in OpenGL).
    \item choose which component to plot
  \end{itemize}
   In 3D options include
  \begin{itemize}
    \item plot shaded surface contours on arbitrary planes that cut through the grid
    \item plot shaded surface contours on boundaries.
    \item plot (colour) contour lines on the planes or boundaries
  \end{itemize}

\item[u (input)]  function to plot contours of
\item[parameters (input)]  supply optional parameters

\item[Author]  WDH

\end{Lentry}
\subsection{StreamLines of a realMappedGridFunction}
 
\begin{flushleft} \textbf{%
\settowidth{\argIndent}{void streamLines(}%
void streamLines(const realMappedGridFunction \& uv, \\ 
\hspace{\argIndent}             PlotStuffParameters \& parameters  = defaultPlotStuffParameters)
}\end{flushleft}
\begin{Lentry}

\item[Description] 
  Plot stream lines of a two-dimensional vector field.
  Optionally supply parameters that define the plot characteristics.
  This routine draws lines that are parallel to a vector field defined by
  two components of the grid function {\ff uv}. By default component values
  0 and 1 of the first component of {\ff uv} are used for ``u'' and ``v''.
  Plotting options include
  \begin{itemize}
    \item choose the components to use for ``u'' and ``v''.
    \item choose new plot bounds (to zoom in on a particular region). In this case
          new streamlines are drawn on the new region as opposed to the plot
          being simply magnified.
  \end{itemize}

\item[uv (input)]  function to plot streamlines of.
\item[parameters (input)]  supply optional parameters

\item[Remarks] 
  \begin{itemize}
    \item The streamlines are coloured by the relative value of $u^2+v^2$
    \item Streamlines that move too slowly are stopped  
    \item There is a maximum number of steps used to integrate any streamline.
    \item To plot streamlines to cover a CompositeGrid, I make a rectangular
          background grid that covers some region (this region could be
          smaller than the entire grid if we are zooming). The number of
          points on this grid is nxg*nyg. The intArray ig(nxg,nyg) is used
          to mark cells in the background grid. I draw streamlines starting
          at the midpoints of the background grid. Whenever a streamline
          pass through a cell of the background grid I increase the value
          of ig(i,j) by one. Only two streamlines are allowed per cell
          or else the streamline is stopped (or never started).
            In this way streamlines cover the domain in a reasonably
          uniform manner.
  \end{itemize}

\item[Author]  WDH

\end{Lentry}
\subsection{Plot a CompositeGrid}
 
\begin{flushleft} \textbf{%
\settowidth{\argIndent}{void plot(}%
void plot(const CompositeGrid \& cg, \\ 
\hspace{\argIndent}      PlotStuffParameters \& parameters  = defaultPlotStuffParameters)
}\end{flushleft}
\begin{Lentry}
\item[Description] 
   Plot a CompositeGrid and optionally supply parameters that define the plot characteristics.
   In two-dimensions grid-lines are plotted. Interpolation points can be plotted with
   small circles.
   In three dimensions, by default only the block boundaries of the grid are plotted.
   You may also plot grid lines on boundaries and/or plot shaded boundary surfaces.
   

   Grids and boundary conditions are plotted with different colours. Grids are numbered
   and boundary conditions are numbered. For each number there corresponds a colour.
   The colour associated with each number is plotted in the lower left corner.
 

\item[cg (input)]  CompositeGrid to plot.
\item[parameters (input/output)]  Supply optional parameters to alter plot characteristics.

\item[Author]  WDH

\end{Lentry}
