\subsubsection{getVariableName}
 
\newlength{\NameListIncludeArgIndent}
\begin{flushleft} \textbf{%
void  \\ 
\settowidth{\NameListIncludeArgIndent}{getVariableName(}%
getVariableName( aString \& answer, aString \& name )
}\end{flushleft}
\begin{description}
\item[{\bf Description:}] 
   Parse the aString "answer" and return the variable name:
 \begin{verbatim}
     answer: "method=4"         -> name="method"
           : "array(5,6)=56.7"  -> name="array"
 \end{verbatim}
\item[{\bf answer (input) :}]  string to parse.
\item[{\bf name (output) :}]  string before the first "=" sign
\end{description}
\subsubsection{intValue}
 
\begin{flushleft} \textbf{%
int  \\ 
\settowidth{\NameListIncludeArgIndent}{intValue(}%
intValue( aString \& answer )
}\end{flushleft}
\begin{description}
\item[{\bf Description:}] 
  Return an int for a string of the form "name=int"
\item[{\bf answer (input) :}]  a aString of the form "name=int"
\item[{\bf Return value:}]  The value of the rhs in "name=int"
\end{description}
\subsubsection{realValue}
 
\begin{flushleft} \textbf{%
real  \\ 
\settowidth{\NameListIncludeArgIndent}{realValue(}%
realValue( aString \& answer )
}\end{flushleft}
\begin{description}
\item[{\bf Description:}] 
  Return a real for a string of the form "name=real"
\item[{\bf answer (input) :}]  a aString of the form "name=real"
\item[{\bf Return value:}]  The value of the rhs in "name=real"
\end{description}
\subsubsection{getIntArray}
 
\begin{flushleft} \textbf{%
int  \\ 
\settowidth{\NameListIncludeArgIndent}{getIntArray(}%
getIntArray( aString \& answer, IntegerArray \& a )
}\end{flushleft}

 
\begin{flushleft} \textbf{%
int  \\ 
\settowidth{\NameListIncludeArgIndent}{getIntArray(}%
getIntArray( aString \& answer, IntegerArray \& a, int \& i0 )
}\end{flushleft}

 
\begin{flushleft} \textbf{%
int  \\ 
\settowidth{\NameListIncludeArgIndent}{getIntArray(}%
getIntArray( aString \& answer, IntegerArray \& a, int \& i0, int \& i1 )
}\end{flushleft}

 
\begin{flushleft} \textbf{%
int  \\ 
\settowidth{\NameListIncludeArgIndent}{getIntArray(}%
getIntArray( aString \& answer, IntegerArray \& a, int \& i0, int \& i1, int \& i2 )
}\end{flushleft}

 
\begin{flushleft} \textbf{%
int  \\ 
\settowidth{\NameListIncludeArgIndent}{getIntArray(}%
getIntArray( aString \& answer, IntegerArray \& a, int \& i0, int \& i1, int \& i2, int \& i3 )
}\end{flushleft}
\begin{description}
\item[{\bf Description:}]  
 Assign the value in an IntegerArray from a string of
 one of the following forms
 \begin{verbatim}
      name=value
      name(i0)=value  
      name(i0,i1)=value
      name(i0,i1,i2)=value
      name(i0,i1,i2,i3)=value
 \end{verbatim}
\item[{\bf answer (input) :}]  a aString of one of the above forms
\item[{\bf a (output) :}]  an array that is to be assigned
\item[{\bf i0,i1,i2,i3 (output) :}]  Return the values for the indices used in evaluating the array.
\item[{\bf Return value:}]  Return TRUE if successful
\end{description}
\subsubsection{getRealArray}
 
\begin{flushleft} \textbf{%
int  \\ 
\settowidth{\NameListIncludeArgIndent}{getRealArray(}%
getRealArray( aString \& answer, RealArray \& a )
}\end{flushleft}

 
\begin{flushleft} \textbf{%
int  \\ 
\settowidth{\NameListIncludeArgIndent}{getRealArray(}%
getRealArray( aString \& answer, RealArray \& a, int \& i0 )
}\end{flushleft}

 
\begin{flushleft} \textbf{%
int  \\ 
\settowidth{\NameListIncludeArgIndent}{getRealArray(}%
getRealArray( aString \& answer, RealArray \& a, int \& i0, int \& i1 )
}\end{flushleft}

 
\begin{flushleft} \textbf{%
int  \\ 
\settowidth{\NameListIncludeArgIndent}{getRealArray(}%
getRealArray( aString \& answer, RealArray \& a, int \& i0, int \& i1, int \& i2 )
}\end{flushleft}
\begin{description}
\item[{\bf Description:}]  Assign values of a RealArray. see the documentation for
  getIntArray.
\end{description}

 
\begin{flushleft} \textbf{%
int  \\ 
\settowidth{\NameListIncludeArgIndent}{getRealArray(}%
getRealArray( aString \& answer, RealArray \& a, int \& i0, int \& i1, int \& i2, int \& i3 )
}\end{flushleft}
\begin{description}
\item[{\bf Description:}]  Assign values of a RealArray. see the documentation for
  getIntArray.
\end{description}
\subsubsection{intArrayValue}
 
\begin{flushleft} \textbf{%
int  \\ 
\settowidth{\NameListIncludeArgIndent}{intArrayValue(}%
intArrayValue( aString \& answer, int \& value, int \& i0 )
}\end{flushleft}

 
\begin{flushleft} \textbf{%
int  \\ 
\settowidth{\NameListIncludeArgIndent}{intArrayValue(}%
intArrayValue( aString \& answer, int \& value, int \& i0 , int \& i1)
}\end{flushleft}

 
\begin{flushleft} \textbf{%
int  \\ 
\settowidth{\NameListIncludeArgIndent}{intArrayValue(}%
intArrayValue( aString \& answer, int \& value, int \& i0, int \& i1, int \& i2)
}\end{flushleft}

 
\begin{flushleft} \textbf{%
int  \\ 
\settowidth{\NameListIncludeArgIndent}{intArrayValue(}%
intArrayValue( aString \& answer, int \& value, int \& i0, int \& i1, int \& i2, int \& i3)
}\end{flushleft}
\begin{description}
\item[{\bf Description:}] 
 Return value and indices i0,i1,i2,i3 from a string of the form
 \begin{verbatim}
   name(i0)         =value  : intArrayValue(answer, value, i0 )
   name(i0,i1)      =value  : intArrayValue(answer, value, i0, i1)
   name(i0,i1,i2)   =value  : intArrayValue(answer, value, i0, i1, i2)
   name(i0,i1,i2,i3)=value  : intArrayValue(answer, value, i0, i1, i2, i3)
 \end{verbatim}
\item[{\bf answer (input) :}]  string to parse.
\item[{\bf value (output) :}]  value found on the rhs of the string
\item[{\bf i0,i1,i2,i3 (output) :}]  index values
\item[{\bf Return value :}]   Return TRUE if successful
\end{description}
\subsubsection{realArrayValue}
 
\begin{flushleft} \textbf{%
int  \\ 
\settowidth{\NameListIncludeArgIndent}{realArrayValue(}%
realArrayValue( aString \& answer, real \& value, int \& i0 )
}\end{flushleft}

 
\begin{flushleft} \textbf{%
int  \\ 
\settowidth{\NameListIncludeArgIndent}{realArrayValue(}%
realArrayValue( aString \& answer, real \& value, int \& i0 , int \& i1)
}\end{flushleft}

 
\begin{flushleft} \textbf{%
int  \\ 
\settowidth{\NameListIncludeArgIndent}{realArrayValue(}%
realArrayValue( aString \& answer, real \& value, int \& i0, int \& i1, int \& i2)
}\end{flushleft}

 
\begin{flushleft} \textbf{%
int  \\ 
\settowidth{\NameListIncludeArgIndent}{realArrayValue(}%
realArrayValue( aString \& answer, real \& value, int \& i0, int \& i1, int \& i2, int \& i3)
}\end{flushleft}
\begin{description}
\item[{\bf Description:}] 
 Return value and indices i0,i1,i2,i3 from a string of the form
 \begin{verbatim}
   name(i0)         =value  : realArrayValue(answer, value, i0 )
   name(i0,i1)      =value  : realArrayValue(answer, value, i0, i1)
   name(i0,i1,i2)   =value  : realArrayValue(answer, value, i0, i1, i2)
   name(i0,i1,i2,i3)=value  : realArrayValue(answer, value, i0, i1, i2, i3)
 \end{verbatim}
\item[{\bf answer (input) :}]  string to parse.
\item[{\bf value (output) :}]  value found on the rhs of the string
\item[{\bf i0,i1,i2,i3 (output) :}]  index values
\item[{\bf Return value :}]   Return TRUE if successful
\end{description}
\subsubsection{arrayEqualsName}
 
\begin{flushleft} \textbf{%
int  \\ 
\settowidth{\NameListIncludeArgIndent}{arrayEqualsName(}%
arrayEqualsName(aString \& answer, \\ 
\hspace{\NameListIncludeArgIndent}const aString nameList[], \\ 
\hspace{\NameListIncludeArgIndent}IntegerArray \& a, \\ 
\hspace{\NameListIncludeArgIndent}int \& i0 optional argument,\\ 
\hspace{\NameListIncludeArgIndent}int \& i1 optional argument,\\ 
\hspace{\NameListIncludeArgIndent}int \& i2 optional argument,\\ 
\hspace{\NameListIncludeArgIndent}int \& i3   optional argument)
}\end{flushleft}
\begin{description}
\item[{\bf Description:}] 
    The aString answer should be of the form of one of
    \begin{itemize}
       \item arrayName(i0)=name
       \item arrayName(i0,i1)=name
       \item arrayName(i0,i1,i2)=name
       \item arrayName(i0,i1,i2,i3)=name
    \end{itemize}
   and the result of this function will be to set
    \begin{itemize}
       \item a(i0)=value where nameList[value]==name
       \item a(i0,i1)=value where nameList[value]==name
       \item a(i0,i1,i2)=value where nameList[value]==name
       \item a(i0,i1,i2,i3)=value where nameList[value]==name
    \end{itemize}

    
\item[{\bf answer (input) :}]  a aString that should be of the form shown above.
\item[{\bf nameList (input) :}]  a null terminated array of names. These names will
    appear on the right hand side of the equals sign.
\item[{\bf a (output):}]  assign a value into this array.
\item[{\bf i0,i1,i2,i3 (ouput) :}]  optional arguments, return the values used in assigning a.
\item[{\bf Return values:}]  return TRUE if successful
\end{description}
\subsubsection{arrayOfNameEqualsValue}
 
\begin{flushleft} \textbf{%
int  \\ 
\settowidth{\NameListIncludeArgIndent}{arrayOfNameEqualsValue(}%
arrayOfNameEqualsValue(aString \& answer, \\ 
\hspace{\NameListIncludeArgIndent}const aString nameList[], \\ 
\hspace{\NameListIncludeArgIndent}IntegerArray \& a, \\ 
\hspace{\NameListIncludeArgIndent}int \& i0 optional argument,\\ 
\hspace{\NameListIncludeArgIndent}int \& i1 optional argument,\\ 
\hspace{\NameListIncludeArgIndent}int \& i2 optional argument,\\ 
\hspace{\NameListIncludeArgIndent}int \& i3   optional argument)
}\end{flushleft}
\begin{description}
\item[{\bf Description:}] 
    The aString answer should be of the form of one of
    \begin{itemize}
       \item arrayName(name0)=value
       \item arrayName(name0,name1)=value
       \item arrayName(name0,name1,name2)=value
       \item arrayName(name0,name1,name2,name3)=value
    \end{itemize}
   and the result of this function will be to set
    \begin{itemize}
       \item a(i0)=value where nameList[i0]==name0
       \item a(i0,i1)=value where nameList[i0]==name0, nameList[i1]==name1
       \item a(i0,i1,i2)=value where nameList[i0]==name0, nameList[i1]==name1,...
       \item a(i0,i1,i2,i3)=value where nameList[i0]==name0, nameList[i1]==name1,...
    \end{itemize}

    
\item[{\bf answer (input) :}]  a aString that should be of the form shown above.
\item[{\bf nameList (input) :}]  a null terminated array of names. These names will
    appear as array arguements in answer.
\item[{\bf a (output):}]  assign a value into this array.
\item[{\bf i0,i1,i2,i3 (ouput) :}]  optional arguments, return the values used in assigning a.
\item[{\bf Return values:}]  return TRUE if successful
\end{description}
\subsection{arrayOfNameEqualsValue}
 
\begin{flushleft} \textbf{%
int  \\ 
\settowidth{\NameListIncludeArgIndent}{arrayOfNameEqualsValue(}%
arrayOfNameEqualsValue(aString \& answer, \\ 
\hspace{\NameListIncludeArgIndent}const aString nameList[], \\ 
\hspace{\NameListIncludeArgIndent}RealArray \& a, \\ 
\hspace{\NameListIncludeArgIndent}int \& i0, \\ 
\hspace{\NameListIncludeArgIndent}int \& i1, \\ 
\hspace{\NameListIncludeArgIndent}int \& i2, \\ 
\hspace{\NameListIncludeArgIndent}int \& i3)
}\end{flushleft}
\begin{description}
\item[{\bf Description:}] 
    The aString answer should be of the form of one of
    \begin{itemize}
       \item arrayName(name0)=value
       \item arrayName(name0,name1)=value
       \item arrayName(name0,name1,name2)=value
       \item arrayName(name0,name1,name2,name3)=value
    \end{itemize}
   and the result of this function will be to set
    \begin{itemize}
       \item a(i0)=value where nameList[i0]==name0
       \item a(i0,i1)=value where nameList[i0]==name0, nameList[i1]==name1
       \item a(i0,i1,i2)=value where nameList[i0]==name0, nameList[i1]==name1,...
       \item a(i0,i1,i2,i3)=value where nameList[i0]==name0, nameList[i1]==name1,...
    \end{itemize}

    
\item[{\bf answer (input) :}]  a aString that should be of the form shown above.
\item[{\bf nameList (input) :}]  a null terminated array of names. These names will
    appear as array arguements in answer.
\item[{\bf a (output):}]  assign a value into this array.
\item[{\bf i0,i1,i2,i3 (ouput) :}]  optional arguments, return the values used in assigning a.
\item[{\bf Return values:}]  return TRUE if successful
\end{description}
