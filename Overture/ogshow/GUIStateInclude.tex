\subsection{Constructor}
 
\newlength{\GUIStateIncludeArgIndent}
\begin{flushleft} \textbf{%
\settowidth{\GUIStateIncludeArgIndent}{GUIState(}% 
GUIState()
}\end{flushleft}
\begin{description}
\item[{\bf Description:}]  Default constructor.
\item[{\bf Author:}]  AP
\end{description}
\subsection{setUserMenu}
 
\begin{flushleft} \textbf{%
void  \\ 
\settowidth{\GUIStateIncludeArgIndent}{setUserMenu(}%
setUserMenu(const aString menu[], const aString \& menuTitle)
}\end{flushleft}
\begin{description}
\item[{\bf Description:}] 
    Sets up a user defined pulldown menu in the graphics windows. The menu will
    appear after the routine pushGUI() has been called.
\item[{\bf menu (input):}] 
    The {\ff menu} is
    an array of aStrings (the menu choices) with an empty aString
    indicating the end of the menu choices. If menu == NULL, any existing user defined 
    menu will be removed.
\item[{\bf menuTitle (input):}]  A aString with the menu title that will appear on the menu bar. Note 
    that a menuTitle must be provided even when menu == NULL.
\item[{\bf Return value:}]  none.
\item[{\bf Author:}]  AP
\end{description}
\subsection{setUserButtons}
 
\begin{flushleft} \textbf{%
void  \\ 
\settowidth{\GUIStateIncludeArgIndent}{setUserButtons(}%
setUserButtons(const aString buttons[][2])
}\end{flushleft}
\begin{description}
\item[{\bf Description:}] 
    This function builds user defined push buttons in the graphics windows. The buttons will
    appear after the routine pushGUI() has been called.
\item[{\bf buttons (input):}]  A two-dimensional array of Strings, terminated by an empty aString.
    For example,
    \begin{verbatim}
    aString buttons[][2] = {{"plot shaded surfaces", "Shade"}, 
                           {"erase",                "Erase"},
                           {"exit",                 "Exit"},
                           {"",                     ""}};
    \end{verbatim}
 The first entry in each row is the aString that will be passed as a command when the 
 button is pressed. The second aString in each row is the name of the button that will
 appear on the graphics window. There can be at most MAX\_BUTTONS buttons, where 
 MAX\_BUTTONS is defined in mogl.h, currently to 15.

 If buttons == NULL, any existing buttons will be removed 
 from the current window.
  

\item[{\bf Return value:}]  none.
\item[{\bf Author:}]  AP
\end{description}
\subsection{buildPopup}
 
\begin{flushleft} \textbf{%
void  \\ 
\settowidth{\GUIStateIncludeArgIndent}{buildPopup(}%
buildPopup(const aString menu[])
}\end{flushleft}
\begin{description}
\item[{\bf Description:}] 
    Sets up a user defined popup menu in all graphics windows and in the command window. 
    The menu will appear after the routine pushGUI() has been called.
\item[{\bf menu (input):}]  
    The {\ff menu} is
    an array of Strings (the menu choices) with an empty aString
    indicating the end of the menu choices. Optionally, a title can be put on top of the menu by 
    starting the first aString with an `!'. For example,
    \begin{verbatim}
       PlotStuff ps;
       aString menu[] = { "!MenuTitle",
                         "plot",
                         "erase",
                         "exit",
                         "" };
       aString menuItem;
       int i=ps.getMenuItem(menu,menuItem);
    \end{verbatim}

  To create a cascading menu, begin the string with an '$>$'.
  To end the cascade begin the string with an '$<$'.
  To end a cascade and start a new cascade, begin the string with '$<$' followed by '$>$'.
  Here is an example:
    \begin{verbatim}
        char *menu1[] = {  "!my title",
                           "plot",
                           ">component",
                                        "u",
                                        "v",
                                        "w",
                           "<erase",
                           ">stuff",
                                    "s1",
                                    ">more stuff", 
                                                  "more1",
                                                  "more2", 
                                    "<s2", 
                          "<>apples", 
                                    "apple1", 
                          "<exit",
                          NULL };  
    \end{verbatim}

\item[{\bf Return value:}]  none.
\item[{\bf Author:}]  AP
\end{description}
\subsection{getDialogSibling}
 
\begin{flushleft} \textbf{%
DialogData \&  \\ 
\settowidth{\GUIStateIncludeArgIndent}{getDialogSibling(}%
getDialogSibling(int number  =-1)
}\end{flushleft}
\begin{description}
\item[{\bf Description:}]  If number==-1 (default), allocate a sibling (dialog) window, otherwise return
   sibling \# 'number'. Note that the sibling window
 will appear on the screen after pushGUI() has been called for this (GUIState) object 
 and showSibling() has been called for the DialogData object returned from this function.
 See the DialogData function description for an example.

\item[{\bf number (input):}]  by default return a new sibling (if number==-1), otherwise return the
    sibling specified by number.
\item[{\bf Returnvalues:}]  The function returns an alias to the DialogData object. There 
  is currently space for 10 siblings (0,...,9) for each GUIState object.

\item[{\bf Author:}]  AP
\end{description}
