\section{Integrate: integrate grid functions on overlapping grids} \label{Integrate}
\index{integrate!grid functions on overlapping grids}
\index{surface integrals}\index{volume integrals}

The {\tt Integrate} class has functions that can be used to integrate a grid
function over a domain or over the boundary (or a subset of the boundary).
For example, one may want to compute the total mass found in a domain 
or compute the force on a body.

Integrating a function on an overlapping grid is non-trivial since care must
be taken in the region where grids overlap.

% The method used by the Integrate Class involves the computation 

The most important member functions are
\begin{description}
  \item[volumeIntegral(u)] : compute the volume integral of a RealCompositeGridFunction u.
  \item[surfaceIntegral(u)] : compute the surface integral of a RealCompositeGridFunction u.
  \item[defineSurface(s,...)] : define a sub-surface `s' as a collection of sides of grids. For example,
     for the `sphere in a box' grid you could define a sub-surface that represents the two surface
     grids on the sphere.
  \item[surfaceIntegral(u,s)] : compute the surface integral on the surface `s'.
\end{description}


The integral of a function $f$ over a domain $\Omega$ in $\Real^d$ is
\[
  \mathcal{I} = \int_\Omega f(\xv) ~d\xv
\]
If the domain is parameterized by $\xv=\xv(\rv)$ for $\rv\in [0,1]^d$ then
\begin{align*}
  \mathcal{I} &= \int_{[0,1]^d}  f(\xv(\rv)) \left\vert {\partial \xv \over \partial \rv}\right\vert ~d\rv \\
              &= \int_{[0,1]^d}  f(\xv(\rv)) J(\rv) ~d\rv \\
              &= \int_{[0,1]^d}  F(\rv)~d\rv
\end{align*}
Thus the integral on a curvilinear grid can be converted to an integral on a uniform grid
by integrating the function $F(\rv) = f(\xv(\rv)) J(\rv)$.


Consider the one-dimensional case
\[
    \mathcal{I} = \int_0^1 F(r) dr
\]
This integral can, for example, be approximated to second-order accuracy by the trapezoidal rule
\begin{align*}
    \int_0^1 F(r) dr &= \sum_{i=0}^{N-1}  {1\over 2}( F_{i+1} + F_i ) \Delta r + O(\Delta r^2) \\
                     &= {1\over 2} F_0\Delta r + \sum_{i=1}^{N-1} F_i \Delta r + {1\over 2} F_N\Delta r+ O(\Delta r^2)
\end{align*}
where $r_i=i\Delta r$, $i=0,1,2,\ldots,N$ and $\Delta r = 1/N$ and $F_i=F(r_i)$.




If the region $\Omega$ is covered by an overlapping grid, the situation is more complicated.
One cannot simply add contributions from each component grid independently since where the grids
overlap some contributions will be added multiple times.

The Integrate class provides two approaches for integration in the case of an overlapping grid
\begin{description}
  \item[Laplace approximation] : the integral is approximated using the left null vector
     of a discretization to the Laplace equation with Neumann boundary conditions. The left null vector,
    appropriately scaled, provides integration weights for each grid point. In this case there is no need
    to eliminate the region of overlap.
  \item[Hybrid approximation] : in this method the overlap is removed, leaving a gap, and an unstructured grid
     is generated to fill the gap. An integration formula is defined on the hybrid grid. This formula
     is transformed into weights applied to the points on the original overlapping grids. NOTE: Currently this
    option is only available for surface grids in 3D.
\end{description}


% ------------------------------------------------------------------------
\newcommand{\Sc}{\mathcal{S}}
\subsection{Surface Integrals}

Assume that we are given a mapping for a volume, $\xv=\Gv(\rv)$, with $\rv=(r,s,t)$ being the 
unit cube coordinates.

We want to integrate a function $f(\xv)$ on a surface, $\Sc$
\begin{align*}
  \int_\Sc f(\xv) d\xv  &= \int_\Sc f(\xv) (\xv_r\times\xv_s)\cdot \nv ~d\rv 
         \equiv \int_\Sc f(\xv(\rv)) \gamma(\rv) ~d\rv
\end{align*}
where $\gamma\equiv(\xv_r\times\xv_s)\cdot \nv$.
Assume we a given a tesselation for $\Sc = \bigcup_{k=1}^N \sigma_k$  
for the surface consisting of triangles and quadrilaterals ($\sigma_k$ represents either a triangle
or quadrilateral on the surface).
Then
\begin{align*}
  \int_\Sc f(\xv) d\xv  &= \sum_k \int_{\sigma_k} f(\xv) \gamma(\rv) ~d\rv \\
\end{align*}
We must now define approximations to surface integrals over triangles or quaridlaterals. 

For triangles we can define the approximation
\begin{align}
\int_{\sigma_k} f(\xv) \gamma(\rv) ~d\rv 
         \approx I_1 \equiv {1\over M}\left\{\sum_{i=1}^M f(\xv_i) \gamma(\xv_i)\right\} ~\Delta_\rv(\sigma_k)
\end{align}
where $\{ \xv_i \}_{i=1}^M$ are the $M=3$ vertices of $\sigma_k$ and 
where $\Delta_\rv(\sigma_k)$ is the area of $\sigma_k$ in the $(r,s)$ plane. 
Some other possible approximations are
\begin{align}
\int_{\sigma_k} f(\xv) \gamma(\rv) ~d\rv 
        & \approx I_2 \equiv f(\xv_b) \gamma(\xv_b)~\Delta_\rv(\sigma_k)  \\
        &  \approx I_3 \equiv {1\over M}\left\{ \sum_{i=1}^M f(\xv_i) \right\} ~\Delta_\xv(\sigma_k) \\
        &  \approx I_4 \equiv f(\xv_b)~\Delta_\xv(\sigma_k)
\end{align}
where $\xv_b$ is the barycenter (centroid) and $\Delta_\xv(\sigma_k)$ is the area $\sigma_k$ in the $\xv$ plane. .
Apparently all the approximations $I_m$ lead to globaly second-order accurate approximations
to the surface integral~\cite{GeorgTausch93}. Note that $I_3$ and $I_4$ do not require the mapping derivatives. 


\subsubsection{Higher-order approximations to integrals}

  High-order approximations to integrals can be defined by first introducing additional nodes
on each element $\sigma$. A polynomial interpolant, $I_{\sigma}(f)$, can be defined in terms of these new
degrees of freedom. One can define an interpolant for the metric term as well, $\gamma\approx I_{\sigma}(\gamma)$. The
approximation to the integral on $\sigma$ can then be defined as the exact integral of the
interpolants,
\begin{align}
  \int_{\sigma} f(\xv) \gamma(\rv) ~d\rv  \approx \int_{\sigma} I_{\sigma}(f) I_{\sigma}(\gamma) ~d\rv~.
\end{align}

Suppose that we are initially only given the values of $f(\xv)$ on some set of
points (e.g. the nodes of the elements).  The values of $f$ can be obtained at the additional 
nodes introduced above by defining another interpolant that spans multiple elements. Alternatively, if we know the
derivatives of $f$ at points on the element then an Hermite interpolant $I_{\sigma}(f)$ can be defined.



\subsubsection{Relating the surface area element to the volume element}

We can related the volume Jacobian $J$ to the surface area function, 
$\gamma=(\xv_r\times\xv_s)\cdot \nv$, 
as follows. 
% 
The volume Jacobian is $J=\vert \partial \xv/\partial \rv \vert$,
or $J=(\xv_r\times\xv_s)\cdot\xv_t$ (assuming a righ-handed coordinate system). 
Consider a surface $t=0$ with normal $\nv=\grad_\xv t/\vert \grad_\xv t\vert$.
The surface area function is 
\begin{align*}
    \gamma &= (\xv_r\times\xv_s)\cdot \nv 
           = (\xv_r\times\xv_s)\cdot \grad_\xv t/\vert \grad_\xv t \vert
\end{align*}         
But $\xv_t = \alpha \nv + \beta \xv_r + \gamma \xv_s$ where 
\begin{align*}
  \alpha&=\xv_t\cdot\nv =\xv_t\cdot\grad_\xv t/\vert \grad_\xv\vert =1/\vert \grad_\xv t\vert~.
\end{align*} 
We have used $\xv_t\cdot\grad_\xv t =1$ (since $\partial t(\xv)/\partial t=1$). 
Thus
\begin{align*}
   J & =(\xv_r\times\xv_s)\cdot\xv_t 
     = (\xv_r\times\xv_s)\cdot(\alpha \nv + \beta \xv_r + \gamma \xv_s) \\
     &= \alpha (\xv_r\times\xv_s)\cdot\nv \\
     &= \alpha \gamma
\end{align*}
Therefore $\gamma = J \vert \grad_\xv t\vert$.


\subsubsection{Integration weights for an overlapping surface grid from a stitched hybrid grid}

In order to compute integrals on an overlapping surface mesh we can first build a hybrid mesh that
fills a gap between the surface grids with an unstructured grid. The function {\tt SurfaceStitcher}
will perform this operation. 

Given a hybrid mesh for the surface we can define a numerical formula (quadrature)
of the form 
\begin{align}
  \int_\Sc f(\xv) d\xv  &\approx  \sum_j f(\xv_j) w_j \label{eq:quadrature}
\end{align}
where the function is evaluated at some points $\xv_j$ (e.g. cell centroids or cell nodes) 
and where $w_j$ are some weights. 

From this formula, we can define another quadrature that only uses values of $f$ on the
grid points of the original overlapping (structured) grid, denoted by $f_\iv^k$ for a point $\iv$ 
on grid $k$. 
To do this we determine how to interpolate $f(\xv_j)$ from valid points on the overlapping
surface grid, 
\begin{align}
  f(\xv_j) \approx \sum_{\jv\in J_j} \alpha_{\jv}^j f_{\jv}^k  ~. \label{eq:interp}
\end{align}
For example if $\xv_j$ is the centroid of a quadrilateral on the original structured grid then
$f(\xv_j)$ could be the average of the 4 neighbours. 
Substituting the interpolation formula~\eqref{eq:interp}
for $f(\xv_j)$ into the quadrature formula~\eqref{eq:quadrature}
results in a new quadrature formula for the overlapping grid:
\begin{align*}
  \int_\Sc f(\xv) d\xv  &\approx  \sum_{\jv} f_{\jv}^k \hat{w}_{\jv}
\end{align*}




\clearpage
% -----------------------------------------------------------------------------
\subsection{Results}

Table~\ref{tab:cic} shows some results for the circle-in-a-channel grid which
is a circle of radius ${1\over2}$ embedded in an square $[-2,2]^2$.
Note that the surface area of the embedded circle is computed very accurately since 
the trapezoidal rule is exponentially accurate on a periodic region.

Table~\ref{tab:cic} also shows results for a sphere-in-a-box grid.


% ************ sib1 ***********
% 
% volume = 6.362080e+01
% surfaceArea = 9.893653e+01
% Error in volume = 1.444027e-01
% Error in surface area = 2.050587e-01
% 
% *********** sib2 **********
% 
% volume = 6.350912e+01
% surfaceArea = 9.908690e+01
% Error in volume = 3.272155e-02
% Error in surface area = 5.469631e-02
% 
% *********** sib2x2 ***********
% 
% volume = 6.348473e+01
% surfaceArea = 9.912758e+01
% Error in volume = 8.332455e-03
% Error in surface area = 1.401757e-02


\begin{table}[hbt]
\begin{center}
\begin{tabular}{|c|c|c|c|} \hline 
  grid  &  $h_0/h$   & err-vol       & err-surf    \\   \hline\hline 
 cic2   &    1       &  $6.71e-2$    & $3.6e-15$     \\ 
 cic3   &    2       &  $1.59e-2$    & $1.1e-14$  \\ 
 cic4   &    4       &  $4.24e-3$    & $1.8e-14$     \\ \hline 
\end{tabular}	
\qquad
\begin{tabular}{|c|c|c|c|} \hline 
  grid  &  $h_0/h$   & err-vol       & err-surf    \\   \hline\hline 
 sib1   &    1       &  $1.44e-1$    & $2.05e-1$     \\ 
 sib2   &    2       &  $3.27e-2$    & $5.47e-2$  \\ 
 sib2x2 &    4       &  $8.33e-3$    & $1.40e-2$     \\ \hline 
\end{tabular}	
\qquad

\end{center}		
\caption{Left: Errors in computing the volume and surface-area of a circle-in-a-channel grid which
is a circle of radius ${1\over2}$ embedded in a square $[-2,2]^2$. The 
volume is $16-\pi/4\approx 15.2146$ while the surface-area is $16+\pi\approx 19.14159$.
Right: Errors in computing the volume and surface area for a sphere-in-a-box grid which
consists of a sphere of radius ${1\over2}$ embedded in a box $[-2,2]^2$. These results
were computed with the left-null vector approach.       }
 \label{tab:cic} 
\end{table}

% Grid sibe2.order2.hdf: surfaceArea for sphere = 3.131655e+00, (true=3.14159) error=9.9e-03
%  Integral(x0)   =  2.8384e-03, true= 0.0000e+00, err=2.8e-03
%  Integral(x0^2) =  2.6222e-01, true= 2.6180e-01, err=4.2e-04
%  Integral(x1)   = -1.0083e-03, true= 0.0000e+00, err=1.0e-03
%  Integral(x1^2) =  2.6192e-01, true= 2.6180e-01, err=1.3e-04
%  Integral(x2)   = -1.3129e-02, true= 0.0000e+00, err=1.3e-02
%  Integral(x2^2) =  2.5877e-01, true= 2.6180e-01, err=3.0e-03
% 
% Grid sibe4.order2.hdf: surfaceArea for sphere = 3.139267e+00, (true=3.14159) error=2.3e-03
%  Integral(x0)   =  5.4249e-04, true= 0.0000e+00, err=5.4e-04
%  Integral(x0^2) =  2.6176e-01, true= 2.6180e-01, err=4.3e-05
%  Integral(x1)   = -4.2214e-04, true= 0.0000e+00, err=4.2e-04
%  Integral(x1^2) =  2.6181e-01, true= 2.6180e-01, err=8.5e-06
%  Integral(x2)   = -3.5119e-03, true= 0.0000e+00, err=3.5e-03
%  Integral(x2^2) =  2.6125e-01, true= 2.6180e-01, err=5.5e-04
% 
% Grid sibe8.order2.hdf: surfaceArea for sphere = 3.140966e+00, (true=3.14159) error=6.3e-04
%  Integral(x0)   =  2.5992e-04, true= 0.0000e+00, err=2.6e-04
%  Integral(x0^2) =  2.6181e-01, true= 2.6180e-01, err=8.5e-06
%  Integral(x1)   = -1.4777e-04, true= 0.0000e+00, err=1.5e-04
%  Integral(x1^2) =  2.6182e-01, true= 2.6180e-01, err=2.5e-05
%  Integral(x2)   = -1.2296e-03, true= 0.0000e+00, err=1.2e-03
%  Integral(x2^2) =  2.6161e-01, true= 2.6180e-01, err=1.9e-04

% ** trouble generating the hybrid grid for this next case: (FOUND A HANGING FACE)
% Grid sibe16.order2.hdf: surfaceArea for sphere = 3.139531e+00, (true=3.14159) error=2.1e-03
%  Integral(x0)   =  5.8189e-05, true= 0.0000e+00, err=5.8e-05
%  Integral(x0^2) =  2.6169e-01, true= 2.6180e-01, err=1.1e-04
%  Integral(x1)   = -8.6810e-04, true= 0.0000e+00, err=8.7e-04
%  Integral(x1^2) =  2.6144e-01, true= 2.6180e-01, err=3.6e-04
%  Integral(x2)   = -3.7687e-04, true= 0.0000e+00, err=3.8e-04
%  Integral(x2^2) =  2.6176e-01, true= 2.6180e-01, err=4.2e-05

% -- increased angle to 89, reduced tol by 10
% Grid sibe16.order2.hdf: surfaceArea for sphere = 3.141421e+00, (true=3.14159) error=1.7e-04
%  Integral(x0)   =  8.6202e-05, true= 0.0000e+00, err=8.6e-05
%  Integral(x0^2) =  2.6179e-01, true= 2.6180e-01, err=1.3e-05
%  Integral(x1)   = -4.4121e-05, true= 0.0000e+00, err=4.4e-05
%  Integral(x1^2) =  2.6180e-01, true= 2.6180e-01, err=1.8e-07
%  Integral(x2)   = -2.5876e-04, true= 0.0000e+00, err=2.6e-04
%  Integral(x2^2) =  2.6177e-01, true= 2.6180e-01, err=3.0e-05


\begin{table}[hbt]
\begin{center}
\begin{tabular}{|c|c|c|c|c|c|c|c|c|} \hline 
  grid  &  $h_0/h$ &    1      &   $x$ & $y$ & $z$  &   $x^2$ & $y^2$ & $z^2$      \\   \hline\hline 
 sibe2  &    1     & $9.9e-03$ & $2.8e-03$&$1.0e-03$&$1.3e-02$ & $4.2e-04$&$1.3e-04$&$3.0e-03$    \\ 
 sibe4  &    2     & $2.3e-03$ & $5.4e-04$&$4.2e-04$&$3.5e-03$ & $4.3e-05$&$8.5e-06$&$5.5e-04$    \\ 
 sibe8  &    4     & $6.3e-04$ & $2.6e-04$&$1.5e-04$&$1.2e-03$ & $8.5e-06$&$2.5e-05$&$1.9e-04$    \\ \hline 
 sibe16 &    4     & $1.7e-04$ & $8.6e-05$&$4.4e-05$&$2.6e-04$ & $1.3e-05$&$1.8e-07$&$3.0e-05$    \\ \hline 
\end{tabular}	
\qquad
\end{center}		
\caption{Left: Errors in the numerical integration of the functions $1,x,y,z,x^2,y^2,z^2$
on the surface of a sphere.
These results where computed with the hybrid-grid approach.}
 \label{tab:sib-hyb} 
\end{table}


% -----------------------------------------------------------------------------
\clearpage
\subsection{Sample usage}

To use the {\tt Integrate} class you should follow the example given below.
(file {\ff \examples/ti.C})
{\footnotesize
\listinginput[1]{1}{\otherStuff/ti.C}
}


\subsection{Member Functions}

\input IntegrateInclude.tex

