\subsection{constructor}
 
\newlength{\FortranIOIncludeArgIndent}
\begin{flushleft} \textbf{%
\settowidth{\FortranIOIncludeArgIndent}{FortranIO(}% 
FortranIO()
}\end{flushleft}
\begin{description}
\item[{\bf Description:}] 
   Build a FortranIO object. 
\end{description}
\subsection{open}
 
\begin{flushleft} \textbf{%
int   \\ 
\settowidth{\FortranIOIncludeArgIndent}{open(}%
open(const aString \& fileName, \\ 
\hspace{\FortranIOIncludeArgIndent}const aString \& fileForm, \\ 
\hspace{\FortranIOIncludeArgIndent}const aString \& fileStatus,\\ 
\hspace{\FortranIOIncludeArgIndent}const int \& fortranUnitNumber  =25)
}\end{flushleft}
\begin{description}
\item[{\bf Description:}] 
   Open a fortran file with a fortran statement of the form:
 \begin{verbatim}
    open (unit=io, file=fileName,form=fileForm,status=fileStatus)
 \end{verbatim}
\item[{\bf fileName (input) :}]  name of the file.
\item[{\bf fileForm (input) :}]  a valid fortran file format such as "formatted" or "unformatted". *** only "unformatted"
  is currently supported.
\item[{\bf fileStatus (input) :}]  a valid file status such as "old", "new", "unknown"
\item[{\bf fortranUnitNumber (input) :}]  a positive integer.
\end{description}
\subsection{close}
 
\begin{flushleft} \textbf{%
int   \\ 
\settowidth{\FortranIOIncludeArgIndent}{close(}%
close()
}\end{flushleft}
\begin{description}
\item[{\bf Description:}] 
   Close a fortran file.
\end{description}
\subsection{print( int )}
 
\begin{flushleft} \textbf{%
int  \\ 
\settowidth{\FortranIOIncludeArgIndent}{print(}%
print(const int \& i)
}\end{flushleft}
\begin{description}
\item[{\bf Description:}] 
   Save an int in the file.
\end{description}
\subsection{print( float )}
 
\begin{flushleft} \textbf{%
int  \\ 
\settowidth{\FortranIOIncludeArgIndent}{print(}%
print(const float \& f)
}\end{flushleft}
\begin{description}
\item[{\bf Description:}] 
\end{description}
\subsection{print( double )}
 
\begin{flushleft} \textbf{%
int  \\ 
\settowidth{\FortranIOIncludeArgIndent}{print(}%
print(const double \& d)
}\end{flushleft}
\begin{description}
\item[{\bf Description:}] 
\end{description}
\subsection{print( int* )}
 
\begin{flushleft} \textbf{%
int  \\ 
\settowidth{\FortranIOIncludeArgIndent}{print(}%
print(const int *a, const int \& count)
}\end{flushleft}
\begin{description}
\item[{\bf Description:}] 
   Save an array of values.
\end{description}
\subsection{print( float* )}
 
\begin{flushleft} \textbf{%
int  \\ 
\settowidth{\FortranIOIncludeArgIndent}{print(}%
print(const float *a, const int \& count)
}\end{flushleft}
\begin{description}
\item[{\bf Description:}] 
   Save an array of values.
\end{description}
\subsection{print( double* )}
 
\begin{flushleft} \textbf{%
int  \\ 
\settowidth{\FortranIOIncludeArgIndent}{print(}%
print(const double *a, const int \& count)
}\end{flushleft}
\begin{description}
\item[{\bf Description:}] 
   Save an array of values.
\end{description}
\subsection{print( aString )}
 
\begin{flushleft} \textbf{%
int  \\ 
\settowidth{\FortranIOIncludeArgIndent}{print(}%
print(const aString \& label )
}\end{flushleft}
\begin{description}
\item[{\bf Description:}] 
\end{description}
\subsection{print( intArray )}
 
\begin{flushleft} \textbf{%
int  \\ 
\settowidth{\FortranIOIncludeArgIndent}{print(}%
print(const intArray \& u)
}\end{flushleft}
\begin{description}
\item[{\bf Description:}] 
\end{description}
\subsection{print( floatArray )}
 
\begin{flushleft} \textbf{%
int  \\ 
\settowidth{\FortranIOIncludeArgIndent}{print(}%
print(const floatArray \& u)
}\end{flushleft}
\begin{description}
\item[{\bf Description:}] 
\end{description}
\subsection{print( doubleArray )}
 
\begin{flushleft} \textbf{%
int  \\ 
\settowidth{\FortranIOIncludeArgIndent}{print(}%
print(const doubleArray \& u)
}\end{flushleft}
\begin{description}
\item[{\bf Description:}] 
\end{description}
\subsection{print( intArray,floatArray )}
 
\begin{flushleft} \textbf{%
int  \\ 
\settowidth{\FortranIOIncludeArgIndent}{print(}%
print(const intArray \& u, const floatArray \& v)
}\end{flushleft}
\begin{description}
\item[{\bf Description:}] 
   Output an int and float array.
\end{description}
\subsection{print( intArray,doubleArray )}
 
\begin{flushleft} \textbf{%
int  \\ 
\settowidth{\FortranIOIncludeArgIndent}{print(}%
print(const intArray \& u, const doubleArray \& v)
}\end{flushleft}
\begin{description}
\item[{\bf Description:}] 
\end{description}
\subsection{read( int )}
 
\begin{flushleft} \textbf{%
int  \\ 
\settowidth{\FortranIOIncludeArgIndent}{read(}%
read(const int \& i)
}\end{flushleft}
\begin{description}
\item[{\bf Description:}] 
   Save an int in the file.
\end{description}
\subsection{read( float )}
 
\begin{flushleft} \textbf{%
int  \\ 
\settowidth{\FortranIOIncludeArgIndent}{read(}%
read(const float \& f)
}\end{flushleft}
\begin{description}
\item[{\bf Description:}] 
\end{description}
\subsection{read( double )}
 
\begin{flushleft} \textbf{%
int  \\ 
\settowidth{\FortranIOIncludeArgIndent}{read(}%
read(const double \& d)
}\end{flushleft}
\begin{description}
\item[{\bf Description:}] 
\end{description}
\subsection{read( int* )}
 
\begin{flushleft} \textbf{%
int  \\ 
\settowidth{\FortranIOIncludeArgIndent}{read(}%
read(const int *a, const int \& count)
}\end{flushleft}
\begin{description}
\item[{\bf Description:}] 
   Save an array of values.
\end{description}
\subsection{read( float* )}
 
\begin{flushleft} \textbf{%
int  \\ 
\settowidth{\FortranIOIncludeArgIndent}{read(}%
read(const float *a, const int \& count)
}\end{flushleft}
\begin{description}
\item[{\bf Description:}] 
   Save an array of values.
\end{description}
\subsection{read( double* )}
 
\begin{flushleft} \textbf{%
int  \\ 
\settowidth{\FortranIOIncludeArgIndent}{read(}%
read(const double *a, const int \& count)
}\end{flushleft}
\begin{description}
\item[{\bf Description:}] 
   Save an array of values.
\end{description}
\subsection{read( aString )}
 
\begin{flushleft} \textbf{%
int  \\ 
\settowidth{\FortranIOIncludeArgIndent}{read(}%
read(const aString \& label )
}\end{flushleft}
\begin{description}
\item[{\bf Description:}] 
\end{description}
\subsection{read( intArray )}
 
\begin{flushleft} \textbf{%
int  \\ 
\settowidth{\FortranIOIncludeArgIndent}{read(}%
read(const intArray \& u)
}\end{flushleft}
\begin{description}
\item[{\bf Description:}] 
   Read in an array -- the array must be dimensioned to the correct size.
\end{description}
\subsection{read( floatArray )}
 
\begin{flushleft} \textbf{%
int  \\ 
\settowidth{\FortranIOIncludeArgIndent}{read(}%
read(const floatArray \& u)
}\end{flushleft}
\begin{description}
\item[{\bf Description:}] 
   Read in an array -- the array must be dimensioned to the correct size.
\end{description}
\subsection{read( doubleArray )}
 
\begin{flushleft} \textbf{%
int  \\ 
\settowidth{\FortranIOIncludeArgIndent}{read(}%
read(const doubleArray \& u)
}\end{flushleft}
\begin{description}
\item[{\bf Description:}] 
   Read in an array -- the array must be dimensioned to the correct size.
\end{description}
