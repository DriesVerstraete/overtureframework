\subsection{constructor}
 
\newlength{\TridiagonalSolverIncludeArgIndent}
\begin{flushleft} \textbf{%
\settowidth{\TridiagonalSolverIncludeArgIndent}{TridiagonalSolver(}% 
TridiagonalSolver() 
}\end{flushleft}
\begin{description}
\item[{\bf Description:}] 
   Use this class to solve a tridiagonal system or a pentadiagonal system. 
  The system may be block tridiagonal. There may
  be multiple independent tridiagonal (pentadiagonal) systems to be solved.
  The basic tridiagonal system is ({\tt type=normal})
 \begin{verbatim}
            | b[0] c[0]                     |
            | a[1] b[1] c[1]                |
        A = |      a[2] b[2] c[2]           |
            |            .    .    .        |
            |                a[.] b[.] c[.] |
            |                     a[n] b[n] |

 \end{verbatim}
 We can also solve the {\tt type=periodic}
 \begin{verbatim}
            | b[0] c[0]                a[0] |
            | a[1] b[1] c[1]                |
        A = |      a[2] b[2] c[2]           |
            |            .    .    .        |
            |                a[.] b[.] c[.] |
            | c[n]                a[n] b[n] |

 \end{verbatim}
 and the {\tt type=extended}
 \begin{verbatim}
            | b[0] c[0] a[0]                |
            | a[1] b[1] c[1]                |
        A = |      a[2] b[2] c[2]           |
            |            .    .    .        |
            |                a[.] b[.] c[.] |
            |                c[n] a[n] b[n] |

 \end{verbatim}
 which may occur with certain boundary conditions.

 This class expects the matrices a,b,c to be passed separately and to be of the 
 form
 \begin{itemize}
  \item a(I1,I2,I3), b(I1,I2,I3), c(I1,I2,I3) : if the block size is 1.
  \item a(b,b,I1,I2) : if the block size is b$>1$.
 \end{itemize}
  The `{\tt axis}' argument to the member functions indicates which of I1,I2 or I3
  represents the axis along which the tridiagonal matrix extends. The other axes
  can be used to hold independent tridiagonal systems. Thus if {\tt axis=0} then
  {\tt a(i1,i2,i3) i1=0,1,2,...,n} are the entries in the tridiagonal matrix for
  each fixed i2 and i3. If {\tt axis=1} then {\tt a(i1,i2,i3) i2=0,1,2,...,n} 
 are the entries in the tridiagonal matrix for fixed i1 and i3.
\end{description}
\subsection{factor}
 
\begin{flushleft} \textbf{%
int  \\ 
\settowidth{\TridiagonalSolverIncludeArgIndent}{factor(}%
factor(RealArray \& a\_, \\ 
\hspace{\TridiagonalSolverIncludeArgIndent}RealArray \& b\_, \\ 
\hspace{\TridiagonalSolverIncludeArgIndent}RealArray \& c\_, \\ 
\hspace{\TridiagonalSolverIncludeArgIndent}const SystemType \& type\_  =normal,\\ 
\hspace{\TridiagonalSolverIncludeArgIndent}const int \& axis\_  =0,\\ 
\hspace{\TridiagonalSolverIncludeArgIndent}const int \& block  =1)
}\end{flushleft}
\begin{description}
\item[{\bf Description:}] 
   Factor the tri-diagonal (block) matrix defined by (a,b,c).
   NOTE: This routine keeps a reference to (a,b,c) and factors in place. 
\item[{\bf a,b,c (input/output) :}]  on input the 3 diagonals, on output the LU factorization
\item[{\bf type (input) :}]  One of {\tt normal}, {\tt periodic} or {\tt extended}.
\item[{\bf axis (input) :}]  0, 1, or 2. See the comments below.
\item[{\bf block (input) :}]  block size. If block=2 or 3 then the matrix is block tridiagonal.
\item[{\bf Notes:}] 
 This class expects the matrices a,b,c to be of the form
 \begin{itemize}
  \item a(I1,I2,I3), b(I1,I2,I3), c(I1,I2,I3) : if the block size is 1.
  \item a(b,b,I1,I2), b(b,b,I1,I2), c(b,b,I1,I2) : if the block size is b$>1$.
 \end{itemize}
  The `{\tt axis}' argument to the member functions indicates which of I1,I2 or I3
  represents the axis along which the tridiagonal matrix extends. The other axes
  can be used to hold independent tridiagonal systems. Thus if {\tt axis=0} then
  {\tt a(i1,i2,i3) i1=0,1,2,...,n} are the entries in the tridiagonal matrix for
  each fixed i2 and i3. If {\tt axis=1} then {\tt a(i1,i2,i3) i2=0,1,2,...,n} 
 are the entries in the tridiagonal matrix for fixed i1 and i3.
\end{description}
\subsection{factor}
 
\begin{flushleft} \textbf{%
int  \\ 
\settowidth{\TridiagonalSolverIncludeArgIndent}{factor(}%
factor(RealArray \& a\_, \\ 
\hspace{\TridiagonalSolverIncludeArgIndent}RealArray \& b\_, \\ 
\hspace{\TridiagonalSolverIncludeArgIndent}RealArray \& c\_, \\ 
\hspace{\TridiagonalSolverIncludeArgIndent}RealArray \& d\_, \\ 
\hspace{\TridiagonalSolverIncludeArgIndent}RealArray \& e\_, \\ 
\hspace{\TridiagonalSolverIncludeArgIndent}const SystemType \& type\_  =normal,\\ 
\hspace{\TridiagonalSolverIncludeArgIndent}const int \& axis\_  =0,\\ 
\hspace{\TridiagonalSolverIncludeArgIndent}const int \& block  =1)
}\end{flushleft}
\begin{description}
\item[{\bf Description:}] 
   Factor the penta-diagonal (block) matrix defined by (a,b,c,d,e).
   NOTE: This routine keeps a reference to (a,b,c,d,e) and factors in place. 
\item[{\bf a,b,c,d,e (input/output) :}]  on input the 5 diagonals, on output the LU factorization
\item[{\bf type (input) :}]  One of {\tt normal}, {\tt periodic} or {\tt extended}.
\item[{\bf axis (input) :}]  0, 1, or 2. See the comments below.
\item[{\bf block (input) :}]  block size. If block=2 or 3 then the matrix is block tridiagonal.
\item[{\bf Notes:}] 
 This class expects the matrices a,b,c to be of the form
 \begin{itemize}
  \item a(I1,I2,I3), b(I1,I2,I3), c(I1,I2,I3), d(I1,I2,I3), e(I1,I2,I3) : if the block size is 1.
  \item a(b,b,I1,I2), b(b,b,I1,I2), c(b,b,I1,I2) d(b,b,I1,I2), e(b,b,I1,I2) : if the block size is b$>1$.
 \end{itemize}
  The `{\tt axis}' argument to the member functions indicates which of I1,I2 or I3
  represents the axis along which the tridiagonal matrix extends. The other axes
  can be used to hold independent tridiagonal systems. Thus if {\tt axis=0} then
  {\tt a(i1,i2,i3) i1=0,1,2,...,n} are the entries in the tridiagonal matrix for
  each fixed i2 and i3. If {\tt axis=1} then {\tt a(i1,i2,i3) i2=0,1,2,...,n} 
 are the entries in the tridiagonal matrix for fixed i1 and i3.
\end{description}
\subsection{solve}
 
\begin{flushleft} \textbf{%
real  \\ 
\settowidth{\TridiagonalSolverIncludeArgIndent}{sizeOf(}%
sizeOf( FILE *file  =NULL) const 
}\end{flushleft}
\begin{description}
\item[{\bf Description:}]  
   Return number of bytes allocated by Oges; optionally print detailed info to a file

\item[{\bf file (input) :}]  optinally supply a file to write detailed info to. Choose file=stdout to
 write to standard output.
\item[{\bf Return value:}]  the number of bytes.
\end{description}
\subsection{solve}
 
\begin{flushleft} \textbf{%
int  \\ 
\settowidth{\TridiagonalSolverIncludeArgIndent}{solve(}%
solve(const RealArray \& r\_ this is not really const,\\ 
\hspace{\TridiagonalSolverIncludeArgIndent}const Range \& R1  =nullRange, \\ 
\hspace{\TridiagonalSolverIncludeArgIndent}const Range \& R2  =nullRange, \\ 
\hspace{\TridiagonalSolverIncludeArgIndent}const Range \& R3  =nullRange)
}\end{flushleft}
\begin{description}
\item[{\bf Description:}] 
    Solve a set of tri-diagonal systems (or penta-diagonal systems). 
\item[{\bf r\_ (input/output) :}]  rhs vector on input, solution on output. This is declared const to avoid compiler
    warnings.
\item[{\bf R1,R2,R3:}]  Specifies which systems to solve. By default all the systems are solved.
  These Ranges must be a subset of the collection of
  systems that are found in the matrices passed to the {\tt factor} function.
  One of these is arguments is ignored, the one corresponding to the axis along which the
  tridiagonal system extends. 
\end{description}
