\section{OGFunction: A class for defining Twilight-zone flows}
\index{OGFunction}\index{twilight zone!defining functions}\index{method of analytic solutions|see{twilight zone}}

The class {\tt OGFunction} and derived classes such as
{\tt OGTrigFunction}, {\tt OGPolyfunction} and {\tt OGPulseFunction} can be used
to define ``exact solutions'' for PDE solvers in the 
manner that has come to be known as ``twilight-zone flow''
(Brown, 1985).

Basically this class defines an analytic function (such 
as a polynomial) and provides functions for evaluating
this function and derivatives in convenient ways. Note that 
as of version 14 it is possible to evaluate any partial
derivative using the {\tt gd} (general derivative) member function.

It is often difficult to come up with exact solutions 
to test a PDE code. \index{twilight zone!how to test PDE codes}
To over come this difficulty one
can add forcing functions to the PDE and boundary
conditions that will make any given function an
exact solution. For example given the time
dependent PDE for vector function $\uv$ 
$$
  \uv_t + a \uv_x + b \uv_y = \fv(x,y,t)
$$
with boundary conditions
$$
  B(\uv)=\gv(x,y,t)
$$
one can make any given function $\Uv(x,y,t)$ an exact solution
by defining
$$
 \fv(x,y,t) = \Uv_t + a \Uv_x + b \Uv_y
$$
and
$$
  \gv(x,y,t)=B(\Uv(x,y,t))
$$

The class {\tt OGPolyFunction} can be used to define the function 
$\Uv(x,y,z,t)$
which is defined to be a polynomial in $x$, $y$, [$z$] and $t$.
The class {\tt OGTrigfunction} can be used to define a function $\Uv(x,y,z,t)$
which is a trigonometric function such as $\cos(2\pi x)\cos(2\pi y)\cos(2\pi t)$.


The base class {\tt OGFunction} does not define any functions and thus you should
never construct one of these. You might, however, keep a pointer to the
base class. This pointer can point to an object of a derived class.


% ----------------------------------
\input OGFunctionInclude.tex
% ----------------------------------

\vfill\eject 
\subsection{\Index{OGPolyFunction}}

This class is derived from the {\tt OGFunction} class and defines a function
that is a polynomial in space times a polynomial in time of the form
$$
U(x,y,z,n,t) = ( \sum_{i,j,k} c(i,j,k,n) x^i y^j z^k ) \sum_m a(m,n) t^m
$$
Each component is a polynomial in space and time.
The coefficient matrices, $c(i,j,k,n)$ and $a(m,n)$ can be specified or else
default values can be used.

\input OGPolyFunctionInclude.tex


\vfill\eject 
\subsection{\Index{OGTrigFunction}}
 
\input OGTrigFunctionInclude.tex

% 
% This class is derived from the {\tt OGFunction} class and defines a function
% and defines a function that is a trigonometric polynomial:
% $$
%   \cos(f_x(n) \pi x) \cos( f_y(n) \pi y) \cos( f_z(n) \pi z) \cos( f_t(n) \pi t)
% $$
% where $f_x(n)$, $f_y(n)$ and $f_z(n)$ can be given values for each component n.
% 
% \noindent Use this constructor to use default values for
% $f_x(n)$, $f_y(n)$, $f_z(n)$ and $f_t(n)$ or to give equal values for all $n$.
% 
% \shadowbox{\parbox{4in}{\begin{flushleft}
%   OGTrigFunction(const real fx0=1.,  \\
% ~~~~~~~~~~~~~~~~~~~~~~~const real fy0=1.,  \\
% ~~~~~~~~~~~~~~~~~~~~~~~const real fz0=0.,   \\
% ~~~~~~~~~~~~~~~~~~~~~~~const real ft0=0.,  \\
% ~~~~~~~~~~~~~~~~~~~~~~~const int maximumNumberOfComponents=10);
% \end{flushleft}}}
% 
% \noindent Use this constructor to give different values to
% $f_x(n)$, $f_y(n)$, $f_z(n)$ and $f_t(n)$ for different values of $n$.
% 
% \shadowbox{\parbox{4in}{\begin{flushleft}
%   OGTrigFunction(const realArray \& fx0, \\
% ~~~~~~~~~~~~~~~~~~~~~~~~~~const realArray \& fy0, \\
% ~~~~~~~~~~~~~~~~~~~~~~~~~~const realArray \& fz0, \\
% ~~~~~~~~~~~~~~~~~~~~~~~~~~const realArray \& ft0);
% \end{flushleft}}}



% {\footnotesize
% \listinginput[1]{1}{\otherStuff/tz.C}
% }

\vfill\eject 
\subsection{\Index{OGPulseFunction}}
 
This class defines a {\it pulse} like function that can be useful for testing
adaptive mesh refinement codes.

The pulse is defined as a generalized Gaussian,
\begin{align*}
  u(\xv,t) &=  a_0 \exp( - a_1 | \xv-\bv(t) |^{2p} )  \\
  \bv(t) &= \cv_0 + \vv t
\end{align*}
where $p>{1\over 2}$.

The derivatives of $u$ are determined as follows. 
Letting
\begin{align*}
   r &= (x-b_0)^2 + (y-b_1)^2 + (z-b_2)^2 \\
   f &= r^p \\
   u &= a_0 \exp(-a_1 f )
\end{align*}
then
\begin{align*}
   u_x &= a_0 \exp(-a_1 f )\left[ -a_1 f_x \right] \\
   u_{xx} &= a_0 \exp(-a_1 f )\left[ (a_1 f_x)^2 - a_1 f_{xx} \right] \\
   u_{xy} &= a_0 \exp(-a_1 f )\left[ a_1 f_x a_1 f_y - a_1 f_{xy} \right] \\
   u_{xxxx} &= a_0 \exp(-a_1 f )\left[ -(a_1 f_x)^3 + 3 f_x f_{xx} - f_{xxx}\right] \\
   u_{xxxx} &= a_0 \exp(-a_1 f )\left[ (a_1 f_x)^4 + 3 f_{xx}^2 -6 f_x^2 f_{xx} +4 f_x f_{xxx} -f_{xxxx}\right] 
\end{align*}
where
\begin{align*}
  f_x & = 2 p r^{p-1} (x-b_0) \\
  f_{xx} &= 2 p r^{p-1} ( 2(p-1) {(x-b_0)^2\over r} + 1 ) \\
  f_{xy} &= 4 p(p-1) r^{p-2} ( (x-b_0)(y-b_1) ) \\
  f_{xxx} &= 4 p (p-1) r^{p-2} (x-b_0)\left[ (p-2) {(x-b_0)^2\over r} + 3  \right] \\
  f_{xxxx} &= 4 p (p-1) r^{p-2}\left[ 4(p-2){(x-b_0)^2\over r}( (p-3){(x-b_0)^2\over r}+3) + 3 \right]
\end{align*}
\input OGPulseFunctionInclude.tex


\subsection{Examples}

The file {\tt Overture/tests/tz.C} is an example code that uses the {\tt OGPolyFunction},
{\tt OGTrigFunction} and {\tt OGPulseFunction} classes. See also the exmaples in the
primer directory.
