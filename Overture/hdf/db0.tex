
\usepackage{verbatim}
\usepackage{moreverb}
\usepackage{graphics}    
% \input{psfig}  % PSFIG macros

% ---- we have lemmas and theorems in this paper ----
\newtheorem{assumption}{Assumption}
\newtheorem{definition}{Definition}

\newcommand{\dataBase}{/users/henshaw/res/hdf}

\begin{document}


% -----definitions-----
\def\R      {{\bf R}}
\def\Dv     {{\bf D}}
\def\bv     {{\bf b}}
\def\fv     {{\bf f}}
\def\Fv     {{\bf F}}
\def\gv     {{\bf g}}
\def\iv     {{\bf i}}
\def\jv     {{\bf j}}
\def\kv     {{\bf k}}
\def\nv     {{\bf n}}
\def\rv     {{\bf r}}
\def\tv     {{\bf t}}
\def\uv     {{\bf u}}
\def\Uv     {{\bf U}}
\def\vv     {{\bf v}}
\def\Vv     {{\bf V}}
\def\xv     {{\bf x}}
\def\yv     {{\bf y}}
\def\zv     {{\bf z}}
\def\lt     {{<}}
\def\grad    {\nabla}
\def\comma  {~~~,~~}
\def\uvd    {{\bf U}}
\def\ud     {{    U}}
\def\pd     {{    P}}
\def\calo{{\cal O}}

\def\ff {\tt} % font for fortran variables
\def\neq {{\ff neq }}
\def\nze {{\ff nze }}
\def\nsave {{\ff nsave}}
\def\nfill {{\ff nfill }}
\def\nqs {${\ff nze}^*$}
\def\flags {{\ff flags }}
\def\icf {{\ff icf}}
\def\idebug {{\ff idebug}}
\def\zratio {{\ff zratio }}
\def\fratio {{\ff fratio }}
\def\icg {{\ff icg }}
\def\ipc {{\ff ipc }}
\def\ipcf {{\ff ipcf }}
\def\rpcf {{\ff rpcf }}
\def\icf {{\ff icf }}
\def\nd {{\ff nd }}
\def\nv {{\ff nv }}
\def\ng {{\ff ng }}
\def\bc {{\ff bc}}

%---------- Title Page for a Research Report----------------------------
\vglue5\baselineskip
\begin{flushleft}
{\Large
GenericDataBase: A C++ Interface to Scientific Data-Bases for Use With A++\\
\vskip .25\baselineskip
HDF\_DataBase: An Implementation of GenericDataBase Using HDF \\
\vskip .5\baselineskip
User Guide, Version 1.00
}
\vskip2\baselineskip
William D. Henshaw                        \\
Computing, Information and Communications Division               \\
Los Alamos National Laboratory    \\
Los Alamos, NM, 87545
\footnote{
        This work was partially
        supported by grant N00014-93-C-0200 from the Office of Naval
        Research}
\vskip2\baselineskip
\today
\vskip\baselineskip
LA-UR-96-4447

\vskip4\baselineskip

\noindent{\bf Abstract:}
We describe a simple C++ interface that can be used to save and retrieve objects
from a hierarchical data-base. Objects are stored in a tree of ``directories'',
much like a unix file system.
Each directory has a name and a class-name to identify it. Any directory can
contain other ``sub-directories'' as well as any of the following types:
\begin{itemize}
  \item int, float, double, String or ``c'' arrays of these types.
  \item A++ arrays: intArray, floatArray, doubleArray
\end{itemize}
The class {\ff GenericDataBase} is a virtual base class (i.e. it declares functions
but does not implement them) that can be used as a generic interface to a data-base.
Applications should be written primarly with a {\ff GenericDataBase} so that they
do not depend on any particular data-base format.
The class {\ff HDF\_DataBase} is derived from {\ff GenericDataBase} and implements
the data-base functions using the Hierarchical Data Format (HDF) from the National Centre
for Super-Computing Applications (NCSA). A user interested in using HDF formatted files
can create an object of the type {\ff HDF\_DataBase} (in a main program for example) and pass
this object to functions that are written in terms of the {\ff GenericDataBase}.

\end{flushleft}

\vfill\eject
\tableofcontents

%---------- End of title Page for a Research Report

\newcommand{\MGF}{Mapped\-Grid\-Function}
\newcommand{\GCF}{Grid\-Collection\-Function}
\newcommand{\CGF}{Composite\-Grid\-Function}

\newcommand{\MGO}{Mapped\-Grid\-Operators}
\newcommand{\GCO}{Grid\-Collection\-Operators}
\newcommand{\CGO}{Composite\-Grid\-Operators}

\newcommand{\primer}{/users/henshaw/res/primer}
\newcommand{\gf}{/users/henshaw/res/gf}
\newcommand{\figures}{../docFigures}
\newcommand{\mapping}{/users/henshaw/res/mapping}
\newcommand{\ogshow}{/users/henshaw/res/ogshow}
\newcommand{\oges}{/users/henshaw/res/oges}
\newcommand{\cguser}{/users/henshaw/cgap/cguser}


\section{Introduction}


We describe a simple C++ interface that can be used to save and retrieve objects
from a hierarchical data-base. Objects are stored in a tree of ``directories'',
much like a unix file system.
Each directory has a name and a class-name to identify it. Any directory can
contain other ``sub-directories'' as well as any of the following types:
\begin{itemize}
  \item int, float, double, String
  \item A++ arrays: intArray, floatArray, doubleArray
  \item arrays of Strings.
\end{itemize}
Using these functions the user can create a hierarchical tree of information in which
user-derived class's can be conveniently stored. 

The class {\ff GenericDataBase} is a virtual base class (i.e. it declares functions
but does not implement them) that can be used as a generic interface to a data-base.
Applications should be written primarly with a {\ff GenericDataBase} so that they
do not depend on any particular data-base format.

The class {\ff HDF\_DataBase} is derived from {\ff GenericDataBase} and implements
the data-base functions using the Hierarchical Data Format (HDF) from the National Centre
for Super-Computing Applications (NCSA). A user interested in creating HDF files
can create an object of the type {\ff HDF\_DataBase} (in a main program for example) and pass
this object to functions that are written in terms of the {\ff GenericDataBase}.

There is also a {\bf streaming mode} where the objects are not saved in a tree structure 
but rather they are collected together and saved in a a few big buffers. In streaming mode 
the creation of directories and the names of objects are ignored.  In streaming mode the
data must be read back in exactly the same order it was written. This mode is faster and
requires less storage than the normal mode. The draw back is that the objects saved,
cannot be located individually by name. Streaming mode can be selectively turned on and off 
(although it should only be turned on and off once with any given directory).


We recommend that each class that needs to be saved to a data-base
implement a {\ff get} and {\ff put} member function using the {\ff
GenericDataBase} class, as shown in example 2.  The class should be
saved in a directory with a given name. The class-name for the
directory should be the name of the class that is being stored.


For more information about HDF, consult the HDF home page at {\tt http://hdf.ncsa.uiuc.edu}.

\vfill\eject
\section{Examples}

\subsection{Example 1: Using the HDF\_DataBase}

In this first example we show how to use the {\ff HDF\_DataBase} class to
save and retrieve data from a file.


This example will create a data-base file that schematically 
has the form shown in figure~\ref{figd}.
%-----------------------------------------------------------------
%                +------+
%                : root :
%                +==+===+
%                   :
%              +----+------+-------------+
%              :           :             |
%           +--+---+    +--+---+      +--+--+
%           :  k   :    : a    :      | bin |
%           +======+    +======+      +==+==+
%                                        |
%                                    +---+---+
%                                    | input |
%                                    +=======+

% \def\dskitem#1{\begin{picture}(40,25)     % root in a box
%   \put(-20,8){\framebox(40,17){#1}}
%   \put(-20,0){\framebox(40, 5){}  }    % data block
%  \end{picture}}

\def\dskitem#1#2{\begin{picture}(50,25)     % root in a box
  \put(-25,8){\framebox(50,17){#1}}
  \put(-25,-7){\framebox(50,15){#2}  }    % data block
 \end{picture}}

% define a DSK item box without data block beneath
%  bottom centre of box is at (0,8)
\def\dskitemw#1{\begin{picture}(40,25)     % root in a box
  \put(-20,8){\framebox(40,17){#1}}
 \end{picture}}
\begin{figure}\begin{center}
 \begin{picture}(180,150)

  \put( 80,102){\dskitemw{(root)}}
  \put( 80, 45){\dskitem{num}{int}}
  \put( 20, 45){\dskitem{x}{floatArray}}
  \put(140, 45){\dskitem{stuff}{directory}}
  \put(140,  0){\dskitem{label1}{String}}
  \put( 20, 70){\line(0,1){20}}
  \put( 80, 70){\line(0,1){40}}
  \put(140, 70){\line(0,1){20}}
  \put(140, 25){\line(0,1){13}}
  \put( 20, 90){\line(1,0){120}}

 \end{picture}

 \caption{Data-base structure for example 1. Each node has a name and a class-name}
 \label{figd}
\end{center} \end{figure}
%---------------------------------------------------------------

(file {\ff \dataBase/ex1.C})
{\footnotesize
\listinginput[1]{1}{/users/henshaw/res/hdf/ex1.C}
}

\vfill\eject
\subsection{Example 2: Writing get and put functions for a class using GenericDataBase}

In this example we show how to write {\ff get} and {\ff put} functions for a class
using the {\ff GenericDataBase}.  The {\ff put } function creates a directory of
a given name into which it stores the data needed by the class. The class name for
the directory is set equal to the name of the class, ``myClass''. 
The {\ff get} function looks for a directory of a given name and class and retrieves
the data needed by the class.


% The {\ff virtualConstructor} member function
% of the data-base class will create 

%-----------------------------------------------------------------
%                +------+
%                : root :
%                +==+===+
%                   :
%                +--+---+
%                :  m1  :
%                +======+
%                   |
%             --------------
%             |            |
%         +------+     +------+
%         |  a1  |     |  a2  |
%         +------+     +------+
%
\begin{figure}\begin{center}
 \begin{picture}(120,160)

  \put( 50,113){\dskitemw{(root)}}
  \put( 50, 75){\dskitem{m1}{myClass}}
  \put( 20,  0){\dskitem{a1}{float}}
  \put( 80,  0){\dskitem{a2}{float}}

  \put( 50,100){\line(0,1){20}}

  \put( 50, 45){\line(0,1){22}}
  \put( 20, 45){\line(1,0){60}} % ---

  \put( 20, 25){\line(0,1){20}}
  \put( 80, 25){\line(0,1){20}}

 \end{picture}

 \caption{Data-base structure for example 2.}
 \label{figb}
\end{center} \end{figure}
%---------------------------------------------------------------

(file {\ff \dataBase/ex2.C})
{\footnotesize
\listinginput[1]{1}{/users/henshaw/res/hdf/ex2.C}
}


\vfill\eject
\subsection{Example 2a: Writing get and put functions for a class using streaming}

In this example we show how to use the {\bf streaming mode} to put and get an object.
In this mode only the data for each {\tt put} is saved into a long buffer. The name
of the object is ignored and no new directories are created. 
Saving data with this mode saves space and is faster. However, the data must be read
back in exactly the way it was written. A {\sl magic number} separates each object
in the buffer so that if you make a mistake when reading in streaming mode it will
most likely be detected.

The previous example is changed so that
the data base mode is set to {\tt streamOutputMode} for the {\tt put} function
and  {\tt streamInputMode} for the {\tt get} function. Only these two new lines
need be added to the class.

In this example {\tt streamInputMode} is turned on in the sub directory created
in the {\tt MyClass} {\tt put} function. When the sub-directory is initially created it inherits
the mode from its parent which in this case is the default mode of {\tt normalMode}.
With the mode set to {\tt streamInputMode} the data saved from subsequent put's 
will be streamed into 3 buffers (float, int and double).
When this sub-directory is deleted the buffers will be saved (since the buffers were
originally opened in this sub-directory). 
Setting the mode back to {\tt normalMode} would also cause the buffers to be saved. 


There is also a {\tt noStreamMode}
which can be set. In {\tt noStreamMode} any attempt to set the mode to {\tt streamInputMode} or
{\tt streamOutputMode} will be ignored. 
This can be used
to force all objects to save themselves in the standard fashion. Thus the main program
could call {\tt db.setMode(GenericDataBase::noStreamMode);} in which case the class
would not be saved in a streaming mode.
To override {\tt noStreamMode} (not normally suggested) 
one must first set the mode back to {\tt normalMode}.


(file {\ff \dataBase/ex2a.C})
{\footnotesize
\listinginput[1]{1}{/users/henshaw/res/hdf/ex2a.C}
}

\section{GenericDataBase}

 This is a class to support access to and from a data-base.
This class knows how to get and put the types
\begin{itemize}
  \item int, float, double, String
  \item A++ arrays, intArray, floatArray, doubleArray
  \item "c" arrays of Strings.
\end{itemize}

\input GenericDataBaseInclude.tex

\vfill\eject
\section{HDF\_DataBase}

HDF\_DataBase is derived from GenericDataBase and implements the
functions using the HDF library from NCSA. HDF stands for
Hierarchical Data Format and NCSA is the National Centre for
Super-Computing Applications.


\subsection{Implementation notes}

\begin{itemize}
 \item   In HDF directories (nodes) in the hierarchy are called vgroups.
  When a vgroup is opened (attached) it is assigned a unique {\ff vgroup\_id}. 
  Every time we open a vgroup we need to close it. Thus in order to {\ff umount}
  a file we must be able to find all open vgroup's and close them. 
  Therefore we keep a list of objects that contain the {\ff vgroup\_id}
  of all the open vgroups. These objects are called {\ff HDF\_DataBaseRCData}
  since they hold Reference-Counted Data.
  A vgroup may be accessed mutliple times, but we only open it once. Each time
  it is accessed we increase the reference count for that vgroup.
 When the reference
 count goes to zero we can close the vgroup. The list that holds the 
 {\ff HDF\_DataBaseRCData} objects
  is called {\ff dbList}.
 Every element in the
 list will refer to a different vgroup so that the {\ff vgroup\_id}'s will
 be different for all elements in the list.

 \item A++ arrays are stored as SDS (Scientfic-Data-Sets). Since the SDS
   interface accesses files in a different way from the vgroup interface
  we need to keep two file identifiers, {\ff file\_id} and {\ff sds\_id}.
  The SDS interface does not have the notion of lower bounds to arrays
  other than zero so we stored the lower bounds for each of the A++
  array dimensions as an SDS ``attribute'' called ``arrayBase''.

 \item {\ff float}'s, {\ff double}'s, {\ff int}'s, {\ff String}'s 
  and arrays of {\ff Strings}'s are stored as HDF vdata objects. 
  The array of {\ff String's} is concatenated and stored as a
  single list of characters (with the different array elements separated
 by the null character).
\end{itemize}


\subsection{LIMITATIONS}

\begin{itemize}
 \item There is no way to delete items from an HDF file, this is a limitation of HDF.
 \item Currently I {\bf do not support the over-writing of data}. If you put something
     twice with the same name it will just create a new item with that same name. The get
     routine will never find it since it finds the first one it encounters. 
\end{itemize}

\end{document}


\bibliography{henshaw}
\bibliographystyle{siam}

