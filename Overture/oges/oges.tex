%-----------------------------------------------------------------------
% Oges: An Overlapping Grid Equation Solver
%
%  This version has a long page length for printing out
%-----------------------------------------------------------------------
\documentclass[11pt]{article}
\usepackage[bookmarks=true]{hyperref}

% \input documentationPageSize.tex
\hbadness=10000 
\sloppy \hfuzz=30pt

\usepackage{calc}
% set the page width and height for the paper (The covers will have their own size)
\setlength{\textwidth}{7in}  
\setlength{\textheight}{9.5in} 
% here we automatically compute the offsets in order to centre the page
\setlength{\oddsidemargin}{(\paperwidth-\textwidth)/2 - 1in}
% \setlength{\topmargin}{(\paperheight-\textheight -\headheight-\headsep-\footskip)/2 - 1in + .8in }
\setlength{\topmargin}{(\paperheight-\textheight -\headheight-\headsep-\footskip)/2 - 1in -.2in }

\input homeHenshaw

\hyphenation{con-ju-gate-Gra-dient-Pre-condition-er}

%\documentclass{article}

\usepackage{verbatim}
\usepackage{moreverb}
\usepackage{graphics}    
\usepackage{epsfig}    
% \input{psfig}  % PSFIG macros
\usepackage{calc}
\usepackage{ifthen}
\usepackage{float}
% \usepackage{fancybox}
\usepackage{makeidx} % index
\makeindex
\newcommand{\Index}[1]{#1\index{#1}}

% ---- we have lemmas and theorems in this paper ----
\newtheorem{assumption}{Assumption}
\newtheorem{definition}{Definition}


\begin{document}


% -----definitions-----
\input wdhDefinitions


\newcommand{\figures}{\homeHenshaw/OvertureFigures}


%---------- Title Page for a Research Report----------------------------
\vspace{5\baselineskip}
\begin{flushleft}
{\Large
Oges User Guide, Version 2.0, \\
A Solver for Steady State Boundary Value Problems \\
on Overlapping Grids \\
}
\vspace{2\baselineskip}
William D. Henshaw, \\
Petri Fast \\
Centre for Applied Scientific Computing \\
Lawrence Livermore National Laboratory    \\
Livermore, CA, 94551   \\
henshaw@llnl.gov \\
http://www.llnl.gov/casc/people/henshaw \\
http://www.llnl.gov/casc/Overture
\vspace{\baselineskip}
\today
\vspace{\baselineskip}
% LA-UR-96-3468
UCRL-MA-132234

\vspace{4\baselineskip}

\noindent{\bf Abstract:} 
We describe the Overture class Oges, an
``Overlapping Grid Equation Solver'', that can be used for the
solution of sparse matrix equations on overlapping grids such that
those created by the grid generator Ogen.  Oges acts as a front end to
a variety of sparse matrix solvers including direct sparse solvers
such as those from Yale or Harwell or iterative solvers (from SLAP and
PETSc) that use algorithms such as conjugate gradient or GMRES.

To use Oges one must first generate a a system of equations (usually
defining a PDE boundary value problem) using the `coefficient matrix'
grid functions and the Overture operator classes. Oges will take a
coefficient matrix generated in this way and then call the appropriate
sparse matrix solver.
Oges can be easily extended to use a new Sparse matrix package.

This document is available from the {\bf Overture} home page,
{\tt http://\-www.llnl.gov/\-casc/\-Overture}.

\end{flushleft}

\tableofcontents

\vfill\eject

%---------- End of title Page for a Research Report

\section{Introduction}

\begin{figure} 
  \begin{center}
  \epsfig{file=\figures/toges.ps,width=.75\linewidth}
  \caption{The `toges' program can be used to test Oges}
  \end{center} 
\end{figure}

We describe the Overture class Oges, an
``Overlapping Grid Equation Solver'', that can be used for the
solution of sparse matrix equations on overlapping grids such that
those created by the grid generator Ogen.  Oges acts as a front end to
a variety of sparse matrix solvers. Currently we have support for
\begin{description}
  \item[\Index{Yale}] : direct sparse matrix package (no pivoting).
  \item[\Index{Harwell}] : direct sparse matrix package with partial pivoting.
  \item[\Index{SLAP}] : The Sparse Linear Algebra Package
from Greenbaum and Seager, an  iterative solver package, includes conjugate gradient and gmres solvers.
  \item[\Index{PETSc}] : The Portable Extensible Toolkit for Scientific computations\cite{petsc-manual} from
     iterative solver package, includes conjugate gradient and gmres solvers
     in addition to many others. Thanks to Petri Fast for writing the interface (PETScEquationSolver) 
     to the serial version of PETSc. There is also a newer interface (PETScSolver) for use with PETSc in parallel. 
\end{description}
By changing one or two parameters
the user may easily try a different solver. For example,
although Yale is in general faster than Harwell, the latter,
 which does pivoting, may be better for some problems.
The SLAP and PETSc iterative solvers may be especially useful for very large
problems when storage is at a premium.

Oges can be easily extended by you to use a new Sparse matrix package. 

To use Oges one must first generate a a system of equations (usually
defining a PDE boundary value problem) using the `coefficient matrix'
grid functions and the Overture operator classes \cite{GF}\cite{OPERATORS}\cite{brown:1996}. 
A `coefficient matrix'\index{grid function!coefficient matrix}
is stored in a {\tt realCompositeGridFunction}. Typically the creation of
a PDE boundary value problem will look something like
{\footnotesize
\begin{verbatim}
   CompositeGrid cg(...);
   realCompositeGridFunction coeff(...);
   CompositeGridOperators op(cg); 
   ...

   coeff=op.laplacianCoefficients();         // form the laplace operator 

   coeff.applyBoundaryConditionCoefficients(0,0,dirichlet,  allBoundaries);
   coeff.applyBoundaryConditionCoefficients(0,0,extrapolate,allBoundaries);
   coeff.finishBoundaryConditions();
\end{verbatim}
}
Oges will take a
coefficient matrix generated in this way and then call the appropriate
sparse matrix solver. Usually this will involve converting the `coefficient matrix'
representation to some other representation such as a compressed-row storage format (this is
done automatically by Oges).

Given a coefficient matrix, Oges can be used as follows
{\footnotesize
\begin{verbatim}
   Oges solver(cg);                       // build a solver
   // use the yale solver:
   solver.set(OgesParameters::THEsolverType,OgesParameters::yale);   
   // ...or.. use PETSc
   solver.set(OgesParameters::THEsolverType,OgesParameters::PETSc);   

   solver.setCoefficientArray( coeff );   // supply coefficients
   realCompositeGridFunction u(cg),f(cg); // build solution and right-hand-side

   ... assign f ...

   u=0.;                                  // initial guess for iterative solvers
   solver.solve(u,f);                     // solve the equations
   ...
\end{verbatim}
}
Generally one must also set other parameters such as the convergence tolerance, preconditioner, etc, when using
iterative solvers such as SLAP or PETSC.

The global variable {\tt Oges::debug} is a bit flag that generates various diagnostic output
from Oges. Setting {\tt Oges::debug=63} (63=1+2+4+8+16+32) will generate lots of debugging output. Setting
{\tt Oges::debug=3} will generate only some debugging output. 


\section{Example Codes}

The file {\tt Overture/tests/toges.C} is a test program for {\tt Oges}.
See the files {\tt Overture/tests/tcm.C}, {\tt tcm2.C}, {\tt tcm3.C}, {\tt tcm4.C} for
examples of working codes. See also the examples presented in the Overture primer\cite{PRIMER}.
The flow solver {\tt OverBlown} \cite{OverBlownUserGuide} also uses Oges. One could look at the source
files for OverBlown for further examples.


\section{Initial guesses for iterative solvers}

When using an iterative solver, the grid function holding the solution should
contain an initial guess. Choose zero if you don't have any idea. If you do
provide and initial guess then all points used in the discretization should be
given initial values including interpolation points and ghost points.


% Oges generates a matrix (stored in one of a number of possible sparse
% matrix formats) corresponding to the discretization of the PDE
% boundary value problem (or any set of equations defined on an
% overlapping grid)
% $$
%      A \Uv = \Fv
% $$
% 
% derivatives to 2nd or 4th order accuracy and how
% to discretize problems in conservation form. 



% The basic procedure for solving a problem with Oges is illustrated
% in the following code ({\ff \oges/sample.C}). This example will become more
% clear after you have read the rest of this manual.
% {\footnotesize
% \listinginput[1]{1}{\oges/sample.C}
% }
% 
% \begin{figure} 
%   \begin{center}
%   \includegraphics{\figures /Oges.idraw.ps}
%   \caption{Class diagram for Oges}
%   \end{center} \label{fig:OgesClassDiagram}
% \end{figure}

\vfill\eject
\input ogesResults


\vfill\eject
\section{Oges Parameters} \label{sec:OgesParameters}\index{OgesParameters}

Solver dependent parameters are found in the {\tt OgesParameters} class.
It is a container class for such parameters are the type of solver, type of preconditioner,
convergence tolerance etc.
Oges contains an {\tt OgesParameters} object to hold these parameters.
Parameters can be set by directly using the Oges {\tt set} functions. This will indirectly
set the values in an {\tt OgesParameters} object contained in an Oges object.
Alternatively  one can first create an {\tt OgesParameters} object, set parameters in that
object and then provide the {\tt OgesParameters} object to Oges using the
{\tt setParameters} function (which will copy you values
into it's local version). Parameters can also be set interactively by calling the Oges
{\tt update} function or the {\tt OgesParameters} update function.


\input OgesParametersInclude.tex

\section{Convergence criteria}

There are many ways to define \Index{convergence criteria} for iterative methods.
The trick for Oges is to have a reasonable uniform way of defining a 
convergence tolerance for the different methods.



The standard PETSc convergence test is
\[
    \| r_k \|_2 < \max ( rtol \star \| r_0 \|_2 <, atol ) \qquad \mbox{PETSc}
\]
where 
\[
   \| x \|_2 = \sqrt{ \sum_i x_i^2 }
\]

The SLAP convergence test is somewhat different:
{\footnotesize
\begin{verbatim}
C   ******************* SLAP ******************
C *Description:
C       SGMRES solves a linear system A*X = B rewritten in the form:
C
C        (SB*A*(M-inverse)*(SX-inverse))*(SX*M*X) = SB*B,
C
C       with right preconditioning, or
C
C        (SB*(M-inverse)*A*(SX-inverse))*(SX*X) = SB*(M-inverse)*B,
C
C       with left preconditioning, where A is an N-by-N real matrix,
C       X  and  B are N-vectors,   SB and SX   are  diagonal scaling
C       matrices,   and M is  a preconditioning    matrix.   It uses
C       preconditioned  Krylov   subpace  methods  based     on  the
C       generalized minimum residual  method (GMRES).   This routine
C       optionally performs  either  the  full     orthogonalization
C       version of the  GMRES  algorithm or an incomplete variant of
C       it.  Both versions use restarting of the linear iteration by
C       default, although the user can disable this feature.
C
C       The GMRES  algorithm generates a sequence  of approximations
C       X(L) to the  true solution of the above  linear system.  The
C       convergence criteria for stopping the  iteration is based on
C       the size  of the  scaled norm of  the residual  R(L)  =  B -
C       A*X(L).  The actual stopping test is either:
C
C               norm(SB*(B-A*X(L))) .le. TOL*norm(SB*B),
C
C       for right preconditioning, or
C
C               norm(SB*(M-inverse)*(B-A*X(L))) .le.
C                       TOL*norm(SB*(M-inverse)*B),
C
C       for left preconditioning, where norm() denotes the euclidean
C       norm, and TOL is  a positive scalar less  than one  input by
C       the user.  If TOL equals zero  when SGMRES is called, then a
C       default  value  of 500*(the   smallest  positive  magnitude,
C       machine epsilon) is used.  If the  scaling arrays SB  and SX
C       are used, then  ideally they  should be chosen  so  that the
C       vectors SX*X(or SX*M*X) and  SB*B have all their  components
C       approximately equal  to  one in  magnitude.  If one wants to
C       use the same scaling in X  and B, then  SB and SX can be the
C       same array in the calling program.
C
\end{verbatim}
}


% \input OgesParametersInclude.tex
\vfill\eject
\section{Linking to PETSc}\index{PETSc!linking to}

An example of linking to the PETSc libraries can be found in the {\tt Overture/tests}
directory. Type 'make tcm3p' to build the {\tt tcm3.C} test code with PETSc. Type
'tcm3p cic.hdf -solver=petsc' to run the example on the grid {\tt cic.hdf} with PETSc.
This example assumes that the {\tt PETSC\_DIR} {\tt PETSC\_ARCH} and {\tt PETSC\_LIB} environmental
variables have been defined per instructions with the PETSc installation.

Here is an explanation of the steps required to build an Overture application with PETSc
(as implemented in the above example).
By default, the {\tt Overture} library is unaware whether  PETSc solvers are available. 
To use PETSc you should
\begin{enumerate}
  \item Build or locate a version of PETSc. I have only built and linked Overture to
the non-parallel version of PETSc. Link to the PETSc libraries (and {\tt lapack}). I link to
\begin{verbatim}
petscLib = -L$(PETSC_LIB) -lpetscsles -lpetscdm -lpetscmat -lpetscvec -lpetsc  \
           -L/usr/local/lib -llapack  -L$(PETSC_LIB) -lmpiuni 
\end{verbatim}
where {\tt \$(PETSC\_LIB)} is the location of the PETSc libraries.
  \item Copy the files {\tt Oges/buildEquationSolvers.C} and {\tt Oges/PETScEquationSolver.C}
    to your application directory
    and compile this file with the flags {\tt -DOVERTURE\_USE\_PETSC} (or edit the file and
   define this variable inside with {\tt \#define OVERTURE\_USE\_PETSC}).
  \item Link these new files, {\tt buildEquationSolvers.o} and {\tt PETScEquationSolver.o}
      with your application (ahead of the
     Overture library so that you get the new version) along with the PETSc libraries.
\end{enumerate}

\section{Adding a new sparse matrix solver to Oges}\index{sparse matrix solver}

If you want to add a new sparse matrix solver to Oges you should look at one of the
existing solvers, {\tt YaleEquationSolver}, {\tt HarwellEquationSolver}, {\tt SlapEquationSolver}
or {\tt PETScEquationSolver}. These classes all derive from the base class {\tt EquationSolver}.
{\tt Oges} contains a list of pointers to these {\tt EquationSolver}'s. You will be able to add
a new solver to this list. It will be known as {\tt OgesParameters::userSolver1}, 
or {\tt OgesParameters::userSolver2} etc., depending on how many new solvers have been added.


You should
\begin{enumerate}
  \item Derive a new class from {\tt EquationSolver}, copying one of the existing solvers (which ever
is closest) to your new solver. Hopefully you can reuse parameters that already exist in {\tt OgesParameters}.
  \item Change the {\tt Oges/buildEquationSolvers.C} file to `new' the solver to have defined and add it
    to the list of {\tt EquationSolver}'s. Change the other functions in {\tt Oges/buildEquationSolvers.C}
    as appropriate.
  \item Compile your files and the new version of {\tt buildEquationSolvers.C} and link to these
    ahead of the Overture library when you build an executable. 
\end{enumerate}

\section{Some More Details about Oges}

In general, Oges expects that the user wants to solve one or more
equations at each valid grid point on an overlapping grid.  The number
of equations that are given at each grid point is called the {\ff
numberOfComponents}.  In the simple case only one equation, such as a
discrete Laplace operator, is specified at each grid point, {\ff
numberOfComponents=1}.  In a more complicated case there will be a
system of equations at each grid point. For example, one may want to
solve the biharmonic equation as a system of two Possion equations in
which case {\ff numberOfComponents=2}.

Oges will create a large sparse matrix where each unknown 
for the sparse matrix will correspond to a particular
component {\ff n},
% ${\ff 0\le n < numberOfComponents}$,
at a particular
grid point ({\ff i1,i2,i3}) on a particular component
grid, {\ff grid}.
Thus there is a mapping from 
${\ff (n,i1,i2,i3,grid)}$ to a unique equation number.
The member functions 
$$
\begin{array}{c}
{\ff int~equationNo(int~n, int~i1, int~i2, int~i3, int~grid)}  \\
{\ff intArray~equationNo(int~n, Index~\&~I1, Index~\&~I2, Index~\&~I3, int~grid)} 
\end{array}
$$
give the equation number(s) for each grid point.
%  Here $n=0,\ldots,{\ff numberOfComponents-1}$ specifies
% the component. For example, if we solve the biharmonic eqaution as
% a system of two Possion equations then {\ff numberOfComponents=2}.
For now the function {\ff equationNo} is defined to use all the grid points
in a given order. In the future, a user should be able to define this
function in a different way. There is also a member function
$$
{\ff void~equationToIndex( int~eqnNo, int~n, int~i1, int~i2, int~i3, int~grid)}
$$ 
that maps an equation number, {\ff eqnNo},  back to a
grid point and component, ${\ff (n,i1,i2,i3,grid)}$ (i.e. it is the
inverse of {\ff equationNo}).



% In general a user defines a problem by giving the discrete stencil for
% each grid point, such as the discrete Laplacian or discrete
% approximations to boundary conditions.  Usually, however, Oges will
% automatically supply the stencils for interpolation, periodicity and
% extrapolation equations.  Thus the user defined discrete stencils can
% ignore interpolation points, extrapolation points and periodic
% boundary conditions (i.e. the user defined stencil does not have to
% correct for periodicity).  The user defined coefficients are supplied
% to Oges in one of two ways.  In simple cases the user can supply just
% the coefficient array using {\ff setCoefficientArray( coeff )}. In
% this case the coefficients are assumed to be numbered in a default
% manner (see section \ref{USC} for more details). A second way that a
% user can supply coefficients is through the virtual member function
% {\ff assignDiscreteCoefficients}.  (Thus the user would have to derive
% a new class from Oges.)  The stencil is defined by a set of
% coefficients and a set of equation numbers on a particular component
% grid
% $$
% \begin{array}{lcl}
%   {\ff coeff(i,i1,i2,i3)} &=& \mbox{coefficients, $i=0,1,2,...$} \\
%   {\ff equationNumber(i,i1,i2,i3)} &=& \mbox{equation number $i=0,1,2,...$} \\
% \end{array}
% $$
% For example, in two dimensions, one standard discretization uses a $3\times 3$
% stencil and thus coefficients will be supplied for $i=0,1,2,...,8$.
% The {\ff equationNumber} array associates an equation number with 
% each coefficient. The function {\ff equationNo} can be used to fill
% in the {\ff equationNumber} array (see the examples).

Sometimes extra unknowns and extra equations are required in order to 
specify a problem.
For example, an eignvalue problem has an extra unknown, the eigenvalue.
An extra unknown may be added to the singular Neumann problem in order
to create a nonsingular system.
Extra unknowns are associated with grid points that are not used.
The number of extra equations is specified with {\ff setNumberOfExtraEquations}.
Oges will find unused points that can be used for extra equations; the
equation numbers for these points will be saved in {\ff extraEquationNumber(i)}.


% The {\ff classify} array is an integer array that indicates 
% the classification of each grid point:
% $$
% {\ff classify[grid](i1,i2,i3,n)} = \left\{
%     \begin{array}{ll}
%           >0 & 
%      =\left\{ \begin{array}{ll} \mbox{interior} & \mbox{point is in the interior}\\
%                                 \mbox{boundary} & \mbox{point is on the boundary}\\
%                                 \mbox{ghost1} & \mbox{point is on ghost line 1} \\
%                                 \mbox{ghost2} & \mbox{point is on ghost line 2} \\
%                                 \mbox{ghost3} & \mbox{point is on ghost line 3} \\
%                                 10+i & \mbox{extra equation $i$} 
%                         \end{array} \right. \\
%           <0 &
%                =\left\{ \begin{array}{ll} \mbox{interpolation} \\
%                                           \mbox{extrapolation} \\
%                                           \mbox{periodic}      \\
%                         \end{array} \right. \\
%            0 & \mbox{unused point}
%     \end{array} \right. 
% $$
% The names {\ff interior}, {\ff boundary}, {\ff ghost1} etc. are 
% defined by the enumerator {\ff classifyTypes} declared in the 
% {\ff OgesOG} class.
% The user is responsible to supply discrete stencils for all points
% with {\ff classify[grid](i1,i2,i3,n)>0}.
% Oges will supply the stencil
% for all points with ${\ff classify[grid](i1,i2,i3,n)\le 0}$. 
% Values in the {\ff classify} array are determined by various parameters
% such as the {\ff numberOfGhostLines}, {\ff ghostLineOptions},
% the {\ff mask} array and the boundary conditions.
% For example, in some discretizations both the equation and the 
% boundary condition are applied on the boundary. The boundary equation
% may then be associated with the first ghost line in which case
% ({\ff classify[grid]>0}) on the first ghostline. In other cases
% or the values on the ghost points may just 
% be extrapolated ({\ff classify[grid]<0}) (see {\ff setGhostLineOption}).
% In general you may (if you are careful)
% change the entries in the {\ff classify} array from 
% their default values and this will change how the sparse matrix is generated.
% For example, you may want to change certain extrapolation points to use
% different equations. (To do this you must first call the {\ff initialize}
% function which will create the {\ff classify array}, and then you can change the
% values.)
% 
% The classify array should be used when assigning the right-hand-side.
% If {\ff realCompositeGridFunction f} is the name of the user's right-hand-side
% then the user is responsible to assign values to {\ff f} at {\bf all} points.
% At points where ${\ff classify[grid](i1,i2,i3,n) > 0 }$ 
% the user will give
% the appropriate values corresponding to the data in his equations.
% At all other points ${\ff classify[grid](i1,i2,i3,n) \le 0 }$ the user 
% should set {\ff f=0} (unless for some strange reason the user wants 
% a nonzero right-hand-side for an interpolation, extrapolation or
% periodicity equation!).



%---------------------------------------------------------------------------------------------
\section{Oges Function Descriptions}

\input OgesInclude.tex


% \section{Examples of using Oges}


% \subsection{Example: Solving a Poisson equation}
% Here is an example of using Oges to solve Laplace's equation (file {\ff toges1.C}).
% 
% {\footnotesize
% \listinginput[1]{1}{\oges/toges1.C}
% }
% 
% \vfill\eject
% \subsection{Example: Assigning the right-hand-side}
% Here is an example of how to assign the right-hand-side using the
% classify array (file {\ff assignRHS.C})
% 
% {\footnotesize
% \listinginput[1]{1}{\oges/assignRHS.C}
% }
% 
% \vfill\eject
% \subsection{Example: User Supplied Coefficients} \label{USC}
% 
% If you would like to discretize some equations that are different
% than the predefined ones you can supply a matrix of coefficients
% for Oges to use.
% Some important points to remember
% when supplying our own coefficients
% \begin{itemize}
% \item The coefficients appear as the first index in the array.
% \item The stencil size is by default $3\times3$ in 2D or
% $3\times3\times3$ in 3D. The stencil size can be changed
% by changing the {\ff orderOfAccuracy}. The stencil width
% is one more than the order of accuracy. 
% \item NOTE that Oges will store the interpolation coefficients 
% in YOUR coefficient array -- thus you must leave space for
% the interpolation equations. This usually means that you must
% leave space for ONE EXTRA coefficient. Also see remarks about the
% compatibility constraint below.
% \item By default the values on the ghost lines are extrapolated.
% If you want to supply equations for a ghost line you should
% use {\ff setGhostlineOption}.
% \item Oges will automatically supply the interpolation 
% equations and periodic boundary conditions. You do not need
% to worry about periodic boundary conditions when forming
% your coefficients.
% \item If your problem is singular such that the constant vector
% is the right null vector then you can de-singularize you system
% with {\ff setCompatibilityConstraint(TRUE)}. In this case you
% must allow space for ONE ADDITIONAL coefficient in your matrix.
% \end{itemize}
% 
% 
% \subsubsection{User supplied coefficients -- vertex-centred grid}
% Here is an example of how to supply your own discrete equations
% for a vertex-centered grid
% (file {\ff togesUSC.C}) 
% 
% {\footnotesize
% \listinginput[1]{1}{\oges/togesUSC.C}
% }
% 
% \subsubsection{User supplied coefficients -- cell-centred grid}
% Here is an example of how to supply your own discrete equations
% for a cell-centered grid
% (file {\ff togesUSCC.C}) 
% 
% {\footnotesize
% \listinginput[1]{1}{\oges/togesUSCC.C}
% }


% \subsection{Example: Refactoring the matrix when the coefficients change}
% 
% In some applications the coefficients of the matrix will change over
% time. One should use the option {\ff setRefactor(TRUE)} to tell Oges
% that the coefficients have changed. For the Yale solver this will mean
% that the matrix is refactored. For the GMRES solver this will mean
% that the preconditionner is recomputed if necessary(?). In the case of
% the Yale solver one could also {\ff setReorder(TRUE)} to tell the 
% solver to reorder the rows to minimize fillin. The re-order option
% is NOT recommended if the matrix system has not changed it's
% STRUCTURE dramatically since re-ordering can take quite a long time. 
% The orginal re-ordering is probably good enough.
% 
% 
% In the example that follows (file {\ff togesRF.C}) one should note
% that a copy of the original coefficients has been saved. This is necessary
% since Oges changes the coefficients (by adding interpolation, extrapolation
% and periodicity equations) -- ***I should fix this so it is not necessary
% to save the old coefficients ****
% {\footnotesize
% \listinginput[1]{1}{\oges/togesRF.C}
% }


% \vfill\eject
% \section{GhostLineOptions}  \label{secGhostLineOptions}
% 
% Here we illustrate how the {\ff ghostLineOptions} 
% affect the {\ff classify} array. We also show that
% ghostline values next to interpolation points
% are always extrapolated.
% 
% 
% % \newcommand{\interiorMark}{\circle{2}}
% % \newcommand{\interiorMark}{\oval(3,3)}
% \newcommand{\interiorMark}{{\makebox(0,0){$\otimes$}}}
% 
% \def\Dmark {\interiorMark}
% \def\Bmark {\interiorMark}
% % \def\Bmark {\interiorMark}
% \def\Emark {\makebox(0,0){\framebox(4,4){}}}
% \def\Imark {\makebox(0,0){$\Delta$}}
% \begin{figure}[hbt]
% \begin{center}
%  \setlength{\unitlength}{2pt}
%  \begin{picture}(180,160)(0,-70)
% 
% %.............................................................
%  \thicklines
%  \put( 20, 20){\line(1,0){60}}   % horizontal lines
%  \put( 20, 80){\line(1,0){60}}
% 
%  \put( 20, 20){\line(0,1){60}}   % vertical lines
%  \put( 80, 20){\line(0,1){60}}
%  \thinlines
% 
%  \put(  0,  0){\dashbox{1.}(10,80){}} % dashed boxes
%  \put( 10,  0){\dashbox{1.}(10,80){}}
%  \put( 20,  0){\dashbox{1.}(10,80){}}
% %  \put( 30,  0){\dashbox{1.}(10,80){}}
%  \put( 40,  0){\dashbox{1.}(10,80){}}
%  \put(  0,  0){\dashbox{1.}(80,10){}}
%  \put(  0, 10){\dashbox{1.}(80,10){}}
%  \put(  0, 20){\dashbox{1.}(80,10){}}
% %  \put(  0, 30){\dashbox{1.}(80,10){}}
%  \put(  0, 40){\dashbox{1.}(80,10){}}
% 
% 
%  \put(  0,  0){\Emark}
%  \put( 10,  0){\Emark}
%  \put( 20,  0){\Emark}
%  \put( 30,  0){\Emark}
%  \put( 40,  0){\Emark}
%  \put( 50,  0){\Emark}
%  \put( 60,  0){\Emark}
% 
%  \put(  0, 10){\Emark}
%  \put( 10, 10){\Emark}
%  \put( 20, 10){\Bmark}
%  \put( 30, 10){\Bmark}
%  \put( 40, 10){\Bmark}
%  \put( 50, 10){\Emark}
%  \put( 60, 10){\Emark}
% 
%  \put(  0, 20){\Emark}
%  \put( 10, 20){\Bmark}
%  \put( 20, 20){\Dmark}
%  \put( 30, 20){\Dmark}
%  \put( 40, 20){\Dmark}
%  \put( 50, 20){\Imark}
%  \put( 60, 20){\Imark}
% 
%  \put(  0, 30){\Emark}
%  \put( 10, 30){\Bmark}
%  \put( 20, 30){\Dmark}
%  \put( 30, 30){\Dmark}
%  \put( 40, 30){\Dmark}
%  \put( 50, 30){\Imark}
%  \put( 60, 30){\Imark}
% 
%  \put(  0, 40){\Emark}
%  \put( 10, 40){\Bmark}
%  \put( 20, 40){\Dmark}
%  \put( 30, 40){\Dmark}
%  \put( 40, 40){\Dmark}
%  \put( 50, 40){\Imark}
%  \put( 60, 40){\Imark}
% %.............................................................
%  \thicklines
%  \put(120, 20){\line(1,0){60}}   % horizontal lines
%  \put(120, 80){\line(1,0){60}}
% 
%  \put(120, 20){\line(0,1){60}}   % vertical lines
%  \put(180, 20){\line(0,1){60}}
%  \thinlines
% 
%  \put(100,  0){\dashbox{1.}(10,80){}} % dashed boxes
%  \put(110,  0){\dashbox{1.}(10,80){}}
%  \put(120,  0){\dashbox{1.}(10,80){}}
% %  \put(130,  0){\dashbox{1.}(10,80){}}
%  \put(140,  0){\dashbox{1.}(10,80){}}
%  \put(100,  0){\dashbox{1.}(80,10){}}
%  \put(100, 10){\dashbox{1.}(80,10){}}
%  \put(100, 20){\dashbox{1.}(80,10){}}
% %  \put(100, 30){\dashbox{1.}(80,10){}}
%  \put(100, 40){\dashbox{1.}(80,10){}}
% 
% 
%  \put(100,  0){\Emark}
%  \put(110,  0){\Emark}
%  \put(120,  0){\Emark}
%  \put(130,  0){\Emark}
%  \put(140,  0){\Emark}
%  \put(150,  0){\Emark}
%  \put(160,  0){\Emark}
% 
%  \put(100, 10){\Emark}
%  \put(110, 10){\Emark}
%  \put(120, 10){\Emark}
%  \put(130, 10){\Dmark}
%  \put(140, 10){\Dmark}
%  \put(150, 10){\Emark}
%  \put(160, 10){\Emark}
% 
%  \put(100, 20){\Emark}
%  \put(110, 20){\Emark}
%  \put(120, 20){\Bmark}
%  \put(130, 20){\Bmark}
%  \put(140, 20){\Bmark}
%  \put(150, 20){\Imark}
%  \put(160, 20){\Imark}
% 
%  \put(100, 30){\Emark}
%  \put(110, 30){\Dmark}
%  \put(120, 30){\Bmark}
%  \put(130, 30){\Dmark}
%  \put(140, 30){\Dmark}
%  \put(150, 30){\Imark}
%  \put(160, 30){\Imark}
% 
%  \put(100, 40){\Emark}
%  \put(110, 40){\Dmark}
%  \put(120, 40){\Bmark}
%  \put(130, 40){\Dmark}
%  \put(140, 40){\Dmark}
%  \put(150, 40){\Imark}
%  \put(160, 40){\Imark}
% 
% %.............................................................
%   \put( 40,-10){\makebox(0,0){useGhostLineExceptCorner}}
%   \put(140,-10){\makebox(0,0){useGhostLineExceptCornerAndNeighbours}}
% 
%  \put(0,-25){
% 
%   \put( 80,  0){\makebox(-4,0)[r]{\Dmark}}
%   \put( 80,  0){\makebox(0,0)[l]{ : Discretized Equation}}
%   \put( 80,-10){\makebox(-4,0)[r]{\Emark}}
%   \put( 80,-10){\makebox(0,0)[l]{ : Extrapolation}}
%   \put( 80,-20){\makebox(-4,0)[r]{\Imark}}
%   \put( 80,-20){\makebox(0,0)[l]{ : Interpolation}}
%     } % end put
% 
%  \end{picture}
% \caption{Illustrating two ghostLineOption's}
%       \label{DBC4}
% \end{center}\end{figure}
% 
% 
% 
% \section{Examples of Discretizing Equations}
% 
% 
% Here is some examples taken from {\ff discreteVertex.C} showing how
% to define the discretization for Oges. 
% 
% The first example shows how the Laplace
% Operator with Dirichlet boundary conditions is defined for a
% vertex centred grid (non-conservative discretization).  In this example
% we apply the Dirchlet boundary condition on the boundary (the value
% on the first ghostline will be extrapolated).
% {\footnotesize\begin{verbatim}
%       
%       int m1,m2,m3;
%       int n=0;  // component number
%       Index I1,I2,I3,R[3];
%       getIndex(cg[grid].indexRange,I1,I2,I3);  // get Index's based on indexRange
% 
%       // Assign equation Numbers - default values:
%       ForStencil(m1,m2,m3)  // loop over the stencil: m1=-1,0,1 m2=-1,0,1 m3=0
%         EQUATIONNUMBER(m1,m2,m3,n,I1,I2,I3)=equationNo(n,I1+m1,I2+m2,I3+m3,grid);
% 
%       ForStencil(m1,m2,m3)
%         COEFF(m1,m2,m3,n,I1,I2,I3)=UXX2(I1,I2,I3,m1,m2,m3)
%                                   +UYY2(I1,I2,I3,m1,m2,m3);
%       ForBoundary(side,axis)  // loop over each boundary
%       {
%         if( cg[grid].boundaryConditions(side,axis) > 0 )
%         {
%           getBoundaryIndex( side,axis,Ib1,Ib2,Ib3);
%           ForStencil(m1,m2,m3)
%             COEFF(m1,m2,m3,n,Ib1,Ib2,Ib3)=delta(m1)*delta(m2);
%         }
%       }
% 
% \end{verbatim}
% }
% In this example there are a number of Macros that have been used to
% improve the readability of the code. The {\ff COEFF} macro expands the
% first index of the {\ff coeff} array to appear as if there are 4
% indices {\ff (m1,m2,m3,n)}, corresponding to the stencil in each
% of 3 dimensions and the component number. The
% {\ff EQUATIONNUMBER} macro performs a similar function for the
% {\ff equationNumber} array.
% The macros {\ff UXX2} and {\ff UYY2} (found in ogesux2.h) define
% second order approximations to the second derivatives. 
% {\ff getBoundaryIndex} is a macro that returns the Index operators
% for a given side of a grid.
% 
% 
% In this second example we discretize the Laplace operator with
% Neumann boundary conditions. The Laplace operator will be
% applied on the boundary and the Neumann boundary condition
% will be associated with the first ghostline.
% 
% {\footnotesize\begin{verbatim}
% 
%       ForStencil(m1,m2,m3)
%       COEFF(m1,m2,m3,n,I1,I2,I3)=UXX2(I1,I2,I3,m1,m2,m3)
%                                 +UYY2(I1,I2,I3,m1,m2,m3);
%       ForBoundary(side,axis)
%       {
%         if( c.boundaryConditions(side,axis) > 0 )
%         {
%           getBoundaryIndex(side,axis,Ib1,Ib2,Ib3);
%           getGhostIndex1(side,axis,Ig1,Ig2,Ig3);   // Index's for ghost points
%           ForStencil(m1,m2,m3)
%           {
%             COEFF(m1,m2,m3,n,Ig1,Ig2,Ig3)=
%                  NORMAL(Ib1,Ib2,Ib3,axis1)*UX2(Ib1,Ib2,Ib3,m1,m2,m3)
%                 +NORMAL(Ib1,Ib2,Ib3,axis2)*UY2(Ib1,Ib2,Ib3,m1,m2,m3);
%              //assign eqn numbers (different from default)
%             EQUATIONNUMBER(m1,m2,m3,n,Ig1,Ig2,Ig3)=
% 		             equationNo(n,Ib1+m1,Ib2+m2,Ib3+m3,grid);
%           }
%         }
%       }
% 
% \end{verbatim}
% }
% The {\ff NORMAL} macro is used to access the {\ff normal } array which
% holds the normal vector on the boundary. The {\ff getGhostIndex1} 
% macro returns the Index operators for the first ghost-line.
% Since the normal derivative boundary condition is centred on the
% boundary we have to set the equation numbers to be different
% from their default values. Thus we set the equation numbers
% associated with the ghost-line {\ff EQUATIONNUMBER(m1,m2,m3,n,Ig1,Ig2,Ig3)} to have
% equation numbers based on the boundary {\ff equationNo(n,Ib1+m1,Ib2+m2,Ib3+m3,grid)}.
% 
% {\footnotesize\begin{verbatim}
% 
% \end{verbatim}
% }
% 
% \section{Computing Integration Weights and Integrating Functions}
% 
% Oges can be use to compute the weights for integration functions
% on overlapping grids. As described in \cite{CGIC} the integration
% weights can be obtained from the left null vector of the matrix
% that discretizes the Laplace operator with Neumann boundary
% conditions.
% 
% There are three steps to integrate a function on an overlapping
% grid. First you must solve for the left null vector using
% Oges. Next the null vector must be scaled by a constant 
% so that its entries become the integration weights. Now a 
% function can be integrated using the appropriate entries
% in the null vector. Some of the entries are used to compute
% the integral of a function over the entire domain while other
% entries are used to compute the integral of a function over
% the boundary.
% 
% Here is a program that computes the integration weights.
% {\footnotesize
% \listinginput[1]{1}{\oges/integrationWeights.C}
% }
% 

% You should use the predefined equationType  {\ff LaplaceNeumann}
% and solve for the transpose:
% {
% \footnotesize
% \begin{verbatim}
% 
%   Oges solver;
%   ...
%   solver.setEquationType(OgesOG::LaplaceNeumann);
%   solver.setTranspose(TRUE);
%   solver.initialize();
%   ...
% 
% \end{verbatim}
% }
% The right
% hand side function should be zero except for the value
% of the extra equation which you can set to 1 (or any
% other non-zero value).
% This value gives a scaling to the integration weights. This scaling will
% be corrected in step 2.
% {
% \footnotesize
% \begin{verbatim}
% 
%   CompositeGrid cg(...);
% 
%   realCompositeGridFunction f(cg);
%   ...
%   for( int grid=0; grid<f.numberOfComponentGrids; grid++ )
%     f[grid]=0.; 
%   // assign extra equation associated with Neumann problem
%   int n,i1,i2,i3;
%   og.equationToIndex( og.extraEquationNumber(0),n,i1,i2,i3,grid );
%   f[grid](i1,i2,i3,n)=1.;
% 
%   realCompositeGridFunction integrationWeights(cg);
%   og.solve( integrationWeights,f );
% 
% \end{verbatim}
% }
% 
% The second step is to scale the left null vector
% with the function {\ff scaleIntegrationWeights}. 
% {
% \footnotesize
% \begin{verbatim}
% 
%   og.scaleIntegrationWeights( integrationWeights );
% 
% \end{verbatim}
% }
% 
% Now you are
% ready to integrate a function using, for example, the 
% function {\ff integrate}:
% {
% \footnotesize
% \begin{verbatim}
% 
%   ...
%   // Here is the function we will integrate
%   for( grid=0; grid<f.numberOfComponentGrids; grid++ )
%     f[grid]=1.;
%   real volumeIntegral,surfaceIntegral;
%   og.integrate( integrationWeights,f,volumeIntegral,surfaceIntegral );
%   ...
% 
% \end{verbatim}
% }

% 
% The {\ff integrate} function is quite simple as shown below 
% ({\ff ogif.C})
% 
% {\footnotesize
% \listinginput[1]{1}{\oges/ogif.C}
% }

\bibliography{/homeHenshaw/papers/henshaw}
\bibliographystyle{siam}

\printindex

\end{document}
