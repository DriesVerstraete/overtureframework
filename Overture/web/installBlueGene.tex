\section{Installing Overture on the Blue Gene}  \label{sec:installBlueGene}

Here are  notes on installing Overture on a Blue Gene/Q at RPI.

I used the gnu compilers:
\begin{verbatim}
  module load gnu
\end{verbatim}

\subsection{Installing parallel HDF}

Set {\tt RUNSERIAL} to be the command to run a program on the Blue Gene.

\begin{verbatim}
tar -xzf hdf5-1.8.8.tar.gz
mv hdf5-1.8.8 hdf5-1.8.8-parallel
cd hdf5-1.8.8-parallel

unsetenv CC
unsetenv cc

./bin/yodconfigure configure

setenv RUNSERIAL "srun -N1 -n1 --time 1 --partition debug"
setenv LDFLAGS "-dynamic"

configure --prefix=`pwd` --enable-parallel


make
make install
\end{verbatim}


\subsection{Installing P++} 

\begin{verbatim}
tar xzf AP-0.8.2.tar.gz
mv A++P++-0.8.2 A++P++-0.8.2-gcc.4.4.6-parallel

./configure --enable-PXX --enable-SHARED_LIBS --prefix=`pwd` \
             --with-CC=mpicc --with-CXX=mpicxx --disable-mpirun-check --without-PADRE
\end{verbatim}

\subsection{Installing Overture} 

Overture was built without graphics and without support for Perl. 

\begin{verbatim}
configure bg  parallel CC=mpicxx bCC=g++ cc=mpicc bcc=gcc FC=mpif90 bFC=gfortran\
          --disable-X11 --disable-perl --disable-gl
\end{verbatim}
