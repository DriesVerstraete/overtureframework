%-----------------------------------------------------------------------
%   Overture figures and movies
%-----------------------------------------------------------------------
\documentclass{article}

% \setlength{\textwidth}{9in}      % page width

\usepackage{verbatim}
%\usepackage{moreverb}
\usepackage{graphics}    
\usepackage{epsfig}    
%\usepackage{calc}
%\usepackage{ifthen}
\usepackage{float}
%\usepackage{fancybox}

\usepackage{html}

\newcommand{\documentation}{../documentation}
\newcommand{\figures}{../figures}


\begin{document}

\begin{center}
{\Large Overture Figures and Movies} \\
~~ \\
LLNL Overlapping Grid Project \\
\htmladdnormallink{Centre for Applied Scientific Computing}{http://www.llnl.gov/casc} \\
\htmladdnormallink{Lawrence Livermore National Laboratory}{http://www.llnl.gov}  \\
\htmladdnormallink{Overture}{http://www.llnl.gov/CASC/Overture} 
\end{center}


All figures shown here we generated with the Overture graphics routines.
Many more figures can be found in the documentation for Overture, especially
the {\bf OverBlown}, {\bf Ogen} and {\bf Mapping} documentation.

\section{Using plotStuff to display results from a moving grid computation}

Here is a figure showing the results from moving grid computation using the {\bf OverBlown} flow solver. 
The figure is displayed in a post-processing
mode using {\bf plotStuff}
% <p align=center><img src="../figures/plotStuff.gif"></p>
\begin{rawhtml}
<p align=center><img src="../figures/plotStuff.gif"></p>
\end{rawhtml}

\section{Some sample overlapping grids}

The  grid generator {\bf ogen} can be used to
build overlapping grids. It has extensive capabilities
for building the individual component grids and an automatic
algorithm for cutting holes and computing the interpolation 
points.

\subsection{{\bf ogen} in action}

\begin{rawhtml}
<p align=center><img  src="../figures/ogenScreenSub.gif"></p>
\end{rawhtml}
\begin{flushleft}
{\LARGE A figure showing the graphical user interface to {\bf ogen}.}
\end{flushleft}
ogenScreenSub.gif

\subsection{Submarine grid}

\begin{rawhtml}
<p align=center><img src="../figures/sub.gif"></p>
\end{rawhtml}
\begin{flushleft}
{\LARGE 3D overlapping grid for a submarine demonstrating capabilities
   for joining surfaces such as where the sail joins the hull.} 
\end{flushleft}

\subsection{Airfoil grid built with elliptic grid generation.}

\begin{rawhtml}
<p align=center><img src="../figures/naca-elliptic.gif"></p>
\end{rawhtml}
\begin{flushleft}
{\LARGE Airfoil grid initially built with transfinite interpolation and
   then smoothed elliptic grid generation.} 
\end{flushleft}


\subsection{Grids for a oceanography}

Thanks to Lotta Olsson for creating these grids.


\begin{rawhtml}
<p align=center><img src="../figures/ocean.gif"></p>
\end{rawhtml}
\begin{flushleft}
{\LARGE 2D overlapping grid for the gulf of Mexico} 
\end{flushleft}

\subsection{2D wave equation}

This example from the primer (version 16 or later)
shows the solution of the 2D wave equation.

\begin{rawhtml}
<p align=center><img src="../figures/wave0.gif"></p> \\
<p align=center><img src="../figures/wave1.gif"></p>  \\
<p align=center><img src="../figures/wave2.gif"></p>
\end{rawhtml}
\begin{flushleft}
{\LARGE Results from the 2D wave equation.}
\end{flushleft}

\section{Some sample results from OverBlown}

{\bf OverBlown} can be used to solve the Navier-Stokes equations. It 
has a number of algorithms including ones for incompressible flow,
compressible flow and slightly compressible flow.

\subsection{Incompressible flow past a cylinder.}
\begin{rawhtml}
<p align=center><img src="../figures/ob.ins.cylinder.sl50.gif"></p>
\end{rawhtml}
\begin{flushleft}
{\LARGE Incompressible flow past a cylinder.}
\end{flushleft}

\subsection{Incompressible flow through some pipes.}
\begin{rawhtml}
<p align=center><img src="../figures/ins.pipes.p.gif"></p>
\end{rawhtml}
\begin{flushleft}
{\LARGE Incompressible flow through some pipes.}
\end{flushleft}


\subsection{Compressible flow past a triangle}
\begin{rawhtml}
<p align=center><img src="../figures/triangleShock.rp6.gif"></p> \\
<p align=center><img src="../figures/triangleShock.r1p0.gif"></p> 
\end{rawhtml}
\begin{flushleft}
{\LARGE Compressible flow past a triangle. Note how there is few noticeable affects
    coming from the overlapping grid.}
\end{flushleft}

\subsection{Incompressible flow past a NACA 0012 airfoil.}
\begin{rawhtml}
<p align=center><img src="../figures/ins.naca.p.gif"></p>
\end{rawhtml}
\begin{flushleft}
{\LARGE Incompressible flow past a NACA 0012 airfoil.}
\end{flushleft}


\section{A moving grid stirring stick}

These figures show results from an {\bf OverBlown} computation using moving overlapping
grids solving the incompressible Navier-Stokes equations. The central
grid (stirring stick) rotates about its centre. The boundary conditions
are no-slip on the four walls and the stirring stick.

Here is a 
\htmladdnormallink{movie (MPEG)}{../movies/stir.mpg}.

\begin{rawhtml}
<href="../movies/stir.mpg">
\end{rawhtml}


\begin{rawhtml}
<p align=center><img src="../figures/stirg.gif"></p>
\end{rawhtml}
\begin{flushleft}
{\LARGE Moving overlapping grid for a stirring stick} \\
~ \\   % do this to add vertical space -- must be an easier way?
~ \\
~ \\
\end{flushleft}

\begin{rawhtml}
<p align=center><img src="../figures/stir7.gif"></p>
\end{rawhtml}
\begin{flushleft}
{\LARGE Streamlines for the stirring stick} \\
~ \\
~ \\
~ \\
\end{flushleft}

\section{A moving valve}

These figures show results from an {\bf OverBlown} computation using moving overlapping
grids solving the incompressible Navier-Stokes equations. A valve opens
and closes. 


Here is a 
\htmladdnormallink{movie (MPEG)}{../movies/slValve.mpg}.
showing stream-lines.
Here is another
\htmladdnormallink{movie (MPEG)}{../movies/uValve.mpg}
showing the {\tt u} component of the velocity.
Yet another
\htmladdnormallink{movie (MPEG)}{../movies/vortValve.mpg}
showing the {\tt vorticity} -- a regular multiplex cinema here!

\begin{rawhtml}
<href="../movies/valve.mpg">
\end{rawhtml}


\begin{rawhtml}
<p align=center><img src="../figures/valve02.gif"></p>
\end{rawhtml}
\begin{flushleft}
{\LARGE Streamlines for the valve as it begins to open}
\end{flushleft}

\section{A two-stroke engine}

These figures show the results from solving the incompressible
Navier-Stokes equations on a model of a two-stroke engine. In the
current computation the geometry is fixed. A pressure
gradient drives the fluid from the 3 inlet ports into 
the larger outlet port.

In the near future we hope
to solve the problem with a moving geometry in which case the
lower face of the cylinder will move up and down.

Here is a 
\htmladdnormallink{movie (MPEG)}{../movies/tse.mpg}
showing the piston moving.

Here is a 
\htmladdnormallink{movie (MPEG)}{../movies/tseSpeed.mpg}
of a compressible flow with the piston moving.

% \noindent
% Here is a static image of the piston at different heights

\begin{rawhtml}
<p align=center><img src="../figures/tse.gif"></p>
\end{rawhtml}
\begin{flushleft}
{\LARGE Grid for a model two-stroke engine} \\
~ \\   % do this to add vertical space -- must be an easier way?
~ \\
~ \\
\end{flushleft}

% \begin{rawhtml}
% <p align=center><img  src="../figures/tsevp9.gif"></p>
% \end{rawhtml}
% \begin{flushleft}
% {\LARGE Flow in a two-stroke engine computed with CGINS} \\
% ~ \\
% ~ \\
% ~ \\
%\end{flushleft}

\section{Shock hitting some rigid bodies.}

Here are some movies of a shock hitting a collection of cylinders that are free to
move as rigid bodies. These computations were performed with the compressible Euler solver in
OverBlown using the Godunov solver from Don Schwendeman (RPI). 
Adaptive mesh refinement is combined with moving grids in these computations.

Here is a Schlieren style \htmladdnormallink{movie (MPEG)}{../movies/randomCylSchlieren.mpg}.

Here is a \htmladdnormallink{movie (MPEG)}{../movies/randomCylGrids.mpg} 
showing a coarsened version of the grids. The overlapping grid
is regenerated at every time step. The AMR grids are completely rebuilt every 8 time steps.


\section{Falling rigid bodies in an incompressible flow}

Here a \htmladdnormallink{movie (MPEG)}{../movies/drops-speed.mpg} showing 
a collection of cylinders falling in a channel. There is
an upward flow that is nearly equal to the terminal velocity of the cylinders.
The solution
is computed with the incompressible flow algorithm in OverBlown. A multigrid
solver is used to solve the pressure Poisson equation. 




\section{Random movies}

Here is a movie of a 
\htmladdnormallink{rotating frissbee (MPEG)}{../movies/pib.mpg}. There is a
cross flow from left to right.


\section{Grid generation from CAD}

We are now working on some grid generation tools to create overlapping
grids for surfaces defined from IGES files
(as produced by proEngineer, for example).

The next figure shows a geometry that was created using proEngineer and output
as an IGES file. The IGES file was then read into Overture, interpreted and plotted.
The surface is represented as a collection of trimmed NURBS (basically
spline surfaces with parts removed.) Each trimmed NURBS is shown as a different colour.

\begin{rawhtml}
<p align=center><img  src="../figures/cat201.gif"></p>
\end{rawhtml}
\begin{flushleft}
{\LARGE Composite surface of trimmed NURBS, read from an IGES file output from proEnginner 
    and plotted with Overture} \\
~ \\
~ \\
~ \\
\end{flushleft}

The main step is to generate a fewer number of overlapping surface patches
using hyperbolic grid generation.

% \begin{rawhtml}
% <p align=center><img  src="../figures/valveSurf.gif"></p>
% \end{rawhtml}
% \begin{flushleft}
% {\LARGE A valve taken from the above geometry, consisting of 10 trimmed nurbs.} \\
% ~ \\
% ~ \\
% ~ \\
% \end{flushleft}
% \begin{rawhtml}
% <p align=center><img  src="../figures/valve-hs.gif"></p>
% \end{rawhtml}
% \begin{flushleft}
% {\LARGE A surface grid for the valve computed with hyperbolic surface grid generation.} \\
% ~ \\
% ~ \\
% ~ \\
% \end{flushleft}

Here are some grids generated for parts of the CAD geometry shown above. These can be generated
with the tools available in Overture.v17.

% <p align=center><img  src="../figures/valveExit-cs.gif"></p>
\begin{rawhtml}
<p align=center><img src="../figures/valveExit.gif"></p>
<p align=center><img src="../figures/buildPortsPlus2.gif"></p>
\end{rawhtml}
\begin{flushleft}
{\LARGE Overlapping grid generated with hyperbolic grid generation and other tools available in Overture.} \\
~ \\
~ \\
~ \\
\end{flushleft}


\section{Future work}


  We are beginning to add support for unstructured grids into Overture. 

   Current work involves the generation of hybrid grids from overlapping grids. Hybrid grids
will replace the overlap with a region of unstructured elements. Kyle Chand has developed 
algorithms for generating two-dimensional hybrid grids. 


\begin{rawhtml}
<p align=center><img src="../figures/interior_bdy_overlap.gif"></p>
<p align=center><img src="../figures/interior2_stretch.gif"></p>
\end{rawhtml}
\begin{flushleft}
{\LARGE Hybrid grid generated from an overlapping grid.} \\
~ \\
~ \\
~ \\
\end{flushleft}



% \begin{figure}
% \htmlimage{thumbnail=0.25}  % make a thumb-nail sketch
% \end{figure}

\end{document}
