%-----------------------------------------------------------------------
%   Overture installation instructions
%-----------------------------------------------------------------------
\documentclass{article}
\usepackage[bookmarks=true,colorlinks=true,linkcolor=blue]{hyperref}

% \input documentationPageSize.tex
\hbadness=10000 
\sloppy \hfuzz=30pt

% \voffset=-.25truein
% \hoffset=-1.25truein
% \setlength{\textwidth}{7in}      % page width
% \setlength{\textheight}{9.5in}    % page height

\usepackage{calc}
\usepackage[lmargin=1.in,rmargin=1.in,tmargin=1.in,bmargin=1.in]{geometry}


% \usepackage{html}

\newcommand{\documentation}{../documentation}
\newcommand{\software}{../software}

\begin{document}

\begin{center}
{\Large Overture Installation Instructions} \\
~~ \\
LLNL Overlapping Grid Project \\
\htmladdnormallink{Centre for Applied Scientific Computing}{http://www.llnl.gov/casc} \\
\htmladdnormallink{Lawrence Livermore National Laboratory}{http://www.llnl.gov}  \\
\htmladdnormallink{Overture}{http://www.llnl.gov/CASC/Overture} 
\end{center}


\begin{abstract}
% \large Overture is not easy to install.  Overture is a research code. Overture is a
% large code. If you don't know C++ or you don't know how to use a debugger then
% Overture is not for you.
% 
% If you are still reading at this point then this document contains the
% wonderfully easy step-by-step instructions on how to install Overture and the
% supporting packages. Let us know where there are mistakes in these instructions.
This document contains step-by-step instructions on how to install Overture and the
supporting packages. 
\end{abstract}

\tableofcontents

\clearpage
\input installedMachines

\section{Installing Open Motif}

Note: Mac OS/X users should skip ahead to Section~\ref{sec:installMac}.
\vskip\baselineskip

You may need to install Motif if you do not already have a version. 
Most Linux systems now come with Motif in {\tt /usr/lib/libXm.so} (or {\tt /usr/lib64/libXm.so} for 64 bit
machines) in which case you can skip this step.

The Motif library
is called {\tt libXm.a} or {\tt libXm.so}. In older versions of Linux the default version of
Motif is probably LessTif. I have had trouble with LessTif 
so you should install open Motif which you can get for free from 
\htmladdnormallink{open Motif}{http://www.openmotif.com/}. I have tried version 2.2.1.
There is an rpm available with precompiled binaries.
Here are the steps I took to install open Motif from the source files:
\begin{enumerate}
    \item {\tt gzip -d openMotif-2.2.1.tar.gz}
    \item {\tt tar -xf openMotif-2.2.1.tar}
    \item {\tt cd openMotif-2.2.1}
    \item {\tt setenv CC gcc} ~:~set CC to your favourite C compiler, usually gcc on linux
%  configure --prefix=`pwd`
    \item {\tt ./configure -\,-prefix="this directory"} ~:~ where "this directory" is the current directory
       (or the place you want to install Motif).
    \item {\tt make}
    \item {\tt make install}
\end{enumerate}   


\clearpage
\section{Installing Mesa OpenGL}

You may need to install \htmladdnormallink{Mesa}{http://www.mesa3d.org/} (OpenGL),
Most Linux systems should already have OpenGL. However, if you would like off-screen rendering
support (for making high-resolution hard copies) then you should install Mesa. I normally run with
Mesa rather than the native OpenGL.

% I currently use version Mesa-6.2.1.
% I currently use version Mesa-6.4.2.
I currently use version Mesa-7.2.

Assuming you have downloaded {\tt MesaDemos-XXX.tar.gz} and {\tt  MesaLib-XXX.tar.gz}
in directory MESA (e.g. /home/henshaw/Mesa)
 \begin{enumerate}
   \item {\tt tar xzf MesaLib-XXX.tar}
   \item {\tt tar xzf MesaDemos-XXX.tar.gz} (optional)
   \item {\tt tar xzf MesaGLUT-XXX.tar.gz} (optional)
   \item {\tt cd Mesa-XXX}
   \item {\tt make} (to see a list of available machines)
   \item {\tt make <machine-name>} where {\tt <machine-name>} is the appropriate choice, such as
        linux-x86 (linux-x86-64). I usually use dynamic libraries.
   \item Goto to one of the example directories (such as demos) and run a test. If you have built
       dynamic libraries then you will first need to set your LD\_LIBRARY\_PATH as indicated by
       the message from Mesa when it finished building.
   \item In previous versions of Overture one needed to build the libGLw library in the
        {\tt cd widgets-sgi} directory. This is no longer necessary since these widgets
      will be built with Overture.
   \item Mesa has off-screen rendering support (for creating hard-copy's) found in libOSMesa.a which should
        automatically be built with Mesa. Overture also needs access to some of the source files in Mesa
        (such as Mesa-XXX/src/mesa/main/context.h)  so your OpenGL environmental variable should point to the
        Mesa source (as opposed to /usr/lib) if you want this off screen rendering. Overture can also
        do off-screen rendering using X pixmaps, but this may not work if you have a graphics card that
        does direct rendering.
%   \item You will need to make some motif widgets that come with Mesa since Overture
%       uses the libGLw library. To make this library:
%      \begin{enumerate}
%        \item {\tt cd widgets-sgi} (i.e. goto the Mesa-3.0/widgets-sgi directory. 
%        \item Edit Makefile.X11 (I hope you use emacs) and change the line in the Makefile:
%         \begin{verbatim}
%     INCDIRS = -I../include -I/usr/include/Motif1.2
%         \end{verbatim}
%         to point to the location of your motif, for example:
%         \begin{verbatim}
%     INCDIRS = -I../include -I$(MOTIF)/include
%         \end{verbatim}
%         where \$MOTIF is the environmental variable as defined for Overture (see below).
%         You should also change the lines
%         \begin{verbatim}
%     #SOURCES = GLwDrawA.c GLwMDrawA.c
%     SOURCES = GLwDrawA.c
%         \end{verbatim}
%         to
%         \begin{verbatim}
%     SOURCES = GLwDrawA.c GLwMDrawA.c
%     #SOURCES = GLwDrawA.c
%         \end{verbatim}
%         so the GLwMDrawA.c file is also compiled.
%        \item Type "make $<$machine-name$>$" (use the same name as above)
%            This will make the libGLw.a library and copy it to the Mesa/lib directory.
%        \item {\tt cp *.h ../include/GL} (Copy the .h files from Mesa/widgets-sgi to the Gl include directory)
%      \end{enumerate}
   \item You may increase the maximum possible resolution that an image can be saved as.
         Mesa is compiled with certain default maximum values
      and the highest resolution depends on the parameters
      that were used to compile Mesa. To increase the Mesa resolution you should
      change the definitions for {\tt MAX\_WIDTH} and {\tt MAX\_HEIGHT} 
      in {\tt Mesa/src/config.h}. For example you may set these to be $2048$ or more.
      Mesa must be re-compiled after these changes are made.
%   \item {\tt setenv OpenGL MESA/Mesa-3.0} (e.g. {\tt setenv /home/henshaw/Mesa-3.0}
%             and add to your .cshrc)
 \end{enumerate}

\clearpage
\section{Installing HDF}

HDF ({\tt http://www.hdfgroup.org/HDF5/}) is a data base package that you may need to install. It may
already be installed on your system.

% Beginning with version 22, Overture can use either HDF4 or HDF5 (decided at the time one configures Overture). 
% HDF5 will be needed if you want to
% run parallel jobs with a grid that is too large to fit in the memory of one processor.

Here are the installation steps for HDF5. I currently use hdf5-1.8.8 (1.6.5 should also work). You will need to build a 
serial version and/or parallel version depending on which version of Overture you are building.
\begin{enumerate}
  \item {\tt tar xzf hdf5-1.8.8.tar.gz}
  \item optionally rename the directory hdf5xxx
  \item {\tt cd hdf5xxx}
  \item {\tt unsetenv CC}
  \item {\tt unsetenv cc}
  \item For the serial version : {\tt setenv CC gcc} (if you want to use gcc).
  \item For the parallel version : {\tt setenv CC \$MPI\_ROOT/bin/mpicc}.  % setenv CC $MPI_ROOT/bin/mpicc
  \item {\tt ./configure --prefix=`pwd`}
  \item Sometimes in parallel I instead need to use {\tt setenv CC gcc} and then {\tt ./configure --prefix=`pwd` --enable-parallel}.
  \item {\tt make}
  \item {\tt make install}
\end{enumerate}


% Be sure to install a release of HDF4 (and not HDF5), I currently use HDF4.2r0.
% Be sure to install HDF4.1r1 or HDF4.1r3. Here is 
% \htmladdnormallink{HDF4.1r3.tar.gz}{\software/HDF4.1r3.tar.gz}. 
% HDF5 has completely new function names and cannot be used.

%- Here are the install notes if you choose to use HDF4 (I currently use HDF4.2r1).
%- \begin{enumerate}
%-    \item {\tt gzip -d HDFxxx.tar.gz}
%-    \item {\tt tar -xf HDFxxx.tar}
%-    \item {\tt cd HDFxxx}
%-    \item {\tt unsetenv CC} 
%-    \item {\tt setenv LDFLAGS -lm} -- this may be necessary to avoid an error about not finding ``sqrt''
%-        when you run the ``make'' step below.
%-    \item The INSTALL file tells you where to get libjpeg.a and libz.a if you don't already have them, the next
%-          step will complain if it can't find these libraries.
%- %    configure --prefix=`pwd`
%-    \item {\tt configure -\,-prefix="this directory"} where "this directory" is the current directory
%-        (or the place you want to install hdf).
%-    \item {\tt make}
%-    \item {\tt make install}
%-    \item At this stage you should see some libraries built in the HDF lib directory.
%- %   \item {\tt cd hdf/test}, {\tt testhdf} ~:~Goto a test directory and run a test
%- \end{enumerate}

\clearpage
\section{Installing A++}

A++/P++ is the serial/parallel array class.
To get A++/P++ you will need to go to the \htmladdnormallink{Overture home page}{http://www.llnl.gov/CASC/Overture}
and register under software.

The A++ installation process requires the gnu autoconf package and gmake, available from the
gnu site \htmladdnormallink{http://www.gnu.org/order/ftp.html}{http://www.gnu.org/order/ftp.html}.

If you have downloaded AP-nnn.tar.gz into directory XXX then the basic steps 
are as follows (if something goes wrong, consult the documentation that comes with A++). 
\begin{enumerate}
    \item {\tt cd XXX}
    \item {\tt gzip -d AP-nnn.tar.gz}
    \item {\tt tar -xf AP-nnn.tar}
    \item {\tt cd A++P++nnn}
%            configure --enable-SHARED_LIBS --prefix=`pwd`
    \item {\tt configure -\,-enable-SHARED\_LIBS -\,-prefix=`pwd`}
%     \item {\tt ./configure --with-CC=insure --with-CXX=insure --enable-CXX_OPT= --enable-C_OPT= --enable-MDI_C_OPT= --enable-SHARED\_LIBS --prefix=`pwd`}
% ./configure --enable-SHARED\_LIBS --prefix=`pwd` --with-CC=xlc --with-CXX=xlC
% ./configure --enable-SHARED\_LIBS --prefix=`pwd` --with-CC=xlc --with-CXX=xlC --enable-CXX_OPT=-g --enable-C_OPT=-g
    \item {\tt make} 
    \item {\tt make install}
    \item Optionally type {\tt make check} to run the A++ test codes.
\end{enumerate}
The libraries will be installed in {\tt XXX/A++P++nnn/A++/install/lib}.
You should use {\tt setenv APlusPlus XXX/A++P++nnn/A++/install } when building Overture.

\clearpage
\section{Installing P++}

P++ is the parallel version of the array class. NOTE: many but not all features of Overture run in parallel. 
{\bf Note} that we build the version of P++ without the newer PADRE implementation. 

To build the P++ library (only required if you are going to build the parallel version of Overture)
\begin{enumerate}
    \item {\tt cd A++P++nnn} (\qquad goto the A++P++ directory)
    \item  Set the enviroment variable {\tt MPI\_ROOT} to point to the mpi directory such as
            {\tt setenv MPI\_ROOT /usr/local/mpi})
    \item Here is the configure statement I used to compile with gcc version 4.3.2 on my dual processor work-station
\begin{verbatim}
configure --enable-PXX --prefix=`pwd` --enable-SHARED_LIBS \
          --with-mpi-include=-I${MPI_ROOT}/include \
          --with-mpi-lib-dirs="-Wl,-rpath,/usr/apps/gcc/4.3.2/lib -L${MPI_ROOT}/lib" \
          --with-mpi-libs="-lpmpich -lmpich -lc -lm"  \
          --with-mpirun=${MPI_ROOT}/bin/mpirun --without-PADRE
\end{verbatim}
    \item Here is a configure statement I used on a Linux cluster that did not use mpirun to start parallel
      jobs. P++ expects to find mpirun by default but you can overide this and provide another script. In the example below 
      I use {\tt mpirun-wdh}, a script that I wrote which calls the appropriate routine to launch parallel jobs. 
\begin{verbatim}
configure --enable-PXX --prefix=`pwd` --enable-SHARED_LIBS \
          --with-mpi-include=-I${MPI_ROOT}/include \
          --with-mpi-lib-dirs="-Wl,-rpath,${MPI_ROOT}/lib -L${MPI_ROOT}/lib" \
           --with-mpi-libs="-lmpi++ -lmpi" \
           --with-mpirun=${HOME}/bin/mpirun-wdh --without-PADRE
\end{verbatim}
    \item Here is the configure statement I used when I couldn't run mpirun on a front end machine so I disabled
the checks that parallel jobs can be executed: 
\begin{verbatim}
configure --enable-PXX --prefix=`pwd` --enable-SHARED_LIBS \
          --with-mpi-include=-I${MPI_ROOT}/include \
          --with-mpi-lib-dirs="-L${MPI_ROOT}/lib" \
          --with-mpi-libs="-lmpi"\
          --disable-mpirun-check \
          --without-PADRE
\end{verbatim}
Here is the configure statement Robert Read used to install P++ on his laptop running fedora and Open MPI
\begin{verbatim}
configure --enable-PXX --prefix=`pwd` --enable-SHARED_LIBS \
          --with-mpi-include=-I/usr/include/openmpi-x86_64 \
          --with-mpi-lib-dirs="-Wl,-rpath,${MPI_ROOT}/lib -L${MPI_ROOT}/lib" \
          --with-mpi-libs="-lmpi -lmpi_cxx -lc -lm" \
          --with-mpirun=${MPI_ROOT}/bin/mpirun --without-PADRE
\end{verbatim}

%
% **** note: change the version of gcc in the next line as appropriate ***
% **** note 2: include the rpath flag for any .so libraries in the FORTARN_LIBS otherwise mpirun tests by configure may fail

% NOTE: Don set the machine file using
%       --with-mpi-machinefile=${MPI_ROOT}/share/machines.ch_mx.LINUX

% *************** for making with gcc 3.4.3 and NO PADRE **************************
% for making with gcc 3.4.3
% configure --enable-PXX --prefix=`pwd` --enable-SHARED_LIBS --with-mpi-include=-I${MPI_ROOT}/include --with-mpi-lib-dirs="-Wl,-rpath,/usr/apps/gcc/3.4.3/lib -L${MPI_ROOT}/lib" --with-mpi-libs="-lpmpich -lmpich ${FORTRAN_LIBS}" --with-mpirun=${MPI_ROOT}/bin/mpirun --without-PADRE

% *************** for making with gcc 3.4.3 -g and NO PADRE **************************
% configure --enable-PXX --prefix=`pwd` --enable-SHARED_LIBS --with-mpi-include=-I${MPI_ROOT}/include --with-mpi-lib-dirs="-Wl,-rpath,/usr/apps/gcc/3.4.4/lib -L${MPI_ROOT}/lib" --with-mpi-libs="-lpmpich -lmpich ${FORTRAN_LIBS}" --with-mpirun=${MPI_ROOT}/bin/mpirun --enable-CXX_OPT=-g --enable-C_OPT=-g --without-PADRE

% *********** for mcr: NO PADRE *************
% configure --enable-PXX --prefix=`pwd` --enable-SHARED_LIBS --with-mpi-include=-I${MPI_ROOT}/include --with-mpi-lib-dirs="-Wl,-rpath,${MPI_ROOT}/lib -L${MPI_ROOT}/lib" --with-mpi-libs="-lmpi" --with-mpirun=${HOME}/bin/mpirun-wdh --without-PADRE
%%

% ============================== OLD =======================================


% with gcc 3.4.4 (ifort64) (no FORTRAN_LIBS needed)
%  configure --enable-PXX --prefix=`pwd` --enable-SHARED_LIBS --with-mpi-include=-I${MPI_ROOT}/include --with-mpi-lib-dirs="-Wl,-rpath,/usr/apps/gcc/3.4.3/lib -L${MPI_ROOT}/lib" --with-mpi-libs="-lpmpich -lmpich " --with-mpirun=${MPI_ROOT}/bin/mpirun --enable-CXX_WARNINGS=""

% for making with gcc 3.4.4
% configure --enable-PXX --prefix=`pwd` --enable-SHARED_LIBS --with-mpi-include=-I${MPI_ROOT}/include --with-mpi-lib-dirs="-Wl,-rpath,/usr/apps/gcc/3.4.4/lib -L${MPI_ROOT}/lib" --with-mpi-libs="-lpmpich -lmpich ${FORTRAN_LIBS}" --with-mpirun=${MPI_ROOT}/bin/mpirun

% for making with gcc 3.4.3 -g 
% configure --enable-PXX --prefix=`pwd` --enable-SHARED_LIBS --with-mpi-include=-I${MPI_ROOT}/include --with-mpi-lib-dirs="-Wl,-rpath,/usr/apps/gcc/3.4.4/lib -L${MPI_ROOT}/lib" --with-mpi-libs="-lpmpich -lmpich ${FORTRAN_LIBS}" --with-mpirun=${MPI_ROOT}/bin/mpirun --enable-CXX_OPT=-g --enable-C_OPT=-g





%configure --enable-PXX --prefix=`pwd` --enable-SHARED_LIBS --with-mpi-include=-I${MPI_ROOT}/include --with-mpi-lib-dirs="-Wl,-rpath,/usr/apps/gcc/3.3.3/lib -L${MPI_ROOT}/lib" --with-mpi-libs="-lpmpich -lmpich ${FORTRAN_LIBS}" --with-mpirun=${MPI_ROOT}/bin/mpirun
%
% configure --enable-PXX --prefix=`pwd` --enable-SHARED_LIBS --with-mpi-include=-I${MPI_ROOT}/include --with-mpi-lib-dirs="-Wl,-rpath,${MPI_ROOT}/lib -L${MPI_ROOT}/lib" --with-mpi-libs="-lmpi" --with-mpirun=${MPI_ROOT}/bin/mpirun
% 
% for mcr:
% configure --enable-PXX --prefix=`pwd` --enable-SHARED_LIBS --with-mpi-include=-I${MPI_ROOT}/include --with-mpi-lib-dirs="-Wl,-rpath,${MPI_ROOT}/lib -L${MPI_ROOT}/lib" --with-mpi-libs="-lmpi" --with-mpirun=${HOME}/bin/mpirun-wdh
% 
% from the P++ directory
% configure --enable-PXX --enable-SHARED_LIBS --prefix=`pwd` --with-mpich=${MPI_ROOT}
%
% edit the file P++/install/Makefile.user.defs
%  MPIINCLUDE     		= -I/usr/local/mpi/include
%  MPILIBDIRS     		= -L/usr/local/mpi/lib
%  MPILIBS        		= -lmpich
%
%%    \item {\tt configure -\,-enable-PXX -enable-SHARED\_LIBS -\,-prefix=`pwd` -\,-with-mpich=\${MPI\_ROOT}}
    \item {\tt make} 
    \item {\tt make install}
    \item Optionally type {\tt make check} to run the P++ test codes.
\end{enumerate}
The libraries will be installed in {\tt XXX/A++P++nnn/P++/install/lib}.
You should use {\tt setenv PPlusPlus XXX/A++P++nnn/P++/install } and {\tt setenv APlusPlus XXX/A++P++nnn/P++/install } 
when building the parallel version of Overture.

{\bf Note:} To change the maximum number of parallel processes that can be used by P++, change the
   file  A++P++nnn/P++/include/maxProcessors.h before building the P++ library.

\vskip\baselineskip
Here are more example builds of P++:
\begin{itemize}
  \item To build on a Linux cluster at LLNL I first ran {\tt mpic++} on a C++ file with the ``-v'' option
    to see all the libraries that were being linked to:
\begin{verbatim}
   mpic++ -v -o conftest -I/usr/local/tools/mvapich2-gnu/include conftest.C
\end{verbatim}
     I then configured P++ using these libraries:
\begin{verbatim}
configure --enable-PXX --prefix=`pwd` --enable-SHARED_LIBS \
   --with-mpi-include=-I${MPI_ROOT}/include \
   --with-mpi-lib-dirs="-Wl,-rpath,${MPI_ROOT}/lib -L${MPI_ROOT}/lib" \
   --with-mpi-libs="-lmpichcxx -lmpich -lpmi -lopa -lmpl -lpthread -lrdmacm \
                    -libverbs -libumad -ldl -lrt -lnuma -lstdc++ -lm" \
   --with-mpirun=${HOME}/bin/mpirun-wdh --without-PADRE
\end{verbatim}
\end{itemize}

%- \clearpage
%- \section{Using gfortran or installing f2c and fort77}
%- 
%- Many of the fortran files in Overture use a fortran compile option that converts single precision code
%- to double precision. Most fortran compilers support this option with the -r8 flag. The gnu fortran
%- compiler g77 does NOT support this option but the newer gnu fortran compiler gfortran does with the
%- {\tt -fdefault-real-8} flag. gfortran is available starting with gcc 4.x.x.
%- 
%- If you do not have gfortran or some other commerical compiler and want to compile Overture 
%- in double precision then you can use f2c and fort77 as described below.
%- 
%- % If you are running on a linux machine I suggest that you 
%- 
%- % Note: We recommend that you use a commercial fortran compiler on Linux
%- 
%-  When you use the gnu fortran compiler g77 and ask for double precision, Overture uses fort77 (a script
%- that uses f2c) to compile fortran files (since the -r8 flag is not supported by g77). 
%- f2c is a nice program that converts fortran to c.
%- f2c and fort77 may be distributed will your version of linux. Type ``{\tt which f2c}'' to see if it
%- is in your path.
%- The f2c library can be obtained from the web, at \htmladdnormallink{netlib}{http://www.netlib.org}, for example.
%- fort77 is a perl script that can be downloaded from the web if you do not
%- already have it. I had trouble with fort77 version 1.15 but 1.14a seemed to work ok except
%- for some warnings that I was able to remove by editing the fort77 file.
%- 
%- Here are versions that worked for me  \htmladdnormallink{f2c.tar.gz}{\software/f2c.tar.gz},
%- \htmladdnormallink{fort77.tar.gz}{\software/fort77.tar.gz}. Here is how to install them
%- \begin{enumerate}
%-    \item {\tt gzip -d f2c.tar.gz}
%-    \item {\tt tar -xf f2c.tar}
%-    \item {\tt cd f2c}
%-    \item edit the makefile and set the compiler and compile flags.
%-    \item {\tt make} ~:~ this should build the {\tt f2c} executable used by fort77
%-    \item {\tt gzip -d fort77.tar.gz}
%-    \item {\tt tar -xf fort77.tar}
%-    \item change your path variable or add some symbolic links so that {\tt f2c} and {\tt fort77}
%-      can be found. If you type ``{\tt which f2c}'' from any directory it should locate {\tt f2c}.
%- \end{enumerate}
% My version of gcc already came with the library {\tt libg2c.a} 

% \clearpage
% \section{Installing STL}
% 
% All current compilers come with STL so there should be no need to install it.
% 
% However, for people who have an older compiler that doesn't support STL (such as Sun 4.2 compilers) you will
%   need to install STL (if you don't have a version already). Most new compilers including gcc support STL,
% which is part of the C++ standard.
% When you configure Overture it will check
%   to see if the compiler supports STL and complain if it doesn't.
%   Here is a version that works for the Sun with CC version 4.2, 
%   \htmladdnormallink{stl.tar.gz}{\software/stl.tar.gz} 
% \begin{enumerate}
%    \item {\tt gzip -d stl.tar.gz}
%    \item {\tt tar -xf stl.tar} ~:~ this completes the installation since STL is all header files.
% %   \item {\tt setenv STL -I$<$this directory$>$/STL/old\_hp -I$<$this directory$>$/STL/SC5} ~:~ tell Overture where to look for STL header files (add this to your .cshrc)
% \end{enumerate}

\clearpage
\section{Installing PETSc (optional)}

Overture can optionally use PETSc to solve sparse matrix problems. You will probably want 
to get PETSc if you want to solve the 3D incompressible Navier-Stokes equations with cgins.

The version of PETSc that works
with Overture.v25 or v22 is PETSc 2.3.2\htmladdnormallink{petsc-lite-2.3.2-p6.tar.gz}{\software/petsc-lite-2.3.2-p6.tar.gz}.
Here are the steps I took to install the (uni-processor version)
on my Linux machine where I use the gcc and PGI fortran compilers:
%   setenv PETSC_DIR `pwd`
%   setenv PETSC_ARCH linux-gnu-opt
%   setenv PETSC_LIB $PETSC_DIR/lib/$PETSC_ARCH
%   ./config/configure.py --PETSC\_ARCH=linux-gnu-opt --with-debugging=0 --with-fortran=0 --with-matlab=0 --with-mpi=0 --with-shared=1 --with-dynamic=1 
% if BLAS is needed: 
%   ./config/configure.py --PETSC\_ARCH=linux-gnu-opt --with-debugging=0 --with-fortran=0 --with-matlab=0 --with-mpi=0 --with-shared=1 --with-dynamic=1 --download-c-blas-lapack=1
\begin{enumerate}
   \item {\tt tar xzf petsc-lite-2.3.2-p6.tar.gz}
   \item {\tt mv petsc-2.3.2-p6 petsc-2.3.2}
   \item {\tt cd petsc-2.3.2}
   \item {\tt setenv PETSC\_DIR `pwd`}
   \item {\tt setenv PETSC\_ARCH linux-gnu-opt}
   \item {\tt setenv PETSC\_LIB \$PETSC\_DIR/lib/\$PETSC\_ARCH}
   \item {\tt setenv CC gcc}
   \item {\tt (for help type: ./config/configure.py --help)}
   \item {\tt ./config/configure.py  --PETSC\_ARCH=linux-gnu-opt --with-debugging=0 --with-fortran=0 \\    
                              --with-matlab=0 --with-mpi=0 --with-shared=1 --with-dynamic=1}
   \item {\tt make}
\end{enumerate}

{\bf Notes for Mac:} Follow the above instructions but use:
\begin{enumerate}
   \item {\tt setenv PETSC\_ARCH macx}
   \item {\tt ./config/configure.py  --PETSC\_ARCH=linux-gnu-opt --with-debugging=0 --with-fortran=0 \\    
                              --with-matlab=0 --with-mpi=0 --with-shared=0 --with-dynamic=0}
\end{enumerate}

% NOTES: MAC

% setenv PETSC_DIR `pwd`
% setenv PETSC_ARCH macx
% setenv PETSC_LIB $PETSC_DIR/lib/$PETSC_ARCH



%  ./config/configure.py  --PETSC\_ARCH=linux-gnu-opt --with-debugging=0 --with-fortran=0 --with-matlab=0 --with-mpi=0 --with-shared=1 --with-dynamic=1


% \vskip2\baselineskip
% {\bf Older version for Overture.v21:} 
% 
% The version of PETSc that works
% with Overture.v21 is PETSc 2.2.1\htmladdnormallink{petsc-2.2.1.tar.gz}{\software/petsc-2.2.1.tar.gz}.
% See also the \htmladdnormallink{PETSc home page}{http://www-fp.mcs.anl.gov/petsc/} where the most recent
% version of PETSc can be found.
% 
% Here are the steps I took for petsc-2.2.1 (uni-processor version)
% on my Linux machine where I use the gcc and PGI fortran compilers.
% \begin{enumerate}
%    \item {\tt gunzip -c petsc-2.2.1.tar.gz | tar xof -}
%    \item {\tt cd petsc-2.2.1}
%    \item {\tt setenv PETSC\_DIR `pwd`}
%    \item {\tt ./config/configure.py --with-mpi=0 --with-matlab=0 --with-fc=pgf77}
%    \item {\tt make BOPT=O}
%    \item {\tt make BOPT=O test}
% \end{enumerate}
% 
% making petsc-lite-2.3.2-p3.tar.gz on zeus, 070518
% 
% *** PARALLEL ****
%     (zeus, 070518)
% 
%     module p (sets PETSC_DIR PETSC_ARCH)
%     tar xzf petsc-lite-2.3.2-p3.tar.gz
%     mv petsc-2.3.2-p3 petsc-2.3.2.gcc3.4.4x86_64.parallel
%     cd petsc-2.3.2.gcc3.4.4x86_64.parallel
%   
%     setenv CC gcc
%     (for help: ./config/configure.py --help)
% 
%     setenv PETSC_ARCH linux-gnu-opt  (done in module)
% 
%  here we have petsc build hypre: 
%     ./config/configure.py  --with-cc=gcc --PETSC_ARCH=linux-gnu-opt --with-debugging=0 --with-fortran=0 --with-matlab=0 --with-mpi-dir=${MPI_ROOT} --LIBS="-Wl,-rpath,${MPI_ROOT}/lib/shared -L${MPI_ROOT}/lib/shared -lmpich" --with-hypre=1 --download-hypre=ifneeded
% 
%   make all test


% 
% \vskip2\baselineskip
% {\bf Older version for Overture.v20:} 
% Here are the steps I took for petsc-2.1.3 on my Linux machine where I use the gcc and PGI fortran compilers.
% \begin{enumerate}
%    \item {\tt gunzip -c petsc.tar.gz | tar xof -}
%    \item {\tt cd petsc-2.1.3}
%    \item Change the following lines in the file bmake/linux\_gcc\_pgf90/packages (assuming you are using the PGI
%      fortran compilers) (here I am using the single processor version of petsc)
% \begin{verbatim}
% #
% # Location of MPI (Message Passing Interface) software
% #
% # wdh MPI_HOME       = /soft/apps/packages/mpich-gnucc-pgf-1.2.1
% # wdh MPI_LIB        = -L${MPI_HOME}/lib -lmpich
% # wdh MPI_INCLUDE    = -I${MPI_HOME}/include
% # wdh MPIRUN         = mpirun -machinefile ${PETSC_DIR}/maint/hosts.local
% #MPIRUN         = /home/bsmith/petsc/bin/mpirun.chiba -showtimesoff -shared
% # wdh added:
% MPI_LIB = ${PETSC_DIR}/lib/lib${BOPT}/${PETSC_ARCH}/libmpiuni.a
% MPI_INCLUDE = -I${PETSC_DIR}/src/sys/src/mpiuni 
% MPIRUN = ${PETSC_DIR}/src/sys/src/mpiuni/mpirun 
% \end{verbatim}
%    \item Set the PETSC enviromental variables (note I am using gcc and the PGI compiler pgf90)
% \begin{verbatim}
% setenv PETSC_DIR /home/henshaw/petsc/petsc-2.1.3
% setenv PETSC_ARCH "linux_gcc_pgf90"
% setenv PETSC_LIB $PETSC_DIR/lib/libO/$PETSC_ARCH
% \end{verbatim}
%    \item {\tt make BOPT=O all} ~~~(to build the PETSc libraries).
% \end{enumerate}

\clearpage
\section{Installing Overture}\label{sec:installingOverture}

Here are instructions to install the Overture library. See also the README file that comes with Overture.
To get Overture you will need to go to the \htmladdnormallink{Overture home page}{http://www.llnl.gov/CASC/Overture}
and register under software.

Overture requires perl for configuration and running tests. If you don't have perl you
can get it for free from \htmladdnormallink{http://www.perl.com}{http://www.perl.com}


Overture needs a collection of environment variables to be set. These variables tell Overture
where to look for other libraries. You can edit the
file {\tt\bf Overture/defenv} to define these variables and then type ``{\tt source defenv}'' to actually set the
variables.
You might want to add them to your {\tt .cshrc} so they get assigned when you open a window (shell).
\begin{itemize}
   \item {\tt setenv XLIBS xxxx} ~:~ Directory holding the {\tt include} and {\tt lib} directories
       for the X window system. For example you might type ``{\tt setenv XLIBS /usr}''.
        If you then type "{\tt ls \$XLIBS/lib}" you should see the X libraries
          such as {\tt libX11.so} or {\tt libX11.a} 
   \item {\tt setenv OpenGL xxxx} ~:~ If you type "{\tt ls \$OpenGL/lib}" you should see the OpenGL libraries
          such as {\tt libGL.so} ({\tt libGL.a}), or {\tt libMesaGL.so} ({\tt libMesaGL.a}) if you use Mesa.
   \item {\tt setenv MOTIF xxxx} ~:~ If you type "{\tt ls \$MOTIF/lib}" you should see the Motif library
          {\tt libXm.so} or {\tt libXm.a} 
   \item {\tt setenv HDF xxx} ~:~ If you type "{\tt ls \$HDF}" you should see the HDF directories including
        {\tt include} and {\tt lib}. If the HDF variable contains the sub-string ``hdf5'' in it, then the
         Overture configure script will know to use HDF5 (otherwise you will need to use the configure option useHDF5).
   \item {\tt setenv APlusPlus xxxx } ~:~ if you type "{\tt ls \$APlusPlus}" you should see the A++
       directories {\tt include} and {\tt lib}; if you type "{\tt ls \$APlusPlus/lib}" you should
       see the A++ libraries and if you type "{\tt ls \$APlusPlus/include}" you should see the A++ include
       files. Here is an example:
       {\tt setenv APlusPlus /home\-/dilbert\-/A++P++\-/A++\-/install}.
   \item {\tt setenv Overture xxxx } ~:~ location of the Overture directory that we are just going
          to install. Here is an example : {\tt Overture /home/henshaw/Overture.v23.d}. 
   \item {\tt setenv CG xxxx } ~: If you install the CG solvers this is the location of the CG source directory.
          Here is an example : {\tt setenv CG /home/henshaw/cg.v23}. 
   \item {\tt setenv CGBUILDPREFIX xxxx} ~: This is the location of the CG executable and object files. You 
           can make this the same as {\tt CG} or choose a different directory location. This option is used
           to compile multiple versions of CG (e.g. serial and parallel) from the same source tree. 
              Here is an example : {\tt setenv CGBUILDPREFIX /home/henshaw/cg.v23.d}
   \item {\tt setenv LAPACK xxx} : CG needs the LAPACK and BLAS libraries. This env variable is the
           directory that holds the {\tt liblapack.a} and {\tt libblas.a} libraries. 
           Example: {\tt setenv LAPACK /home/henshaw/lapack}
%   \item {\tt setenv F2C\_INCLUDE xxx}  ~:~ if you use f2c and fort77 to get double precision, this is 
%      the location of the f2c include files. If you type "{\tt ls \$F2C\_INCLUDE}" you should see
%       the f2c files such as {\tt f2c.h}
%   \item {\tt setenv STL xxx}  ~:~ for users with a C++ compiler that does not include the STL libraries
%      (such as Sun CC version 4.2) this is a list of include paths to look for STL such as
% \begin{verbatim}
%       setenv STL "-I/home/henshaw/STL"
% \end{verbatim}
    \item Set the {\tt LD\_LIBRARY\_PATH} environment variable which is used by the loader when you run an Overture
   program in order to find dynamic libraries (such as libX11.so or libOverture.so):
{\footnotesize
\begin{verbatim}
setenv LD_LIBRARY_PATH ${MOTIF}/lib:${XLIBS}/lib:${HDF}/lib:${OpenGL}/lib:$Overture/lib:${APlusPlus}/lib
\end{verbatim}
}
    For machines with the ``-rpath'' like load flag, it may not be necessary to set the {\tt LD\_LIBRARY\_PATH}
    since this information is stored with the executable. 
\end{itemize}

\noindent Here now are the steps for installation:
\begin{enumerate}
    \item {\tt tar xzf Overture.xxx.tar}
    \item {\tt cd Overture.xxx}
    \item {\tt configure} ~~:~(or {\tt configure [option1] [option2]} ...) (or type {\tt configure -\,-help}).
          If perl is in a funny place then type {\tt perl configure}. The default is to build a debug version
          with double precision (recommended). 
    \item {\tt make} ~:~this will build Overture, takes about 1/2 hour for the default options, a couple of hours
                 for an optimized version. You can also type ``{\tt make -j 2}'' on machines with multiple processors
                 in order to perform a parallel make.
\end{enumerate}

If you would like to {\bf test the distribution} type 
\begin{flushleft}
  ~~~~{\tt check.p} 
\end{flushleft}
from the main Overture directory to build a collection of grids and run a set of regression tests. 
You can also run these test separately (as well as other tests) by: 
\begin{enumerate}
    \item {\tt cd sampleGrids} ~:~goto the sample grids directory
    \item {\tt make} ~:~will build a set of overlapping grids and check to see if they are correctly built.
                This is a good test that Overture is installed and working. 
    \item {\tt cd ../primer}
    \item {\tt make}  ~:~to make the primer examples. Now run some of the primer examples, such as
    {\tt mappedGridExample1} or {\tt mappedGridExample2} etc.
    \item {\tt cd ../tests} ~:~go to the tests directory which holds other tests
    \item {\tt make} ~:~make the test routines. Type {\tt checkop.p} to run a perl script
         to test the operators. 
    \item There are other demos and tests in the {\tt sampleGrids}, {\tt sampleMappings} and {\tt examples}
          directories.
\end{enumerate}
 

\subsection{More comments on installing Overture}

Overture uses the perl script {\tt configure} to build configuration files. Type
\begin{verbatim}
    configure --help
\end{verbatim}
to see the configure options. If perl is not installed where I think it should be then you may have
to type
\begin{verbatim}
    perl configure --help
\end{verbatim}
The default options are ``precision=double'' for double-precision and
``debug'' for compiling with debug information so that you can diagnose problems with a 
debugger. In most cases performance is not seriously degraded by the ``debug'' option although some
programs such as the grid generator ``ogen'' will run quite a bit faster if you configure
with the ``opt'' option for optimization:
\begin{verbatim}
    configure opt
\end{verbatim}

Generally the configure script will automatically determine the machine type. If not you may have
to type
\begin{verbatim}
       configure <machine-name> [options]
\end{verbatim}
The configure script will read the file config/Makedefs.$<$machine-name$>$
and build Makefile's in each directory. If you need to change the Makefile's for your setup
then you should usually change the config/Makedefs.xxx file and re-configure, rather than changing all the
Makefile's.


% Some routines such as
% \begin{enumerate}
%    \item Mapping/DataPointMapping.C
%    \item Mapping/dmpScalar.C
%    \item GridFunction/LaplaceFDCoefficients.C
% \end{enumerate}
% can take a long time to compile with some compilers if the -O optimization flag is
% on. You may have to compile these separately with -g. Files 1-2 I am working on simplifying 
% but if you still cannot compile them with -g then look for the \#ifdef in these files
% where I don't compile the 4th order accurate interpolation and turn this \#ifdef on for your machine
% so that the code for 4th-order interpolation is not compiled.

% \subsection{Notes for compiling on an SGI}
% 
% \begin{enumerate}
%   \item Overture will build a 64 bit version (-64 flag)
%   \item You should set LD\_LIBRARY64\_PATH instead of LD\_LIBRARY\_PATH as in
% \begin{verbatim}
% setenv LD_LIBRARY64_PATH ${MOTIF}/lib:${XLIBS}/lib:${HDF}/lib:${OpenGL}/lib64:$Overture/lib:${APlusPlus}/lib
% \end{verbatim}
%   \item Build the 64 bit version of HDF
%   \item Normally I link to the native OpenGL but if you need to run on an SGI and display on another
%      machine you may want to build Overture with Mesa (depending on your graphics capabilities you may not
%      be able to display graphics on a non-sgi machine). Build the 64 bit version of Mesa if you use it.
% \end{enumerate}

\clearpage
\section{General comments}

Here are some steps that I would take to track down bugs.

\begin{enumerate}
  \item Do the A++ test codes run?
  \item If you are having trouble with the graphics try building the {\tt paperplane} example
     in the {\tt Overture/tests} directory. This example tests whether OpenGL and Motif are working (and does
     not even link to any Overture libraries). 
  \item The perl script {\tt Overture/check.p} will run a series of regression tests. 
  \item Some Overture test codes are in the directory Overture/tests. 
        The perl script {\tt Overture/tests/checkop.p} will
        run regression tests on the operators (these tests are also run by Overture/check.p). 
%     The perl script testGridFunctions.p is supposed to run some tests and
%     compare the results to stored answers (in files *.check). This test
%     routine still needs some work so you may have to interpret the results.
 \item Use gdb or your favourite debugger to see what the problem is. If you don't
    know how to use a debugger you've come to the wrong party.
\end{enumerate}

% ========================================================================================
\clearpage
% ----------- Installation on the Mac ----------------------


\input macInstall.tex

% Overture and cg have been build on Mac OS/X although we are still working on getting
% the configuration files properly set. 
% 
% Here are some comments from Daniel Appelo on the packages he used: 
% {\small
% \begin{verbatim}
% The other packages I use are easily obtainable by fink (built from source):
% bash-3.2$ fink list hdf5
%  i   hdf5                         1.6.5-1012     Library and file format for scientific data
%  i   hdf5-bin                     1.6.5-1012     Library and file format for scientific data
%  i   hdf5-shlibs                  1.6.5-1012     Library and file format for scientific data
% bash-3.2$ fink list openmotif
% (i)  openmotif3                   2.2.4-1        Official Implementation of OSF/Motif
% (i)  openmotif3-bin               2.2.4-1        Official Implementation of OSF/Motif
% (i)  openmotif3-shlibs            2.2.4-1        Official Implementation of OSF/Motif
% bash-3.2$ fink list jpeg
%  i   libjpeg                      6b-17          JPEG image format handling library
%  i   libjpeg-bin                  6b-17          Executables for libjpeg package
%  i   libjpeg-exif                 6b-17          Utilities for manipulating JPEG and EXIF
%  i   libjpeg-shlibs               6b-17          Shared libraries for libjpeg package
% bash-3.2$ fink list gcc
% (i)  gcc43                        4.3.2-1000     GNU Compiler Collection Version 4.3
% (i)  gcc43-shlibs                 4.3.2-1000     Shared libraries for gcc4
% \end{verbatim}


% }
% ========================================================================================
\clearpage
% ----------- Installation on the Blue Gene ----------------------


\input installBlueGene.tex



% ========================================================================================
\clearpage
\section{Installing LAPACK}

CG needs the LAPACK and BLAS libraries. If you need to build these libraries you
should find a version {\tt lapack.tgz} from the web. 

\noindent Here are the steps for installation:
\begin{enumerate}
    \item {\tt tar xzf lapack.tgz} :  Untar the lapack library.
    \item {\tt cd LAPACK}
    \item Copy the appropriate {\tt make.inc.XXX} file from the INSTALL directory into
           {\tt make.inc} in the main directory. Example: {\tt cp INSTALL/make.inc.LINUX make.inc}
    \item Edit the {\tt make.inc} file and define the compiler and compiler options.
    \item Type {\tt make blaslib} to make BLAS.
    \item Type {\tt make lapacklib} to make LAPACK.
    \item CG expects the libraries to be named {\tt liblapack.a} and {\tt libblas.a}
          so you may need to rename the libraries that are built in the main directory.
\end{enumerate}

% ***************************************************************************************
% ******************************* CG ****************************************************
% ***************************************************************************************
\clearpage
\section{Installing CG} \label{sec:installingCG}

Here are instructions to install the CG solvers. CG uses the same enviromental variables 
as Overture. There is no need to configure CG since it finds all the information it
needs from Overture. See the cg {\tt Readme} file for more information. 

\noindent Here are the steps for installation:
\begin{enumerate}
    \item {\tt tar xzf cg.xxx.tar} : untar the cg files. 
    \item Define the CG and CGBUILDPREFIX environmental variables as described in Section~\ref{sec:installingOverture}.
    \item {\tt cd cg.xxx} 
    \item {\tt make} ~:~this will build the default CG solvers.
                 You can also type ``{\tt make -j 2}'' on machines with multiple processors
                 in order to perform a parallel make. Sometimes the parallel make gets
                 confused so you may have to type {\tt make} again after the first one finishes.
    \item Instead of making all the default CG solvers as above, you can also just build individual ones. 
       If you only want to build cgins, for example, go to the {\tt \$CG/ins} directory and type {\tt make}. 
    \item By default the Maxwell solver is not built. Go to the {\tt \$CG/mx} directory and type {\tt make} if you
          want to use it. 
\end{enumerate}

Here are some tests you can run 
\begin{enumerate}
    \item From the main cg directory type {\tt make check} 
   %  (or {\tt make check-quiet} if you want to see less output)
     to run regression tests for each solver.
    \item To test just one solver, say cgins, go to the {\tt \$CG/ins/check} directory and type {\tt make}. 
\end{enumerate}

If you want to change some CG files and rebuild a solver you can use the {\tt \$CG/user} directory
to do this. See the {\tt \$CG/user/Readme} file for more information.


% ========================================================================================
\clearpage
\section{Notes for running Overture in parallel}


The parallel version of Overture is in beta release to friendly users. Not everything works in
parallel but many things do.

To configure and build the parallel version (you need to install P++ first):
\begin{enumerate}
  \item {\tt setenv PPlusPlus xxxx } where xxxx is the location of the P++ installation.
  \item {\tt setenv APlusPlus xxxx } where xxxx is ALSO the location of the P++ installation.
  \item {\tt configure parallel} 
  \item {\tt make} 
\end{enumerate}


\noindent{\bf General notes:}
\begin{itemize}
  \item You will need to have both a serial and parallel version of Overture built.
  \item It is safer to use the serial version of plotStuff to plot grids and plot results in show files, although
        some of the graphics, such as plotting grids and 2d/3d contours, should work in parallel.
  \item Build grids with the serial version of ogen unless you have to build a very large grid. (The parallel version of ogen
        works in many cases but is still under development).
  \item Parallel moving grids work (but not yet parallel deforming grids).
  \item Parallel AMR works (but not yet parallel AMR and parallel moving grids).
  \item Some Mapping's don't work in parallel since I haven't gotten around to writing the
     new mapS and inverseMapS functions that take serial arrays as arguments 
     -- these are usually easy to write. 
  \item Generally it is best to convert some Mapping's (e.g. DataPointMapping, HyperbolicMapping)
       into NurbsMappings for use with parallel (see, for example, {\tt Overture/sampleGrids/loftedBox.cmd}). 
%  \item Overture graphics ``works'' in parallel but should only be used for smaller examples since the
%        graphics routines copy the distributed grids/grid-functions to processor 0 for plotting (and thus there
%        must be enough memory on processor 0 to hold everything).
\end{itemize}


\noindent{\bf PDE solvers that should run in parallel:}
\begin{itemize}
  \item The primer example pwave is a good example of a small parallel code. Many of the
        other primer examples may run in parallel but are generally not good examples of how to write efficient
        parallel code.
  \item The cg solvers cgcns, cgins, cgad, cgmx, cgsm, and cgmp run in parallel but not all options will work. 
        cgasf does not run in parallel yet. 
%   \item There are some issues with the parallel I/O (through hdf5) that we don't fully understand. 
%         There are collective and non-collective read/write modes in hdf5. Non-collective mode is currently the 
%         default and seems more stable. Collective writes, however, are generally much faster. When running cg codes
%         you can turn on the collective modes as follows:
% \begin{verbatim}
%            mpirun -np 2 cgcns -readCollective -writeCollective cicShockg.cmd
% \end{verbatim}
\end{itemize}


\noindent{\bf Developer notes:} If you are writing Overture code to run in parallel here are some notes:
\begin{itemize}
  \item Avoid scalar indexing of a parallel array, $u(i1,i2,i3)$ since this will use communication by default.
        Instead, access the local serial array (see examples in pwave.C for example). 
  \item Avoid P++ array operations except for simple copies.  It is more efficient and robust to grab the serial
        arrays, operate on the serial arrays and then call updateGhostBoundaries. 
  \item There are a variety of parallel utility routines for dealing with parallel array operations,
        grid functions and parallel copies etc. See the files {\tt App.\{h,C\}} and {\tt ParallelUtility.\{h,C\}}.
\end{itemize}


\end{document}
















Before installing Overture you will need to get the supporting software.
You will need A++, OpenGL and HDF. See the instructions on the web site 
for more details. 


--------------------------------------------------------------------------------------------------
(1)  Setting Environmental variables.

Before making Overture you should define the following environmental variables:
    
Overture  = directory for Overture
APlusPlus = directory for A++, the array class library -- this directory should contain a lib and include directory
OpenGL    = directory for OpengGL -- this directory should conatin a lib and include directory
HDF       = directory where HDF exists  -- this directory should contain a lib and include directory
MOTIF     = directory in which the Motif lib and include directories sit
XLIBS     = directory in which the X window lib and include directories sit

PARALLEL_LIB   = libraries needed by the parallel version of Overture (future versions only)
F2C_INCLUDE = only needed for compiling with gcc and double precision. This is the location
              of the f2c include files.
    
For example, in your .cshrc you can put something like   
   setenv Overture  /home/henshaw/Overture.v18
   setenv APlusPlus /users/dquinlan/A++P++.v1.2.34/A++/lib/solaris_cc_CC
   setenv OpenGL    /users/henshaw/OpenGL/Mesa 
   setenv HDF       /users/henshaw/hdf
   setenv MOTIF     /usr/dt 
   setenv XLIBS     /usr/openwin
    
   setenv PPlusPlus /users/dquinlan/A++P++.v1.2.34/P++/lib/solaris_cc_CC
   setenv PARALLEL_LIB  

In future (as yet unreleased) versions you will need something like 
the following for running in parallel:
   setenv PARALLEL_LIB   -L/usr/local/mpi/mpich-1.1.1 -lmpi

Remember to run 'source .cshrc' after changing the .cshrc file so the new variables are defined.
NOTE: See also the primer documentation, the section on Makefiles and
.cshrc files, for some examples on how to set environmental variables. 

You will need to set your LD_LIBRARY_PATH so that the loader will know where to  look
for the dynamically loaded Overture libraries. This may look something like

setenv LD_LIBRARY_PATH  ${MOTIF}/lib:${XLIBS}/lib:${HDF}/lib:${OpenGL}/lib:$Overture/lib:${APlusPlus}/lib

--------------------------------------------------------------------------------------------------

(2) Make Overture
  Type "configure" with no arguments and read the instructions there. Or type "perl configure"
  if perl is not located in "/usr/local/bin/perl".
    Then type 
       configure [options]
    Then type
       make

I usually configure with
       configure double debug
to build a double precision version with debugging. The debug version compiles faster (approximately 1/2 hour)
compared to an optimized version (1-2 hours). 

The compile options for different machines are defined in the files
in the config directory.

The Overture/bin directory contains the overlapping grid generator,
ogen, and the plotting routine, plotStuff, which can be used to plot
results ("show" files) or grids.

--------------------------------------------------------------------------------------------------

(3) Make the primer examples (optional)

   Go to the Overture/primer directory and type
      make 
   to build the primer examples. Or you can type 
      make <name>
   to build an example with a given name.

A good place to start is to compile and run the examples in the primer
directory. See the primer documentation that can be obatined from
Bill's home page under the Documentation directory.


-----------------------------------------------------------------------------------------------

Note for compiling on an HP:
  
  In A++P++.v0.6.8f I had to change src/array.C :

array.C: (3 places, intArray and floatArray too)
// *wdh * int doubleArray::DISPLAY_FORMAT = doubleArray::SMART_DISPLAY_FORMAT;
int doubleArray::DISPLAY_FORMAT = 2;   











======General install notes===============================================================

 o first type 
       configure
   to see the configure options. Then type something like
       configure <machine-name> [options]

   The <machine-name> is normlly not required as configure will figure this out in most cases.
   You may have to type "perl configure" to run the script as it is a perl script.
   The configure script will read the file config/Makedefs.<machine-name>
   and build Makefile's in each directory. If you need to change the Makefile's for your setup
   then you should usually change the config/Makedefs.xxx file and re-configure, rather than changing all the
   Makefile's.



Some routines such as
     1.  Mapping/DataPointMapping.C
     2.  Mapping/dmpScalar.C
     3.  GridFunction/LaplaceFDCoefficients.C
   can take a long time to compile with some compilers if the -O optimization flag is
on. You may have to compile these separately with -g. Files 1-2 I am working on simplifying 
but if you still cannot compile them with -g then look for the #ifdef in these files
where I don't compile the 4th order accurate interpolation and turn this #ifdef on for your machine
so that the code for 4th-order interpolation is not compiled.




------------------------------------------------------------------------------------------
 
 NOTES on installing Mesa (OpenGL)

If you are using the Mesa libraries for OpenGL then:

  (1) I suggest that you make dynamic libraries (such as sunos5-sl on a sun).
  (2) You will need to make some motif widgets that come with Mesa since Overture
      uses the libGLw library. To make this library:
      (a) Goto the Mesa/widgets-sgi directory. (You may need to download the MesaDemos
          part of Mesa, depending on the version of Mesa)  Change the line in the Makefile:
               INCDIRS = -I../include -I/usr/include/Motif1.2
          to point to the location of your motif, for example:
               INCDIRS = -I../include -I$(MOTIF)/include
          where $MOTIF is the environmental variable as defined for Overture (see below).
      (b) Type "make <machine-name>" 
          This will make the GLw library and copy it to the Mesa/lib directory.
          
      (c) Copy the .h files from Mesa/widgets-sgi to the Gl include directory,
              cp *.h ../include/GL
------------------------------------------------------------------------------------------


NOTES on installing HDF:

 (0) ./configure -v --prefix="this directory"
     To compile the library and utilities type:
        make
    To install the libraries, utilities, includes and man pages type: e.g.
        make install

 (1) To create dynamic libraries (this is optional) for HDF I did the following steps: 
   (a) Changed the CFLAGS macro in the appropriate configure file in the hdf config directory
       to compile for position-independent-code, on a Sun ultra I changed config/mh-solaris:
          CFLAGS= -Xc -O -KPIC
   (b) built the hdf libraries
   (c) copied Overture/DataBase/Makefile.hdf to hdf/lib/Makefile
   (d) typed "make" in hdf/lib to build the dyaynmic libraries from the static libraries.


   (e) On a pentium machine running linux I had to change the CFLAGS macro in config/mh-linux to
         CFLAGS=-ansi -DINTEL86

------------------------------------------------------------------------------------------


NOTES on f2c and fort77 - double precision and gcc:

   (1) When you use gcc (g++) and ask for double precision, Overture uses fort77 (a script
       that uses f2c) to compile fortran files (since the -r8 flag is not supported by g77). 
       The f2c library can be obtained from the web, at www.netlib.org, for example.
       fort77 is a perl script that can be downloaded from the web if you do not
       already have it. I had trouble with version 1.15 but 1.14a seemed to work ok except
       for some warnings that I was able to remove by editing the fort77 file.






======Linux install notes:===============================================================

 *******Newer versions of Linux (glibc linux or Red-hat 5.0 or later) ********

   o There are some problems with using dynamic libraries and fortran with g77.
     I had to compile with -g to get it to work. This is the default in config/Makedef.linux

