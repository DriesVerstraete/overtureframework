\section{Installing Overture on Mac OS/X}  \label{sec:installMac}

\subsection{OS X 10.9 - Mavericks} 

\vskip\baselineskip
\noindent{\bf Stage I:}
The default compiler for Mavericks is {\em clang} which replaces gcc and g++. DO NOT install gcc. 
\begin{enumerate}
  \item Install Xcode from the apple web site.
  \item Install the command line tools in Xcode: 
    \begin{verbatim}
       xcode-select --install
    \end{verbatim}
    This command will ask you whether you want to install the command lines tools. You will know if you have the
    command lines tools if you can issue the command {\tt gcc --version}. 
  \item Install gfortran from {\tt http://gcc.gnu.org/wiki/GFortranBinaries\#MacOS}
  \item Install XQuartz (if you do not already have it). This should install X11 libraries in {\tt /opt/X11/lib}. This is needed
                 so you can pop up X windows but since there is no Motif here we will need to install another version using Macports.
  \item Install Macports
  \item Use Macports to install Open Motif and OpenGL:
     \begin{verbatim}
         sudo port install openmotif
         sudo port install mesa
    \end{verbatim}
    This will install another version of X11 in {\tt /opt/local/lib} in addition to the Open Motif and Mesa libraries. 
\end{enumerate}


\vskip\baselineskip
\noindent{\bf Stage II:} 
\begin{enumerate}
  \item Install HDF following the normal instructions.
  \item Install A++ following the normal instructions but,  
    you may need to make a symbolic link for gmake (somewhere in your path)
    \begin{verbatim}
       ln -s /usr/bin/make /home/dilbert/bin/gmake
    \end{verbatim}
    and configure with 
    \begin{verbatim}
       ./configure  --disable-SHARED_LIBS  --prefix=`pwd`
    \end{verbatim}
  \item Install Overture and cg following the regular instructions and using
    \begin{verbatim}
       setenv XLIBS /opt/local
       setenv MOTIF /opt/local
       setenv OpenGL /opt/local
    \end{verbatim}
    as the locations for X11, Motif and OpenGL  (do NOT accidently link to /opt/X11).
\end{enumerate}

\clearpage
\subsection{Installing Overture on Mac OS/X -- *OLDER VERSIONS*}  

Overture and cg have been build on Mac OS/X. Thanks to Kyle for working this out.
The installation process is very similar to the process for Linux after some
initial steps to install the appropriate compilers etc.. 

\vskip\baselineskip
\noindent{\bf Stage I:}
\begin{enumerate}
  \item Install and update the Apple developer tools (an extra DVD that came with the mac).
  \item Install macports, see, http://www.macports.org/
  \item Using macports, install hdf, mesa, glw, petsc, etc using:
  \begin{enumerate}
    \item port install gcc4.3 hdf5 petsc mesa glw motif
  \end{enumerate}   
  Here we have installed version 4.3 of gcc. 
%
 \item It is convenient to make soft links for the 4.3 compilers that macports installs.  Go to the
    MacPorts bin directory and make links like
   \begin{enumerate}
     \item {\tt ln -s gcc-mp-4.3 gcc}
   \end{enumerate}
   Do this for gcc, g++ and gfortran.
\end{enumerate}

\noindent{\bf Stage II: Install A++P++:}
\begin{enumerate}
  \item Download A++P++, Overture and cg from the Overture web site.
  \begin{enumerate}
    \item Unpack A++: {\tt tar xzf AP-nnn.tar.gz}
    \item Rename A++ directory if desired: {\tt mv A++P++-nnn  A++P++-nnn-gcc4.3}
    \item {\tt setenv CC gcc; setenv CXX g++; setenv FC gfortran}

    \item {\tt ./configure --with-CC="\$CC -I/usr/include/malloc -mlongcall" --with-CXX="\$CXX -I/usr/include/malloc -mlongcall" --disable-SHARED\_LIBS  --prefix=`pwd`}
    \item {\tt make MAKE=make install}
    \item {\tt setenv APlusPlus `pwd`/A++/install; setenv PPlusPlus \$APlusPlus}
  \end{enumerate}
\end{enumerate}

\noindent{\bf Stage III:} Install Overture and CG: follow the instructions in Sections~\ref{sec:installingOverture} and~\ref{sec:installingCG}
for installing Overture and CG. 

Notes:
\begin{enumerate}
    \item You should add the Overture/lib directory to your dynamic library path:
       {\tt setenv DYLD\_LIBRARY\_PATH \${DYLD\_LIBRARY\_PATH}:\${Overture}/lib}
\end{enumerate}


% \begin{enumerate}
%   \item Unpack Overture
%   \item Assign Overture environmental variables (see Section~\ref{sec:installingOverture} of these instructions).
%   \item configure Overture and make.
% \end{enumerate}
%- 
%- 5) configure and install Overture
%- 
%- I recently purchased a new macbook pro and installed Overture and cg on it. This time
%- I used MacPorts instead of fink to install the supporting libraries and tools.  Here
%- is how I did it:
%- 
%- 1) Install macports, see
%- http://www.macports.org/
%- 
%- 2) Using macports, install hdf, mesa, glw, petsc, etc using :
%- 
%- % port install gcc4.3 hdf5 petsc mesa glw motif
%- 
%- 2.1) it is convenient to make soft links for the 4.3 compilers that macports installs.  Go to the
%- MacPorts bin directory and make links like
%- % ln -s gcc-mp-4.3 gcc
%- do this for gcc, g++ and gfortran
%- 
%- 3) download A++P++, Overture and cg
%- 
%- 4) Install A++P++
%- 4.1) unpack A++P++ somewhere and configure it using
%- % configure --prefix=`pwd`/gcc4.3
%- ./configure --with-CC="/Users/chand1/MacPorts/bin/gcc -I/usr/include/malloc" --with-CXX="/Users/chand1/MacPorts/bin/g++ -I/usr/include/malloc" --prefix=/Users/chand1/Overture/A++P++/A++P++-0.7.9d/gcc4.3.3
%- 
%- 
%- 5) configure and install Overture
%- 
%- mailto: chand1@llnl.gov
%- phoneto: (925) 422 7740 
%- 
%- 
%- ===============================
%- 
%- 
%- Here are some instructions for building Overture.v22 and cg.v22 on a mac.  
%- You will need to have fink installed (and fink Commander if you wish).   Note that
%- there are two tar files attached: 
%- ov22_osx_build_patch.tar - should be untared in the Overture.v22 directory
%- cgv22_osx_build_patch.tar - should be intared in the cg.v22 directory
%- 
%- good luck!
%- Kyle
%- 
%- Overture and cg installation instructions for Mac OS X
%- 
%- system info:
%- Mac OS X 10.4.10 (PPC G4)
%- Fink 0.8.1.rsync powerpc
%- FinkCommander 0.5.4 (.release)
%- 
%- steps:
%- 0. install and update the Apple developer tools (an extra DVD that came with the mac)
%- 1. download A++P++-0.7.9d.wdh061229.tar.gz, Overture v22 and cg.v22
%- 2. install hdf5 and openmotif using fink or fink commander (and X11 if you did not get it with the developer tools)
%- 3. install gcc42 and gcc42-shlibs using fink (NOTE: I had to do a Source->selfupdate-rsync to get gcc42 to appear in fink commander).  The compilers will be /sw/bin/g++-4 and /sw/bin/gfortran.
%-    3.1 to make Overture's configure work:
%-         % cd /sw/bin; 
%-         % sudo ln -s gcc-4 gcc; 
%-         % sudo ln -s g++-4 g++
%-         % set path = ( /sw/bin $path ) ; rehash
%-         - you need to make "g++" default to g++-4, 
%-           ie `which g++` should return something that points to /sw/bin/g++-4
%- 4. unpack and install A++P++
%-    4.1 untar/gz A++P++
%-    4.2 % cd A++P++-0.7.9d
%-    4.3 % setenv CC /sw/bin/gcc-4; setenv CXX /sw/bin/g++-4; setenv FC gfortran
%-    4.4 % configure --with-CC="$CC -I/usr/include/malloc -mlongcall" --with-CXX="$CXX -I/usr/include/malloc -mlongcall" --disable-SHARED_LIBS  --prefix=`pwd`/gcc-4.2
%-    4.5 % make MAKE=make install
%-    4.6 % setenv APlusPlus `pwd`/gcc-4.2/A++/install; setenv PPlusPlus $APlusPlus
%- 5. unpack and install Overture
%-    5.1 untar/gz Overture.v22.tar.gz
%-    5.2 % cd Overture.v22
%-    5.3 % setenv HDF /sw; setenv MOTIF /sw; setenv XLIBS /usr/X11R6; setenv OpenGL $XLIBS; setenv Overture `pwd`
%-    5.4 untar ov22_osx_build_patch.tar in the $Overture directory
%-    5.5 % configure darwin precision=double CC=g++ cc=gcc FC=gfortran useHDF5
%-    5.6 % make
%-    5.7 % setenv DYLD_LIBRARY_PATH ${DYLD_LIBRARY_PATH}:${Overture}/lib
%-    5.8 to generate some grids: % cd sampleGrids; perl generate.p 
%- 6. unpack and install cg
%-    6.1 untar/gz 
%-    6.2 % cd cg.v22
%-    6.3 untar cgv22_osx_build_patch.tar in  the cg.v22 directory
%-    6.3 % setenv OV_OPENGL_HEADERS "-I$XLIBS/include"
%-    6.4 % make mode=concise OS=Darwin
%-    6.5 % setenv DYLD_LIBRARY_PATH ${DYLD_LIBRARY_PATH}:`pwd`/common/lib:`pwd`/cns/lib:`pwd`/ins/lib:`pwd`/asf/lib:`pwd`/ad/lib:`pwd`/mp/lib
%-    6.6 % make OS=Darwin check 
%- 
%- Notes: 
%- - % indicates the following text is a shell (csh) command. You can
%-     copy and paste the text following % into your shell to execute 
%-     the command.
%- - The instructions assume the csh shell.  To set variables in sh, replace
%-   setenv VARIABLE VALUE
%-   with
%-   export VARIABLE=VALUE
%- - On my G4 machine it took all day to build gcc42, so be patient...
%- 
%- -- Kyle K. Chand mailto:chand1@llnl.gov phoneto: (925) 422 7740 
